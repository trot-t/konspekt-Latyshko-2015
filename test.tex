\documentclass[a4paper,11pt]{report}
\usepackage[T1,T2A]{fontenc}
\usepackage{ucs}
\usepackage[utf8x]{inputenc}
\usepackage{latexsym}
%\usepackage{amsmath}
%\usepackage{mathtext}
\usepackage[english,ukrainian,russian]{babel}
\usepackage{tikz}
\usepackage{siunitx}
\usetikzlibrary{patterns}
\usetikzlibrary{positioning}
\usetikzlibrary{decorations.text}
\usetikzlibrary{decorations.pathmorphing}
\usetikzlibrary{snakes}
\usepackage[american,cuteinductors,smartlabels]{circuitikz}
\usepackage{enumitem}
\usepackage{relsize}
\usepackage{wrapfig}
\usepackage{float}
\usepackage{graphicx}
\graphicspath{{./images/}}
\usepackage{hyperref}
\usepackage{cancel}
\usepackage{pgfplots}
\usetikzlibrary{arrows.meta,calc,decorations.markings,math,arrows.meta}
\usepackage{pdfpages}
\usepackage{booktabs}
% http://www1.combinatorics.org/Information/article.html
\author{ Латышко Владимир Данилович \\
Санкт-Петербургский государственный электротехнический\\университет "ЛЭТИ"\\
%\small \texttt{lvd@lvd.spb.ru }}
Прокшин Артем \\
Санкт-Петербургский государственный электротехнический\\университет "ЛЭТИ"
%\small \texttt{taybola@gmail.com}
}

\usetikzlibrary{calc}
\usetikzlibrary{decorations.pathreplacing,intersections,calc}
\ctikzset{bipoles/thickness=1}
\ctikzset{bipoles/length=0.8cm}
\ctikzset{bipoles/diode/height=.375}
\ctikzset{bipoles/diode/width=.3}
% \ctikzset{tripoles/thyristor/height=.8}
% \ctikzset{tripoles/thyristor/width=1}
\ctikzset{bipoles/vsourceam/height/.initial=.7}
\ctikzset{bipoles/vsourceam/width/.initial=.7}
\tikzstyle{every node}=[font=\small]
\tikzstyle{every path}=[line width=0.8pt,line cap=round,line join=round]

\AddEnumerateCounter{\Asbuk}{\@Asbuk}{\CYRM}
\AddEnumerateCounter{\asbuk}{\@asbuk}{\cyrm}
\renewcommand{\theenumi}{(\asbuk{enumi})}
\renewcommand{\labelenumi}{\asbuk{enumi})}


\title{Конспект лекций по дисциплине Силовая электроника}
%\date{}
\renewcommand{\chaptername}{Лекция}
\begin{document}

\tikzset{component/.style={draw,thick,circle,fill=white,minimum size =0.75cm,inner sep=0pt}}
%http://tex.stackexchange.com/questions/132076/how-to-draw-a-dc-motor-in-circuitikz

%http://tex.stackexchange.com/questions/132076/how-to-draw-a-dc-motor-in-circuitikz`
% prepare to create bipoles

\makeatletter
\def\TikzBipolePath#1#2{\pgf@circ@bipole@path{#1}{#2}}
\def\CircDirection{\pgf@circ@direction}

%\pgf@circ@Rlen = \pgfkeysvalueof{/tikz/circuitikz/bipoles/length}

\makeatother 

\newlength{\ResUp} \newlength{\ResDown}
\newlength{\ResLeft} \newlength{\ResRight}

% set default motor size

\ctikzset{bipoles/motor/height/.initial=.8}
\ctikzset{bipoles/motor/width/.initial=.8}

% create motor shape

\pgfcircdeclarebipole{}
 {\ctikzvalof{bipoles/motor/height}}
 {motor}
 {\ctikzvalof{bipoles/motor/height}}
 {\ctikzvalof{bipoles/motor/width}}
 {
    \pgfsetlinewidth{\ctikzvalof{bipoles/thickness}\pgfstartlinewidth}
    \pgfextractx{\ResRight}{\northeast}
    \pgfextracty{\ResUp}{\northeast}
    \pgfextractx{\ResLeft}{\southwest}
    \pgfextracty{\ResDown}{\southwest}

  \pgfpathmoveto{\pgfpoint{0.775\ResLeft}{0.2\ResDown}}
\pgfpathlineto{\pgfpoint{\ResLeft}{0.2\ResDown}}
  \pgfpathlineto{\pgfpoint{\ResLeft}{0.2\ResUp}}
  \pgfpathlineto{\pgfpoint{0.775\ResLeft}{0.2\ResUp}}
\pgfpathellipse{\pgfpointorigin}{\pgfpoint{0.8\ResRight}{0cm}}
    {\pgfpoint{0cm}{0.8\ResUp}}
  \pgfpathmoveto{\pgfpoint{0.775\ResRight}{0.2\ResDown}}
    \pgfpathlineto{\pgfpoint{\ResRight}{0.2\ResDown}}
  \pgfpathlineto{\pgfpoint{\ResRight}{0.2\ResUp}}
  \pgfpathlineto{\pgfpoint{0.775\ResRight}{0.2\ResUp}}
  \pgfusepath{draw} %draw motor
    \pgftext[rotate=-\CircDirection]{\textsf{M}}
 }

% create motorto-path style

\def\motorpath#1{\TikzBipolePath{motor}{#1}}
\tikzset{motor/.style = {\circuitikzbasekey, /tikz/to path=\motorpath, l=#1}}

% end of setup

 

%\maketitle
\hspace{-3 cm}
\begin{tikzpicture}
  \begin{scope}[xscale=2,yscale=3]
    \newcommand{\alphaa}{106.0 * pi / 180}
    \newcommand{\gammaa}{21.0 * pi / 180}

    \draw[thin, ->] (-0.2, 0) -- (5.8,0) node[right] {$\omega t$};
    \draw[thin, ->] (0,0) -- (0,1.3) node[left] {$U$};
    
    \foreach \x/\xtext in {0, {pi/3}/{\frac{\pi}{m}},{pi}/\pi}
    \draw (\x,0.1) -- (\x,-0.1) node [below] {$\xtext$};

    % A,B,C
    \draw[domain=-0.2:5.8, smooth, yellow]
    plot (\x,{cos((\x) r)}); \node[above] at (0.2,1) {$e_k$};
    \draw[domain=-0.2:5.8, help lines, smooth, green]
    plot (\x,{cos((\x-2/3*pi) r)}); \node[above] at (2*pi/3+0.2,1) {$e_{k+1}$};
    \draw[domain=-0.2:5.8, help lines, smooth]
    plot (\x,{cos((\x-4/3*pi) r)}); \node[above] at (4*pi/3+0.2,1) {$e_{k-1}$};
    % (A+B)/2 
    \draw[domain={pi/3:pi/3+pi}, loosely dotted]
    plot (\x,{cos(\x r)/2 + cos((\x - 2/3*pi) r)/2});


    
    \foreach \qq [evaluate=\qq as \qqshft using \qq*2*pi/3] in {0,...,1}
     {
      \begin{scope}[xshift=\qqshft cm,
          every path/.style={color=red}]
        
        %Ed
%        \draw[domain={-pi/3 + \alphaa + \gammaa}:{pi/3},ultra thick]
%        plot (\x, {cos(\x r)});
        \draw[domain={\alphaa+\gammaa}:{\alphaa + pi/3},ultra thick]
        plot (\x,{cos(\x r)})
        -| (\alphaa+pi/3, {cos((\alphaa + pi/3) r)/2 + cos((\alphaa-pi/3) r)/2})
        [domain={\alphaa+pi/3}:{\alphaa+pi/3+\gammaa}]
        plot (\x,{cos(\x r)/2 + cos((\x-2/3*pi) r)/2})
        -| (\alphaa+pi/3+\gammaa, {cos((\alphaa +pi/3+ \gammaa-2*pi/3) r) })

        [domain={\alphaa+\gammaa-pi+2*pi/3}:{\alphaa-pi/3+2*pi/3}]
          plot(\x,  {cos((\x) r)})
        ;        
       \end{scope}
     }

     \node[below] at (pi+0.3, -0.6) {${\displaystyle \frac{e_k + e_{k+1}}{2}}$};
     \draw[thin,->] (pi+0.2,-0.6) --
     (pi+0.3,{cos((pi+0.3) r)/2 + cos((pi-2*pi/3+0.3) r)/2 -0.1});
     % alpha
     \draw[thin] (pi/3, 0.35) -- (pi/3,0.8);
     \draw[thin] (pi/3+\alphaa,0.1) -- (pi/3+\alphaa, 0.8);
     \draw[thin,->] (pi/3, 0.75) -- (pi/3+\alphaa,0.75);
     \node[left] at (pi/3, 0.75) {$\alpha$};
     % beta
     \draw[thin] (4*pi/3,-0.4) -- (4*pi/3,0.8);
     \draw[thin,<->] (pi/3+\alphaa, 0.6)--(4*pi/3, 0.6) node[right]
           {$\beta=\pi-\alpha$};
     % U_инвертора
     \draw[thin,color=red] (pi/2,-0.36) -- (5*pi/6+pi,-0.36) node[right]
     {$U_{\textcyrillic{инв}}$};

  \end{scope}
  \end{tikzpicture}    

За счет чего течёт ток?
\begin{figure}[H]
\begin{circuitikz}\draw
(0,4) to[L] (0,2)
node[left] at (0,3.5) {-}
node[left] at (0,2.5) {+}
node[left] at (0,2) {${\textcyrillic{было}}$}
node[right] at (0,3.5) {(+)}
node[right] at (0,2.5) {(-)}
node[right] at (0,2) {${\textcyrillic{стало}}$}
to[Ty,mirror] (0,0)
(2,4) to[L,*-] (2,2)
to[Ty,mirror] (2,0)
(4,4) to[L,*-] (4,2)
to[Ty,mirror] (4,0)
(0,4) to[short,-*] (2,4)
to[short,-*] (4,4)
to[short] (8,4)
(0,0) to[short,-*] (2,0)
to[short,-*] (4,0)
to[short] (8,0)
to (8,1)
to[american voltage source] (8,3)
to[short] (8,4)
node[right] at (8,2.5) {(+)}
node[right] at (8,1.5) {(-)}
node[right] at (8.2,2) {$E_{\textcyrillic{н}}{\textcyrillic{ЭДС нагрузки}}$}
node[right] at (8,0.5) {$
\begin{array}{c}
{\textcyrillic{все напряжения}}\\
{\textcyrillic{отсчитываем}}\\
{\textcyrillic{относительно}}\\
{\textcyrillic{нулевого провода}}
\end{array}
$}
;\end{circuitikz}
\end{figure}
Все напряжения отсчитываются относительно нулевого провода и поэтому
 ток течёт за счёт ЭДС нагрузки!

$\alpha$ двигаем дальше, от этого положительная часть уменьшается,
а отрицательная увеличивается. Если энергия течёт в сеть, значит
\begin{circuitikz}\draw
node[right] at (0,0) {$e_{\textcyrillic{н}}$}
node[left] at (0,0.2) {+}
node[left] at (0,-0.2) {-}
;\end{circuitikz}.

Теперь рисуем картинку при $\alpha>90^\circ$ Введем угол $\beta$:
$$
\beta = \pi - \alpha
$$

$\alpha$ -- угол запаздывания, $\beta$ -- угол опережения.
Рисуем ток.
\begin{figure}[H]
\begin{tikzpicture}
  \begin{scope}[xscale=2,yscale=3]
      \newcommand{\alphaa}{106.0 * pi / 180}
          \newcommand{\gammaa}{21.0 * pi / 180}

    \draw[thin, ->] (-0.2, 0) -- (5.8,0) node[right] {$\omega t$};

    \draw[domain=-0.2:1.8, smooth, yellow]
        plot (\x,0)
    [domain=1.8:2.2, smooth, yellow]
         plot (\x, {1/0.4*(\x-1.8)})
    [domain=2.2:3.8, smooth, yellow]
         plot (\x,1)
    [domain=3.8:4.2, smooth, yellow]
         plot (\x, {-1/0.4*(\x-4.2)})
    [domain=4.2:5.8, smooth, yellow]
         plot (\x,0);
    \draw[domain=-0.2:1.8, help lines, smooth, green]
         plot (\x,1)
    [domain=1.8:2.2, help lines, smooth, green]
        plot (\x,{-1/0.4*(\x-2.2)})
    [domain=2.2:3.8, help lines, smooth, green]
         plot (\x,0)
    [domain=3.8:4.2, help lines, smooth, green]
        plot (\x, {1/0.4*(\x-3.8)})
    [domain=4.2:5.8, help lines, smooth, green]
         plot (\x,1);
    %
    \draw[domain=1.8:3.6, smooth, dashed, yellow]
    plot (\x, {cos(1.4*(\x - 2.7) r) - cos(1.4*(1.8 - 2.7) r)});
    \draw[domain=1.8:3.6, help lines, smooth, loosely dashed,green]
    plot (\x, {-cos(1.4*(\x - 2.7) r) + cos(1.4*(1.8 - 2.7) r)+1});
    %
    \draw (2.7,0.1) -- (2.7,-0.1) node[below] {$\pi$};
\end{scope}
\end{tikzpicture}        
\end{figure}

$\beta>0$ Если не выключить $k$-й вентиль до точки пересечения, то он уже
никогда не выключится.
Если $L$ большая, коммутация не кончится. Темп изменения тока пропорционален
ЭДС. Но ЭДС, а значит и скорость нарастания тока уменьшаются, а дальше
скорость станет отрицательной, а значит, коммутация не будет продолжаться.

Коммутация вентилей должна закончиться до момента отсчета угла $\beta$.
В противном случае переключения фаз не произойдёт. $\beta$ не просто
больше 0, а $\beta>\gamma$. Предположим, что разность $\beta-\gamma$ мала.
Сравним её со временем выключения вентиля. Вентиль может самопроизвольно
включиться $I\approx n$ носителей. $I$ убывает, но рекомбинации носителей
не произошло(вентиль не восстановился).
$\beta-\gamma>{\textcyrillic{времени запирания тиристора}}$.
$\omega t_{\textcyrillic{выкл}}= \sigma$ -- угол запирания или выключения вентиля.

$\beta>\gamma+\sigma$ -- каждый раз условие увеличивается.

Практически, учитывается наличие несимметрии напряжения сети, несимметрии
углов регулирования $\alpha$ (несимметрия СИФУ), с учётом несинусоидальности
и разброса параметров тиристора необходим запас по углу для устойчивой
работы инвертора ($\psi$ -- угол запаса)

\begin{equation}
\beta \ge (\gamma+\beta+\psi) = \beta_{min}
\label{steady_invertor}
\end{equation} -- условие устойчивой работы инвертора.
отсюда, $\alpha_{max} = \pi - \beta_{min}$


\begin{tikzpicture}
  \begin{scope}[xscale=1.5,yscale=1.5]
      \newcommand{\alphaa}{106.0 * pi / 180}
      \newcommand{\gammaa}{21.0 * pi / 180}
    % A,B,C
        \draw[domain=0.1:1.4, help lines, smooth, black]
            plot (\x,{cos((\x) r)});
        \draw[domain=0.1:1.4, help lines, smooth, green]
            plot (\x,{cos((\x-2/3*pi) r)});
        \draw[ultra thick] (1.6,0.6) -- (2,0.6);
        \draw[ultra thick] (1.6,0.4) -- (2,0.4);
        \draw[ultra thick] (1.9,0.7) -- (2.1,0.5);
        \draw[ultra thick](1.9,0.3) -- (2.1,0.5);
        \draw[domain=0.1:1.4, help lines, smooth, dashed, black]
                    plot ({2+\x},{cos((\x) r)});
       \draw[domain=0.1:1.4, help lines, smooth, black]
                           plot ({1.8+\x},{cos((\x) r)});                           
        \draw[domain=0.1:1.4, help lines, smooth, green]
                    plot ({2+\x},{cos((\x-2/3*pi) r)});
                                        
\end{scope}
\end{tikzpicture}
Если сместилась фаза, $\beta$ тоже сместилась. Несимметрия СИФУ - средний угол
18,22. Там где не хватит $\beta$.
Несинусоидальность фазы - опять приводит к необходимости увеличить $\beta$.
От температуры, тока, который был, напряжение приложенное.
${\displaystyle \Delta}_\alpha + {\displaystyle \Delta}\phi$ -- несимметрия
сети + $ {\displaystyle \Delta}\phi$ --несинусоидальность +
 ${\displaystyle \Delta}$ (разброс на углы включения).
 Эмпирически $\beta_{min} = 15...30^\circ$, $\alpha_{max}=150,165^\circ$
К чему приведет невыполнение (\ref{steady_invertor})? Невыполнение (\ref{steady_invertor}) может привести к ``опрокидыванию'' инвертора: вместо инвертора получим
выпрямительный режим -- аварийный режим, связанный с переходом преобразователя
в выпрямительный режим с резким возрастанием выпрямленного тока. Выпрямленный ток
может возрастать до значений, близких к току К.З, Иногда называют током
``двойного'' К.З., т.е. к К.З. одновременно и преобразователя и источника.


\begin{figure}[H]
\begin{circuitikz}\draw
(0,4) to[L] (0,2)
node[left] at (0,3.5) {-}
node[left] at (0,3) {$U_d$}
node[left] at (0,2.5) {+}
node[left] at (0,2) {${\textcyrillic{выпрямленное}}$}
%node[right] at (0,3.5) {(+)}
%node[right] at (0,2.5) {(-)}
%node[right] at (0,2) {${\textcyrillic{стало}}$}
to[Ty,mirror] (0,0)
(1,4) to[L,*-] (1,2)
to[Ty,mirror] (1,0)
(2,4) to[L,*-] (2,2)
to[Ty,mirror] (2,0)
(0,4) to[short,-*] (1,4)
to[short,-*] (2,4)
to[short] (8,4)
(0,0) to[short,-*] (1,0)
to[short,-*] (2,0)
to[short] (8,0)
to (8,1)
to[american voltage source] (8,3)
to[short] (8,4)
%
(3,3.9) to[short,v^=$U_{\textcyrillic{инв}}$] (3,0.1)
(2.9,0.2) -- ++ (0.1,-0.1)
-- ++ (0.1,0.1)
(2.9,3.8) -- ++ (0.1,0.1)
-- ++ (0.1,-0.1)
%
node[right] at (8,2.5) {+}
node[right] at (8,1.5) {-}
node[right] at (8.2,2) {$E_{\textcyrillic{н, противо-ЭДС}}$} 
node[right] at (8,0.5) {}
% I
(6,3.8) to[short,i^>={$I_d$}] (5,3.8)
;\end{circuitikz}
\end{figure}

$$
I_d = \frac{\overbrace{E_{\textcyrillic{н}}}^{\textcyrillic{согласно с током}} -
\overbrace{\mid E_d\mid}^{\textcyrillic{само отрицательно}}
}{R_{\textcyrillic{н}} +R_{\textcyrillic{эквив}}}
$$
Была разность ЭДС, а станет сумма. $U_{\textcyrillic{инв}}$ -- аккумулятор,
выпрямитель, солнечная батарея. $R$ -- маленькая, $R_{\textcyrillic{н}} - 5\%$
и если напряжение не в плюсе а в минусе, то в 19 раз вырастет ток. Получаем
КЗ и для инвертора и для нагруки. Не двойной ток, а ``удвоение'' явления.
А ток $I \approx I_{\textcyrillic{К.З.}}/2$.

Опрокидывание инвертора приводит к аварийному отключению преобразователя. А есть и электронные средства.

Инверторный режим принципиально менее надежен чем выпрямительный режим.

Опрокидывание инвертора может происходить в случае
\begin{itemize}
\item кратковременного исчезновения или резкого уменьшения напряжения питающей сети;
\item в случае пропуска (даже одиночного) управляющего импульса.
\item в случае ложного несвоевременного срабатывания (даже одиночного) какого-либо
вентиля.
\end{itemize}

4 причины: 1) -- невыполнение условий.
пропустили импульс -- пропал контакт.
Ложное отпирание может произойти вследствие сбоя СИФУ -- ток растет лавиной.

Где используется инверторный режим.

Солнечные батареи $-\;\leftarrow;=$. Где инверторы крайне необходимы.
Гидрогенераторы мощностью 100МВт возбуждаются с помощью 10МВт.
Для управления генераторами а обмотке возбуждения гигантская магнитная энергия,
её и нужно передать в сеть. Для включения-выключения генератора требуется форсировка в 5-8-14 раз. На синхронных компенсаторах
в 10-13 раз, 1300В вместо 100в. И такая же скорость снижения должна быть.

Отключение линий.
Разгон -- выпрямительный режим, Самое экономное торможение -- рекуперация,инверторный режим.

\subsection{Прерывистый режим работы преобразователя}
Режим прерывистого выпрямленного тока. Ток нагрузки, правильный ток $I_d$. А если
он(ток) $I_d$ начинает прерываться -- это прерывистый режим.
Пульсация большая. 1й случай -- это нонсенс. А вторая ситуация, когда ток маленький
и ток сравним с пульсацией.

$$
\frac{E_d - E_{\textcyrillic{н}}}{R_\Sigma} = I_d 
$$
В этой формуле $I_d$ -- постоянный, варьируется за счет $E_d - E_{\textcyrillic{н}}$.

$$
e_d = E_d + \left(e_d\right) =const
$$
$\alpha$ не меняется.
$$
\frac{\left(e_d\right)}{{\textcyrillic{переменная}} ``Z''} = const
$$
а ток разложить в ряд Фурье.

Двигатель в холостом режиме $E_{\textcyrillic{двиг}} \approx
E_{\textcyrillic{источника}}$

\begin{tikzpicture}
  \begin{scope}[xscale=1,yscale=1]
  \newcommand{\alphaa}{106.0 * pi / 180}
  \newcommand{\gammaa}{21.0 * pi / 180}
  \draw[thin, ->] (-0.2, 0) -- (9,0) node[right] {$\omega t$};
  \draw[thin,dashed] (-0.2, 2) -- (3.5,2);
  \draw[thin] (0,2) -- (0.5,2.5) -- (1.5,1.5) -- (2.5,2.5) -- (3.5,1.5);
% A,B,C
  % =>
  \draw[ultra thick] (3.8,1.1) -- (4.5,1.1);
  \draw[ultra thick] (3.8,0.9) -- (4.5,0.9);
  \draw[ultra thick] (4.4,1.2) -- (4.6,1.0);
  \draw[ultra thick](4.4,0.8) -- (4.6,1.0);

 \draw[thin,dashed] (5, 0.2) -- (8.5,0.2);
 \draw[thin,red] (5,0.2) -- (5.5,0.7) -- (6.2,0) -- %(6.5,-0.3) -- (7.5,0.7)
 (6.8,0) -- (7.5,0.7)-- (8.2,0) -- (8.5,0);


\end{scope}
\end{tikzpicture}

Пересечь ``нуль'' не может, потому что тиристоры не проводят ток в обратном направлении. Такая
ситуация бывает в при работе двигателя режиме Х.Х. Ток плохой(маленький), но управлять двигателем
нужно и в этом режиме. Для установок,например, для того чтобы вывести двигатель в нужную точку
(позиционирование х,у), это может иметь важное значение.

\hspace{-3 cm}
\begin{tikzpicture}
  \begin{scope}[xscale=2,yscale=3]
    \newcommand{\alphaa}{106.0 * pi / 180}
    \newcommand{\gammaa}{21.0 * pi / 180}
    \newcommand{\En}{0.37}

    \draw[thin, ->] (-0.2, 0) -- (5.8,0) node[right] {$\omega t$};
    \draw[thin, ->] (0,0) -- (0,1.3) node[left] {$U$};

    \draw[thin, ->] (-0.2, -2) -- (5.8,-2) node[right] {$\omega t$};
    \draw[thin, ->] (0,-2) -- (0,-1) node[left] {$I$};


    \foreach \x/\xtext in {0, {pi/3}/{\frac{\pi}{m}},{pi}/\pi}
    \draw (\x,0.1) -- (\x,-0.1) node [below] {$\xtext$};

    % A,B,C
    \draw[domain=-0.2:5.8, smooth, yellow]
    plot (\x,{cos((\x) r)}); \node[above] at (0.2,1) {$e_{k-1}$};
    \draw[domain=-0.2:5.8, help lines, smooth, green]
    plot (\x,{cos((\x-2/3*pi) r)}); \node[above] at (2*pi/3+0.2,1) {$e_k$};
    \draw[domain=-0.2:5.8, help lines, smooth]
    plot (\x,{cos((\x-4/3*pi) r)}); \node[above] at (4*pi/3+0.2,1) {$e_{k+1}$};

    % E_н
    \draw[thin] (-0.2,\En) -- (5.8,\En);
    \draw[thin,<->] (5.6,\En)--(5.6,0)node[midway,right]
    {$E_\textcyrillic{н}$};     

    \foreach \qq [evaluate=\qq as \qqshft using \qq*2*pi/3] in {0,...,1}
     {
      \begin{scope}[xshift=\qqshft cm,
          every path/.style={color=red}]
        
        %Ed
        \draw[domain={pi/3-0.2}:{pi/3 + 0.4},ultra thick]
        plot (\x, {cos(\x r)})
        -| ({pi/3+0.4}, \En);
        \draw[domain={pi/3 + 0.4}:{pi-0.2},ultra thick]
        plot (\x, \En)
        -| ({pi-0.2},{cos((pi-0.2-2/3*pi) r)});

        %Id
        \draw[domain={pi/3-0.2}:{pi/3 + 0.4},ultra thick]
plot(\x, {-2 + (pi/0.6*cos((\x-pi/3-0.1) r) - pi/0.6*cos((0.3) r) });
%plot(\x, {-3*(\x-pi/3+0.7)*(\x-pi/3+0.2)*(\x-pi/3-0.4)-2});
        \draw[domain={pi/3 + 0.4}:{pi-0.2},ultra thick]
        plot (\x, -2);
     \end{scope}

      % подписи под Id
      \draw[thin] (pi/3-0.2,-2)-- (pi/3-0.2,-2-0.12);
       \draw[thin] (pi-0.2,-2)-- (pi-0.2,-2-0.12);
        \draw[thin,<->] (pi/3-0.2,-2.1)--(pi-0.2,-2.1)
        node[midway,below] {$2\pi/m$};
        %alpha
        \draw[thin] (pi/3, 0.6) -- (pi/3, 1);
        \draw[thin] (pi/3-0.1, 0.7) -- (pi/3+0.1, 0.9)
          node[above right] {$\alpha$};
        \draw[thin] (pi-0.2, 0.7) -- (pi-0.2, 1);
%        \draw[thin,<-] (pi-0.2, 0.9) -- (pi+0.1, 0.9) node[right] {$\alpha$};
        \draw[thin,-latex] (pi/3,0.8) -- (pi-0.2, 0.8);
        % \lamdba - угол проводимости
        \draw[thin] (pi-0.2, \En-0.1) -- (pi-0.2, -0.5);
        \draw[thin] (pi+0.4, 0.1) -- (pi+0.4, -0.5);
         \draw[thin,<->] (pi-0.2, -0.4) -- (pi+0.4, -0.4) node[right]
         {$\lambda{\textcyrillic{ -- угол проводимости}}$};
         \node[right] at (pi/3,-1.1) {$
         \begin{array}{c}         
         {\textcyrillic{производная}} \\
         {\textcyrillic{пропорциональна}} \\
         {\textcyrillic{разности напряжений}}
         \end{array}$};
         \draw[thin,->] (pi-0.3,-1.3) -- (pi-0.1,-1.8);
          \draw[thin,->] (2*pi/3,-0.9) -- (pi-0.25, 0.5);
          %
          \draw[thin,dashed] (pi+0.14,-1.75)--(pi+0.14,0.38);
          \draw[thin,<-] (pi+0.1,-2.05) -- (pi+0.1,-2.2) node[right]
          {$\textcyrillic{максимум немного левее из-за } i\cdot r_\phi$};

          %заштрихованная площадка
          \draw[thin,domain={pi-0.2}:{pi + 0.14},pattern=north west lines,
          pattern color=blue]
          (pi-0.2,\En)--
          plot (\x,{cos((\x-2/3*pi) r)});
          \draw[thin,blue] (pi,\En +0.2) --(pi+0.5,\En +1.2)
          node[above] {
$\begin{array}{c}
\textcyrillic{заштрихованная площадка }\sim
          \textcyrillic{ току}\\
          e_\phi -E\textcyrillic{н} -\xcancel{i\cdot r_\pi = L
          \frac{\partial i}{\partial t}}
          \end{array}$};
          
   }
\end{scope}
\end{tikzpicture}
Если бы не было $i\cdot r_\phi$, то максимум тока был бы где
$U=E_\textcyrillic{н}$ (производная в точке максимума равна 0).

Угол проводимости $\lambda$ -- угол, в течении которого ток больше 0.

Никакого угла коммутации не будет. Ток проводит -- пауза -- проводит
следующий вентиль.

Постоянная ЭДС нагрузки
$$
\underbrace{E_d + (e_d)}_{\textcyrillic{выпрямленное}} -
E_{\textcyrillic{н}} = \underbrace{
L \frac{\partial i_d}{\partial t} +R_\Sigma i_d}_
{\textcyrillic{и нагрузка и преобразователь}}
$$
$U_0$ включили в $E_{\textcyrillic{н}}$. Разность
$E_d + (e_d) -E_{\textcyrillic{н}}$ больше 0, чтобы протекал ток.



\begin{figure}[H]
\begin{circuitikz}\draw
(0,4) to[L] (0,2)
to[Ty,mirror] (0,0)
(1,4) to[L,*-] (1,2)
to[Ty,mirror] (1,0)
(0,4) to[short,-*] (1,4)
to[short] (4,4)
(0,0) to[short,-*] (1,0)
to[short] (4,0)
to[american voltage source] (4,1.5)
to[L] (4,2.75)
to[R] (4,4)
%
node[right] at (4.2,1.3) {-}
node[right] at (4.2,0.2) {+}
node[right] at (4.2,0.75) {$E_{\textcyrillic{н}}
{\textcyrillic{ положительное}}$}
;\end{circuitikz}
\end{figure}
В паузе будет ЭДС нагрузки, все вентили выключены.

$$
U_d = \frac{1}{2\pi/m}\left[
\int\limits_{-\frac{\pi}{m} + \alpha}^{-\frac{\pi}{m} + \alpha + \lambda}
\sqrt{2}E_{2\phi}cos(\omega t) d\omega t +
E_{\textcyrillic{н}} \left(\frac{2\pi}{m} - \lambda \right)
\right] =
$$

В вырaжении для $U_d$ выбор начала отсчета и слагаемые изображены на рисунке


\begin{tikzpicture}
  \begin{scope}[xscale=2,yscale=3]
    \newcommand{\alphaa}{106.0 * pi / 180}
    \newcommand{\gammaa}{21.0 * pi / 180}
    \newcommand{\En}{0.37}

    \draw[thin, ->] (-0.2, 0) -- (5.8,0) node[right] {$\omega t$};
    \draw[thin, ->] ({2/3*pi},0) -- ({2/3*pi},1.3) node[left] {$U$};

    % A,B,C
%    \draw[domain=-0.2:5.8, smooth, yellow]
%    plot (\x,{cos((\x) r)}); \node[above] at (0.2,1) {$e_{k-1}$};
    \draw[domain=-0.2:5.8, help lines, smooth, green]
    plot (\x,{cos((\x-2/3*pi) r)}); \node[above] at (2*pi/3+0.2,1) {$e_k$};
    \draw[domain=-0.2:5.8, help lines, smooth]
    plot (\x,{cos((\x-4/3*pi) r)}); \node[above] at (4*pi/3+0.2,1) {$e_{k+1}$};

    % E_н
    \draw[thin] (-0.2,\En) -- (5.8,\En);
    \draw[thin,<->] (5.6,\En)--(5.6,0)node[midway,right]
    {$E_\textcyrillic{н}$};

      % стрелка к -pi/2
        \draw[thin,latex-] (pi/3,0.7) -- (2*pi/3, 0.7) node[midway, below] {$\pi/2$};
      %alpha
        \draw[thin] (pi/3, 0.6) -- (pi/3, 1);
        \draw[thin] (pi/3-0.1, 0.7) -- (pi/3+0.1, 0.9)
          node[above right] {$\alpha$};
        \draw[thin] (pi-0.5, 0.7) -- (pi-0.5, 1);
        \draw[thin,-latex] (pi/3,0.8) -- (pi-0.5, 0.8);
      % \lamdba - угол проводимости
        \draw[thin] (pi-0.5, \En-0.1) -- (pi-0.5, -0.5);
        \draw[thin] (pi+0.4, 0.1) -- (pi+0.4, -0.5);
         \draw[thin,<->] (pi-0.5, -0.4) -- (pi+0.4, -0.4) node[right]
         {$\lambda{\textcyrillic{ -- угол проводимости}}$};

         %заштрихованная площадка
          \draw[thin,domain={pi-0.5}:{pi + 0.4},pattern=north west lines,
          pattern color=blue]
          (pi-0.5,0)--
          plot (\x,{cos((\x-2/3*pi) r)}) -- (pi+0.4,0);
       
      % начало следующего угла проводимости
         \newcommand{\lambdaNext}{(pi+2/3*pi-0.3)}
         \draw[thin] (pi+2/3*pi-0.5,0) -- (pi+2/3*pi-0.5, {cos(((pi+2/3*pi-0.5) -4/3*pi) r)} ) ;
   
      % штрихованная площадка
        \draw[thin,domain={pi+0.4}:{pi+2/3*pi-0.5},pattern=north east lines, pattern color=gray]
         (pi+0.4,0) -- plot (\x,\En) -- (pi+2/3*pi-0.5,0);

  \end{scope}
\end{tikzpicture}

${\displaystyle \lambda < \frac{2\pi}{m}}$ -- означает, что ток прерывистый.

${\displaystyle \lambda > \frac{2\pi}{m} => \lambda = \frac{2\pi}{m} + \gamma}$
-- режим непрерывного тока.

при ${\displaystyle \lambda = \frac{2\pi}{m}}$ -- граница, должны сливаться оба режима, сливаются на отрезке длиной $\gamma=0$.

$$
= \frac{m}{2\pi} \sqrt{2} E_{2\phi}\left[ sin\left(-\frac{\pi}{m} +
\alpha + \lambda\right) - sin\left(-\frac{\pi}{m} +\alpha\right)
\right]
+ E_{\textcyrillic{н}} \left( 1 - \frac{\lambda m}{2\pi} \right) =
$$

$$
\frac{m}{2\pi} \sqrt{2} E_{2\phi} sin \frac{\lambda}{2}
cos\left( \alpha - \frac{\pi}{m} +\frac{\lambda}{2} \right)
+ E_{\textcyrillic{н}} \left( 1 - \frac{\lambda m}{2\pi}\right) =
$$

$$
U_d = E_{d0} \frac{sin\frac{\lambda}{2}}{sin\frac{\pi}{m}}
cos\left(\alpha-\frac{\pi}{m} + \frac{\lambda}{2}\right) +
E_{\textcyrillic{н}} \left( 1 - \frac{\lambda m}{2\pi} \right)
$$

${\displaystyle \lambda = \frac{2\pi}{m}}$ -- граничный режим $\frac{\pi}{m}?$

Частный случай режима работы управляемого преобразователя
на чисто активную нагрузку.
$$
= \int\limits_{-\frac{\pi}{2} + \alpha}^{\frac{\pi}{m}} \cos\omega t\; d\omega t
$$

а если $\alpha=0$ -- тогда не будет прерывистого режима. Прерывистый режим
будет в случае когда $\alpha>0$

$$
\underbrace{\frac{\pi}{m} + \alpha}_{\begin{array}{c}
{\textcyrillic{момент включения}}\\
{\textcyrillic{следующей фазы}}
\end{array}} > \frac{\pi}{2}
$$

Должно выполнятся неравенство, иначе нет прерывистого режима. 
Прерывистый режим при чисто активной нагрузке изображен на рисунке:

\begin{tikzpicture}
  \begin{scope}[xscale=2,yscale=3]
    \newcommand{\alphaa}{106.0 * pi / 180}
    \newcommand{\gammaa}{21.0 * pi / 180}
    \newcommand{\En}{0.37}

    \draw[thin, ->] (-0.2, 0) -- (5.8,0) node[right] {$\omega t$};
    \draw[thin, ->] ({2/3*pi},0) -- ({2/3*pi},1.3) node[left] {$U$};

    % A,B,C
%    \draw[domain=-0.2:5.8, smooth, yellow]
%    plot (\x,{cos((\x) r)}); \node[above] at (0.2,1) {$e_{k-1}$};
    \draw[domain=-0.2:5.8, help lines, smooth, green]
    plot (\x,{cos((\x-2/3*pi) r)}); \node[above] at (2*pi/3+0.2,1) {$e_k$};
    \draw[domain=-0.2:5.8, help lines, smooth]
    plot (\x,{cos((\x-4/3*pi) r)}); \node[above] at (4*pi/3+0.2,1) {$e_{k+1}$};

      % стрелка к -pi/2
        \draw[thin,latex-] (pi/3,0.7) -- (2*pi/3, 0.7) node[midway, below] {$\pi/2$};

      %alpha
        \draw[thin] (pi/3, 0.6) -- (pi/3, 1);
        \draw[thin] (pi/3-0.1, 0.7) -- (pi/3+0.1, 0.9)
          node[above right] {$\alpha$};
        \draw[thin] (pi-0.5, 0.7) -- (pi-0.5, 1);
        \draw[thin,-latex] (pi/3,0.8) -- (pi-0.5, 0.8);
      % \lamdba - угол проводимости
        \draw[thin] (pi-0.5, \En-0.1) -- (pi-0.5, -0.5);
        \draw[thin] (2/3*pi + pi/2, 0.1) -- (2/3*pi + pi/2, -0.5);
         \draw[thin,<->] (pi-0.5, -0.4) -- (2/3*pi + pi/2, -0.4) node[right]
         {$\lambda{\textcyrillic{ -- угол проводимости}}$};

      % начало следующего угла проводимости
         \newcommand{\lambdaNext}{(pi+2/3*pi-0.3)}
         \draw[red] (pi+2/3*pi-0.5,0) -- (pi+2/3*pi-0.5, {cos(((pi+2/3*pi-0.5) -4/3*pi) r)} ) ;

      \draw[red] (pi-0.5, 0) -- (pi-0.5, {cos((pi-0.5-2/3*pi) r)});
       \draw[red,domain=pi-0.5:2/3*pi+ pi/2] plot (\x,{cos((\x-2/3*pi) r)});
       \draw[red] (2/3*pi+ pi/2,0) -- (pi+2/3*pi-0.5,0);
\end{scope}
\end{tikzpicture}

\begin{equation}
\frac{m}{2\pi}\sqrt{2}E_{2\phi}\left[1 - sin\left(\alpha-\frac{\pi}{m}\right)
\right] =
E_{d0} \frac{1 - sin\left(\alpha-\frac{\pi}{m}\right)}{2sin\frac{\pi}{m}}
\end{equation}

$$
\alpha>\frac{\pi}{2} -\frac{\pi}{m}
$$
\begin{equation}
\frac{\pi}{2} -\frac{\pi}{m} < \alpha < \frac{\pi}{2} +\frac{\pi}{m}
\label{7a}
\end{equation}
для разных m разные неравенства (\ref{7a})

\end{document}
