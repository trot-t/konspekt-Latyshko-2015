\documentclass[a4paper,11pt]{report}
\usepackage[T1,T2A]{fontenc}
\usepackage{ucs}
\usepackage[utf8x]{inputenc}
\usepackage{latexsym}
%\usepackage{amsmath}
%\usepackage{mathtext}
\usepackage[english,ukrainian,russian]{babel}
\usepackage{tikz}
\usepackage{siunitx}
\usetikzlibrary{patterns}
\usetikzlibrary{positioning}
\usetikzlibrary{decorations.text}
\usetikzlibrary{decorations.pathmorphing}
\usetikzlibrary{snakes}
\usepackage[american,cuteinductors,smartlabels]{circuitikz}
\usepackage{enumitem}
\usepackage{relsize}
\usepackage{wrapfig}
\usepackage{float}
\usepackage{graphicx}
\graphicspath{{./images/}}
\usepackage{hyperref}
\usepackage{cancel}
\usepackage{pgfplots}
\usetikzlibrary{arrows.meta,calc,decorations.markings,math,arrows.meta}
\usepackage{pdfpages}
\usepackage{booktabs}
% http://www1.combinatorics.org/Information/article.html
\author{ Латышко Владимир Данилович \\
Санкт-Петербургский государственный электротехнический\\университет "ЛЭТИ"\\
%\small \texttt{lvd@lvd.spb.ru }}
Прокшин Артем \\
Санкт-Петербургский государственный электротехнический\\университет "ЛЭТИ"
%\small \texttt{taybola@gmail.com}
}

\usetikzlibrary{calc}
\usetikzlibrary{decorations.pathreplacing,intersections,calc}
\ctikzset{bipoles/thickness=1}
\ctikzset{bipoles/length=0.8cm}
\ctikzset{bipoles/diode/height=.375}
\ctikzset{bipoles/diode/width=.3}
% \ctikzset{tripoles/thyristor/height=.8}
% \ctikzset{tripoles/thyristor/width=1}
\ctikzset{bipoles/vsourceam/height/.initial=.7}
\ctikzset{bipoles/vsourceam/width/.initial=.7}
\tikzstyle{every node}=[font=\small]
\tikzstyle{every path}=[line width=0.8pt,line cap=round,line join=round]

\AddEnumerateCounter{\Asbuk}{\@Asbuk}{\CYRM}
\AddEnumerateCounter{\asbuk}{\@asbuk}{\cyrm}
\renewcommand{\theenumi}{(\asbuk{enumi})}
\renewcommand{\labelenumi}{\asbuk{enumi})}


\title{Конспект лекций по дисциплине Силовая электроника}
%\date{}
\renewcommand{\chaptername}{Лекция}
\begin{document}

\tikzset{component/.style={draw,thick,circle,fill=white,minimum size =0.75cm,inner sep=0pt}}
%http://tex.stackexchange.com/questions/132076/how-to-draw-a-dc-motor-in-circuitikz

%http://tex.stackexchange.com/questions/132076/how-to-draw-a-dc-motor-in-circuitikz`
% prepare to create bipoles

\makeatletter
\def\TikzBipolePath#1#2{\pgf@circ@bipole@path{#1}{#2}}
\def\CircDirection{\pgf@circ@direction}

%\pgf@circ@Rlen = \pgfkeysvalueof{/tikz/circuitikz/bipoles/length}

\makeatother 

\newlength{\ResUp} \newlength{\ResDown}
\newlength{\ResLeft} \newlength{\ResRight}

% set default motor size

\ctikzset{bipoles/motor/height/.initial=.8}
\ctikzset{bipoles/motor/width/.initial=.8}

% create motor shape

\pgfcircdeclarebipole{}
 {\ctikzvalof{bipoles/motor/height}}
 {motor}
 {\ctikzvalof{bipoles/motor/height}}
 {\ctikzvalof{bipoles/motor/width}}
 {
    \pgfsetlinewidth{\ctikzvalof{bipoles/thickness}\pgfstartlinewidth}
    \pgfextractx{\ResRight}{\northeast}
    \pgfextracty{\ResUp}{\northeast}
    \pgfextractx{\ResLeft}{\southwest}
    \pgfextracty{\ResDown}{\southwest}

  \pgfpathmoveto{\pgfpoint{0.775\ResLeft}{0.2\ResDown}}
\pgfpathlineto{\pgfpoint{\ResLeft}{0.2\ResDown}}
  \pgfpathlineto{\pgfpoint{\ResLeft}{0.2\ResUp}}
  \pgfpathlineto{\pgfpoint{0.775\ResLeft}{0.2\ResUp}}
\pgfpathellipse{\pgfpointorigin}{\pgfpoint{0.8\ResRight}{0cm}}
    {\pgfpoint{0cm}{0.8\ResUp}}
  \pgfpathmoveto{\pgfpoint{0.775\ResRight}{0.2\ResDown}}
    \pgfpathlineto{\pgfpoint{\ResRight}{0.2\ResDown}}
  \pgfpathlineto{\pgfpoint{\ResRight}{0.2\ResUp}}
  \pgfpathlineto{\pgfpoint{0.775\ResRight}{0.2\ResUp}}
  \pgfusepath{draw} %draw motor
    \pgftext[rotate=-\CircDirection]{\textsf{M}}
 }

% create motorto-path style

\def\motorpath#1{\TikzBipolePath{motor}{#1}}
\tikzset{motor/.style = {\circuitikzbasekey, /tikz/to path=\motorpath, l=#1}}

% end of setup

 

%\maketitle
%\chapter{Схемы выпрямителей}

Мы занимаемся статическими преобразователями в противовес электромашинным. Статические преобразователи управляются ключами.
Диод, вообще говоря, имеет управление по силовой цепи: в тот момент когда ток спадает до 0, диод выключается.
Диод управляется полярностю сети.
Для тиристоров лучше употреблять термин "запираемый" или "незапираемый", а не управляемый.

Кроме "классических" приборов существуют модули, в которых несколько приборов интегрированы в одном корпусе.
\begin{figure}[H]
%  \includegraphics[scale=1]{l4a.eps}
\centering
\begin{circuitikz}\draw
  (0,2.5) to[Ty] (0,0.5)
  (1.7,0.5) to[Tr] (1.7,2.5)
  node[left] at (2.9,1.5) {$\approx$}
  (4,0) to[short,-*] (4,0.5)
  -- (4.5,0.5)
  to[Ty,mirror] (4.5, 2.5)
  -- (3.5,2.5)
  to[Ty,mirror] (3.5,0.5)
  -- (4,0.5)
  (4,2.5) to[short,*-] (4,3)  
  ;
  \draw[thin, ->] (3.1, 0) -- (3.1, 0.9);
  \draw[thin, ->] (4.9, 0) -- (4.9, 1.5);
  \draw node[below] at (4,-0.1) {это разные цепи управления};
  \draw node[below] at (4,-0.5) {и между ними могут быть киловольты};
\end{circuitikz}
\caption{Семистор}
\end{figure}

\begin{figure}[H]
\centering
%  \includegraphics[scale=1]{l4_1.eps}
\begin{circuitikz}\draw
(1,3) node[npn,rotate=90] (npn1) {}
(npn1) node[anchor=south] {VT1}
(1,0) to[R] (npn1.B)
(3,3) node[npn,yscale=-1,rotate=-90] (npn2) {}
(npn2) node[anchor=south] {VT2}
(npn2.C) to[short] (4,3)
to[short,l_=b] (4.5,3)
(1,0) to[short] (3,0)
      to[R] (npn2.B)
(npn1.E) to[short] (npn2.E)
(2,3) to[short,*-*] (2,4)
(4,3) to[short,*-] (4,4)
(2,4) to[Do, l=VD2] (4,4)
(2,4) to[short] (4,4) % ГОСТ
(0,3) to[short] (0,4)
(2,4) to[Do, l_=VD1] (0,4)
(2,4) to[short] (0,4) % ГОСТ
(0,3) to[short] (npn1.C)
(-0.5,3) to[short,l_=a] (0,3) % put label
(2,3) to[short,*-] (2,-0.3)
;\end{circuitikz}

  \caption{аналог запираемого тиристора}
\end{figure}


Диоды $VD1$ и $VD2$ шунтируют транзисторы. Если к концу $a$ приложен положительный потенциал, а к концу $b$ приложен отрицательный,
то ток идет по цепи $VT1$ -- $VD2$. При противоположной полярности ток идет по цепи $VT2$ -- $VD1$

%\begin{figure}[H]
%  \includegraphics[scale=1]{l4_2.eps}
%  \caption{}
%  \end{figure}

\begin{figure}[H]
\centering
\begin{circuitikz}\draw
(0,0) node[npn] (npn1) {}
(npn1.B) node[anchor=east] {+} %{$\begin{array}{c}VT1\\+\end{array}$}
%(npn1) node[anchor=west] {VT1}
(1,2) -- (0,2) to[Do] (npn1.C)
(0,2) to[short] (npn1.C) % ГОСТ
(1,0) node[pnp,xscale=-1,yscale=-1] (pnp1) {} 
(pnp1.B) node[anchor=west] {-}
(pnp1.C) to[Do] (1,2)
(pnp1.C) to[short] (1,2) % ГОСТ
(npn1.E) to[short] (pnp1.E)
;\end{circuitikz}
\caption{}
\end{figure}


Диоды параллельны. Чтобы открыть транзисторы нужно подать разнополярные сигналы. Но можно схитрить:

\begin{figure}[H]
%  \includegraphics[scale=1]{l4_3.eps} 
\centering
\begin{circuitikz}\draw
(2,0) node[npn] (npn1) {}
(0,0) to[short] (npn1.B)
(0,-2) to[short,o-] (0,0)
(0,-2) node[anchor=north] {+}
(1,-2) to[R] (1,0)
(3,2) -- (2,2) to[Do] (npn1.C)
(2,2) to[short] (npn1.C) % ГОСТ
(3,0) node[pnp,xscale=-1,yscale=-1] (pnp1) {} 
(pnp1.B) to[short] (5,0)
(4,-2) to[R] (4,0)
%(pnp1.B) node[anchor=west] {-} 
(pnp1.C) to[Do] (3,2)
(pnp1.C) to[short] (3,2) % ГОСТ
(npn1.E) to[short] (pnp1.E)
(5,-2) to[short, o-] (5,0)
(5,-2) node[anchor=north] {-}
(2.5,-2) to[short,*-*] (2.5,-0.75)
(1,-2) to[short] (4,-2)
;\end{circuitikz}
\caption{}
\end{figure}

В этой схеме оба транзистора открыты. При противоположной полярности -- закрыты.

\begin{figure}[H]
\centering
%  \includegraphics[scale=1]{l4_4.eps}
%  \caption{}
%  \end{figure}

\begin{circuitikz}[american voltages]\draw
(0,0) node[npn] (npn1) {}
(npn1.C)  to[short, i<=$i$] (0,2)
(0,2) to[short] (1,2)
(1,2) to[short] (2.25, 0.75)
(2.25, 0.75) to[Do] (1,2)
(2.25, 3.25) to[Do, i=$i$] (1,2)
(2.25, 3.25) to[short] (1,2)
(3.5, 2) to[Do,i=$i$] (2.25, 0.75)
(2.25, 0.75) to[short] (3.5, 2)
(3.5, 2) to[Do] (2.25, 3.25)
(2.25, 3.25) to[short] (3.5, 2)
(3.5,2) to[short] (4,2)
        to[short,i<=$i$] (4,-0.75)
        to[short] (0,-0.75)
(2.25, 0.75) to[short, -o] (2.25, 0)
(2.25, 0) node[anchor=north] {$\sim$}
(2.25, 3.25) to[short,-o] (2.25, 4)
(2.25, 4) node[anchor=south] {$\sim$}
;\end{circuitikz}
  \caption{}
  \end{figure}

На этой схеме используется биполярный транзистор, но можно использовать MOSFET, IGBT, pnp. 
Недостаток: ток всегдa будет протекать по трем приборам. В предыдущей схеме ток протекал по двум приборах.
А в случае двух встечных запираемых тиристорах -- один (правда пятислойный).

Самым распространённым типом преобразовательного прибора является выпрямитель.
Затем по распространённости следут НПЧ, тиристорный регулятор переменного
напряжения.

Силовую структуру выпрямителя мы уже знаем. Дальше идет информационная электроника.

Фильтр - как правило, используется индуктивный фильтр и комбинации $LC$, $CL$, $CLC$, резонансные фильтры.

Трансформаторы - схемы включения \rotatebox[origin=c]{180}{$Y$},$\Delta$,$Z$ и обратные $Y$, \rotatebox[origin=c]{180}{$\Delta$}, $Z$, в
которых соединение концов обмоток поменяно местами по сравнению с
прямой конфигурацией. Такие же схемы соединения обмоток используются
и на первичной стороне.

Трансформатор, вентильная группа, фильтр. Фильтр, как правило,
индуктивный фильтр. Конденсаторы не любят быстроменяющегося
напряжения. Емкость эффективна при холостом ходе и малых токах.
Конденсаторы используются в звене постоянного тока.
В инверторах напряжения используется конденсатор,
если инвертор тока -- индуктивность. Кроме $C$-фильтра могут
быть $CL$ и $CLC$ фильтры.

Основные схемы -- "нулевые". Если "нуль" не выведен, то можно
восстановить "искусственный нуль".

\begin{figure}[H]
  \centering
  \begin{circuitikz}[ american voltages]\draw
    (0,4) to[short] (1,4)
    (0,4) node[anchor=east] {A}
    (1,0) to[R, l=Z] (1,2)
    (1,2) to[short] (1,4)
    (0,3) to[short] (2,3)
    (0,3) node[anchor=east] {B} 
    (2,0) to[R, l=Z] (2,2)
    (2,2) to[short] (2,3)
    (0,2) to[short] (3,2)
    (0,2) node[anchor=east] {C}
    (3,0) to[R, l=Z] (3,2)
    (4.5,0) to[lamp] (4.5,2)
    (4.5,2) to[short, -*] (3,2)
    (1,0) to[short,-*] (2,0)
    (2,0) to[short,-*] (3,0)
    (3,0) to[short] (4.5,0)
    ;\end{circuitikz}
  \caption{схема восстановления искусственного нуля}
\end{figure}
При линейном напряжении между фазами 380V получаем напряжение
на лампе 220V. Активное сопротивление $Z$ должно быть на порядок,
или хотя бы в несколько раз меньше сопротивления нагрузки
$R_{\textcyrillic{н}}$
Использовать $R$ вместо $L$ или $С$ - глупость.

\section{нулевые схемы выпрямителей}
\subsection{Однофазная нулевая схема}
\begin{figure}[H]
  \centering
  \begin{circuitikz}[ american voltages]\draw
    (0,2) to[Ty, o-,v^=$ $] (2,2)
    (0,0) to[short, o-] (2,0)
    ;\end{circuitikz}
\caption{однофазная нулевая схема}
\end{figure}

Подавать на тиристор управляющий сигнал, когда на тиристоре
обратное напряжение крайне нежелательно.

При $U_{VS}<0$ и $|I_S| \approx I_{\textcyrillic{упр}} $ возникает
так называемый транзисторный эффект тиристора, который приводит к
нагреву тиристора. Эти токи имеют порядок $10^{-2}...10^{-1}A$
при температуре около $40^\circ$, но могут достигать $1-2A$
при низких температурах $\approx -40^\circ$. Потому
тиристорные(транзисторные) преобразователи требуют подогрева
при низких температурах.
Максимальный управляющий ток при нормальных условиях достигает
$600 mA$. К чему приводит $I_{\textcyrillic{упр}} = 10^{-2}$,
а $U=1000V$? Тиристор рассеивает сотни ватт, а к ним добавится
полкиловатта. А если в схеме 6-12 тиристоров? Поэтому при
отрицательном напряжении на тиристоре управляющий сигнал не подают.

Электрохимическая нагрузка эквивалентна аккумулятору--встречной ЭДС.
В нагрузке обязательно бывает $L$. $e_{\textcyrillic{н}}$, полярность
может быть согласна, а может быть противоположна приложенному
напряжению.
\begin{circuitikz}[ american voltages]\draw
  (0,0) to[R,l=R] (2,0)
  to[L,l=L]  (4,0)
  to[voltage source, l=e] (6,0)
;\end{circuitikz}  
В частном случае роль $E$ может выполнять конденсатор $C$.
Но один $C$ быть не может, он обязательно должен быть в паре c $R$.
\begin{circuitikz}[ american voltages]\draw
  (0,0) to[R,l=R] (0,2)
        to[short] (1,2)
        to[C,l=C] (1,0)
        to[short] (0,0)  
        ;\end{circuitikz}


\subsection{симметричная двухфазная схема}
% http://tex.stackexchange.com/questions/207770/transformers-in-circuitikz
\begin{figure}[H]
    \centering
\begin{circuitikz}[american]
\draw (0,0) node [transformer core](T){}  % reminded by @PaulGessler, thanks.
      (T.A1) node[above] {+}%A1}
      (T.A2) node[below] {-}%A2}
      (T.B1) node[above] {+}%B1} 
      (T.B2) node[below] {-}%B2}
%(T.base) node{K};
(T.B1) to[Ty] ++ (2,0)
to[short] ++ (1,0)
to[R, l=''$Z_{\textcyrillic{н}}$''] ($(T.B1)!0.515!(T.B2)+(3.0,0)$)
(T.B2) to[Ty] ++ (2,0)
to[short, -*] ++ (0,2.1)
;
% 2 new lines for neutral line on the secondary side.
\draw[thick] ($(T.B1)!0.515!(T.B2)-(0.7,0)$)--node[pos=0.5,above,inner sep=0pt](n){$ $}++ (3.7,0);
;\end{circuitikz}
\end{figure}
\subsection{Двухполупериодная однофазная нулевая схема}
\begin{figure}[H]
      \centering
\begin{circuitikz}[american]
  \draw
  (0,0) to[short,o-*] (1,0)
  to[C,-*] (1,1)
  to[C,-*] (1,2)
  to[short,-o] (0,2)
  (1,0) to[Do] (4,0)
  (1,0) to[short] (4,0)
  to[short] (4,2)
  to[short] (1,2)
  to[Do] (4,2)
  (1,1) to[short,-o] (2,1)
  (2,1) node[anchor=south] {-}
  (4,1) to[short,-o] (3,1)
  (3,1) node[anchor=south] {+}
  ;\end{circuitikz}
\end{figure}
Потечет ли ток? Ток не потечёт, поскольку конденсатор не пропускает постоянный ток.
По ``нулевому'' проводу должен протекать постоянный ток.
Создадим искусственный ток:

\begin{figure}[H]
  \centering
  \begin{circuitikz}[american]
    \draw
    (0,0) to[short,o-*] (1,0)
    to[C,-*] (1,1)
    to[C,-*] (1,2)
    to[short,-o] (0,2)
    (1,0) to[short] ++ (2,0)
    (3,0)  to[Do] ++ (3,0)
    (3,0) to[short] ++ (3,0)
    to[short] ++ (0,2)
    to[short] (1,2)
    (3,2) to[Do] ++ (3,0)
    (1,1) to[short,-o] ++ (3,0)
    node[anchor=south] {-}
    (6,1) to[short,-o] ++ (-1,0)
    node[anchor=south] {+}
    (2,0) to[R,*-*,color=red] (2,1)
    (2,1) to[R,-*,color=red] (2,2)
    (3,0) to[L,*-*,color=red] (3,1)
    (3,1) to[L,-*,color=red] (3,2)
    ;\end{circuitikz}
\end{figure}

Для постоянного тока $Z$ не может быть емкостным. ``$Z$'' должен проводить
постоянный ток.

В ``нулевых'' схемах число вентилей равно числу фаз.
\begin{figure}[H]
\caption{$m=3$}
\centering
\begin{circuitikz}[american]
  \draw
  (0,1) to[short] ++ (2,0)
  (0,3) to[Ty,*-*] ++ (0,-2)
  (1,3) to[Ty,*-*] ++ (0,-2)
  (2,3) to[Ty,*-*] ++ (0,-2)
  (1,1) -- ++ (0,-1)
  -- ++ (1,0)
  to[R,l=$Z_{\textcyrillic{н}}$] ++ (2,0)
  -- ++ (0,3)
  -- ++ (-4,0)
  ;\end{circuitikz}
\end{figure}
В $Z_{\textcyrillic{н}}$ должны быть учтены параметры провода. ``Искусственный ноль''
должен пропускать постоянный ток и $Z_{\textcyrillic{н}}$ должен быть на порядок
больше $Z$ от ``искусственного ноля''

%\begin{figure}[H]
  \begin{circuitikz}[american]\draw
    (1,2) to[Ty,mirror,-*] (1,0)
    (2,2) to[Ty,mirror,-*] (2,0)
    (3,2) to[Ty,mirror,-*] (3,0)
    to[short,-o] (0,0)
    node[anchor=east] {-} 
    (0.7,0.6) to[short,-o] (0, 0.6)
    node[anchor=east] {+}
    ;\end{circuitikz}
%\end{figure}

$U_{\textcyrillic{управления}} \approx 10V$, может быть 5-8-10,max 15
  
  \begin{circuitikz}[american]\draw
    (1,1) to[Ty] ++ (0,2)
    node[anchor=south] {A}
    (1,1) to[short] ++ (2,0)
    (2,1) to[short,*-] ++ (0,-1)
    to[R,l=$Z_{\textcyrillic{н}}$] ++ (2,0)
%    (0,2.4) to[short] ++ (0.7,0)
    (0.7,2.4) to[short,-o] (0,2.4)
    node[anchor=east] {+}
    (1,2.7) to[short,*-o] ++ (-1,0)
    node[anchor=east] {-}
    ;\end{circuitikz}
  В этой схеме сигнал управления подается относительно фазы $A$.
  Это означает, что сигналы управления нужно развязывать, например, посредством
  импульсного трансформатора.

  6-фазная схема $m=6$, забирать будем с таких обмоток
Y, \rotatebox[origin=c]{180}{$Y$}
  , рисовать не буду

мостовая схема -- это объединение двух нулевых. Ток будет протекать в два раза дольше. Если это трансформатор, то обмотка лучше используется.
Для того чтобы идти дальше, вернемся к частному случаю. Нулевая
однополупериодная схема.
\begin{figure}[H]
  \begin{circuitikz}[american]\draw
    (0,0) to[short,l_=$\textcyrillic{как бы ``ноль''}$] (4,0)
    (0,2) to[short,o-] ++ (1,0)
    to[Ty] ++ (2,0)
    to[C,*-*] ++ (0,-2)
    (4,0) to[R,*-,color=red] (4,2)
    -- ++ (-1,0)
    (5,0) to[short] ++ (1,0)
    to[C,*-] ++ (0,3)
    -- (3,3)
    to[Ty] ++ (-2,0)
    to[short,-*] ++ (0,-1)
    (4.25,0) node[circ] {}
    (4.5,0) node[circ] {}
    (4.75,0) node[circ] {}
    (6,3) node[anchor=south] {-}
    (4,2) node[anchor=south] {+}
    ;\end{circuitikz}
\end{figure}
Напряжение между $(+)$ и $(-)$ равно удвоенному напряжению на конденсаторе.
$u_{d1}$ поставлена маленькими, поскольку это функция времени.
Сумма $u_{d1} + u_{d2}$. По среднему значению будет в два раза больше чем
$U$. Это схема удвоения, $2U$ -- это напряжения на двух конденсаторах.
Эта схема Латура. Однофазная мостовая -- это схема Греса.
Трехфазная мостовая -- Ларионова. Имена возвращаются.

Рассмотрим еще одну схему удвоения:

%\begin{circuitikz}[ american voltages]
%\draw (0,0) to[R, l=$R_1$, i=$i_1$, v_>=$U_1$, o-*] (2,0);
%\end{circuitikz}
%
%\begin{circuitikz} 
%\draw 
%(0,0) node[npn] (npn2) {} 
%(2,0) node[npn,xscale=-1] (npn1) {}
%(npn2.C) to[short] (npn1.C) 
%(npn2.E) to[short] (npn1.E) 
%;\end{circuitikz}
%
%\begin{figure}[ht]
%\centering
%\begin{circuitikz}\draw
%(0,0) node[npn] (npn1) {} 
%%(npn1.B) node[anchor=east] {$\begin{array}{c}VT1\\+\end{array}$}
%%(npn1) node[anchor=west] {VT1}
%(1,2) -- (0,2) to[D*] (npn1.C)
%(1,0) node[pnp,xscale=-1,yscale=-1] (pnp1) {+}
%(pnp1.B) node[anchor=west] {-}
%(pnp1.C) to[D*] (1,2)
%(npn1.E) to[short] (pnp1.E)
%;\end{circuitikz}
%\caption{}
%\end{figure}
%
%\begin{circuitikz}
%    \draw
%    (0,0)
%        to[V, l=$V_s$] ++(0,2.5)
%;
%\end{circuitikz}
  %
%\begin{figure}[H]
%\begin{circuitikz}\draw
%(0,0) to[C, l=$\SI {10}{\micro\farad}$] (0,2) -- (0,3)
%      to[R, l=2.2] (4,3) -- (4,2)
%      to[L, l=12, i=$i_1$] (4,0) -- (0,0)
%(4,2) to[D*, *-*, color=red] (2,0) 
%(0,2) to[R, l=1, *-] (2,2)
%      to[cV, v=$\SI {.3}{\kilo \ohm} i_1$] (4,2)
%(2,0) to[I, i=1] (2,2)
%;\end{circuitikz}
%\end{figure}

%% http://www.texample.net/tikz/examples/power-electronics-rectifier/
% Example 4-7, p. 135 of Hart, discontinuous current in full-wave rect
\begin{tikzpicture}
  \begin{scope}[xscale=1,yscale=1.5]
    \newcommand{\alphaa}{60 * pi / 180}
    \newcommand{\betaa}{216 * pi / 180}

    \draw[thin, ->] (-0.2, 0) -- (14,0) node[right] {$\omega t$};

    \foreach \x/\xtext in {0,{pi}/\pi,
      {2*pi}/{2\pi},{3*pi}/{3\pi}}
    \draw (\x,0.1) -- (\x,-0.1) node [below] {$\xtext$};
    \draw (\betaa,-0.1) -- (\betaa,0.1) node [above] {$\beta$};


    % Vs
    \draw[domain=0:14, help lines, smooth]
    plot (\x,{sin(\x r)});

    % Vo and Io
    \foreach \qq [evaluate=\qq as \qqshft using \qq*2*pi] in {0,...,1}
     {
      \begin{scope}[xshift=\qqshft cm,
          every path/.style={ultra thick, color=red}]
        %Vo

        \draw[domain=pi/6+pi/24:{pi/2 + pi/12}]
        plot(\x,{sin(\x r)})
        [domain={pi/2 + pi/12}:2*pi+pi/6+pi/24]
        plot(\x,{1.17*exp(\x*(-0.1))});
        \draw
        [domain=0:pi + pi/6+pi/24]
        plot(\x,{-0.83*exp(\x*(-0.1))})
        [domain=pi + pi/6+pi/24:{pi + pi/2 + pi/12}]
            plot(\x,{sin(\x r)})
            [domain={pi + pi/2 + pi/12}:2*pi]
            plot(\x,{-1.75*exp(\x*(-0.12))})
            ;
       \end{scope}
     }
     \node[right,color=red] at ({pi/2+pi/12},1.05) {$u_{C1}$};
     \node[right,color=red] at ({pi+pi/2+pi/12},-1.05) {$u_{C2}$};
  \end{scope}
  \end{tikzpicture}

\begin{figure}[H]
  \begin{circuitikz}\draw
    (1,0) to[R,l_={$R_{\textcyrillic{н}}$}] ++ (3,0)
    -- ++ (0,1)
    to[C,l={$C_2$}, *-*] ++ (-3,0)
    -- ++ (0,-1)
    (4,4) to[C,l=$C_1$,v=$U$,color=red] ++ (-3,0)
    (4,4) to[Ty,l=$VD_2$] ++ (0,-3)
    (1,1) to[Ty,l=$VD_1$,v=$ $,color=red,*-*] (4,4)
    (1,1) to[short,-o] ++ (-1,0)
    node[anchor=east] {+}
    (1,4) to[short,-o] ++ (-1,0)
    node[anchor=east] {-}
    ;\end{circuitikz}
\end{figure}

При $-$ $+$ на входе конденсатор $C_1$ зарядится $-$ $С_1$ $+$.
При подаче напряжения на $VD_1$
\begin{circuitikz}\draw
  (0,0) to[Ty,l=$VD_1$] (0,2)
  to[short,v=$ $] ++ (0,-2)
  ;\end{circuitikz}
 откроется $VD_2$
\begin{circuitikz}\draw
  (0,2) to[Ty,l=$VD_2$,v=$ $] (0,0)
%  to[short,v=$ $] ++ (0,-2)
    ;\end{circuitikz}
 

\begin{figure}[H]
  \begin{circuitikz}\draw
    (1,0) to[R,l_={$R_{\textcyrillic{н}}$}] ++ (3,0)
    -- ++ (0,1)
    to[C,l={$C_2$},color=red, *-*] ++ (-3,0)
    -- ++ (0,-1)
    (4,4) to[C,l=$C_1$,v=$U$] ++ (-3,0)
    (4,4) to[Ty,l=$VD_2$,v=$2U$,color=red] ++ (0,-3)
    (1,1) to[Ty,l=$VD_1$,*-*] (4,4)
    (1,1) to[short,-o] ++ (-1,0)
    node[anchor=east] {-}
    (1,4) to[short,-o] ++ (-1,0)
    node[anchor=east] {+}
    ;\end{circuitikz}
\end{figure}

За несколько периодов напряжения сети если пренебрегать
$R_{\textcyrillic{н}}$ конденсатор $C_1$ зарядится до амлитулы $U$,
а конденсатор $C_2$ до двух амплитуд $2U$.

\subsection{вариант схема умножения}

\begin{figure}[H]
  \begin{circuitikz}\draw
%    (1,0) to[R,l_={$R_{\textcyrillic{н}}$}] ++ (3,0)
%    -- ++ (0,1)
    (4,1) to[C,l={$C_2$}, *-*] ++ (-3,0)
%    -- ++ (0,-1)
    (4,4) to[C,l=$C_1$] ++ (-3,0)
    (4,4) to[Ty,l=$VD_2$] ++ (0,-3)
    (1,1) to[Ty,l=$VD_1$,mirror,*-*] (4,4)
    (1,1) to[short,-o] ++ (-1,0)
    node[anchor=east] {-}
    (1,4) to[short,-o] ++ (-1,0)
    node[anchor=east] {+}
    (4,4) to[C,l=$C_3$] ++ (3,0)
    (4,1) to[Ty,l=$VD_3$,mirror] ++ (3,3)
    to[Ty,l=$VD_4$,*-*] ++ (0,-3)
    to[C,l=$C_4$] ++ (-3,0)
    (7,4) to[C,l=$C_5$] ++ (3,0)
    (7,1) to[Ty,l=$VD_5$,mirror] ++ (3,3)
    to[Ty,l=$VD_6$,*-] ++ (0,-3)
    to[C,l=$C_6$] ++ (-3,0)   
    ;\end{circuitikz}
\end{figure}

Конденсатор $C_1$ заряжается на положительных полупериодах порциями.
На верхних конденсаторах $C_1$, $C_3$, $C_5$, ... нечётные напряжения
$U$, $3U$, $5U$. На нижних конденсаторах $C_2$, $C_4$, $C_6$, ... нечётные напряжения $2U$, $4U$, $6U$. Таким способом в середине прошлого века получали
наряжение в 1 миллион вольт, полтора миллиона вольт.
Получали 2 миллиона -- была построена линия постоянного тока
Экибастуз-Москва. Несколько миллионов можно было получить.

В первых цветных телевизорах напряжение анод-катод было 25kV. Получали это
напряжение выпрямлением на электровакуумном приборе. Для этого применялся
импульсный трансформатор строчной развертки. Горели строчные трансформаторы
из-за межвитковых замыканий.
Перешли на схему Латура -- пожары прекратились.

Требуется высокое напряжение и малый ток. Газоочистка в трубах
требует высокое напряжение. Сажа притягивается как лохмотья бумажки
притягиваются к расчёстке.

Пример практического применения выпрямителей.

$7\times\underbrace{220\cdot\sqrt{2}}_{283} \approx 1700,1800V$ 
$10\times 283 = 2830V$

Можно ещё одну схему нарисовать. Её почему-то в учебниках не приводят

\begin{figure}[H]
  \begin{circuitikz}\draw
    (3,0) to[C] ++ (-3,0)
    to[Do] ++ (0,3)
    to[C,*-] ++ (3,0)
    (3,0) to[Do] ++ (-3,3)
    -- ++ (0,-3)
    (3,0) to[Do,*-] ++ (-3,3)
    (3,0) -- ++ (-3,3)
    %
    (6,0) to[C] ++ (-3,0)
    to[Do] ++ (0,3)
    to[C,*-o] ++ (3,0)
    (6,0) to[Do] ++ (-3,3)
    -- ++ (0,-3)
    (6,0) to[Do,o-] ++ (-3,3)
    (6,0) -- ++ (-3,3)
    %
    (6,0) to[Do] ++ (3,3)
    to[Do,-*] ++ (0,-3)
    to[C] ++ (-3,0)
    -- ++ (3,3)
    to[C] ++ (-3,0)
    (9,0) -- ++ (0,3)
    %
    (9,0) to[Do] ++ (3,3)
    to[Do,-*] ++ (0,-3)
    to[C] ++ (-3,0)
    -- ++ (3,3)
    to[C] ++ (-3,0)
    (12,0) -- ++ (0,3)
    ;\end{circuitikz}
\end{figure}

Эта схема как бы двухполупериодная. Частота пульсаций 100Гц, а не 50Гц.

\begin{figure}[H]
  \begin{circuitikz}\draw
    (3,0) to[Do, o-] ++ (-3,3)
    to[C,-o] ++ (3,0)
    to[C] ++ (3,0)
    -- ++ (-3,-3)
    to[Do] ++ (3,3)
    (3,0) -- ++ (-3,3)
    ;\end{circuitikz}
\end{figure}
Получили схему Латура.

До сих пор не писали математических соотношений. В Качестве основной схемы рассмотрим
m-фазную ``нулевую'' схему. Минимум $m=2$. Всегда должна быть предыдущая и последующая фазы.
Также всегда должны быть первая и последняя фазы.

\begin{figure}[H]
  \begin{circuitikz}\draw
    (0,5) -- ++ (0.5,0)
    (0,5) to[L,l=$e_1$] ++ (0,-2)
    to[Ty,l=$\!V_1$] ++ (0,-2)
    -- ++ (0,-1)
    -- ++ (0.5,0)    
    %
    (1.2,5) to[short,-*] ++ (0.5,0)
    to[short,-*] ++ (1.4,0)
    (1.7,5) to[L,l=$e_k$] ++ (0,-2)
    to[Ty,l=$\!V_k$] ++ (0,-2)
    to[short,-*] ++ (0,-1)
    -- ++ (0.8,0)
    (1.2,0) to[short] ++ (0.9,0)
    %
    (3.1,5) -- ++ (0.5,0)
    (3.1,5) to[L,l=$e_{k+1}$] ++ (0,-2)
    to[Ty,l=$\!V_{k+1}$] ++ (0,-2)
    to[short,-*] ++ (0,-1)
    -- ++ (0.5,0)
    (2.5,0) to[short] ++ (0.5,0)
    %
    (5,5) to[L,l=$e_m$] ++ (0,-2)
    to[Ty,l=$\!V_m$] ++ (0,-2)
    -- ++ (0,-1)
%    -- ++ (0.5,0)
    (4.5,0) to[short,-*] ++ (0.5,0)
    %
    -- ++ (1,0)
    to[L,l=$L_\Phi\;R_\Phi$] ++ (2,0)
    to[short] ++ (1,0)
    %
    to[battery1,l=$E_{\textcyrillic{н}}$] ++ (0,2)
    to[L] ++ (0,1)
    to[R,l=$R_{\textcyrillic{н}}$] ++ (0,2)
    %
    to[short,-*] ++ (-4,0)
    -- ++ (-0.5,0) 
    ;\end{circuitikz}
\end{figure}






%% http://www.texample.net/tikz/examples/power-electronics-rectifier/
% Example 4-7, p. 135 of Hart, discontinuous current in full-wave rect
\hspace{-3 cm}
\begin{tikzpicture}
  \begin{scope}[xscale=1.7,yscale=2.8]
    \newcommand{\alphaa}{68 * pi / 180}
    \newcommand{\betaa}{200 * pi / 180}
    \newcommand{\gammaa}{82 * pi / 180}

    \draw[thin, ->] (-3.5, 0) -- (7,0) node[right] {$\omega t$};
    \draw[thin, ->] (0,0) -- (0,1.3) node[left] {$U$};
    
    \foreach \x/\xtext in {{-pi}/{-\pi},{-pi/6}/{-\frac{\pi}{m}}, 0,{pi}/\pi,
      {2*pi}/{2\pi}}
    \draw (\x,0.1) -- (\x,-0.1) node [below] {$\xtext$};

    \draw[thin] (\alphaa, {cos(\alphaa r)}) -- (\alphaa,-0.6) node [below] {};
    \draw[thin, ->] ({pi/6},-0.55) -- (\alphaa,-0.55) node[right] {$\alpha$};
    \draw[thin] ( {pi/6 - 0.05},{-0.55 - 0.05} ) -- ++ (0.1,0.1); 
%    \draw (\betaa,-0.1) -- (\betaa,0.1) node [above] {$\beta$};
    \draw[thin] (\gammaa,-0.40) -- (\gammaa,{cos((\gammaa - pi/3) r)/2 + cos(\gammaa r)/2}) node [above] {};
%    \draw[thin, ->] ({pi/6},-0.35) -- (\gammaa,-0.35) node[right] {$\gamma$};
%    \draw[thin] ( {pi/6 - 0.05},{-0.35 - 0.05} ) -- ++ (0.1,0.1);
    \draw[thin, ->] ({\alphaa-pi/12}, -0.35) -- (\alphaa,-0.35);
    \draw[thin, <-] (\gammaa,-0.35) -- ({\gammaa+pi/12},-0.35) node[below] {$\gamma$};
    
    
     \draw[thin] ({pi/6},{cos((pi/6) r)}) -- ({pi/6},-0.7) node [above] {} ;

    % Vs
    \draw[domain=-3:6.4, help lines, smooth, yellow!97!black]
    plot (\x,{sin((\x + pi/6) r)});

    \draw[domain=-3:6.4, help lines, smooth, green]
    plot (\x,{sin((\x + 60 * pi / 180 + pi/6) r)});

    \draw[domain=-3:6.4, help lines, smooth]
    plot (\x,{sin((\x + 120 * pi / 180 + pi/6) r)});

    \draw[domain=-3:6.4, help lines, smooth]
    plot (\x,{sin((\x + 180 * pi / 180 + pi/6) r)});

    \draw[domain=-3:6.4, help lines, smooth]
    plot (\x,{sin((\x + 240 * pi / 180 + pi/6) r)});

    \draw[domain=-3:6.4, help lines, smooth]
    plot (\x,{sin((\x + 300 * pi / 180 + pi/6) r)});
    
    
    % -Vs

    % Vo and Io
    \foreach \qq [evaluate=\qq as \qqshft using \qq*pi/3] in {-1,...,2}
     {
      \begin{scope}[xshift=\qqshft cm,
          every path/.style={ultra thick, color=red}]
        %Vo
        \draw[domain={pi/6+\gammaa-pi/2}:\alphaa]
        plot (\x,{cos(\x r)})
        -| (\alphaa,{cos((\alphaa - pi/3) r)/2 + cos(\alphaa r)/2})
        [domain=\alphaa:\gammaa]
        plot (\x,{cos((\x - pi/3) r)/2 + cos(\x r)/2 })
        -| (\gammaa, {cos((\gammaa - pi/3) r) })
        ;        
       \end{scope}
     }
     \node[above,color=red] at ({\gammaa + pi/3},1) {$u$};

     % I
     \draw[thin, ->] (-3.5, -1.7) -- (6.4,-1.7) node[right] {$\omega t$};
     \draw[thin, ->] (0,-1.9) -- (0,-1.2) node[left] {$I$};

     \draw[green] ({\gammaa - pi/3},  -1.4) -- (\alphaa, -1.4);
     \draw[green] (\alphaa, -1.4) -- (\gammaa, -1.7);
     \draw[green] (\gammaa, -1.7) -- ({\alphaa + pi/3},-1.7);

     \draw[yellow!97!black] ({\gammaa - pi/3},  -1.7) -- (\alphaa, -1.7);
     \draw[yellow!97!black] (\alphaa, -1.7) -- (\gammaa, -1.4);
     \draw[yellow!97!black] (\gammaa, -1.4) -- ({\alphaa + pi/3},-1.4);

     \draw[thin] (\alphaa,-1.65) -- (\alphaa,-1.9);
     \draw[thin] (\gammaa,-1.65) -- (\gammaa,-1.9);
     \draw[thin, ->] ({\alphaa-pi/12}, -1.8) -- (\alphaa,-1.8);
      \draw[thin, <-] (\gammaa,-1.8) -- ({\gammaa+pi/6},-1.8) node[below] {$\gamma$}; 
  \end{scope}
  \end{tikzpicture}


В $e_k$ содержатся ($E_{2m\phi}$, $L_\phi$, $r_\phi$).
Где $L_\phi$ -- эквивалентная индуктивность приведенная ко вторичной обмотке.
Как определить параметры? Из опыта к.з. Есть измеренные U, I, Z, P.
Из $I^2 R = P$ определим $L_{\textcyrillic{фазы}}$

$X_{\textcyrillic{фазы}}$, $\omega L_{\textcyrillic{фазы}} = 2\pi f L_\phi$.

$U_0$ -- как бы встречная ЭДС.

$U = U_0 + R_D\cdot I$, где $R_D$ -- динамическое сопротивление.

Принимаем допущение $i_d \approx I_d$ -- пренебрегаем пульсациями.
Еще одно допущение $m\ge 2$. Неуправляемые диоды проходили бы по максимуму
волн. Обозначим $e_1$. На периоде $2\pi$ имеем $m$ пульсаций.
Предположим, что тиристоры имеют задержку отпирания на угол $\alpha$
(угол регулирования)

Кривая выпрямленного напряжения (рис). Не спешите делать его жирным, мы
будем его поправлять.

Ток через фильтр ${\displaystyle L\frac{\partial i}{\partial t} = \frac{\partial \Psi}{\partial t}}$. На постоянном токе $i_d \approx I_d$ пульсациями пренебрегаем.
$U_d$ -- подлежит определению! Предполагаем, что до $k$-й фазы всё включалось.

Рисуем токи $i_k$ $i_{k+1}$, рируем графики $I_d=i+d$, ...
$i_k$ изменяются мгновенно? Да, если пренебречь индуктивностью...
А наши обмотки имеют индуктивность. Мгновенно этот процесс не может завершиться.
Индуктивность $L_{\phi k}$ не хочет отдавать ток, а индуктивность $L_{\phi (k+1)}$
-- принимать.
Что заставляет обмотки обмениваться током -- разность эдс $e_k - e_{k+1}$.
Обе обмотки начинают проводить ток. Получается к.з. Если не пренебрагать
сопротивлением, то на малое сопротивленние. есть два эдс и два $L$.
 $e_{k+1} - e_k$ упадёт на внутреннем сопротивлении по закону Кирхгофа.
 $i_R$, $i_R$, ${\displaystyle L\frac{\partial i}{\partial t}}$,
 ${\displaystyle L\frac{\partial i}{\partial t}}$.

Переход тока с одной фазы преобразователя на другую называется процессом
коммутации.

Время коммутации определяется разностью ЭДС и суммы активно-индуктивнях
сопротивлений коммутируемых(коммутирующих) фаз.
Время процесса(время коммутации), а соответствующий угол принято обозначать
$\gamma$. $-e_{k}+$ $-e_{k+1}+$, но у $e_{k+1}$ плюс больше! В $k$-й фазе
протекал ток. Из него вычитается ток к.з. Течёт прямой ток, навстречу ток к.з.
Когда ток $i$ вентиля становится равным нулю вентиль выключается.

Какое напряжение будет на нагрузке-лампочке, когда к ней приложаться
разные ЭДС.
\begin{circuitikz}\draw
(0,0) to[voltmeter] (3,0)
to[short,-*] ++ (0,1)
to[R] ++ (-1.5,0)
to[battery1,v=$9V$] ++ (-1.5,0)
(3,1) to[short] ++ (0,1)
to[R] ++ (-1.5,0)
to[battery1,v=$12V$] ++ (-1.5,0)
(0,0) to[short,-*] ++ (0,1)
to[short] ++ (0,1)
;\end{circuitikz}

Что покажет вольтметр? Если не указано внутреннее сопротивление, то неизвестно.
Если одинаковые внутренние сопротивления, то вольтметр покажет полусумму
напряжений. Разность напряжений упадет на этих сопротивлениях. Поняв эту детскую
задачу двинемся дальше.
Теряется заштрихованная площадка из-за коммутации на индуктивностях.

Нужно взять интеграл. Площадь считаем в угловых единицах.
$$
u_d = \frac{1}{2\pi/m}\int\limits_{-\frac{\pi}{m} + \alpha}^
{\frac{\pi}{m} + \alpha}\left( u_\phi -
\underbrace{\scriptstyle{\Delta} u_{V_S}}_{
\begin{array}{c}
\textcyrillic{падение} \\
\textcyrillic{на вентиле}
\end{array}
} -
\underbrace{\scriptstyle{\Delta} u_\Phi}_{
\begin{array}{c}
\textcyrillic{падение} \\
\textcyrillic{на фильтре}
\end{array}
}
\right) d\omega t =
$$
где
$$
u_\phi = e_\phi - R_\phi \cdot i_\phi - L_\phi \frac{di_\phi}{dt}
$$
и
$
e_\phi = E_k
$ -- косинусоида.

$$
=  \frac{m}{2\pi}
\underbrace{
\int\limits_{-\frac{\pi}{m}+\alpha}^{\frac{\pi}{m}+\alpha}
\sqrt{2}E_{2\Phi}\cos\omega t\: d\,\omega t
}_
{\begin{array}{c}
\textcyrillic{эквивалентное значение} \\
\textcyrillic{выпрямленной ЭДС}
\end{array}} -{\scriptstyle \Delta} U_d
$$
через $U_d$ обозначены все остальные падения напряжения.
А если нет никаких падений, то нет и $U_d$.

$$
E_d = \frac{m}{2\pi}
\int\limits_{-\frac{\pi}{m}+\alpha}^{\frac{\pi}{m}+\alpha}
\sqrt{2}E_{2\Phi}Cos\,\omega t\: d\,\omega t=
\frac{m}{2\pi} \sqrt{2}\left[\sin\left(\frac{\pi}{m} + \alpha\right)
- \sin\left(-\frac{\pi}{m} + \alpha\right)
\right]
$$
$\cos$ полусуммы на $\sin$ полуразности.

$$
\frac{m}{2\pi}\sqrt{2}\:2\sin\frac{\pi}{m}\cos\alpha = E_{d0}  \cos\alpha
$$
Что такое $E_{d0}$ -- это $U_d$ при $\alpha=0$ или
$E_{d0}$ - выпрямленное ЭДС в случае неуправляемых диодов.

\begin{equation}
E_{d0} = \frac{m}{\pi}\sqrt{2} E_{2\Phi} Sin\frac{\pi}{m}
\end{equation}

\begin{equation}
E_d = E_{d0} Cos \alpha
\end{equation}
Еще не закончили интегрировать, но отметим важный момент

% http://mydebianblog.blogspot.ru/2009/01/tables-in-latex.html
\begin{table}
\begin{center}
\begin{tabular}{|c|c|c|c|c|c|c|c|}
\hline
m & ``1'' & 2 & 3 & 4 & 6 & \hspace{2 cm} & $\infty$ \\
\hline
${\displaystyle \frac{E_{d0}}{E_{2\Phi}}}$ & 0.45 & 0.9 & 1.17 & 1.27 & 1.35 & & $\sqrt{2} \approx 1.414$ \\
\hline
\end{tabular}
\end{center}
\end{table}

При $\infty$: если количество горбиков возрастает, то в пределе подойдёт к $\sqrt{2}$.
Формула верна в предположении, что ток всегда протекает.

\begin{figure}[H]
\begin{tikzpicture}
  \begin{scope}[xscale=0.8,yscale=1.6]
   \draw node[above] at (-pi/2,0.9)  {{\bf m=2}};
   \draw[thin, ->] (-5, 0) -- (8.5,0) node[right] {$\omega t$};
    % Vo and Io
    \foreach \qq [evaluate=\qq as \qqshft using \qq*pi] in {-1,...,2}
     {
      \begin{scope}[xshift=\qqshft cm,
          every path/.style={ultra thick, color=red}]
        %Vo
        \draw[domain=-pi/2:pi/2]
        plot (\x,{cos(\x r)})
        ;        
       \end{scope}
     }
\end{scope}
\end{tikzpicture}
\caption{m=2, E = 0.9}
\end{figure}

\begin{figure}[H]
\begin{tikzpicture}
  \begin{scope}[xscale=0.8,yscale=1.6]
   \draw node[above] at (-pi/3,1)  {{\bf m=3}};
   \draw[thin, ->] (-3.5, 0) -- (10,0) node[right] {$\omega t$};
    % Vo and Io
    \foreach \qq [evaluate=\qq as \qqshft using \qq*2*pi/3] in {-1,...,4}
     {
      \begin{scope}[xshift=\qqshft cm,
          every path/.style={ultra thick, color=red}]
        %Vo
        \draw[domain=-pi/3:pi/3]
        plot (\x,{cos(\x r)});        
       \end{scope}
     }
\end{scope}
\end{tikzpicture}
\caption{m=3, E = 0.866}
\end{figure}


\begin{figure}[H]
\begin{tikzpicture}
  \begin{scope}[xscale=0.9,yscale=1.8]
   \draw node[above] at (-pi/2,1.1)  {{\bf m=6}};
   \draw[thin, ->] (-3, 0) -- (7.2,0) node[right] {$\omega t$};
    % Vo and Io
    \foreach \qq [evaluate=\qq as \qqshft using \qq*pi/3] in {-2,...,6}
     {
      \begin{scope}[xshift=\qqshft cm,
          every path/.style={ultra thick, color=red}]
        %Vo
        \draw[domain=-pi/6:pi/6]
        plot (\x,{cos(\x r)})
        ;        
       \end{scope}
     }
\end{scope}
\end{tikzpicture}
\caption{m=6, E = }
\end{figure}

\begin{figure}[H]
\begin{tikzpicture}
  \begin{scope}[xscale=0.9,yscale=1.8]
   \draw node[above] at (-pi/2,1.1)  {{\bf m=4}};
   \draw[thin, ->] (-4, 0) -- (7.5,0) node[right] {$\omega t$};
    % Vo and Io
    \foreach \qq [evaluate=\qq as \qqshft using \qq*pi/2] in {-2,...,4}
     {
      \begin{scope}[xshift=\qqshft cm,
          every path/.style={ultra thick, color=red}]
        %Vo
        \draw[domain=-pi/4:pi/4]
        plot (\x,{cos(\x r)})
        ;        
       \end{scope}
     }
\end{scope}
\end{tikzpicture}
\caption{m=4, ${\displaystyle E = \frac{\sqrt{2}}{2}}$}
\end{figure}

Вернулись к однополупериодной схеме

\begin{figure}[H]
\begin{tikzpicture}
  \begin{scope}[xscale=0.9,yscale=1.8]
   \draw node[above] at (-pi/2,0.9)  {{\bf m=''1''}};
   \draw[thin, ->] (-3.5, 0) -- (8.5,0) node[right] {$\omega t$};
    % Vo and Io
    \foreach \qq [evaluate=\qq as \qqshft using \qq*pi] in {0,2}
     {
      \begin{scope}[xshift=\qqshft cm,
          every path/.style={ultra thick, color=red}]
        %Vo
        \draw[domain=-pi/2:pi/2]
        plot (\x,{cos(\x r)})
        ;        
       \end{scope}
     }
\end{scope}
\end{tikzpicture}
\caption{m=''1'', E = 0.45}
\end{figure}

Что осталось? Досчитать ${\scriptstyle} \Delta U_d$

$$
{\scriptstyle} \Delta U_d = \frac{m}{2\pi}\left\{
\int\limits_{-\frac{\pi}{m} + \alpha}^{\frac{\pi}{m} + \alpha} L_\phi \frac{d i_\phi}{dt} d\omega t +
\int\limits_{-\frac{\pi}{m} + \alpha}^{\frac{\pi}{m} + \alpha} ( i_\phi r_\phi + i_\phi R_d + U_0 + I_d R_\Phi)d\omega t
\right\} =
$$

$$
U_\phi = e_\phi - L_\phi \frac{\partial i_\phi}{\partial t} - i_R i_\phi
$$

1-й интеграл. смотрим на график $i$
$$
\int\limits_{\frac{\pi}{m} + \alpha \rightarrow {\textcyrillic{в момент включения самой фазы i}}}
^{\frac{\pi}{m} + \alpha \rightarrow {\textcyrillic{в момент включения следующей фазы}}}
\omega L_\phi \frac{di_\phi}{d(\omega t)} d\omega t
$$

$$
\frac{m}{2\pi}\left\{
\underbrace{\left. L_\phi i_\phi \right|_{-\frac{\pi}{m} + \alpha}}_
{
\begin{array}{c}
i_\pi = I_d\; {\textcyrillic{ когда фаза}} \\
{\textcyrillic{начала включатся}}
\end{array}
}
-
\underbrace{ L_\phi i_\phi \bigr|_{\frac{\pi}{m} + \alpha}}_
{
\begin{array}{c}
=0\;{\textcyrillic{ когда фаза}} \\
{\textcyrillic{включилась}}
\end{array}
}
\right\}
$$

Делаем допущение: $i_\phi$ -- меньше чем площадь прямоугольника (приближённо на всём промежутка равно $I_d$ )
$$
\int i_\phi (r_\phi + R_D) d\omega t + \frac{m}{2\pi}(U_0 + I_d R_\Phi)\int d\omega t =
$$
$$
= \frac{m}{2\pi} I_d X_\Phi +
\underbrace{\int i_\phi (r_\phi + R_D) d\omega t}_{\textcyrillic{это возьмём приближённо}} +(U_0 + I_D R_\Phi)
=
$$


\begin{tikzpicture}
\begin{scope}
\draw (0.5,0) -- (3.5,0);
\draw (0.5,0) -- (1,1.5);
\draw (1,1.5) -- (3.5,1.5);
\draw (3.5,1.5); -- (3.5,0);
\draw[thin] (0.5, 0.2) -- (0.5, 2.3);
\draw[thin] (1, 1.6) -- (1, 2.3);
\draw[thin, ->] (0.1, 2.1) -- (0.5, 2.1);
\draw[thin, <-] (1, 2.1) -- (1.6, 2.1) node[left,below] {$\gamma$};
\draw[thin]  (3.6,1.5) -- (4.2,1.5);
\draw[thin]  (3.6,0) -- (4.2,0);
\draw[thin, <->] (4.1,0) -- (4.1,1.5);
\node[right] at (4.2, 0.75) {$I_d$};
\node[right] at (2, -0.75) {${\displaystyle \frac{2\pi}{m}}$};
\end{scope}
\end{tikzpicture}
Площадь трапеции равна ${\displaystyle I_d\left( \frac{2\pi}{m} - \frac{\gamma}{2}\right)}$ 

$$
= \frac{m}{2\pi}I_d \left( \frac{2\pi}{m} - \frac{\gamma}{2}\right)\left(r_\phi + R_D\right)
$$

$$
{\scriptstyle \Delta} U_d = \frac{m}{2\pi} I_d X_\Phi + U_0 + I_d R_\Phi +
\left(r_\phi + R_D\right) \left( 1 - \frac{\gamma m}{4 \pi}\right) I_d
$$

$$
U_d = E_{d0} Cos \alpha - {\scriptstyle \Delta} U_d
$$

\begin{equation}
U_d = E_{d} - U_0 - \frac{m}{2\pi} X_\Phi I_d - I_d \left\{
R_\Phi + \left(r_\phi + R_D\right) \left( 1 - \frac{\gamma m}{4 \pi}\right) 
\right\}
\end{equation}

Осталось определить, как $\gamma$ зависит от тока.

%\include{real_lection4d.tex}
%
\begin{equation}
E_{d0} = \frac{m}{\pi}\sqrt{2} E_{2\Phi} Sin\frac{\pi}{m}
\end{equation}
-- выпрямленная ЭДС неуправляемого преобразователя

\begin{equation}
E_d = E_{d0} Cos \alpha
\end{equation}
-- выпрямленная ЭДС управляемого преобразователя

\begin{equation}
U_d = E_{d} - U_0 - \frac{m}{2\pi} X_\Phi I_d - I_d \left\{
R_\Phi + \left(r_\phi + R_D\right) \left( 1 - \frac{\gamma m}{4 \pi}\right) 
\right\}
\label{three}
\end{equation}
-- Выпрямленное напряжение для угла $\alpha$, или максимальное (при $\alpha=0$).

$U_0$ -- напряжение на вентиле, $R_D$ -- дифференциальное сопротивление вентиля.
$r_\phi$ -- полное эквивалентное сопротивление фазы с учетом сопротивления сети,
приведённого ко вторичной обмотке, сопротивление проводников.

Для точности отметим член ${\displaystyle \frac{\gamma}{4\pi}}$.

\hspace{-3 cm}
\begin{tikzpicture}
  \begin{scope}[xscale=2,yscale=3]
    \newcommand{\alphaa}{26.0 * pi / 180}
    \newcommand{\gammaa}{21.0 * pi / 180}

    \draw[thin, ->] (-2.7, 0) -- (5.8,0) node[right] {$\omega t$};
    \draw[thin, ->] (0,0) -- (0,1.3) node[left] {$U$};
    
    \foreach \x/\xtext in {{-pi/3}/{-\frac{\pi}{m}}, 0,
      {pi/3}/{\frac{\pi}{m}},{pi}/\pi}
    \draw (\x,0.1) -- (\x,-0.1) node [below] {$\xtext$};

    % A,B,C
    \draw[domain=-2.7:5.8, smooth, yellow]
    plot (\x,{cos((\x) r)}); \node[above] at (0.2,1) {$e_k$};
    \draw[domain=-2.7:5.8, help lines, smooth, green]
    plot (\x,{cos((\x-2/3*pi) r)}); \node[above] at (2*pi/3+0.2,1) {$e_{k+1}$};
    \draw[domain=-2.7:5.8, help lines, smooth]
    plot (\x,{cos((\x-4/3*pi) r)}); \node[above] at (-2*pi/3+0.2,1) {$e_{k-1}$};
    % (A+B)/2 
    \draw[domain={pi/3:pi/3+pi}, loosely dotted]
    plot (\x,{cos(\x r)/2 + cos((\x - 2/3*pi) r)/2});


    
    \foreach \qq [evaluate=\qq as \qqshft using \qq*2*pi/3] in {0,...,1}
     {
      \begin{scope}[xshift=\qqshft cm,
          every path/.style={color=red}]

    % Ed' Ed''
    \draw[domain={-pi/3 + \alphaa + \gammaa}:
        {pi/3+\alphaa-0.03},loosely dotted,red]
    plot ({\x-0.03}, {cos(\x r)-0.03})
    -| ({pi/3+\alphaa-0.06},  {cos((\alphaa +pi/3-2/3*pi) r)-0.03})
    [domain={pi/3+\alphaa-0.06}:{pi+\alphaa}]
    plot ({\x-0.03}, {cos((\x-2/3*pi)r)-0.03});

    \draw[domain={-pi/3 + \alphaa + \gammaa}:
                 {pi/3+\alphaa+\gammaa+0.03},loosely dashed,red]
    plot ({\x+0.03}, {cos(\x r)+0.03})
    -| ({pi/3+\alphaa+\gammaa+0.03},
                     {cos((\alphaa+\gammaa +pi/3-2/3*pi) r)+0.03})
    [domain={pi/3+\alphaa+\gammaa+0.03}:{pi+\alphaa}]
    plot ({\x+0.03}, {cos((\x-2/3*pi)r)+0.03});
        
        %Ed
        \draw[domain={-pi/3 + \alphaa + \gammaa}:{pi/3},ultra thick]
        plot (\x, {cos(\x r)});
        \draw[domain={pi/3}:{\alphaa + pi/3},ultra thick]
        plot (\x,{cos(\x r)})
        -| (\alphaa+pi/3, {cos((\alphaa + pi/3) r)/2 + cos((\alphaa-pi/3) r)/2})
        [domain={\alphaa+pi/3}:{\alphaa+pi/3+\gammaa}]
        plot (\x,{cos(\x r)/2 + cos((\x-2/3*pi) r)/2})
        -| (\alphaa+pi/3+\gammaa, {cos((\alphaa +pi/3+ \gammaa-2*pi/3) r) })
        [domain={\alphaa+pi/3+\gammaa}:{pi/3+2*pi/3}]
          plot(\x,  {cos((\x-2/3*pi) r)})
        ;        
       \end{scope}
     }
     %hatch
     \draw[domain=pi/3 + \alphaa:pi/3 + \alphaa+\gammaa,red,
         pattern=north east lines,pattern color=red]
       ({pi/3+\alphaa},  {cos((\alphaa +pi/3-2/3*pi) r)}) --
     plot (\x,{cos(\x r)/2 + cos((\x - 2/3*pi) r)/2});
     \draw[domain=pi/3 + \alphaa:pi/3 + \alphaa+\gammaa,red,
       pattern=north east lines,pattern color=red]
     plot (\x,{cos((\x - 2/3*pi) r)})
     -| ({pi/3 + \alphaa+\gammaa},{cos((pi/3 + \alphaa+\gammaa) r)/2
       + cos((pi/3 + \alphaa+\gammaa - 2/3*pi) r)/2});
     %
     \node[above,color=red] at ({\gammaa + pi/3},1) {$e_{d\:'}$};
     \draw[thin,->,color=red,dotted]  ({\gammaa + pi/3},1) --
     ({\gammaa + pi/3},{cos((pi/3+\alphaa - 2*pi/3) r)+0.05}); 
     \node[below,color=red] at ({pi-pi/6},-0.5) {$e_{d\:''}$};
     \draw[thin,->,color=red,dashed] ({pi-pi/6},-0.5) --
     ({pi/3+\alphaa+\gammaa+0.1},{cos((pi/3+\alphaa+\gammaa) r)+0.1});
     \node[below] at (pi+0.3, -0.6) {${\displaystyle \frac{e_k + e_{k+1}}{2}}$};
     \draw[thin,->] (pi+0.2,-0.6) --
     (pi+0.3,{cos((pi+0.3) r)/2 + cos((pi-2*pi/3+0.3) r)/2 -0.1});
     \draw[thin] (pi/3, -0.35) -- (pi/3,-0.8);
     \draw[thin] (pi/3+\alphaa,-0.1) -- (pi/3+\alphaa, -0.8);
     \draw[thin,->] (pi/3, -0.6) -- (pi/3+\alphaa,-0.6);
     \node[left] at (pi/3, -0.6) {$\alpha\:'=\alpha$};
     \draw[thin] (pi/3+\alphaa+\gammaa,-0.35) --
     (pi/3+\alphaa+\gammaa,-0.8);
     \draw[thin] (pi/3-0.05,-0.65) -- (pi/3+0.05,-0.55);
     \draw[thin,->] (pi/3+\alphaa-0.25,-0.75) -- (pi/3+\alphaa,-0.75);
     \draw[thin,<-] (pi/3+\alphaa+\gammaa,-0.75) --
     (pi/3+\alphaa+\gammaa+0.3,-0.75) node[below] {$\gamma$};
     \draw[thin] (pi, 0.55) -- (pi,1.2);
     \draw[thin] (pi +\alphaa + \gammaa,1) -- (pi +\alphaa + \gammaa,1.2);
     \draw[thin,->] (pi,1.15) -- (pi +\alphaa + \gammaa,1.15)
     node[right] {$\alpha\:'\,' = \alpha+\gamma$};
     \draw[thin] (pi-0.05,1.1) -- (pi+0.05,1.2);
  \end{scope}
  \end{tikzpicture}    

Почти в каждом билете будет задачка, в которой придется рисовать.
кривую
\begin{tikzpicture}
  \draw[ultra thick,red] (0,0) -- (1,0);
  \end{tikzpicture}
можно получить как полусумму двух кривых $e_{d\:'}$
\begin{tikzpicture}
  \draw[thick,loosely dotted,red] (0,0) -- (1,0);
\end{tikzpicture}
и кривой $e_{d\:'\:'}$
\begin{tikzpicture}
  \draw[thick,loosely dashed,red] (0,0) -- (1,0);
\end{tikzpicture}.

В прошлый раз было убедительно доказано, что при соединении двух ЭДС
с равными внутренними сопротивлениями суммарная ЭДС будет
полусуммой этих ЭДС.

Из-за индуктивности ток в $k$-й фазе спадает с конечной скоростью, а в
$k+1$ -й нарастает с конечной скоростью. Разность ЭДС $e_{k+1} - e_k$
падает на сопротивлении. Площадка
\begin{tikzpicture}
  \begin{scope}[xscale=2,yscale=3]
      \newcommand{\alphaa}{26.0 * pi / 180}
    \newcommand{\gammaa}{21.0 * pi / 180}
         %hatch
         \draw[color=red,domain=pi/3 + \alphaa:pi/3 + \alphaa+\gammaa,
           pattern=north east lines,pattern color=red]
         ({pi/3+\alphaa},  {cos((\alphaa +pi/3-2/3*pi) r)}) --
         % (A+B)/2
         plot (\x,{cos(\x r)/2 + cos((\x - 2/3*pi) r)/2})
         %B
         plot (\x,{cos((\x - 2/3*pi) r)})
         -| ({pi/3 + \alphaa+\gammaa},{cos((pi/3 + \alphaa+\gammaa) r)/2
           + cos((pi/3 + \alphaa+\gammaa - 2/3*pi) r)/2});
         
  \end{scope}
\end{tikzpicture}
показывает падение напряжения из-за индуктивности фазы.
Напомню, $X_\phi = 2\pi f L_\phi$ (индуктивность это величина отношения
${\displaystyle \frac{\textcyrillic{потока}}{\textcyrillic{к току}}}$,
где магнитный поток--это магнитный поток рассеяния,
обусловленный током нагрузки, а не током намагничивания.
\begin{tikzpicture}
  \begin{scope}[xscale=1.7,yscale=2.8]
          \newcommand{\alphaa}{26.0 * pi / 180}
    \newcommand{\gammaa}{21.0 * pi / 180}
     % I
     \draw[thin, ->] (-1.0, -1.7) -- (2,-1.7) node[right] {$\omega t$};
     \draw[thin, ->] (0,-1.9) -- (0,-1.2) node[left] {$I$};

     \draw[green] ({\gammaa - pi/3},  -1.4) -- (\alphaa, -1.4);
     \draw[green] (\alphaa, -1.4) -- (\gammaa, -1.7);
     \draw[green] (\gammaa, -1.7) -- ({\alphaa + pi/3},-1.7);

     \draw[yellow] ({\gammaa - pi/3},  -1.7) -- (\alphaa, -1.7);
     \draw[yellow] (\alphaa, -1.7) -- (\gammaa, -1.4);
     \draw[yellow] (\gammaa, -1.4) -- ({\alphaa + pi/3},-1.4);

     \draw[thin] (\alphaa,-1.65) -- (\alphaa,-1.9);
     \draw[thin] (\gammaa,-1.65) -- (\gammaa,-1.9);
     \draw[thin, ->] ({\alphaa-pi/12}, -1.8) -- (\alphaa,-1.8);
     \draw[thin, <-] (\gammaa,-1.8) -- ({\gammaa+pi/6},-1.8) node[below] {$\gamma$};
     \draw[thin,<->] ({(\gammaa+\alphaa)/2+0.6},-1.7) --
     ({(\gammaa+\alphaa)/2+0.6},-1.4);
     \node[left] at  ({(\gammaa+\alphaa)/2+0.6},-1.55) {$I_d$};
  \end{scope}
\end{tikzpicture}
Делали допущение $i_d = I_d$ -- мгновенное равно среднему, пульсаций нет.
Уменьшить пульсации до нуля мы не можем, но уменьшить до уровня, когда
пульсациями можем пренебречь технически возможно.

Мгновенное значение ЭДС при угле регулирования $\alpha$ и угле коммутации
$\gamma$ представим в виду полусуммы

$$
e_{\alpha,\gamma} = \frac{e_{\begin{tikzpicture}
  \draw[thick,dotted,red] (0,0) -- (0.5,0);
\end{tikzpicture}} +e_{
\begin{tikzpicture}
  \draw[thick,dashed,red] (0,0) -- (0.5,0);
\end{tikzpicture}}}{2} =
\frac{e_{d\:'} + e_{d\:''}}{2} =
$$
Это значит, что каждая из них кривая мгновенного напржения у которого нет $\gamma$.
$$
=\frac{e_{d\:'}(\alpha\:' = \alpha, \gamma\:' =0) +
e_{d\:''}(\alpha\:'' = \alpha + \gamma, \gamma\:''=0)}{2} =
$$

\begin{tikzpicture}
  \draw[thick,dashed,red] (0,0) -- (0.5,0);
\end{tikzpicture} -- при угле $\alpha\:'' = \alpha + \gamma$

$$
E_d(\alpha,\gamma) = \frac{E_{d'}(\alpha,\gamma=0) +
  E_{d''}(\alpha'' = \alpha + \gamma, \gamma'' = 0)
}{2} 
$$

Для чего это делаем. Нужно проинтегрировать на интервале повторяемости

$$
E_d = \frac{E_{d0} cos \alpha' + E_{d0} cos \alpha''}{2} =
E_{d0}\frac{cos \alpha + cos(\alpha + \gamma)}{2}
$$

$$
  E_d = E_{d0}\frac{cos \alpha + cos(\alpha + \gamma)}{2}
$$

Проанализируем результат.
На каждом интервале повторяемости теряем эту площадку
\begin{tikzpicture}
  \begin{scope}[xscale=2,yscale=3]
      \newcommand{\alphaa}{26.0 * pi / 180}
    \newcommand{\gammaa}{21.0 * pi / 180}
          %hatch
          \draw[color=red,domain=pi/3 + \alphaa:pi/3 + \alphaa+\gammaa,
            pattern=north east lines,pattern color=red]
          ({pi/3+\alphaa},  {cos((\alphaa +pi/3-2/3*pi) r)}) --
          % (A+B)/2
          plot (\x,{cos(\x r)/2 + cos((\x - 2/3*pi) r)/2})
          %B
          plot (\x,{cos((\x - 2/3*pi) r)})
          -| ({pi/3 + \alphaa+\gammaa},{cos((pi/3 + \alphaa+\gammaa) r)/2
            + cos((pi/3 + \alphaa+\gammaa - 2/3*pi) r)/2});          
\end{scope}
\end{tikzpicture} -- ${\scriptstyle \Delta}E_d$ -- разница между $e_{d'}$
и $e_{d''}$

\begin{equation}
  E_d = E_{d0}\frac{cos \alpha + cos(\alpha + \gamma)}{2}
  \end{equation}

$$
{\scriptstyle \Delta}E_d = \underbrace{E_d(\alpha,\gamma=0)}_
{\textcyrillic{без коммутации}} - \underbrace{E_d(\alpha,\gamma\ne 0)}_
{\textcyrillic{при  реальном }\gamma} 
$$

Смотрим внимательно на уравнение (\ref{three}) где учтено ${\scriptstyle \Delta}E_d$.
Отметим $X_\phi$ -- падение на индуктивности. Кажущаяся нелепость:
Постоянный ток умножается на индуктивность. Это есть ЭДС самоиндукции
$\frac{m}{2\pi}X_\phi I_d$ -- коммутационное.
В учебниках пишут ${\scriptstyle \Delta}U$, у нас написано $E$, подчеркивая
что природа этого -- падение на самоиндукции.

\begin{equation}
{\scriptstyle \Delta}U_{d\gamma} = {\scriptstyle \Delta}E_{d\gamma} =
E_{d0}\frac{cos\alpha -cos(\alpha + \gamma)}{2} =\frac{m}{2\pi}X_\phi I_d
\label{stressed}
\end{equation}

Коэффициэнт $R_{K} ={\displaystyle \frac{m}{2\pi}X_\phi}$ мы умножаем на ток
нулевой частоты. Это фикция, но так говорят, мы упоминаем шуткаи ради.
Но это сопротивление не греется от проходящего тока, поэтому оно фиктивное.

Из предыдущего (\ref{stressed}) уравнения можем найти $\gamma$

\begin{equation}
  \gamma = arccos\left[cos \alpha -\left(\frac{\frac{\displaystyle m}
      {\displaystyle \pi}X_\phi I_d}{E_{d0}}\right)\right] - \alpha  
\end{equation}

В уравнении (\ref{stressed}) ${\scriptstyle \Delta}U_{d\gamma}$ индекс $\gamma$
подчёркивает, что это падение напряжения вследствие угла коммутации $\gamma$

\begin{equation}
  \gamma = arccos\left[cos \alpha -2\left(\frac{{\scriptstyle \Delta}U_{d\gamma}}
         {E_{d0}}\right)\right] - \alpha  
\end{equation}

\begin{equation}
  \gamma = arccos\left[cos \alpha -\left(2\frac{R_K I_d}
         {E_{d0}}\right)\right] - \alpha  
\end{equation}

где
\begin{equation}
  R_K = \frac{m}{2\pi}X_\phi
\end{equation}

Еще раз перепишем формулу (\ref{three}):
$$
U_d = E_d - U_0 - I_d\left[R_K +R_{\textcyrillic{эквивалентное}}\right]
$$
При практических расчётах приближённно $U_0$ мало, также пренебрегают
влиянием $\gamma$ на коммутационное сопротивление.
$$
\hspace{-2 cm}
U_d = E_{d0} \;cos\;\alpha - I_d\left(R_K +
\underbrace{r_\phi}_{
  \begin{array}{c}
    \textcyrillic{на стороне переменного} \\
      \textcyrillic{тока выпрямителя}
    \end{array}
} +
\underbrace{R_\Phi}_{
    \begin{array}{c}
      \textcyrillic{на стороне постоянного}\\
      \textcyrillic{тока выпрямителя}
      \end{array}
}
\right)
$$
Эту формулу применяют для практических расчётов.

$$
R_K =\frac{m}{2\pi} X_\phi -\left(r_\phi + R_D\right) \frac{\gamma}{4\pi}
$$
-- эта формула для совсем точных расчётов.
Активное сопротивление на постоянном токе уменьшается.

Уравнение (\ref{three}) -- статические характеристики. В статическом режиме
$\alpha$ и $I$ не меняются. То что присутствует $\gamma$ -- его нужно
исключить, решив уравнение относительно $\gamma$

$
U_d = F(\alpha,I_d) 
$ -- можно рассматривать уравнение как функцию двух переменных.

$U_d = f(\alpha)$ при $I_d = const$ -- {\bf регулировочные характеристики}. Почему
во множественном числе? потому что для разных $I_d$.

$U_d = f(I_d)$ при $\alpha=const$ -- {\bf внешние характеристики}.

Как их строят? Строят семейство регулировочных характеристик и семейство
внешних характеристик.

При $I_d$ малых (пренебрегаем падением на внутренних элементах). Это одна
из регулировочных характеристик, причём основная, а внешних -- много.

\subsection{регулировочные характеристики}
характеристики строятся в относительных единицах.



\begin{tikzpicture}
  \begin{scope}[xscale=3.5,yscale=4.5]
    \newcommand{\alphaa}{26.0 * pi / 180}
    \newcommand{\gammaa}{21.0 * pi / 180}

    \draw[thin, ->] (0, 0) -- (3.5,0) node[right] {$\alpha$};
    \draw[thin, ->] (0,-1.2) -- (0,1.3) node[left] {$\frac{U_d}{E_{d0}}$};
    \draw[thin,loosely dashed] (0,-1) -- (pi,-1);
    
    \foreach \x/\xtext in {{pi/6}/{\frac{\pi}{6}}, {pi/3}/{\frac{\pi}{3}},
      {pi/2}/{\frac{\pi}{2}}, {2*pi/3}/\frac{2\pi}{3},
      {5*pi/6}/\frac{5\pi}{6},
      {pi}/{\pi}}
    \draw (\x,0.1) -- (\x,-0.1) node [below] {$\xtext$};

    \foreach \y/\ytext in {-1/-1,-0.5/-0.5,0.5/0.5,1,1}
    \draw (0.1,\y) -- (-0.1,\y) node [left] {$\ytext$};

    % A
    \draw[domain=0:pi, help lines, smooth]
    plot (\x,{cos((\x) r)});
    % нижняя граница
    \draw[domain=0:{pi-(23/180)*pi}, help lines, smooth,dashed]
    plot (\x,{cos((\x) r) - 0.08});
    %
    \draw[thin,<->] ({pi/4}, {cos((pi/4) r)}) --
    ({pi/4}, {cos((pi/4) r) - 0.08});
    \draw[thin] ({pi/4}, {cos((pi/4) r) - 0.04}) --
    ({pi/4+0.3}, {cos((pi/4) r)}) node[right]
         {${\displaystyle \frac{{\scriptstyle \Delta}U_d}{E_{d0}}}$};
    %
    \node[below] at ({pi/6},0.45)
         {$\begin{array}{c}
             \textcyrillic{выпрямительный}\\
             \textcyrillic{режим}
           \end{array}$
         };
    \node[below] at ({5*pi/6},-0.3)
         {$\begin{array}{c}
             \textcyrillic{инверторный}\\
             \textcyrillic{режим}
           \end{array}$
         };

         \node[below] at (2*pi/3,-0.85) {${\scriptstyle \Delta}U_d>0$};
         \draw[thin,->] (2*pi/3,-0.85) -- (2*pi/3+0.25,-0.8);
         \draw[thin,<-] (5*pi/6-0.1,-0.8) -- (5*pi/6+0.4,-0.7)
         node[right]
         {$\begin{array}{c}
             {\displaystyle \frac{E_d}{E_{d0}} {\textcyrillic{-- всё падение}}}\\
             {\textcyrillic{напряжения}}\approx 0\\
             {\textcyrillic{пренебрегаем всем}}
             \end{array}$};
  \end{scope}
\end{tikzpicture}

$\alpha\ge 0$, потому что не сможем включить раньше. Какой теоретический
предел $\alpha$
\begin{tikzpicture}
  \begin{scope}[xscale=1.5,yscale=3]
    \newcommand{\alphaa}{26.0 * pi / 180}
    \newcommand{\gammaa}{21.0 * pi / 180}
    \draw[thin, ->] (0, 0) -- (pi+pi/6+pi/3,0) node[right] {$\omega t$};
    \draw[thin, ->] (0,0) -- (0,1.3) node[left] {$U$};
    
    \foreach \x/\xtext in {{pi/6}/{\frac{\pi}{6}},
      {pi/3}/{\frac{\pi}{3}}, {pi}/{\pi},{4*pi/3}/{\frac{4\pi}{3}}}
    \draw (\x,0.1) -- (\x,-0.1) node [below] {$\xtext$};

    % A,B,C
    \draw[domain=-0.1:{pi/6+pi+pi/3}, smooth, yellow]
    plot (\x,{cos((\x) r)}); \node[above] at (0.2,1) {$e_k$};
    \draw[domain=-0.1:{pi/6+pi+pi/3}, help lines, smooth, green]
    plot (\x,{cos((\x-2/3*pi) r)}); \node[above] at (2*pi/3+0.2,1) {$e_{k+1}$};

    \draw[thin] ({pi/3},-0.35) -- (pi/3,-1.27);
    \draw[thin] ({4*pi/3},{cos((4*pi/3) r)-0.1}) --  ({4*pi/3},-1.27);
    \draw[thin,<->] ({pi/3},-1.15) -- ({4*pi/3},-1.15);
    \node[below] at ({(pi/3+4*pi/3)/2},-1.25)
         {теоретический диапазон $\alpha=\pi$};
  \end{scope}
\end{tikzpicture}

Обычно строят графики $I_d \approx 0$ и $I_d\approx I_{\textcyrillic{номинальный}}$
(близко к номиналу). Падение
\begin{tikzpicture}
  \begin{scope}
     \draw[thin,<->] (0, 0.25) --
    (0, -0.25);
    \draw[thin,<-] (0.1, 0) --   (0.5, 0) node[right]
         {${\displaystyle \frac{{\scriptstyle \Delta}U_d}{E_{d0}}}$};
  \end{scope}
  \end{tikzpicture}
-- единицы процента, до $10\%$. $30\%$ быть не может, поскольку, 15\% падение
на реактивных и 15\% падение на активных сопротивлениях, а 15\% потерь
недопустимо для КПД выпрямителей. У выпрямителях КПД $\approx$ 95\%.

Выпрямленное напряжение становится меньше 0, как это понять? $I>0$ всегда.
Это не диод, $\alpha \ne 0$, это тиристор. $I\cdot U<0$, значит преобразователь
преобразует в обратном направлении. И КПД $\ne 95\%$, потом поговорим об этом
подробнее.

Когда $U \approx 0$, напряжение примерно 10\% от номинала, КПД уменьшается не
потому что увеличиваются потери, а потому что уменьшается мощность.

$P_d = U_dI_d <0$ -- инверторный режим. Инверторный режим преобразователя обычно имеет место когда $E_d<0$ и $\alpha>90^\circ$.

%\hspace{-3 cm}
\begin{tikzpicture}
  \begin{scope}[xscale=2,yscale=3]
    \newcommand{\alphaa}{106.0 * pi / 180}
    \newcommand{\gammaa}{21.0 * pi / 180}

    \draw[thin, ->] (-0.2, 0) -- (5.8,0) node[right] {$\omega t$};
    \draw[thin, ->] (0,0) -- (0,1.3) node[left] {$U$};
    
    \foreach \x/\xtext in {0, {pi/3}/{\frac{\pi}{m}},{pi}/\pi}
    \draw (\x,0.1) -- (\x,-0.1) node [below] {$\xtext$};

    % A,B,C
    \draw[domain=-0.2:5.8, smooth, yellow]
    plot (\x,{cos((\x) r)}); \node[above] at (0.2,1) {$e_k$};
    \draw[domain=-0.2:5.8, help lines, smooth, green]
    plot (\x,{cos((\x-2/3*pi) r)}); \node[above] at (2*pi/3+0.2,1) {$e_{k+1}$};
    \draw[domain=-0.2:5.8, help lines, smooth]
    plot (\x,{cos((\x-4/3*pi) r)}); \node[above] at (4*pi/3+0.2,1) {$e_{k-1}$};
    % (A+B)/2 
    \draw[domain={pi/3:pi/3+pi}, loosely dotted]
    plot (\x,{cos(\x r)/2 + cos((\x - 2/3*pi) r)/2});


    
    \foreach \qq [evaluate=\qq as \qqshft using \qq*2*pi/3] in {0,...,1}
     {
      \begin{scope}[xshift=\qqshft cm,
          every path/.style={color=red}]
        
        %Ed
%        \draw[domain={-pi/3 + \alphaa + \gammaa}:{pi/3},ultra thick]
%        plot (\x, {cos(\x r)});
        \draw[domain={\alphaa+\gammaa}:{\alphaa + pi/3},ultra thick]
        plot (\x,{cos(\x r)})
        -| (\alphaa+pi/3, {cos((\alphaa + pi/3) r)/2 + cos((\alphaa-pi/3) r)/2})
        [domain={\alphaa+pi/3}:{\alphaa+pi/3+\gammaa}]
        plot (\x,{cos(\x r)/2 + cos((\x-2/3*pi) r)/2})
        -| (\alphaa+pi/3+\gammaa, {cos((\alphaa +pi/3+ \gammaa-2*pi/3) r) })

        [domain={\alphaa+\gammaa-pi+2*pi/3}:{\alphaa-pi/3+2*pi/3}]
          plot(\x,  {cos((\x) r)})
        ;        
       \end{scope}
     }

     \node[below] at (pi+0.3, -0.6) {${\displaystyle \frac{e_k + e_{k+1}}{2}}$};
     \draw[thin,->] (pi+0.2,-0.6) --
     (pi+0.3,{cos((pi+0.3) r)/2 + cos((pi-2*pi/3+0.3) r)/2 -0.1});
     % alpha
     \draw[thin] (pi/3, 0.35) -- (pi/3,0.8);
     \draw[thin] (pi/3+\alphaa,0.1) -- (pi/3+\alphaa, 0.8);
     \draw[thin,->] (pi/3, 0.75) -- (pi/3+\alphaa,0.75);
     \node[left] at (pi/3, 0.75) {$\alpha$};
     % beta
     \draw[thin] (4*pi/3,-0.4) -- (4*pi/3,0.8);
     \draw[thin,<->] (pi/3+\alphaa, 0.6)--(4*pi/3, 0.6) node[right]
           {$\beta=\pi-\alpha$};
     % U_инвертора
     \draw[thin,color=red] (pi/2,-0.36) -- (5*pi/6+pi,-0.36) node[right]
     {$U_{\textcyrillic{инв}}$};

  \end{scope}
  \end{tikzpicture}    

За счет чего течёт ток?
\begin{figure}[H]
\begin{circuitikz}\draw
(0,4) to[L] (0,2)
node[left] at (0,3.5) {-}
node[left] at (0,2.5) {+}
node[left] at (0,2) {${\textcyrillic{было}}$}
node[right] at (0,3.5) {(+)}
node[right] at (0,2.5) {(-)}
node[right] at (0,2) {${\textcyrillic{стало}}$}
to[Ty,mirror] (0,0)
(2,4) to[L,*-] (2,2)
to[Ty,mirror] (2,0)
(4,4) to[L,*-] (4,2)
to[Ty,mirror] (4,0)
(0,4) to[short,-*] (2,4)
to[short,-*] (4,4)
to[short] (8,4)
(0,0) to[short,-*] (2,0)
to[short,-*] (4,0)
to[short] (8,0)
to (8,1)
to[american voltage source] (8,3)
to[short] (8,4)
node[right] at (8,2.5) {(+)}
node[right] at (8,1.5) {(-)}
node[right] at (8.2,2) {$E_{\textcyrillic{н}}{\textcyrillic{ЭДС нагрузки}}$}
node[right] at (8,0.5) {$
\begin{array}{c}
{\textcyrillic{все напряжения}}\\
{\textcyrillic{отсчитываем}}\\
{\textcyrillic{относительно}}\\
{\textcyrillic{нулевого провода}}
\end{array}
$}
;\end{circuitikz}
\end{figure}
Все напряжения отсчитываются относительно нулевого провода и поэтому
 ток течёт за счёт ЭДС нагрузки!

$\alpha$ двигаем дальше, от этого положительная часть уменьшается,
а отрицательная увеличивается. Если энергия течёт в сеть, значит
\begin{circuitikz}\draw
node[right] at (0,0) {$e_{\textcyrillic{н}}$}
node[left] at (0,0.2) {+}
node[left] at (0,-0.2) {-}
;\end{circuitikz}.

Теперь рисуем картинку при $\alpha>90^\circ$ Введем угол $\beta$:
$$
\beta = \pi - \alpha
$$

$\alpha$ -- угол запаздывания, $\beta$ -- угол опережения.
Рисуем ток.
\begin{figure}[H]
\begin{tikzpicture}
  \begin{scope}[xscale=2,yscale=3]
      \newcommand{\alphaa}{106.0 * pi / 180}
          \newcommand{\gammaa}{21.0 * pi / 180}

    \draw[thin, ->] (-0.2, 0) -- (5.8,0) node[right] {$\omega t$};

    \draw[domain=-0.2:1.8, smooth, yellow]
        plot (\x,0)
    [domain=1.8:2.2, smooth, yellow]
         plot (\x, {1/0.4*(\x-1.8)})
    [domain=2.2:3.8, smooth, yellow]
         plot (\x,1)
    [domain=3.8:4.2, smooth, yellow]
         plot (\x, {-1/0.4*(\x-4.2)})
    [domain=4.2:5.8, smooth, yellow]
         plot (\x,0);
    \draw[domain=-0.2:1.8, help lines, smooth, green]
         plot (\x,1)
    [domain=1.8:2.2, help lines, smooth, green]
        plot (\x,{-1/0.4*(\x-2.2)})
    [domain=2.2:3.8, help lines, smooth, green]
         plot (\x,0)
    [domain=3.8:4.2, help lines, smooth, green]
        plot (\x, {1/0.4*(\x-3.8)})
    [domain=4.2:5.8, help lines, smooth, green]
         plot (\x,1);
    %
    \draw[domain=1.8:3.6, smooth, dashed, yellow]
    plot (\x, {cos(1.4*(\x - 2.7) r) - cos(1.4*(1.8 - 2.7) r)});
    \draw[domain=1.8:3.6, help lines, smooth, loosely dashed,green]
    plot (\x, {-cos(1.4*(\x - 2.7) r) + cos(1.4*(1.8 - 2.7) r)+1});
    %
    \draw (2.7,0.1) -- (2.7,-0.1) node[below] {$\pi$};
\end{scope}
\end{tikzpicture}        
\end{figure}

$\beta>0$ Если не выключить $k$-й вентиль до точки пересечения, то он уже
никогда не выключится.
Если $L$ большая, коммутация не кончится. Темп изменения тока пропорционален
ЭДС. Но ЭДС, а значит и скорость нарастания тока уменьшаются, а дальше
скорость станет отрицательной, а значит, коммутация не будет продолжаться.

Коммутация вентилей должна закончиться до момента отсчета угла $\beta$.
В противном случае переключения фаз не произойдёт. $\beta$ не просто
больше 0, а $\beta>\gamma$. Предположим, что разность $\beta-\gamma$ мала.
Сравним её со временем выключения вентиля. Вентиль может самопроизвольно
включиться $I\approx n$ носителей. $I$ убывает, но рекомбинации носителей
не произошло(вентиль не восстановился).
$\beta-\gamma>{\textcyrillic{времени запирания тиристора}}$.
$\omega t_{\textcyrillic{выкл}}= \sigma$ -- угол запирания или выключения вентиля.

$\beta>\gamma+\sigma$ -- каждый раз условие увеличивается.

Практически, учитывается наличие несимметрии напряжения сети, несимметрии
углов регулирования $\alpha$ (несимметрия СИФУ), с учётом несинусоидальности
и разброса параметров тиристора необходим запас по углу для устойчивой
работы инвертора ($\psi$ -- угол запаса)

\begin{equation}
\beta \ge (\gamma+\beta+\psi) = \beta_{min}
\label{steady_invertor}
\end{equation} -- условие устойчивой работы инвертора.
отсюда, $\alpha_{max} = \pi - \beta_{min}$


\begin{tikzpicture}
  \begin{scope}[xscale=1.5,yscale=1.5]
      \newcommand{\alphaa}{106.0 * pi / 180}
      \newcommand{\gammaa}{21.0 * pi / 180}
    % A,B,C
        \draw[domain=0.1:1.4, help lines, smooth, black]
            plot (\x,{cos((\x) r)});
        \draw[domain=0.1:1.4, help lines, smooth, green]
            plot (\x,{cos((\x-2/3*pi) r)});
        \draw[ultra thick] (1.6,0.6) -- (2,0.6);
        \draw[ultra thick] (1.6,0.4) -- (2,0.4);
        \draw[ultra thick] (1.9,0.7) -- (2.1,0.5);
        \draw[ultra thick](1.9,0.3) -- (2.1,0.5);
        \draw[domain=0.1:1.4, help lines, smooth, dashed, black]
                    plot ({2+\x},{cos((\x) r)});
       \draw[domain=0.1:1.4, help lines, smooth, black]
                           plot ({1.8+\x},{cos((\x) r)});                           
        \draw[domain=0.1:1.4, help lines, smooth, green]
                    plot ({2+\x},{cos((\x-2/3*pi) r)});
                                        
\end{scope}
\end{tikzpicture}
Если сместилась фаза, $\beta$ тоже сместилась. Несимметрия СИФУ - средний угол
18,22. Там где не хватит $\beta$.
Несинусоидальность фазы - опять приводит к необходимости увеличить $\beta$.
От температуры, тока, который был, напряжение приложенное.
${\displaystyle \Delta}_\alpha + {\displaystyle \Delta}\phi$ -- несимметрия
сети + $ {\displaystyle \Delta}\phi$ --несинусоидальность +
 ${\displaystyle \Delta}$ (разброс на углы включения).
 Эмпирически $\beta_{min} = 15...30^\circ$, $\alpha_{max}=150,165^\circ$
К чему приведет невыполнение (\ref{steady_invertor})? Невыполнение (\ref{steady_invertor}) может привести к ``опрокидыванию'' инвертора: вместо инвертора получим
выпрямительный режим -- аварийный режим, связанный с переходом преобразователя
в выпрямительный режим с резким возрастанием выпрямленного тока. Выпрямленный ток
может возрастать до значений, близких к току К.З, Иногда называют током
``двойного'' К.З., т.е. к К.З. одновременно и преобразователя и источника.


\begin{figure}[H]
\begin{circuitikz}\draw
(0,4) to[L] (0,2)
node[left] at (0,3.5) {-}
node[left] at (0,3) {$U_d$}
node[left] at (0,2.5) {+}
node[left] at (0,2) {${\textcyrillic{выпрямленное}}$}
%node[right] at (0,3.5) {(+)}
%node[right] at (0,2.5) {(-)}
%node[right] at (0,2) {${\textcyrillic{стало}}$}
to[Ty,mirror] (0,0)
(1,4) to[L,*-] (1,2)
to[Ty,mirror] (1,0)
(2,4) to[L,*-] (2,2)
to[Ty,mirror] (2,0)
(0,4) to[short,-*] (1,4)
to[short,-*] (2,4)
to[short] (8,4)
(0,0) to[short,-*] (1,0)
to[short,-*] (2,0)
to[short] (8,0)
to (8,1)
to[american voltage source] (8,3)
to[short] (8,4)
%
(3,3.9) to[short,v^=$U_{\textcyrillic{инв}}$] (3,0.1)
(2.9,0.2) -- ++ (0.1,-0.1)
-- ++ (0.1,0.1)
(2.9,3.8) -- ++ (0.1,0.1)
-- ++ (0.1,-0.1)
%
node[right] at (8,2.5) {+}
node[right] at (8,1.5) {-}
node[right] at (8.2,2) {$E_{\textcyrillic{н, противо-ЭДС}}$} 
node[right] at (8,0.5) {}
% I
(6,3.8) to[short,i^>={$I_d$}] (5,3.8)
;\end{circuitikz}
\end{figure}

$$
I_d = \frac{\overbrace{E_{\textcyrillic{н}}}^{\textcyrillic{согласно с током}} -
\overbrace{\mid E_d\mid}^{\textcyrillic{само отрицательно}}
}{R_{\textcyrillic{н}} +R_{\textcyrillic{эквив}}}
$$
Была разность ЭДС, а станет сумма. $U_{\textcyrillic{инв}}$ -- аккумулятор,
выпрямитель, солнечная батарея. $R$ -- маленькая, $R_{\textcyrillic{н}} - 5\%$
и если напряжение не в плюсе а в минусе, то в 19 раз вырастет ток. Получаем
КЗ и для инвертора и для нагруки. Не двойной ток, а ``удвоение'' явления.
А ток $I \approx I_{\textcyrillic{К.З.}}/2$.

Опрокидывание инвертора приводит к аварийному отключению преобразователя. А есть и электронные средства.

Инверторный режим принципиально менее надежен чем выпрямительный режим.

Опрокидывание инвертора может происходить в случае
\begin{itemize}
\item кратковременного исчезновения или резкого уменьшения напряжения питающей сети;
\item в случае пропуска (даже одиночного) управляющего импульса.
\item в случае ложного несвоевременного срабатывания (даже одиночного) какого-либо
вентиля.
\end{itemize}

4 причины: 1) -- невыполнение условий.
пропустили импульс -- пропал контакт.
Ложное отпирание может произойти вследствие сбоя СИФУ -- ток растет лавиной.

Где используется инверторный режим.

Солнечные батареи $-\;\leftarrow;=$. Где инверторы крайне необходимы.
Гидрогенераторы мощностью 100МВт возбуждаются с помощью 10МВт.
Для управления генераторами а обмотке возбуждения гигантская магнитная энергия,
её и нужно передать в сеть. Для включения-выключения генератора требуется форсировка в 5-8-14 раз. На синхронных компенсаторах
в 10-13 раз, 1300В вместо 100в. И такая же скорость снижения должна быть.

Отключение линий.
Разгон -- выпрямительный режим, Самое экономное торможение -- рекуперация,инверторный режим.

\subsection{Прерывистый режим работы преобразователя}
Режим прерывистого выпрямленного тока. Ток нагрузки, правильный ток $I_d$. А если
он(ток) $I_d$ начинает прерываться -- это прерывистый режим.
Пульсация большая. 1й случай -- это нонсенс. А вторая ситуация, когда ток маленький
и ток сравним с пульсацией.

$$
\frac{E_d - E_{\textcyrillic{н}}}{R_\Sigma} = I_d 
$$
В этой формуле $I_d$ -- постоянный, варьируется за счет $E_d - E_{\textcyrillic{н}}$.

$$
e_d = E_d + \left(e_d\right) =const
$$
$\alpha$ не меняется.
$$
\frac{\left(e_d\right)}{{\textcyrillic{переменная}} ``Z''} = const
$$
а ток разложить в ряд Фурье.

Двигатель в холостом режиме $E_{\textcyrillic{двиг}} \approx
E_{\textcyrillic{источника}}$

\begin{tikzpicture}
  \begin{scope}[xscale=1,yscale=1]
  \newcommand{\alphaa}{106.0 * pi / 180}
  \newcommand{\gammaa}{21.0 * pi / 180}
  \draw[thin, ->] (-0.2, 0) -- (9,0) node[right] {$\omega t$};
  \draw[thin,dashed] (-0.2, 2) -- (3.5,2);
  \draw[thin] (0,2) -- (0.5,2.5) -- (1.5,1.5) -- (2.5,2.5) -- (3.5,1.5);
% A,B,C
  % =>
  \draw[ultra thick] (3.8,1.1) -- (4.5,1.1);
  \draw[ultra thick] (3.8,0.9) -- (4.5,0.9);
  \draw[ultra thick] (4.4,1.2) -- (4.6,1.0);
  \draw[ultra thick](4.4,0.8) -- (4.6,1.0);

 \draw[thin,dashed] (5, 0.2) -- (8.5,0.2);
 \draw[thin,red] (5,0.2) -- (5.5,0.7) -- (6.2,0) -- %(6.5,-0.3) -- (7.5,0.7)
 (6.8,0) -- (7.5,0.7)-- (8.2,0) -- (8.5,0);


\end{scope}
\end{tikzpicture}

Пересечь ``нуль'' не может, потому что тиристоры не проводят ток в обратном направлении. Такая
ситуация бывает в при работе двигателя режиме Х.Х. Ток плохой(маленький), но управлять двигателем
нужно и в этом режиме. Для установок,например, для того чтобы вывести двигатель в нужную точку
(позиционирование х,у), это может иметь важное значение.

\hspace{-3 cm}
\begin{tikzpicture}
  \begin{scope}[xscale=2,yscale=3]
    \newcommand{\alphaa}{106.0 * pi / 180}
    \newcommand{\gammaa}{21.0 * pi / 180}
    \newcommand{\En}{0.37}

    \draw[thin, ->] (-0.2, 0) -- (5.8,0) node[right] {$\omega t$};
    \draw[thin, ->] (0,0) -- (0,1.3) node[left] {$U$};

    \draw[thin, ->] (-0.2, -2) -- (5.8,-2) node[right] {$\omega t$};
    \draw[thin, ->] (0,-2) -- (0,-1) node[left] {$I$};


    \foreach \x/\xtext in {0, {pi/3}/{\frac{\pi}{m}},{pi}/\pi}
    \draw (\x,0.1) -- (\x,-0.1) node [below] {$\xtext$};

    % A,B,C
    \draw[domain=-0.2:5.8, smooth, yellow]
    plot (\x,{cos((\x) r)}); \node[above] at (0.2,1) {$e_{k-1}$};
    \draw[domain=-0.2:5.8, help lines, smooth, green]
    plot (\x,{cos((\x-2/3*pi) r)}); \node[above] at (2*pi/3+0.2,1) {$e_k$};
    \draw[domain=-0.2:5.8, help lines, smooth]
    plot (\x,{cos((\x-4/3*pi) r)}); \node[above] at (4*pi/3+0.2,1) {$e_{k+1}$};

    % E_н
    \draw[thin] (-0.2,\En) -- (5.8,\En);
    \draw[thin,<->] (5.6,\En)--(5.6,0)node[midway,right]
    {$E_\textcyrillic{н}$};     

    \foreach \qq [evaluate=\qq as \qqshft using \qq*2*pi/3] in {0,...,1}
     {
      \begin{scope}[xshift=\qqshft cm,
          every path/.style={color=red}]
        
        %Ed
        \draw[domain={pi/3-0.2}:{pi/3 + 0.4},ultra thick]
        plot (\x, {cos(\x r)})
        -| ({pi/3+0.4}, \En);
        \draw[domain={pi/3 + 0.4}:{pi-0.2},ultra thick]
        plot (\x, \En)
        -| ({pi-0.2},{cos((pi-0.2-2/3*pi) r)});

        %Id
        \draw[domain={pi/3-0.2}:{pi/3 + 0.4},ultra thick]
plot(\x, {-2 + (pi/0.6*cos((\x-pi/3-0.1) r) - pi/0.6*cos((0.3) r) });
%plot(\x, {-3*(\x-pi/3+0.7)*(\x-pi/3+0.2)*(\x-pi/3-0.4)-2});
        \draw[domain={pi/3 + 0.4}:{pi-0.2},ultra thick]
        plot (\x, -2);
     \end{scope}

      % подписи под Id
      \draw[thin] (pi/3-0.2,-2)-- (pi/3-0.2,-2-0.12);
       \draw[thin] (pi-0.2,-2)-- (pi-0.2,-2-0.12);
        \draw[thin,<->] (pi/3-0.2,-2.1)--(pi-0.2,-2.1)
        node[midway,below] {$2\pi/m$};
        %alpha
        \draw[thin] (pi/3, 0.6) -- (pi/3, 1);
        \draw[thin] (pi/3-0.1, 0.7) -- (pi/3+0.1, 0.9)
          node[above right] {$\alpha$};
        \draw[thin] (pi-0.2, 0.7) -- (pi-0.2, 1);
%        \draw[thin,<-] (pi-0.2, 0.9) -- (pi+0.1, 0.9) node[right] {$\alpha$};
        \draw[thin,-latex] (pi/3,0.8) -- (pi-0.2, 0.8);
        % \lamdba - угол проводимости
        \draw[thin] (pi-0.2, \En-0.1) -- (pi-0.2, -0.5);
        \draw[thin] (pi+0.4, 0.1) -- (pi+0.4, -0.5);
         \draw[thin,<->] (pi-0.2, -0.4) -- (pi+0.4, -0.4) node[right]
         {$\lambda{\textcyrillic{ -- угол проводимости}}$};
         \node[right] at (pi/3,-1.1) {$
         \begin{array}{c}         
         {\textcyrillic{производная}} \\
         {\textcyrillic{пропорциональна}} \\
         {\textcyrillic{разности напряжений}}
         \end{array}$};
         \draw[thin,->] (pi-0.3,-1.3) -- (pi-0.1,-1.8);
          \draw[thin,->] (2*pi/3,-0.9) -- (pi-0.25, 0.5);
          %
          \draw[thin,dashed] (pi+0.14,-1.75)--(pi+0.14,0.38);
          \draw[thin,<-] (pi+0.1,-2.05) -- (pi+0.1,-2.2) node[right]
          {$\textcyrillic{максимум немного левее из-за } i\cdot r_\phi$};

          %заштрихованная площадка
          \draw[thin,domain={pi-0.2}:{pi + 0.14},pattern=north west lines,
          pattern color=blue]
          (pi-0.2,\En)--
          plot (\x,{cos((\x-2/3*pi) r)});
          \draw[thin,blue] (pi,\En +0.2) --(pi+0.5,\En +1.2)
          node[above] {
$\begin{array}{c}
\textcyrillic{заштрихованная площадка }\sim
          \textcyrillic{ току}\\
          e_\phi -E\textcyrillic{н} -\xcancel{i\cdot r_\pi = L
          \frac{\partial i}{\partial t}}
          \end{array}$};
          
   }
\end{scope}
\end{tikzpicture}
Если бы не было $i\cdot r_\phi$, то максимум тока был бы где
$U=E_\textcyrillic{н}$ (производная в точке максимума равна 0).

Угол проводимости $\lambda$ -- угол, в течении которого ток больше 0.

Никакого угла коммутации не будет. Ток проводит -- пауза -- проводит
следующий вентиль.

Постоянная ЭДС нагрузки
$$
\underbrace{E_d + (e_d)}_{\textcyrillic{выпрямленное}} -
E_{\textcyrillic{н}} = \underbrace{
L \frac{\partial i_d}{\partial t} +R_\Sigma i_d}_
{\textcyrillic{и нагрузка и преобразователь}}
$$
$U_0$ включили в $E_{\textcyrillic{н}}$. Разность
$E_d + (e_d) -E_{\textcyrillic{н}}$ больше 0, чтобы протекал ток.



\begin{figure}[H]
\begin{circuitikz}\draw
(0,4) to[L] (0,2)
to[Ty,mirror] (0,0)
(1,4) to[L,*-] (1,2)
to[Ty,mirror] (1,0)
(0,4) to[short,-*] (1,4)
to[short] (4,4)
(0,0) to[short,-*] (1,0)
to[short] (4,0)
to[american voltage source] (4,1.5)
to[L] (4,2.75)
to[R] (4,4)
%
node[right] at (4.2,1.3) {-}
node[right] at (4.2,0.2) {+}
node[right] at (4.2,0.75) {$E_{\textcyrillic{н}}
{\textcyrillic{ положительное}}$}
;\end{circuitikz}
\end{figure}
В паузе будет ЭДС нагрузки, все вентили выключены.

$$
U_d = \frac{1}{2\pi/m}\left[
\int\limits_{-\frac{\pi}{m} + \alpha}^{-\frac{\pi}{m} + \alpha + \lambda}
\sqrt{2}E_{2\phi}cos(\omega t) d\omega t +
E_{\textcyrillic{н}} \left(\frac{2\pi}{m} - \lambda \right)
\right] =
$$

В вырaжении для $U_d$ выбор начала отсчета и слагаемые изображены на рисунке


\begin{tikzpicture}
  \begin{scope}[xscale=2,yscale=3]
    \newcommand{\alphaa}{106.0 * pi / 180}
    \newcommand{\gammaa}{21.0 * pi / 180}
    \newcommand{\En}{0.37}

    \draw[thin, ->] (-0.2, 0) -- (5.8,0) node[right] {$\omega t$};
    \draw[thin, ->] ({2/3*pi},0) -- ({2/3*pi},1.3) node[left] {$U$};

    % A,B,C
%    \draw[domain=-0.2:5.8, smooth, yellow]
%    plot (\x,{cos((\x) r)}); \node[above] at (0.2,1) {$e_{k-1}$};
    \draw[domain=-0.2:5.8, help lines, smooth, green]
    plot (\x,{cos((\x-2/3*pi) r)}); \node[above] at (2*pi/3+0.2,1) {$e_k$};
    \draw[domain=-0.2:5.8, help lines, smooth]
    plot (\x,{cos((\x-4/3*pi) r)}); \node[above] at (4*pi/3+0.2,1) {$e_{k+1}$};

    % E_н
    \draw[thin] (-0.2,\En) -- (5.8,\En);
    \draw[thin,<->] (5.6,\En)--(5.6,0)node[midway,right]
    {$E_\textcyrillic{н}$};

      % стрелка к -pi/2
        \draw[thin,latex-] (pi/3,0.7) -- (2*pi/3, 0.7) node[midway, below] {$\pi/2$};
      %alpha
        \draw[thin] (pi/3, 0.6) -- (pi/3, 1);
        \draw[thin] (pi/3-0.1, 0.7) -- (pi/3+0.1, 0.9)
          node[above right] {$\alpha$};
        \draw[thin] (pi-0.5, 0.7) -- (pi-0.5, 1);
        \draw[thin,-latex] (pi/3,0.8) -- (pi-0.5, 0.8);
      % \lamdba - угол проводимости
        \draw[thin] (pi-0.5, \En-0.1) -- (pi-0.5, -0.5);
        \draw[thin] (pi+0.4, 0.1) -- (pi+0.4, -0.5);
         \draw[thin,<->] (pi-0.5, -0.4) -- (pi+0.4, -0.4) node[right]
         {$\lambda{\textcyrillic{ -- угол проводимости}}$};

         %заштрихованная площадка
          \draw[thin,domain={pi-0.5}:{pi + 0.4},pattern=north west lines,
          pattern color=blue]
          (pi-0.5,0)--
          plot (\x,{cos((\x-2/3*pi) r)}) -- (pi+0.4,0);
       
      % начало следующего угла проводимости
         \newcommand{\lambdaNext}{(pi+2/3*pi-0.3)}
         \draw[thin] (pi+2/3*pi-0.5,0) -- (pi+2/3*pi-0.5, {cos(((pi+2/3*pi-0.5) -4/3*pi) r)} ) ;
   
      % штрихованная площадка
        \draw[thin,domain={pi+0.4}:{pi+2/3*pi-0.5},pattern=north east lines, pattern color=gray]
         (pi+0.4,0) -- plot (\x,\En) -- (pi+2/3*pi-0.5,0);

  \end{scope}
\end{tikzpicture}

${\displaystyle \lambda < \frac{2\pi}{m}}$ -- означает, что ток прерывистый.

${\displaystyle \lambda > \frac{2\pi}{m} => \lambda = \frac{2\pi}{m} + \gamma}$
-- режим непрерывного тока.

при ${\displaystyle \lambda = \frac{2\pi}{m}}$ -- граница, должны сливаться оба режима, сливаются на отрезке длиной $\gamma=0$.

$$
= \frac{m}{2\pi} \sqrt{2} E_{2\phi}\left[ sin\left(-\frac{\pi}{m} +
\alpha + \lambda\right) - sin\left(-\frac{\pi}{m} +\alpha\right)
\right]
+ E_{\textcyrillic{н}} \left( 1 - \frac{\lambda m}{2\pi} \right) =
$$

$$
\frac{m}{2\pi} \sqrt{2} E_{2\phi} sin \frac{\lambda}{2}
cos\left( \alpha - \frac{\pi}{m} +\frac{\lambda}{2} \right)
+ E_{\textcyrillic{н}} \left( 1 - \frac{\lambda m}{2\pi}\right) =
$$

$$
U_d = E_{d0} \frac{sin\frac{\lambda}{2}}{sin\frac{\pi}{m}}
cos\left(\alpha-\frac{\pi}{m} + \frac{\lambda}{2}\right) +
E_{\textcyrillic{н}} \left( 1 - \frac{\lambda m}{2\pi} \right)
$$

${\displaystyle \lambda = \frac{2\pi}{m}}$ -- граничный режим $\frac{\pi}{m}?$

Частный случай режима работы управляемого преобразователя
на чисто активную нагрузку.
$$
= \int\limits_{-\frac{\pi}{2} + \alpha}^{\frac{\pi}{m}} \cos\omega t\; d\omega t
$$

а если $\alpha=0$ -- тогда не будет прерывистого режима. Прерывистый режим
будет в случае когда $\alpha>0$

$$
\underbrace{\frac{\pi}{m} + \alpha}_{\begin{array}{c}
{\textcyrillic{момент включения}}\\
{\textcyrillic{следующей фазы}}
\end{array}} > \frac{\pi}{2}
$$

Должно выполнятся неравенство, иначе нет прерывистого режима. 
Прерывистый режим при чисто активной нагрузке изображен на рисунке:

\begin{tikzpicture}
  \begin{scope}[xscale=2,yscale=3]
    \newcommand{\alphaa}{106.0 * pi / 180}
    \newcommand{\gammaa}{21.0 * pi / 180}
    \newcommand{\En}{0.37}

    \draw[thin, ->] (-0.2, 0) -- (5.8,0) node[right] {$\omega t$};
    \draw[thin, ->] ({2/3*pi},0) -- ({2/3*pi},1.3) node[left] {$U$};

    % A,B,C
%    \draw[domain=-0.2:5.8, smooth, yellow]
%    plot (\x,{cos((\x) r)}); \node[above] at (0.2,1) {$e_{k-1}$};
    \draw[domain=-0.2:5.8, help lines, smooth, green]
    plot (\x,{cos((\x-2/3*pi) r)}); \node[above] at (2*pi/3+0.2,1) {$e_k$};
    \draw[domain=-0.2:5.8, help lines, smooth]
    plot (\x,{cos((\x-4/3*pi) r)}); \node[above] at (4*pi/3+0.2,1) {$e_{k+1}$};

      % стрелка к -pi/2
        \draw[thin,latex-] (pi/3,0.7) -- (2*pi/3, 0.7) node[midway, below] {$\pi/2$};

      %alpha
        \draw[thin] (pi/3, 0.6) -- (pi/3, 1);
        \draw[thin] (pi/3-0.1, 0.7) -- (pi/3+0.1, 0.9)
          node[above right] {$\alpha$};
        \draw[thin] (pi-0.5, 0.7) -- (pi-0.5, 1);
        \draw[thin,-latex] (pi/3,0.8) -- (pi-0.5, 0.8);
      % \lamdba - угол проводимости
        \draw[thin] (pi-0.5, \En-0.1) -- (pi-0.5, -0.5);
        \draw[thin] (2/3*pi + pi/2, 0.1) -- (2/3*pi + pi/2, -0.5);
         \draw[thin,<->] (pi-0.5, -0.4) -- (2/3*pi + pi/2, -0.4) node[right]
         {$\lambda{\textcyrillic{ -- угол проводимости}}$};

      % начало следующего угла проводимости
         \newcommand{\lambdaNext}{(pi+2/3*pi-0.3)}
         \draw[red] (pi+2/3*pi-0.5,0) -- (pi+2/3*pi-0.5, {cos(((pi+2/3*pi-0.5) -4/3*pi) r)} ) ;

      \draw[red] (pi-0.5, 0) -- (pi-0.5, {cos((pi-0.5-2/3*pi) r)});
       \draw[red,domain=pi-0.5:2/3*pi+ pi/2] plot (\x,{cos((\x-2/3*pi) r)});
       \draw[red] (2/3*pi+ pi/2,0) -- (pi+2/3*pi-0.5,0);
\end{scope}
\end{tikzpicture}

\begin{equation}
\frac{m}{2\pi}\sqrt{2}E_{2\phi}\left[1 - sin\left(\alpha-\frac{\pi}{m}\right)
\right] =
E_{d0} \frac{1 - sin\left(\alpha-\frac{\pi}{m}\right)}{2sin\frac{\pi}{m}}
\end{equation}

$$
\alpha>\frac{\pi}{2} -\frac{\pi}{m}
$$
\begin{equation}
\frac{\pi}{2} -\frac{\pi}{m} < \alpha < \frac{\pi}{2} +\frac{\pi}{m}
\label{7a}
\end{equation}
для разных m разные неравенства (\ref{7a})

Завершаем рассмотрение реверсивных преобразователей.
\subsection{Функциональная схема раздельного управления}
\begin{circuitikz}
  \draw[thick,dashed,double,->](0,0)--(2,0);
  \draw[thick,dashed,double,->](0,0)--(0,2)
        node[midway,left]{$U_\textcyrillic{ЗНТ}$};

  \draw
  (2,-0.5) rectangle (3.5,0.5) (2.75,0)node{БРУ}
  (-1,2) rectangle (1,3) (0,2.5)node {САР} (0,3.5)node{(ACP)}
  (1.5,1)rectangle(2.5,5) (2,3)node{ВУ}
  (4,1.5)rectangle(5.5,2.5) (4.75,2)node{СУ2}
  (4,4)rectangle(5.5,5) (4.75,4.5)node{СУ1};
%  \draw[->] (4.75,1)--(4.75,1.5);
%  \draw[->] (4.75,3.5)--(4.75,4);
  \draw[->] (2.5,2)--(4,2)node[midway,above]{$U_\textcyrillic{упр2}$};
  \draw[->] (2.5,4.5)--(4,4.5)node[midway,above]{$U_\textcyrillic{упр1}$};
  \draw[->] (3.5,0)--(3.75,0)--(3.75,3.25)--
  (4.75,3.25)node[midway,above]{$u_1$}--(4.75,4);
  \draw[->] (3.5,0)--(3.75,0)--(4.75,0)node[midway,above]{$u_2$}--(4.75,1.5);
  \draw[->,dashed] (5.5,4.5)--(7,4.5)node[midway,above]{$\alpha_1$}; 
  \draw[->,dashed] (5.5,2)--(9.5,2)--(9.5,4.5)--(11,4.5)node[midway,above]{$\alpha_2$};
  \draw[->] (1,2.5)--(1.5 ,2.5);
  % выпрямитель-инвертор
  \draw
  (7,4)rectangle(9,5)
  (7,5)node[above]{ВГ1}
  (8.5,4.5)to[Do](7.5,4.5)--(8.5,4.5)
  (7.87,4.34)--(7.70,4.17)
  (11,4)rectangle(13,5)
  (13,5)node[above]{ВГ2}
  (12.5,4.5)to[Do](11.5,4.5)--(12.5,4.5)
  (11.87,4.34)--(11.70,4.17)
  %соединения от выпрямителя-инвертора
  
  (7.5,4)--(7.5,1.5)to[L](7.5,0)
  to[american controlled current source,l=$\textcyrillic{ДТ}_1$](10,0)
  to[american controlled current source,l=$\textcyrillic{ДТ}_2$](12.5,0)  
  (8.5,4)--(8.5,3.5)--(11.5,3.5)--(11.5,4)
  (12.5,4)--(12.5,1.5)to[L](12.5,0)
  (8,5)--(8,5.5) -- (10-0.433*21/12,6.65-21/12*0.25)
  (8,5.8)node[above]{m фаз}
  (12,5)--(12,5.5)--(10+0.433*21/12,6.65-21/12*0.25)
  (12,5.8)node[above]{m фаз}
  (10,6.65+0.82)to[ospst,l_=QF](10,8.7)--++(-3,0) % к трехфазной сети
  (8,8.5)--(8.4,8.9)
  (7.8,8.5)--(8.2,8.9)
  (7.6,8.5)--(8.0,8.9)
  %мотор  (10,3.5)--(10,1.5)to[short](10,0) (10,1.25)node[component]{M}
  (10,3.5)to[short,*-](10,1.5)to[motor,-*](10,0);
  \draw[dotted](7.5,0.75)circle(0.5); % выбрасываем L уравнительный
  \draw[thin](7,0.25)--(8,1.25);
  \draw[dotted](12.5,0.75)circle(0.5); % выбрасываем L уравнительный
  \draw[thin](12,0.25)--(13,1.25);
  \draw[->] (8.75,-0.3)--(8.75,-0.75)--(3,-0.75)--(3,-0.5);
  \draw[->] (11.25,-0.3)--(11.25,-1)--(2.5,-1)--(2.5,-0.5);
  %трансформатор (10, 6.65) (-0.433,-0.25)
  \draw({10-0.433*2/3},{6.65-0.25*2/3}) circle(0.5)
  ({10+0.433*2/3},{6.65-0.25*2/3}) circle(0.5)
  (10,{6.65+0.5*2/3}) circle(0.5) 
  ;\end{circuitikz}

Трансформатор имеет особенность преобразовывать число фаз. У Каждой вентильной группы ({\it ВГ}) своя
система управления СИФУ ({\it СУ}) или система управления вентилями ({\it СУВ}).
Каждая ${\textcyrillic{\it СУ}}_N$ генерирует
сигналы с углом $\alpha_N$.
Импульсов должно быть {\it m}. Уравнительные реакторы используются только в случае совместного управления.
На схеме уравнительные реакторы обведены кружками, чтобы подчеркнуть, что в раздельном управлении они не
обязательны. При совместном управлении должно выполнятся условие $\alpha_1+\alpha_2>\pi$, в противном
случае протекает большой уравнительный ток.

\begin{circuitikz}
  \draw (4,1.5)rectangle(5.5,2.5) (4.75,2)node{СУ2};
  \draw[red,<-] (4.75,1.5)--(4.75,1) node[right,color=black]{-- вход для выключения СИФУ.};
\end{circuitikz}

\begin{circuitikz}
  \draw
  (2,-0.5) rectangle (3.5,0.5) (2.75,0)node{БРУ};
  \draw[<-] (2.2,-0.5)--(2.2,-1) node[below]{$\textcyrillic{ДТ}_1$};
  \draw[<-] (3.3,-0.5)--(3.3,-1) node[below]{$\textcyrillic{ДТ}_2$};
  \draw[dashed,double,->] (0,0)--(2,0);
\end{circuitikz}

    {\it БРУ} -- блок раздельного управления. Токи управления от датчиков тока
    $\textcyrillic{ДТ}_1$ и  $\textcyrillic{ДТ}_2$ (10-100mA, мах 10-15V).
    {\it БРУ} $\equiv$ {\it ЛПУ} -- он же логическое переключающее устройство.
  Стрелкой \begin{circuitikz}
    \draw[dashed,double,->] (0,0)--(1,0);\end{circuitikz} обозначен сигнал инициирующий работу
  (вначале отключить, затем включить после паузы).

\begin{figure}[H]  
  \begin{tikzpicture}[scale=1]
  \draw[thin,->,domain={pi/3}:{2*pi},samples=100,red]
  plot (canvas polar cs:angle=\x r,radius= {30*20/sqrt((30*cos(\x r))^2 +(20*sin(\x r))^2)})
  node[above right] {$I_1$};  
  \draw(-0.6,1.2)node {$\textcyrillic{ВГ}_1$};
  \draw(0.5,-1.1) to[motor] (0.5,-0.1);
  \end{tikzpicture}
  \hspace{2cm}
  \begin{tikzpicture}
  \draw[thin,->,domain={pi/2}:{2*pi-pi/4},samples=100,red]
  (4,0) plot (canvas polar cs:angle=\x r,radius= {30*20/sqrt((30*cos(\x r))^2 +(20*sin(\x r))^2)})
  node[above right] {$I_2$};
  \draw(-0.5,-1.1) to[motor] (-0.5,-0.1);
   \end{tikzpicture}
  \caption{направление тока} 
\end{figure}

Одновременно токи $I_1$ и $I_2$ не могут существовать, должна быть пауза, чтобы тиристоры
выключились.

Для измерения переменного тока существует трансформатор тока, изолированный от силовой цепи.
Существуют ``трансформатор постоянного тока'' построенный на принципе подмагничивания.
Чаще всего используется токоизменительный шунт.
%\begin{tikzpicture}
%\end{tikzpicture}
Исторически шунты расчитывались на протекание тока $45mV$, в настоящее время
расчитываются на $75mV$. С помощью шунта превратили сигнал тока в напряжение. Чтобы не было
ошибок в динамике нужно чтобы шунт был безиндуктивный. Представим, что через
преобразователь проходит ток $10kA$, или 1000А, тогда даже миливольты превращаются
в большее ватты.

Заострим проблему: Сумма токов утечек например 10 включенных параллельно вентилей
может превысить ток удержания. Выключаются вентили последовательно, и наконец остается
один последний вентиль в котором
$$
\begin{array}{ccc}
  10kA &-& 75mV\\
  10mA&-&
\end{array}
$$
величина может превысить порог чувствительности для удержания одного.

({\it АСР}) -- автоматическая система регулирования по ГОСТу. ({\it САР}) --
система автоматического регулирования.

\begin{circuitikz}
  \draw[->,dashed,double] (0,0)--(0,1) node[midway,right]{$U_\textcyrillic{знт}$}; 
\end{circuitikz} -- сигнал заданного направления тока, тоже логический (0 или 1)
Отключать можно тогда, когда $I=0$.
Увеличиваем $\alpha$, ток $I$ падает до нуля, в этот момент {\it БРУ} отключает импульсы,
и это же побудительный момент для перемены направления тока. Не показаны выдержки
времени. Кроме того есть запрет на переключения в момент формирования импульсов

\begin{circuitikz}
  \draw[<-,dashed](0,0)--(1,0)--(1,1) node{$\alpha_1$};
  \draw[<-,dashed](0,-0.3)--(2,-0.3)--(2,0.5)node{$\alpha_2$};
  \draw[<-,thin] (2.1,-0.1)--(3,-0.1)node[right]
       {$\begin{array}{c}
           \textcyrillic{сам импульс }\alpha_2\\
           \textcyrillic{запрещает себя выключать}
        \end{array}$};
  \end{circuitikz}

функция необязательная но часто используеммая. Ненадежность токовой логики
шунта с усилителем. Возникает проблема: ток перегрузки двигателя 250\% номинала.
Контролируемый ток 1/1000 доля номинала. Вместо датчика тока используются датчики
напряжения 1-2...4В -- падение напряжения в открытом состоянии.
Вместо датчика тока используются датчик запертого состояния тиристора.
Бывает, что датчик говорить ``0'', но это может оказаться переменный ток проходящий
через нуль. Схемы управления бывают аналоговыми, цифроаналоговыми
Когда система управления реализуется аппаратно нужны ли две систмемы управления.
У трансформатора не работает. Убрали уравнительные реакторы, для этого и была нужна
система раздельного управления, чтобы оптимизировать.
Зачем два СИФУ? Аппаратно и програмно алгоритм может быть выполнен с одной
{\it СУ}. Тогда вместо двух систем импульсов(включения и выключения) используется
переключатель между двумя СИФУ. Если СИФУ одно, $\alpha_1\downarrow$
$\alpha_2\uparrow$ (переключение и на выходе и на входе). Обязательно должен
предусмотреть переключение на входе.

\subsection{Внешние характеристики реверсивных преобразователей}
До этого рисовали внешние характеристики в 2х квадрантах, сейчас нарисуем в
4х квадрантах

\hspace{-2cm}
\begin{tikzpicture}
  \begin{scope}[xscale=1.3,yscale=3]
    \newcommand{\betamin}{160/180}
    % axis x,y
    \draw[thin, ->] (0, 0) -- (6,0) node[right]
         {${\displaystyle \frac{I_d}{I_{d0}}}$};
    \draw[thin, ->] (0, -1) -- (0,1.2) node[left]
         {${\displaystyle \frac{U_d}{E_{d0}} }$};
         \draw[thin, loosely dashed] (0,-1) -- (6,-1);
    % ylabel
    \foreach \y/\ytext in {-1/-1,{-\betamin}/\beta_{min},0/0,1/1}
    \draw (0.1,\y) -- (-0.1,\y) node[left] {$\ytext$};
    % \beta_min
    \draw[thin,loosely dashed] (0, {-\betamin}) -- (6, {-\betamin+0.1});
    \node at (0.3,-0.4) {$'1-0'$};

    % eclipse (1-1)
    \draw[domain=0:0.7, help lines,dotted, smooth]
    plot (\x,{sqrt(1-\x*\x/0.49)});
    \draw[domain=0:0.7, help lines,dotted, smooth]
    plot (\x,{-sqrt(1-\x*\x/0.49)});
    %ellipse слева
    \draw[domain=-0.7:0, help lines,dotted, smooth]
    plot (\x,{sqrt(1-\x*\x/0.49)});
    \draw[domain=-0.7:0, help lines,dotted, smooth]
        plot (\x,{-sqrt(1-\x*\x/0.49)});
% наклон    \draw[thin] (0,1) -- (6,0.8);
    \draw[thin] (0,1) -- (4.76,1-1/30*4.76);
    % 0.2/6*180/pi
%    \draw[domain=4.76:6.79, help lines, smooth]
%    plot (\x, {-0.4*\x*\x +2*0.4*4.76*\x + 1 -1/30*4.76 -0.4*4.76*4.76
%    });


    \node[rotate=-4.45] at (3,1) {$\alpha=0^\circ$};
    \draw[thin] ({sqrt(0.49*(1-0.67*0.67))},{0.67-1/30}) -- (5.2, 0.67-1/30*5.2);
    % рисуем параболу ax^2 + bx + c
    % выбираем a=1
    % из f'(x)=0 = 2ax+b => b=-2 a x0
    % a x0^2 + b x0 + c = y0
    % a x0^2 -2 a x0^2 +c = y0
    % c = y0+a x0^2
    \draw[domain=0:{sqrt(0.49*(1-0.67*0.67))}, help lines, smooth]
    plot (\x, {\x*\x -2*sqrt(0.49*(1-0.67*0.67))*\x + 0.67 +
      0.49*(1-0.67*0.67 ) -0.0335
     });

    \draw[thin] ({-sqrt(0.49*(1-0.67*0.67))},{0.67+1/30}) -- (-5.2, 0.67+1/30*5.2);
%    \draw[domain=0:{sqrt(0.49*(1-0.67*0.67))}, help lines, smooth]
%    plot (\x, {\x*\x -2*sqrt(0.49*(1-0.67*0.67))*\x + 0.67 +
%      0.49*(1-0.67*0.67 ) -0.0335
%    });
    


    \node[rotate=-4.45] at (3.1,0.67) {$\alpha=30^\circ$};
    \draw[thin] ({sqrt(0.49*(1-0.33*0.33))},{0.33-1/30}) -- (5.54, 0.33-1/30*5.54);
    \draw[domain=0:{sqrt(0.49*(1-0.33*0.33))}, help lines, smooth]
    plot (\x, {1.2*\x*\x -2*1.2*sqrt(0.49*(1-0.33*0.33))*\x + 0.33 +
      1.2*0.49*(1-0.33*0.33 ) -0.0335
    });
    \draw[thin] ({-sqrt(0.49*(1-0.33*0.33))},{0.33+1/30}) -- (-5.54, 0.33+1/30*5.54);
%    \draw[domain=0:{sqrt(0.49*(1-0.33*0.33))}, help lines, smooth]
%    plot (\x, {1.2*\x*\x -2*1.2*sqrt(0.49*(1-0.33*0.33))*\x + 0.33 +
%      1.2*0.49*(1-0.33*0.33 ) -0.0335
%    });
    


    \node[rotate=-4.45] at (3.2,0.33) {$\alpha=60^\circ$};
    \draw[thin] ( 0.7, {0-1/30})  -- (5.8, -1/30*5.8);
    \draw[domain=0:{sqrt(0.49*(1-0*0))}, help lines, smooth]
    plot (\x, {1.4*\x*\x -2*1.4*sqrt(0.49*(1-0.*0.))*\x + 0. +
      1.4*0.49*(1-0.*0. ) -0.0335
    });
    \draw[thin] ( -0.7, {0+1/30})  -- (-5.8, 1/30*5.8);
    
    \node[rotate=-4.45] at (5.2,-0.1) {$\alpha=90^\circ$};
    \draw[thin] ({sqrt(0.49*(1-0.33*0.33))},{-0.33-1/30}) -- (5.92, -0.33-1/30*5.92);
    \draw[domain=0:{sqrt(0.49*(1-0.33*0.33))}, help lines, smooth]
    plot (\x, {1.8*\x*\x -2*1.8*sqrt(0.49*(1-0.33*0.33))*\x -0.33 +
      1.8*0.49*(1-0.33*0.33 ) -0.0335
          });
   \draw[thin] ({-sqrt(0.49*(1-0.33*0.33))},{-0.33+1/30}) -- (-5.92, -0.33+1/30*5.92);
    
    \node[rotate=-4.45] at (3.4,-0.33) {$\alpha=120^\circ$};


    \draw[thin] ({sqrt(0.49*(1-0.66*0.66))},{-0.66-1/30}) -- (5,  {-0.66-0.16});
    \draw[domain=0:{sqrt(0.49*(1-0.67*0.67))}, help lines, smooth]
    plot (\x, {3*\x*\x -2*3*sqrt(0.49*(1-0.67*0.67))*\x -0.67 +
      3*0.49*(1-0.67*0.67 ) -0.028
    });
    \draw[thin] ({-sqrt(0.49*(1-0.66*0.66))},{-0.66+1/30}) -- (-5,  {-0.66+0.16});

    \node[rotate=-4.45] at (3.5,-0.66) {$\alpha=150^\circ$};

    \node[rotate=2] at (3.5,-0.9)
         {\textcyrillic{граница устойчивости инверторного режима}};
  \end{scope}
\end{tikzpicture}

Говорили, что область внутри эллипса это область прерывистого тока.
По умолчанию все углы в радианах. $\beta_{min}(\gamma,\delta,\psi)$,
$\alpha_{max} = 150^\circ-160^\circ$.
Забыли про правую полуплоскостью На левой--другой преобразователь.
По отношению к нагрузке этот преобразователь включён ``наоборот''?

Симметрично относительно начала координат внешние характеристики
реверсивных преобразователей с раздельным управлением.

А что с совместным управлением? $\alpha_1+\alpha_2>180^\circ$.
Не учёл падение от активных сопротивлений и $U_0$ и не рисуем область
многовенлильной коммутации.

При совместном управлении различают два способа управления:
\begin{itemize}
\item совместное согласованное  $\alpha_1+\alpha_2\approx \pi$
  с максимально доступной точностью.
  \item совместное несогласованное, когда  $\alpha_1+\alpha_2>\pi$
  \end{itemize}

В инженерном плане равенство $=\pi$ невозможно точно измерить и оно
легко может нарушиться из-за нестабильности. Если же заведомо не стремиться
к равенству нулю, тогда будет заведомо большой уравнительный ток.
<Если импульсы не отключаются то правая ЭДС присутствтвует слева?>

\begin{circuitikz}\draw
  (0,0)to[L](1,0)to[european resistor](2,0)--(2.2,0)(2.5,0)circle(0.3)
  (2.8,0)--(3,0)to[ammeter](4,0)--(4,1.5)
  (2.5,0)node{$E_m$}
  (4,1.5)--
  (4,3)to[ammeter,mirror](3,3)--(2.8,3)(2.5,3)circle(0.3)(2.2,3)--(2,3)
  to[Ty](1,3)to[L](0,3)--(0,0)
  (2.5,3)node{$E_{d1}$}
  (0,1.5)to[L,*-](1,1.5)to[Ty](2,1.5)--(2.2,1.5)
  (2.5,1.5)circle(0.3)(2.8,1.5)--(3,1.5)to[ammeter,-*](4,1.5)
  (2.5,1.5)node{$E_{d2}$};
  \draw[red,->](3,2.7)--(0.8,2.7)arc(90:270:0.45)--(3,1.8);
  \draw[thin,<-](0.3,2.25)--(-0.4,2.25) node[left]
       {$\begin{array}{c}\textcyrillic{уравнительный ток}\\
         \textcyrillic{всегда в одну сторону}\end{array}$}
;\end{circuitikz}

Уравнительный ток всегда идет минуя нагрузку. Рассмотрим пример
$$
\begin{array}{c}
  60A\leftarrow\\
  80A\rightarrow\\
  20A\leftarrow
\end{array}
$$
Какой уравнительный ток? Уравнительный ток равен $60A$
$$
\begin{array}{c}
  350A\leftarrow\\
  90A\rightarrow\\
  260A\rightarrow
\end{array}
$$

Меньший из двух вентильных токов -- уравнительный. А разность есть ток нагрузки.
Увеличили угол значит падение напряжения на вентидлях уменьшились?
Уравнительный ток большой, значит углы раздвигаются.

Совместное управление не миф, а может быть реализовано достаточно точно.

На графике изображен ток в нагрузке, но есть уравнительный ток.
Для реверсивного преобразователя
\begin{tikzpicture}
  \draw[domain=0:1]
  plot(\x,{(\x-1)^2+1});
\end{tikzpicture}
Внутри эллипса ток собственно в вентильной группе.

Поэтому характеристики спрямляются

\begin{tikzpicture}
  \begin{scope}[xscale=1.5,yscale=2]
  \draw[thin,->] (-2,0)--(2,0) node[right]{$\omega t$};
  \draw[thin,->] (0,-0.2)--(0,1);
  \draw[domain=-1.5:-0.25]
  plot(\x,-0.2*\x+0.5);
  \draw[domain=-0.25:0.25]
  plot(\x,-0.4*\x+0.45);
  \draw[domain=0.25:1.5]
  plot(\x,-0.2*\x+0.4);
  \end{scope}
\end{tikzpicture}

Дадим оценку тому что прошли: совместное управление мы его обругали,
но есть спрямление характеристик в области малых токов.
В теории автоматического управления электроприводами.  
Нужен измеритель  уравнительного тока. В одной вентильной группе сумма
уравнительного тока плюс нагрузка, в другом только уравнительный ток.
Спрямление характеристик -- достоинство совместного управления.

\subsection{Пульсации выпрямленного напряжения и тока}
Выпрямленное с дефектами, пульсациями. Нужно количественно оценить амплитуду
пульсаций. Как определить амплитуду переменной составляющей?

\begin{tikzpicture}
  \draw[red](-pi,0.65)--(pi,0.65);
  \draw[domain=-pi:pi,help lines,smooth]
  plot(\x, {cos(\x r)})
  plot(\x, {cos((\x-pi/3) r)})
  plot(\x, {cos((\x+pi/3) r)})
  plot(\x, {cos((\x-2*pi/3) r)})
  plot(\x, {cos((\x+2*pi/3) r)})
  plot(\x, {cos((\x+pi) r)});
  \draw[domain=-pi+0.3:-2*pi/3+0.3-pi/6,red,pattern=north east lines,
  pattern color=red]
  (-pi+0.3, 0.65) --
  plot(\x, {cos((\x+pi) r)});
  \draw[domain=-pi+0.3+pi/6:-2*pi/3+0.3,red,pattern=north east lines,
  pattern color=red]
  plot(\x, {cos((\x+pi) r)})
  -|({-2*pi/3+0.3}, 0.65);
  
  \draw[domain=-2*pi/3+0.3:-pi/3+0.3-pi/6,red,pattern=north east lines,
  pattern color=red]
  (-2*pi/3+0.3, 0.65) --
  plot(\x, {cos((\x+2*pi/3) r)});
  \draw[domain=-2*pi/3+0.3+pi/6:-pi/3+0.3,red,pattern=north east lines,
  pattern color=red]
  plot(\x, {cos((\x+2*pi/3) r)})
    -|({-pi/3+0.3}, 0.65);
  
  \draw[domain=-pi/3+0.3:0*pi/3+0.3-pi/6,red,pattern=north east lines,
  pattern color=red]
  (-pi/3+0.3, 0.65) --
  plot(\x, {cos((\x+pi/3) r)});
  \draw[domain=-pi/3+0.3+pi/6:0*pi/3+0.3,red,pattern=north east lines,
  pattern color=red]
  plot(\x, {cos((\x+pi/3) r)})
  -|({0*pi/3+0.3}, 0.65);
  

  \draw[domain=0*pi/3+0.3:pi/3+0.3-pi/6,red,pattern=north east lines,
  pattern color=red]
  (0*pi/3+0.3, 0.65) --
  plot(\x, {cos((\x) r)});
  \draw[domain=0*pi/3+0.3+pi/6:pi/3+0.3,red,pattern=north east lines,
  pattern color=red]
  plot(\x, {cos((\x) r)})
  -|({pi/3+0.3}, 0.65);
    

  \draw[domain=pi/3+0.3:2*pi/3+0.3-pi/6,red,pattern=north east lines,
  pattern color=red] 
  (pi/3+0.3,0.65) --
  plot(\x, {cos((\x-pi/3) r)});
  \draw[domain=pi/3+0.3+pi/6 : 2*pi/3+0.3,red,
      pattern=north east lines,pattern color=red]
  plot(\x, {cos((\x-pi/3) r)})
  -|({2*pi/3+0.3}, 0.65);
  \draw[red]
  ({-pi+5*pi/3+0.3}, 0.65) -- ({-pi+5*pi/3+0.3}, {cos((-pi+3*pi/3+0.3) r)});  
  ;

  % http://tex.stackexchange.com/questions/54464/hatch-a-rectangle-in-tikz
  %  \draw [thick,pattern=north west lines, pattern color=red] (1,0)--(1,1) to [bend left] (4,4) -- (4,0) --cycle;
  \end{tikzpicture}

Можно представить рядом Фурье m-пульсаций.
$$
f_\textcyrillic{гармоники} = k(mf_c)
$$
где $k=1...k...$, $f_C$ -- частота сети

Найдем $U_k$ гармоники.

При $m=6$, $f_c=50\textcyrillic{Гц}$
$$
\begin{array}{ccc}
  f_\textcyrillic{гармоники}&=&300\\
  &&600\\
  &&900\\
  &&1200
\end{array}
$$

Допустим, мы знаем как выбрать фильтр? Нам нужно фильтровать гармоники тока,
а не гармоники $U$. Зная $U_k$ найти $I_k$, затем сложить. Неблагодарная
задача. Обычно берут первую гармонику, учитывают с коэффициэнтом запаса.

$$
E_{dkm} = \frac{\sqrt{2}E_{d0}}{(km)^2-1}
\sqrt{cos^2\alpha + (km \:sin\alpha)^2}
$$
Чем больше $\alpha$ гармоника возрастает. Максимальная величина гармоники
при $\displaystyle \alpha=\frac{\pi}{2}$, Это понятно из графика и из
формулы

$$
(E_{dkm})_{max} = \frac{km}{(km)^2-1} \sqrt{2} E_{d0}
$$
С ростом $m$ гармоники убывают.

Можно считать, что гармоники обратно пропорционально частоте.

$$
E_{\sim} = \frac{m}{(m)^2-1} \sqrt{2} E_{d0}
$$
$$
f = mf_c
$$
С коэффициэнтом запаса 10-20\%.
Если $\alpha$ не доходит до $90^\circ$, тогда считают для максимального $\alpha$.

Ток возбуждения $I_{min}$, нулём никогда не бывает. По заданному току

$$
I_\sim = \frac{E_\sim}{\omega L_\Sigma}=
$$
   $L_\Sigma$ по всей цепи и постоянного и переменного тока

$$
=\frac{\cancel{m}\sqrt{2}E_{d0}}{(m^2-1)2\pi \cancel{m}F_cL_\Sigma}
(m\textcyrillic{ можно сократить})
$$

$$
L_\Sigma = L_\phi + L_\textcyrillic{н} + L_\Phi
$$
Возникает задача. Если нужно ограничить до заданной величины.

Рассмотрим другой способ решения, нравится больше, но тоже приближенный.

Рассмотрим небольшую задачу:

\begin{circuitikz}\draw
  (0,1)to[battery1,*-*](0,2)
  (0,1)--(1,1)to[voltmeter,l_=$V_1$](1,2)--(0,2)
  (0,1)--(0,0)--(2,0)
  (3,0)to[L](2,0)
  (2,-0.5)--(2,-1.5)--(2.24,-1.5)
  (2,-0.5)to[L](3,-0.5)--(3,-1.5)--(2.76,-1.5)
  (2.5,-1.5)circle(0.26)
  (2.5,-1.5)node{G}
  (2.5,-1.76)node[below]{100V}
  (2,-1.5)node[left]{$f=100\textcyrillic{Гц}$}
  (2,-0.24)rectangle(3,-0.26)
  (0,2)--(0,3)--(3,3)
  (3,0)to[voltmeter,l=$V_2$,*-*](3,3)
  %
  (3,3)to[L](4,3)
  (3,3.24)rectangle(4,3.26)
  (4,4)--(4,3.5)to[L](3,3.5)--(3,4)
  (2.9,3.9)--(3,4)--(3.1,3.9)
  (3.9,3.9)--(4,4)--(4.1,3.9)
  (2.9,4)--(3,4.1)--(3.1,4)
  (3.9,4)--(4,4.1)--(4.1,4) node[right]{$50\textcyrillic{Гц}$}
  (3,4.1)--(3,4.5) node[above right]{$\sim 100V$}
  (4,4.1)--(4,4.5)
  %
  (3,0)--(4,0)
  (5,0)to[L,-*](4,0)
  (4,-0.5)--(4,-1.5)--(4.24,-1.5)
  (4,-0.5)to[L](5,-0.5)--(5,-1.5)--(4.76,-1.5)
  (4,-0.24)rectangle(5,-0.26)
  (4.5,-1.5)circle(0.26)
  (4.5,-1.5)node{G}
  (4.5,-1.76)node[below]{100V}
  (5,-1.5)node[right]{$f=200\textcyrillic{Гц}$}
  (4,0)to[voltmeter,l_=$V_3$,*-*](4,3)
  %
  (5,0)--(7,0)
  (4,3)--(7,3)
  (7,0)to[voltmeter](7,3)
  ;
  \draw[thin,<-] (7.26,1.5)--(8,1.5)
  node[right]{$\begin{array}{c}\textcyrillic{действующее значение}\\
      \textcyrillic{не магнитоэлектрический}\end{array}$};
\end{circuitikz}

Если частоты не одинаковы, то действующее значение равно квадратному корню из
суммы квадратов.

Оценим
$$
E_d=E_{d0}\;cos \alpha
$$
Не учтен угол коммутации в формуле. При больших $\alpha$ угол $\gamma$ уменьшается.
$\gamma$ вызван индуктивностью, для высших гармоник это сопротивление -- гармоники будут
уменьшаться. Должен учитывать худший случай. Поэтому сейчас угол коммутации игнорируем.
Найдем среднеквадратичную составляющую:
$$
\sqrt{2} E_{2\phi} = \frac{E_{d0}}{\frac{m}{\pi} sin \frac{\pi}{m}}
$$
из формулы (1)

$$
E_{d\textcyrillic{средне квадратичное}} = \sqrt{
  \frac{1}{2\pi/m}\int\begin{array}{c}\textcyrillic{от квадрата}\\
  \textcyrillic{мгновенного значения}\end{array}} =
$$

$E_{d\textcyrillic{средне квадратичное}}$ -- это на активную нагрузку(физический смысл)

$$
=\sqrt{
  \frac{1}{2\pi/m}\int\limits_{-\frac{\pi}{m}+\alpha}^{\frac{\pi}{m}+\alpha}
  \frac{E_{d0}}{\frac{m}{\pi} sin \frac{\pi}{m}} cos \omega t}
=
\frac{E_{d0}}{\sqrt{2}\frac{m}{\pi} sin \frac{\pi}{m}}
\sqrt{1+\frac{m}{2\pi}sin\frac{2\pi}{m}cos2\alpha}
$$

$$
E_{d\nu} = \sqrt{E^2_{d\textcyrillic{ср.кв.}} - E_d^2} =
$$

$E_{d\nu}$ -- действующее значение всех гармоник на стороне постоянного тока. $E_d$ -- $E_{d\textcyrillic{среднее}}$

$$
= E_{2\phi}\sqrt{1-\frac{m}{\pi}sin\frac{\pi}{m}
  \left[2\frac{m}{\pi}sin\frac{\pi}{m}\left(cos\alpha\right)^2 -
    cos\frac{\pi}{m}cos2\alpha\right]}
$$

Вместо $E_{2\phi}$ можно подставить
$\displaystyle \frac{E_{d0}}{\sqrt{2}\frac{m}{\pi}sin\frac{\pi}{m}}$, можно преобразовать,
в разной форме можно записать, но эта форма понятнее.

Напряжение может интересовать только при чисто активной нагрузке.

Наибольшие пульсации имеет граничный режим, который можно понимать как случай когда
постоянная составляющая равна пульсациям
\begin{tikzpicture}
  \begin{scope}[scale=0.5]
  \draw[domain=-2*pi:2*pi]
  plot(\x, {sin(\x r)+1.5});
  \draw[thin,loosely dotted](-2*pi,1.5)--(2*pi,1.5)node[right]{};
  \draw[thin,->](-2*pi,0)--(2*pi,0)node[right]{$\omega t$};
  \draw[thin,->](-2*pi+0.2,1.5)--(-2*pi+0.4,0.1) node[below]
       {$\textcyrillic{если среднее будет уменьшаться}$};
\end{scope}
\end{tikzpicture}

Переменная составляющая не влияет на постоянную. Зависит от $\alpha$, но зависит еще от
ЭДС нагрузки. если среднее будет уменьшаться то это никак не отразится на переменной
составляющей.

Прерывистый режим начнётся, когда переменная составляющая коснётся 0. Это и будет
граничный режим.

\begin{tikzpicture}
  \begin{scope}[scale=0.5]
    \draw[domain=-2*pi:2*pi]
    plot(\x, {sin(\x r)+1});
    \draw[thin,loosely dotted](-2*pi,1)--(2*pi,1)node[right]{};
    \draw[thin,->](-2*pi,0)--(2*pi,0)node[right]{$\omega t$};
  \end{scope}
\end{tikzpicture}

6-ти пульсная кривая.

\begin{tikzpicture}
  \begin{scope}[scale=1.5]
    \draw[very thin,->] (-pi-0.1,0) -- (pi+0.1,0) node[right] {$\omega t$};
    \draw[thin,->] (-2*pi/3,-0.1)--(-2*pi/3,1.2) node[left] {$U$};
    
  %  \draw[red](-pi,0.65)--(pi,0.65);
  \draw[domain=-pi/2-0.1:pi/2+0.1,help lines,smooth]
  plot(\x, {cos(\x r)});
  \draw[domain=-pi/2+pi/3-0.1:pi/2+pi/3+0.1,help lines,smooth]
  plot(\x, {cos((\x-pi/3) r)});
  \draw[domain=-pi/2-pi/3-0.1:pi/2-pi/3+0.1,help lines,smooth]
  plot(\x, {cos((\x+pi/3) r)});
  \draw[domain=-pi/2+2*pi/3-0.1:pi-pi/4+0.1,help lines,smooth]  
  plot(\x, {cos((\x-2*pi/3) r)});
  \draw[domain=-pi-0.1:-2*pi+pi/2+2*pi/3+0.1,help lines,smooth]
  plot(\x, {cos((\x-2*pi/3) r)});
  \draw[domain=-pi-0.1:pi/2-2*pi/3+0.1,help lines,smooth]  
  plot(\x, {cos((\x+2*pi/3) r)});
%    \draw[domain=2*pi-pi/2-2*pi/3-0.1: pi-pi/4+0.1,help lines,smooth]
%    plot(\x, {cos((\x+2*pi/3) r)});
  \draw[domain=-pi-0.1:-pi/2+0.1,help lines,smooth]  
  plot(\x, {cos((\x+pi) r)});
  \draw[domain=pi/2-0.1:pi-pi/4+0.1,help lines,smooth]
    plot(\x, {cos((\x+pi) r)});
  
  \draw[red] (-pi+0.3-pi/24,0.65)-- (-pi+0.3+pi/24,0.65);
  
  \draw[domain=-pi+0.3+pi/24:-2*pi/3+0.3-pi/6,red,pattern=north east lines,
    pattern color=red]
  (-pi+0.3+pi/24, 0.65) --
  plot(\x, {cos((\x+pi) r)});
  \draw[domain=-pi+0.3+pi/6:-2*pi/3+0.3-pi/24,red,pattern=north east lines,
    pattern color=red]
  plot(\x, {cos((\x+pi) r)})
  -|({-2*pi/3+0.3-pi/24}, 0.65);

  \draw[red] (-2*pi/3+0.3-pi/24,0.65)-- (-2*pi/3+0.3+pi/24,0.65); 

  \draw[domain=-2*pi/3+0.3+pi/24:-pi/3+0.3-pi/6,red,pattern=north east lines,
    pattern color=red]
  (-2*pi/3+0.3+pi/24, 0.65) --
  plot(\x, {cos((\x+2*pi/3) r)});
  \draw[domain=-2*pi/3+0.3+pi/6:-pi/3+0.3-pi/24,red,pattern=north east lines,
    pattern color=red]
  plot(\x, {cos((\x+2*pi/3) r)})
  -|({-pi/3+0.3-pi/24}, 0.65);

 \draw[red] (-pi/3+0.3-pi/24,0.65)-- (-pi/3+0.3+pi/24,0.65);  
  
  \draw[domain=-pi/3+0.3+pi/24:0*pi/3+0.3-pi/6,red,pattern=north east lines,
    pattern color=red]
  (-pi/3+0.3+pi/24, 0.65) --
  plot(\x, {cos((\x+pi/3) r)});
  \draw[domain=-pi/3+0.3+pi/6:0*pi/3+0.3-pi/24,red,pattern=north east lines,
    pattern color=red]
  plot(\x, {cos((\x+pi/3) r)})
  -|({0*pi/3+0.3-pi/24}, 0.65);
  
 \draw[red] (0*pi/3+0.3-pi/24, 0.65)--(0*pi/3+0.3+pi/24, 0.65);
  
  \draw[domain=0*pi/3+0.3+pi/24:pi/3+0.3-pi/6,red,pattern=north east lines,
    pattern color=red]
  (0*pi/3+0.3+pi/24, 0.65) --
  plot(\x, {cos((\x) r)});
  \draw[domain=0*pi/3+0.3+pi/6:pi/3+0.3-pi/24,red,pattern=north east lines,
    pattern color=red]
  plot(\x, {cos((\x) r)})
  -|({pi/3+0.3-pi/24}, 0.65);

 \draw[red] (pi/3+0.3-pi/24,0.65) -- (pi/3+0.3+pi/24,0.65);
  
  \draw[domain=pi/3+0.3+pi/24:2*pi/3+0.3-pi/6,red,pattern=north east lines,
    pattern color=red]
  (pi/3+0.3+pi/24, 0.65) --
  plot(\x, {cos((\x-pi/3) r)});
  \draw[domain=pi/3+0.3+pi/6 : 2*pi/3+0.3-pi/24,red,
    pattern=north east lines,pattern color=red]
  plot(\x, {cos((\x-pi/3) r)})
  -|({2*pi/3+0.3-pi/24}, 0.65);

  % ток
  \draw[thin,->] (-pi-0.1,-1)--(pi+0.1,-1) node[right] {$\omega t$};
  \draw[thin,->] (-2*pi/3,-1.1)--(-2*pi/3,-0.2) node[left] {$I$};
  \draw[red,domain=-pi/3+0.3+pi/24: 0*pi/3+0.3-pi/24]
  plot(\x, {-1-1.5*(\x-(-pi/3+0.3+pi/24))*(\x-(0*pi/3+0.3-pi/24))*(\x-(-pi/2+0.1))});
  \draw[red,domain=0*pi/3+0.3+pi/24: 1*pi/3+0.3-pi/24]
  plot(\x, {-1-1.5*(\x-(0*pi/3+0.3+pi/24))*(\x-(1*pi/3+0.3-pi/24))*(\x-(pi/3-pi/2+0.1))});

  \draw[red,domain=-2*pi/3+0.3+pi/24: -pi/3+0.3-pi/24]
  plot(\x, {-1-1.5*(\x-(-2*pi/3+0.3+pi/24))*(\x-(-pi/3+0.3-pi/24))*(\x-(-pi/3-pi/2+0.1))});

  \draw[red] (-pi/3+0.3-pi/24,-1)--(-pi/3+0.3+pi/24,-1);
  \draw[red] (0*pi/3+0.3-pi/24,-1)--(0*pi/3+0.3+pi/24,-1);
  
  % линии напряжение-ток-\lambda
  \draw[thin] (-pi/3+0.3+pi/24,0.6)-- (-pi/3+0.3+pi/24,-0.95);
  \draw[thin](-pi/3+0.3+pi/24,-1.05)--(-pi/3+0.3+pi/24,-1.4);
  \draw[thin] (0*pi/3+0.3-pi/24,0.3)-- (0*pi/3+0.3-pi/24,-0.95);
  \draw[thin](0*pi/3+0.3-pi/24,-1.05)--(0*pi/3+0.3-pi/24,-1.4);
  \draw[thin,dashed] (-pi/6+0.3+0.03,0.6)--(-pi/6+0.3+0.03,-0.65);
  % lambda
  \draw[thin,<->] (-pi/3+0.3+pi/24,-1.3) -- (0*pi/3+0.3-pi/24,-1.3) node[midway,below]{$\lambda$};  
  \end{scope}  
\end{tikzpicture}  

Знаю, что ток маленький $R_\phi$, $R_D$, $R_\Phi$. $I_\textcyrillic{малое} R_\Sigma$,
много меньше ЭДС синусоиды

$$
\begin{array}{ccc}
  \textcyrillic{ЭДС нагрузки}& + & \xcancel{\textcyrillic{синусоидa}}\\
  \uparrow&&\\
  \textcyrillic{непренебрежимо}
\end{array}
$$

Есть ЭДС нагрузки плюс индуктивность.

$$
e_d -E_\textcyrillic{н} = L_\Sigma \frac{\partial i_d}{\partial t} + \xcancel{i_dR_\Sigma}
$$

Член $i_dR_\Sigma$ демонстративно зачеркнут, потому что второго порядка малости. Даже
при номинальном токе составляет несколько процентов.

$i_d$ это и есть $i_\textcyrillic{фазы}$.

Проинтегрировав эту разность $i_d = f(\omega t)$, а потом возьмём интеграл

$$
I_{d \textcyrillic{среднее}} = \int i_d\; d(\omega t) =
$$

$\displaystyle \lambda = \frac{2\pi}{m}$ -- граничный режим.

$$
\frac{1}{2\pi/m} \int\limits_{-\frac{\pi}{m}+\alpha}^{\frac{\pi}{m}+\alpha}
i_d\;d(\omega t)
$$
-- среднее значение граничного тока.

$$
I_{d\;\textcyrillic{гр}} = f(m,\alpha) = I(\sim)
$$
-- амплитуда пульсаций выпрямленного тока.

Полагая, что $I_\textcyrillic{среднее} =
I_\textcyrillic{амплитуда пульсаций выпрямленного тока}$, получим зависимоть
максимальных пульсаций с учётом гармонического состава. Этот способ учитывает реальную
кривую без разложения её в ряд Фурье.

Чему равна амплитуда тока?

\end{document}
