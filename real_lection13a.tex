напоминание про сумму двух нерегулируемых инверторов. было д.з. Кто-нибудь сделал?

\begin{tikzpicture}
\draw[->,>=latex,thin] (0,-2) -- (0,2.5);
\draw[->,>=latex,thin] (-1,0) -- (3.5,0);
\draw (-0.5,-1) -- (-0.5,1) -- (0.5,1) -- (0.5,-1) -- (1.5,-1) -- (1.5,1);

\draw[thin,<->,>=latex] (0.5,0.5) -- (1.5,0.5) node[midway, above] {$\tau$};
\draw[thin] (-0.5,-1.2) -- (-0.5,-1.5) (1.5,-1.2) -- (1.5,-1.5);
\draw[<->,>=latex] (-0.5,-1.35) -- (1.5,-1.35) node[midway, below] {$T$};

\draw (2.5,-1.1) node {$\tau = \frac{1}{2}T$};
\end{tikzpicture}

$$
A_{(k)} = \frac{2\cdot 2}{T}
\int\limits_{-\frac{T}{2}}^{\frac{T}{2}} U_\textcyrillic{п} \cos\left(\frac{2\pi}{T}t\right)dt = 
\frac{4}{T} U_\textcyrillic{п} \frac{1}{k\frac{2\pi}{T}} 2\sin\left(k\frac{2\pi}{T}\cdot\frac{T}{4}\right) =
\frac{4}{\pi}\frac{\sin{k\frac{\pi}{2}}}{k} U_\textcyrillic{п}
$$

воспользовались формулой для определения преобразования Фурье, косинусной компоненты:

$$
A_{cos} = \frac{2}{T} \int\limits_{t_1}^{t_1+ T} X(t) \cos\left(\frac{2\pi}{T}\right) dt
$$

Это амплитуда, $A_k(max)$, Отрицательной амплитуда не бывает. Изменение знака определяет фазу. В начале координат
значения отрицательные. Если взать действующее значение первой гармоники

$$
U_{(1)} = \frac{2\sqrt{2}}{\pi} U_\textcyrillic{п}
$$

где коэффициент при $U_\textcyrillic{п}$ -- коэффициент формы синусоиды.

$$
U_{(k)} = \frac{U_{1}}{k}
$$

Трехфазный автономный инвертор напряжения (АИН). Построен на основе 3-х однофазных НПЧ

Подумаем как их хорошо между собой соединить. В случае 1-фазного источник питания был сеть, развязывали
его иногда с помощью трансформатора.

Мы не можем соединить их ни в звезду ни в треугольник. А на вторичной стороне соединим в звезду. 


\begin{tikzpicture}
	\draw (0,0) to[L] (1,0) (1.5,0) to[L] (2.5,0) (3,0) to[L] (4,0)
	 (4,0.5) to[L] (3.1,0.5) (2.5,0.5) to[L] (1.5,0.5) (0.9,0.5) to[L] (0,0.5);
	\draw[ultra thick] (0,.25) -- (1,.25) (1.5,.25) -- (2.5,.25) (3,.25) -- (4,.25);
	\draw (1,0) -- (1,-.5) -- (2.5,-.5) -- (2.5,0) (2.5,-.5) -- (4,-.5) -- (4,0);
%	\draw (2,-1.5) node[elmech](motor){M};
	\draw (0,0) -- (0,-1.5) to [Telmech=M,n=motor] (3,-1.5) --  (3,0);
	\draw (-0.3,1) rectangle (1.1,2) (1.3,1) rectangle (2.7,2) (2.9,1) rectangle (4.3,2);
	\draw (0,0.5) -- (0,1) (0.9,0.5) -- (0.9,1) (1.5,0.5) -- (1.5,1) (2.5,0.5) -- (2.5,1)
	(3.1,0.5) -- (3.1,1) (4,0.5) -- (4,1);
	\draw (0.4,1.5) node {АИН A};
	\draw (2,1.5) node {АИН B};
	\draw (3.6,1.5) node {АИН C};
% фазы
	\draw (-0.5,2.5) -- (4.5,2.5) (-0.5,3) -- (4.5,3); 
	\draw (0.3,2) -- (0.3,3) (0.7,2) -- (0.7,2.5) (1.8,2) -- (1.8,3) (2.2,2) -- (2.2,2.5) (3.3,2) -- (3.3,3) 
	(3.8,2) -- (3.8,2.5);
\end{tikzpicture}

Примечание: можно не использовать трансформатор, если двигательные обмотки подключить шестью концами

\begin{tikzpicture}
\newcommand{\Da}{0.5} 
\newcommand{\Db}{1.5} 
	\draw ({\Da*cos(90)},{\Da*sin(90)}) node (Aa) {} to[L]  ({\Db*cos(90)},{\Db*sin(90)}) node (Ab) {};
	\draw ({\Da*cos(210)},{\Da*sin(210)}) node (Ba) {} to[L]  ({\Db*cos(210)},{\Db*sin(210)}) node (Bb) {};
	\draw ({\Da*cos(330)},{\Da*sin(330)}) node (Ca) {} to[L]  ({\Db*cos(330)},{\Db*sin(330)}) node (Cb) {};

	\draw (Ca.center) --++ (0,0.3) --++ (-0.3,0) --++ (0,-1) --++ (1.5,0) --+ (0,{0.7 + \Da*sin(30) + 2});
	\draw (Cb.center) --++ (0,0.6) --++ (-0.3,0) --++ (0,{-0.6 + \Db*sin(30)+ 2});
	\draw (0.6,2) rectangle (2,3);

	\draw (Ab.center) --++ (-1.5,0) --++ (0,{-\Db+2});
	\draw (Aa.center) --++ (0.3,0) --++ (0,{\Da*sin(30)+1}) --++ (-1.5,0) --++ (0,0.25);
	\draw (-2.05, 2) rectangle (-0.65,3);

	\draw (Ba.center) --++ (0,{\Db*sin(30)+0.85}) --++ (3.5,0) --++ (0,0.65);
	\draw (Bb.center) --++ (0,-0.4) --++ (5,0) --++ (0,{0.4+\Db*sin(30) + 2});
	\draw (2.7,2) rectangle (4.1,3);
\end{tikzpicture}

3 точки где мы можем развязать -- на входе, на выходе и в нагрузке.

\begin{tikzpicture}
	\draw (0,0) rectangle (1.5,1)  (3,0) rectangle (4.5,1) (6,0) rectangle (7.5,1);
	\draw (0,2) rectangle (1.5,3)  (3,2) rectangle (4.5,3) (6,2) rectangle (7.5,3);
	\draw (0.4,1) -- (0.4,2) (1.1,1) -- (1.1,2);
	\draw (0.1,1.5) node {$+$} (1.4,1.5) node {$-$};
	\draw (3.4,1) -- (3.4,2) (4.1,1) -- (4.1,2);
        \draw (3.1,1.5) node {$+$} (4.4,1.5) node {$-$};
        \draw (6.4,1) -- (6.4,2) (7.1,1) -- (7.1,2);
        \draw (6.1,1.5) node {$+$} (7.4,1.5) node {$-$};
        \draw (0.75,2.5) node {B};
        \draw (3.75,2.5) node {B};
        \draw (6.75,2.5) node {B};
	\draw (0.35, 3) -- (0.35, 3.5) to[L] (1.15,3.5) -- (1.15,3) (1.15,4) to[L] (0.35,4);
	\draw[ultra thick] (0.35,3.75) -- (1.15,3.75); 
	\draw (3.35, 3) -- (3.35, 3.5) to[L] (4.15,3.5) -- (4.15,3) (4.15,4) to[L] (3.35,4);
	\draw[ultra thick] (3.35,3.75) -- (4.15,3.75);

	\draw (6.35, 3) -- (6.35, 3.5) to[L] (7.15,3.5) -- (7.15,3) (7.15,4) to[L] (6.35,4);
	\draw[ultra thick] (6.35,3.75) -- (7.15,3.75);
	\draw (8,2) node[right] {\begin{minipage}[t]{0.15\textwidth}нулевая встречно-параллельная схема \end{minipage}};
\end{tikzpicture}

\vspace{1cm}
\begin{tikzpicture}
	\draw (0,1.2) to[Ty-,mirror] (0,0) (0.5,1.1) to[Ty-,mirror] (0.5,0.1) -- (0.5,0) 
	   (1,1) to[Ty-,mirror] (1,0.2) -- (1,0) ;
	\draw (0,0) -- (1,0) ;
	\draw (2,0) -- (2,0.2)     to[Ty-,mirror] (2,1) -- (1,1)  
	      (2.5,0) -- (2.5,0.1) to[Ty-,mirror] (2.5,1.1) -- (0.5,1.1);
	\draw (3,0) -- (3,0)       to[Ty-,mirror] (3,1.2) -- (0,1.2) ;
	\draw (2,0) -- (3,0);
	\draw (0.5,0) -- (0.5,-0.2) -- (2.5,-0.2) -- (2.5,0);

	\draw (4,0) rectangle (4.9,1.3);
	\draw (5.5,0) rectangle (6.4,1.3);
	% фазы
	\draw (0,1.8) -- (6.5,1.8); 
	\draw (0,2.1) -- (6.5,2.1); 
	\draw (0,2.4) -- (6.5,2.4); 

	\draw (1.1,1.2) -- (1.1,2.4) (1.5,1.1) -- (1.5,2.1) (1.9,1) --(1.9,1.8);
	\draw (4.2,1.3) -- (4.2,2.4) (4.45,1.3) -- (4.45,2.1) (4.7,1.3) --(4.7,1.8);
	\draw (5.7,1.3) -- (5.7,2.4) (5.95,1.3) -- (5.95,2.1) (6.2,1.3) --(6.2,1.8);

	\draw (3,-1.3) circle (0.4) circle(0.6) node {M}; 
	\draw (7,1.5) node[right] 
	{\begin{minipage}[t]{0.2\textwidth} когда 3-х фазная нагрузка - 0-й провод не нужен\end{minipage}};
	
	\draw (1.5,-0.2) -- (1.5,-0.5) -- ({3+0.6*cos(150)}, {-1.3 + 0.6*sin(150)});
	\draw (4.45,0) -- (4.45,-0.3) -- ({3+0.6*cos(45)}, {-1.3 + 0.6*sin(45)});
	\draw (5.95,0) -- (5.95,-0.3) -- ({3+0.6*cos(20)}, {-1.3 + 0.6*sin(20)});
\end{tikzpicture}

Здесь нет гальванической развязки для ``0''-х схем. Если мостовая схема, то галваническая развязка.

По недоразумению называется мостовой схемы, хотя она нулевая.

Вспомним однофазный мостовой АИН

\begin{tikzpicture}
	\draw (0,3) -- (0.5,3) to[eC,mirror,-*] (0.5,1.5) to[eC,mirror] (0.5,0) -- (0,0); 
	\draw (0.5,3) -- (1,3) to[european resistor ] (1,1.5) to[european resistor] (1,0) -- (0.5,0);  
	\draw (0.5,1.5) to[short,-*] (1,1.5);
	\draw (1,0) -- (2.5,0) to[Tnigbt] (2.5,1.5)  to[Tnigbt] (2.5,3) -- (1,3);
	\draw (2.5,0) -- (3,0) to[D-] (3,1.5) to[D-] (3,3) -- (2.5,3);
	\draw (1,1.5) -- (1.2,1.3) to[european resistor] (2.3,1.3) -- (2.5,1.3) to[short,*-*] (3,1.3);
	\draw (2.5,0) -- (4.5,0) to[Tnigbt] (4.5,1.5) node (A) {} to[Tnigbt] (4.5,3) -- (2.5,3); 
	\draw (A.center) to[european resistor] ++ (-1.2,0) node (B) {};
	\draw[dashed] (B.center) -- (1,1.5);

	\draw [<-]  (1.8,1) --++ (-1,-2) node[right] {0-я схема -- включаем нагрузку здесь};
	\draw [<-]  (4,1.3) --++(1.5,-1.5) node[right] {здесь рисуем дальше};
\end{tikzpicture}

В мостовой схеме в 2 раза больше напряжений.

\begin{circuitikz}
\draw (0,3) -- (0.5,3) to[eC,mirror,-*] (0.5,1.5) to[eC,mirror] (0.5,0) -- (0,0);
	\draw (3,3) to[Ty-] (3,1.5)  to[Ty-] (3,0);
	\draw (3.5,0) to[D-] (3.5,1.5)  to[D-] (3.5,3);

	\draw (5,3) to[Ty-] (5,1.5)  to[Ty-] (5,0);
	\draw (5.5,0) to[D-] (5.5,1.5)  to[D-] (5.5,3);
	
	\draw (7,3) to[Ty-] (7,1.5)  to[Ty-] (7,0);
	\draw (7.5,0) to[D-] (7.5,1.5)  to[D-] (7.5,3);

	\draw (3,1.8) to[short,*-*] (3.5,1.8) -- (8,1.8) to[european resistor,l=$Z_A$] (9.5,1.8);	
	\draw (5,1.5) to[short,*-*] (5.5,1.5) -- (8,1.5) to[european resistor] (9.5,1.5);	
	\draw (9.5,1.2) to[european resistor,l=$Z_C$] (8,1.2) -- (7.5,1.2) to[short,*-*] (7,1.2);	

	\draw (0.5,0) -- (7.5,0) (0.5,3) -- (7.5,3); % плюсовая и минусовая шины
	\draw (9.5,1.2) -- (9.5,1.8) (9.5,1.5) --++(0.5,0) --++ (0,-2) --++ (-9,0) --++(0,2) -- (0.5,1.5) (10.5,1.5) node[right]{0 -- негрузки}; 

	\draw[ultra thick, red] (2, -0.3) -- (2.4,-0.7) (2,-0.7) -- (2.4,-0.3);
	\draw[ultra thick, red] (4, -0.3) -- (4.4,-0.7) (4,-0.7) -- (4.4,-0.3);
	\draw[ultra thick, red] (6, -0.3) -- (6.4,-0.7) (6,-0.7) -- (6.4,-0.3);
	\draw[<-,thin] (1.2,1.5) -- (0.5,-1) node[right] {это '' 0'' в однофазных нулевых АИН};
\end{circuitikz}
Чтобы короче рисуем на запираемых тиристорах.

Источник должен быть со средней точкой. Меандры сдвинуты на 1/3 периода у трех одно-фазных АИН. Какая величина 1-й гармоники -- она 0 по ''0'' проводу, если интересует 1-я гармоника.
Сумма равна 0 в гармониках кратных трем. Оборвав провод, улучшим состав. Четныз не было $n=3(2k-1)$
 
По сравнению с нулевой модовый состав улучшился. Неправильно называют техфазная ''мостовая''.  Спорить не будем, но будем но правильнее было бы назвать ''... ''
Выполнена на основе 3-х однофазных нулевых.

Нулевая схема плоха, что напряжение в нулевой схеме ''в два раза меньше?''

3-х фазная схема АИН на основе 3 х мостовых

\subsection*{3-х фазная схема АИН на основе 3-х мостовых}

\begin{circuitikz}
\draw (0,3) -- (0.5,3) to[eC,mirror,-*] (0.5,1.5) to[eC,mirror] (0.5,0) -- (0,0);
	\draw (1,3) to[european resistor] (1,1.5) to[european resistor] (1,0);

        \draw (3,3) to[Ty-] (3,1.5)  to[Ty-] (3,0);
        \draw (3.5,0) to[D-] (3.5,1.5)  to[D-] (3.5,3);

        \draw (5,3) to[Ty-] (5,1.5)  to[Ty-] (5,0);
        \draw (5.5,0) to[D-] (5.5,1.5)  to[D-] (5.5,3);

        \draw (7,3) to[Ty-] (7,1.5)  to[Ty-] (7,0);
        \draw (7.5,0) to[D-] (7.5,1.5)  to[D-] (7.5,3);

        \draw (9,3) to[Ty-] (9,1.5)  to[Ty-] (9,0);
        \draw (9.5,0) to[D-] (9.5,1.5)  to[D-] (9.5,3);

	\draw (0.5,1.5) to[short,*-*] (1,1.5);

        \draw (11,3) to[Ty-] (11,1.5)  to[Ty-] (11,0);
        \draw (11.5,0) to[D-] (11.5,1.5)  to[D-] (11.5,3);

        \draw (13,3) to[Ty-] (13,1.5)  to[Ty-] (13,0);
        \draw (13.5,0) to[D-] (13.5,1.5)  to[D-] (13.5,3);

	\draw (3,1.5) to[short,*-*] (3.5,1.5) to[R] (5,1.5) to[short,*-*] (5.5,1.5);
	\draw (7,1.5) to[short,*-*] (7.5,1.5) to[R] (9,1.5) to[short,*-*] (9.5,1.5);
	\draw (11,1.5) to[short,*-*] (11.5,1.5) to[R] (13,1.5) to[short,*-*] (13.5,1.5);


	\draw (0.5,0) -- (13.5,0) (0.5,3) -- (13.5,3); % плюсовая и минусовая шины
\end{circuitikz}

Этот случай -- гальваническая развязка в нагрузке. Её достоинства -- при том же напряжении нагрузки в 2 раза меньше в трансформаторе. Мощность выросла в два раза. Появилась 3-я гармоника!!! -- ухудшилось.

\begin{tabular}{cc}
	3-я гармоника & 33\%\\
	5-я гармоника & 20\%\\
\end{tabular}

Поэтому такую схему применяют очень редко. Гармоники низкой частоты большой амплитуды.

Сеть 380/220в 50 Герц.

$E_{d0} = 2.34U\text{a}\cdot220 = 1.35\cdot 380 \approx 513В$
Добавляем $\pm 10\%$ тогда 564В.

Но бывает перенапряжения. 20\% бывают всегда. 680В, 700В, 750В -- возможное повышение в нормальном режиме.

К.З. тоже нормальный режим. Авария -- если К.З. развивается дальше.
На 500В -- достаточно дорогие. Через конденсатор идет переменный ток 300Гц. А если ШИМ то килогерцы.
100-200мА на 1кВатт мощности на
$$
n = 6k\pm 1 \text{-- без ШИМа и ШИРа}
$$
В графиках нет 7,12,15 отрезков времени.

Потенциаль относительно какой точки? <удем считать что относительно $"0"$ в кавычках.
Пусть 1,2,3,4,5,6 -- моменты отпирания ключей. Период разделен на 6 отрезков по 60 градусов.

$U_a$ -- потенциал относительно точки $"0"$.

\begin{circuitikz}

	\draw (0,1) -- (0,0) to[R] (3,0) -- (3,-1) node[right] {мне неизвестен потенциал $"0"$};
	\draw[thin,<->] (0,0.8) -- (3,0.8) node[midway,above] {$U_{A0}$};
\end{circuitikz}


\begin{circuitikz}
	\foreach \x\y in {1/1,2/2,3/3,4/4,5/5,6/6,7/1,8/2,9/3,10/4,11/5,12/6,13/1,14/2,15/3}
	\draw (\x,2) node {\y};
	\draw (1,0) -- (1,.5) -- (4,.5) -- (4,-.5) -- (7,-.5) -- (7,.5) -- (10,.5) -- (10,-.5) -- (13,-.5);
	\draw (3,-2) -- (3,-1) -- (6,-1) -- (6,-2) -- (9,-2) -- (9,-1) -- (12,-1) -- (12,-2);
	\draw (5,-3.5) -- (5,-2.5) -- (8,-2.5) -- (8,-3.5) -- (11,-3.5) -- (11,-2.5) -- (14,-2.5) -- (14,-3.5);
	\draw[thin,help lines,smooth,->] (0,0) -- (15,0); \draw (0,0) node[left] {$U_A$};
	\draw[thin,help lines,smooth,->] (0,-1.5) -- (15,-1.5); \draw (0,-1.5) node[left] {$U_B$};
	\draw[thin,help lines,smooth,->] (0,-3) -- (15,-3); \draw (0,-3) node[left] {$U_C$};
	\draw[thin,help lines,smooth,->] (0,-4.5) -- (15,-4.5); \draw (0,-4.5) node[left] {$U_{A0}$};
	\draw[thin,help lines,smooth,->] (0,-6) -- (15,-6); \draw (0,-6) node[left] {$U_{B0}$};
	\draw[thin,help lines,smooth,->] (0,-7.5) -- (15,-7.5); \draw (0,-7.5) node[left] {$U_{C0}$};
	\draw[thin,help lines,smooth,->] (0,-9) -- (15,-9); \draw (0,-9) node[left] {$U_{AB}$};
	\draw[thin] (0,-9.3) -- (0,0.7);
\end{circuitikz}

Чтобы понятнее еще раз рисую схему:

\begin{circuitikz}
\draw (0,0) node[left] {$+$}-- (0,1) -- (0.5,1) -- (0.5,0.4) to[european resistor,l_=C] (2.5,0.4) -- (2.5,1) to[european resistor,l=B] (4.5,1)  -- (4.5,0) node[right] {$-$};  
\draw (0.5,1) -- (0.5,1.6) to[european resistor,l=A]  (2.5,1.6) -- (2.5,1);
	\draw (5,0) node[right] {$Z_A=Z_B=Z_C$};
	\draw (0.5,-1.5) to[european resistor,l=$R/2$] (2.5,-1.5) to[european resistor,l=$R$] (4.5,-1.5);
\end{circuitikz}

В 3-х фазных ФИН используются такие же способы регулирования выходного напряжения что и в однофазных
\begin{enumerate}
\item Входное напряжение
\item ШИР
\item ШИМ
\item изменение фазы угла напряжения между 2-х трехфазных АИН, складываемых на нагрузке.
\end{enumerate}

Осуществление ШИР (большое множество ШИМ с ШИР)

\begin{circuitikz}
	\draw (1,0) -- (1,.5) -- (2.4,.5) -- (2.4,-.5) -- (2.6,-.5) -- (2.6,.5) -- (4,.5) -- (4,-.5) -- (7,-.5) -- (7,.5)  --  (8.4,.5) -- (8.4,-.5) -- (8.6,-.5) -- (8.6,.5) -- (10,.5) -- (10,-.5) -- 
        (11.4,-.5) -- (11.4,.5) -- (11.6,.5) -- (11.6,-.5) -- % добавлено по сравнению с лекцией
	(13,-.5);
	\draw (3,-2) -- (3,-1) -- (4.4,-1) -- (4.4,-2) -- (4.6,-2) -- (4.6,-1) -- (6,-1) -- (6,-2) -- 
         (7.4,-2) -- (7.4,-1) -- (7.6,-1) -- (7.6,-2) -- % добавлено по сравнению с лекцией
	(9,-2) -- (9,-1) -- (10.4,-1) -- (10.4,-2) -- (10.6,-2) -- (10.6,-1) -- (12,-1) -- (12,-2);
	\draw (5,-3.5) -- (5,-2.5) -- (6.4,-2.5) -- (6.4,-3.5) -- (6.6,-3.5) -- (6.6,-2.5) -- (8,-2.5) -- (8,-3.5) -- 
	(9.4,-3.5) -- (9.4,-2.5) -- (9.4,-2.5) -- (9.6,-2.5) -- (9.6,-3.5) -- % добавлено по сравнению с лекцией
	(11,-3.5) -- (11,-2.5) --  (12.4,-2.5) -- (12.4,-3.5) -- (12.6,-3.5) -- (12.6,-2.5) -- (14,-2.5) -- (14,-3.5);	


%	\draw[dashed] (4,-4) node[right] {хотел бы получить такое:};
	\draw[dashed] (6.5,-6) -- (5.5,-6) node[left] {$U_A$};	
	\draw (7,-6.5) -- (7,-5.5) -- (7.4,-5.5) -- (7.4,-6) -- (7.6,-6) -- (7.6,-5.5) --  (8,-5.5) -- (8,-5) -- (8.4,-5) -- (8.4,-6) -- (8.6,-6) -- (8.6,-5) -- (9,-5) -- (9,-5.5) -- (9.4,-5.5) -- (9.4,-6) -- (9.6,-6) --
	(9.6,-5.5) -- (10,-5.5) -- (10,-6.5) --  (10.4,-6.5) --  (10.4,-6) -- (10.6,-6) -- (10.6,-6.5) -- (11,-6.5) -- (11,-7) --  (11.4,-7) -- (11.4,-6) -- (11.6,-6) -- (11.6,-7) -- (12,-7) -- (12,-6.5) -- 
	(12.4,-6.5) -- (12.4,-6) -- (12.6,-6) -- (12.6,-6.5) -- (13,-6.5) -- (13,-5.5); 
\draw[dashed] (13.5,-6) -- (14.5,-6);
\end{circuitikz}

Можно не один раз повторить, а два или несколько, сколько позволит быстродействие СПП.

А целесообразность? Как то хорошо регулировать напряжение. Гармонический состав будет изменятся.

Способы рационального ШИМ -- изменение какой-то величины функцией како-то другой функйии.
Оптимальная -- близко к какой-то другой.

Что мы выпрямляем: ток или напряжение? Выпрямитель, работающий на индуктивную нагрузку --
у него есть разнополярное напряжение.

Подразумевается, что полезную работу делает ток. Аккумулятор заряжается током. Количество гальваники
$\sim$ ампер.часов.

Задача: оптимизировать ток.

ШИР -- ухудшая гармонический состав напряжения. В обычном меандре $U_n = \frac{U(1)}{n}$.
25 раз переключил, появилась 25-я гармоника. Гармоника \underline{в напряжении} увеличивается почти в 100 раз.
\vspace{0.3cm}

\begin{circuitikz}
\draw (0,0) -- (0,.5) -- (.3,.5) -- (.3,1) -- (.6,1) -- (.6,0) -- (.9,0) -- (.9,1)
	-- (1.2,1) -- (1.2,.5) -- (1.5,.5) -- (1.5,0);
\draw[<->] (2,0) -- (2,1);
\end{circuitikz}

Ухудшая гармонический состав напряжения может улучшить гармонический состав тока.

\begin{circuitikz}
\draw (0,0) node[left] {Что такое уровни напряжения?};
	\draw  (1,1.5) node[right] {$\frac{2}{3}$}
	(1,0.5) node[right] {$\frac{1}{3}$}
	(1,-0.5) node[right] {$-\frac{1}{3}$}
	(1,-1.5) node[right] {$-\frac{2}{3}$};
\end{circuitikz}

Появился как бы 5-й уровень "0". Если я буду модулировать, чтобы среднее было близко к функции, в
данном случае к синусоиде.
