\documentclass[a4paper,11pt]{report}
\usepackage[T1,T2A]{fontenc}
\usepackage{ucs}
\usepackage[utf8x]{inputenc}
\usepackage{latexsym}
%\usepackage{amsmath}
%\usepackage{mathtext}
\usepackage[english,ukrainian,russian]{babel}
\usepackage{tikz}
\usepackage{siunitx}
\usetikzlibrary{patterns}
\usetikzlibrary{positioning}
%\usepackage[compatibility]{circuitikz}
\usetikzlibrary{decorations.text}
\usetikzlibrary{decorations.pathmorphing}
\usepackage[american,cuteinductors,smartlabels]{circuitikz}
\usepackage{enumitem}
\usepackage{wrapfig}
\usepackage{float}
\usepackage{graphicx}
\graphicspath{{./images/}}
\usepackage{hyperref}
\usepackage{cancel}
\usepackage{pgfplots}
\usetikzlibrary{arrows.meta,calc,decorations.markings,math,arrows.meta}
\usepackage{booktabs}

% http://www1.combinatorics.org/Information/article.html
\author{ Прокшин Артем \\
\small ЛЭТИ\\
\small \texttt{taybola@gmail.com}}
%\usetikzlibrary{circuits}
%\usetikzlibrary{circuits.ee}
%\usetikzlibrary{circuits.ee.IEC}
%\usetikzlibrary{circuits.logic.IEC}

\usetikzlibrary{calc}
\ctikzset{bipoles/thickness=1}
\ctikzset{bipoles/length=0.8cm}
\ctikzset{bipoles/diode/height=.375}
\ctikzset{bipoles/diode/width=.3}
\ctikzset{tripoles/thyristor/height=.8}
%\ctikzset{tripoles/thyristor/width=1}
\ctikzset{tripoles/thyristor/width=0.8}
\ctikzset{bipoles/vsourceam/height/.initial=.7}
\ctikzset{bipoles/vsourceam/width/.initial=.7}
\tikzstyle{every node}=[font=\small]
\tikzstyle{every path}=[line width=0.8pt,line cap=round,line join=round]

\ctikzset{resistor = european}
\ctikzset{inductor = american}
\ctikzset{tripoles/thyristor/height=0.35}
\ctikzset{tripoles/thyristor/height 2=0.4} 
%\ctikzset{tripoles/thyristor/width=0.35} 
\ctikzset{tripoles/thyristor/diode width left=0.35}
\ctikzset{tripoles/thyristor/diode width right=0.35}

\ctikzset{bipoles/cuteindictor/coils/.initial=3}
\ctikzset{bipoles/americanindictor/coils/.initial=3}

\ctikzset{tripoles/Lnigbt/bodydiode width/.initial=2.5}

\makeatletter
%% american inductor

\pgfcircdeclarebipole{}{\ctikzvalof{bipoles/americaninductor/height 2}}{americaninductor}{\ctikzvalof{bipoles/americaninductor/height}}{\ctikzvalof{bipoles/americaninductor/width}}{
    \pgf@circ@res@step=\ctikzvalof{bipoles/americaninductor/width}\pgf@circ@Rlen
    \pgfsetlinewidth{\pgfkeysvalueof{/tikz/circuitikz/bipoles/thickness}\pgfstartlinewidth} 
    \pgftransformationadjustments
    \advance \pgf@circ@res@step by \pgfhorizontaltransformationadjustment\pgflinewidth
    \divide \pgf@circ@res@step by \ctikzvalof{bipoles/americaninductor/coils}
    \divide \pgf@circ@res@step by 2
    \pgf@circ@res@other = \ctikzvalof{bipoles/americaninductor/coil height}\pgf@circ@Rlen

    \pgfpathmoveto{\pgfpoint{\pgf@circ@res@left-\pgfhorizontaltransformationadjustment*0.5*\pgflinewidth}{-\pgfverticaltransformationadjustment*0.4*\pgfstartlinewidth}}%correct value would be 0.5 but arcs are not really flat, therefore 0.4 is better is (almost) all cases
  \foreach \x in {1,...,\ctikzvalof{bipoles/americaninductor/coils}}
    {\pgfpatharc{180}{0}{\pgf@circ@res@step and \pgf@circ@res@other}}
    \pgfsetbuttcap
    \pgfsetbeveljoin
    \pgfusepath{stroke}
}
\makeatother



\AddEnumerateCounter{\Asbuk}{\@Asbuk}{\CYRM}
\AddEnumerateCounter{\asbuk}{\@asbuk}{\cyrm}
\renewcommand{\theenumi}{(\asbuk{enumi})}
\renewcommand{\labelenumi}{\asbuk{enumi})}


%\title{Силовая энергетика}
%\date{}
\begin{document}
\tikzset{component/.style={draw,thick,circle,fill=white,minimum size =0.75cm,inner sep=0pt}}
%http://tex.stackexchange.com/questions/132076/how-to-draw-a-dc-motor-in-circuitikz

%http://tex.stackexchange.com/questions/132076/how-to-draw-a-dc-motor-in-circuitikz`
% prepare to create bipoles

\makeatletter
\def\TikzBipolePath#1#2{\pgf@circ@bipole@path{#1}{#2}}
\def\CircDirection{\pgf@circ@direction}

%\pgf@circ@Rlen = \pgfkeysvalueof{/tikz/circuitikz/bipoles/length}

\makeatother 

\newlength{\ResUp} \newlength{\ResDown}
\newlength{\ResLeft} \newlength{\ResRight}

% set default motor size

\ctikzset{bipoles/motor/height/.initial=.8}
\ctikzset{bipoles/motor/width/.initial=.8}

% create motor shape

\pgfcircdeclarebipole{}
 {\ctikzvalof{bipoles/motor/height}}
 {motor}
 {\ctikzvalof{bipoles/motor/height}}
 {\ctikzvalof{bipoles/motor/width}}
 {
    \pgfsetlinewidth{\ctikzvalof{bipoles/thickness}\pgfstartlinewidth}
    \pgfextractx{\ResRight}{\northeast}
    \pgfextracty{\ResUp}{\northeast}
    \pgfextractx{\ResLeft}{\southwest}
    \pgfextracty{\ResDown}{\southwest}

  \pgfpathmoveto{\pgfpoint{0.775\ResLeft}{0.2\ResDown}}
\pgfpathlineto{\pgfpoint{\ResLeft}{0.2\ResDown}}
  \pgfpathlineto{\pgfpoint{\ResLeft}{0.2\ResUp}}
  \pgfpathlineto{\pgfpoint{0.775\ResLeft}{0.2\ResUp}}
\pgfpathellipse{\pgfpointorigin}{\pgfpoint{0.8\ResRight}{0cm}}
    {\pgfpoint{0cm}{0.8\ResUp}}
  \pgfpathmoveto{\pgfpoint{0.775\ResRight}{0.2\ResDown}}
    \pgfpathlineto{\pgfpoint{\ResRight}{0.2\ResDown}}
  \pgfpathlineto{\pgfpoint{\ResRight}{0.2\ResUp}}
  \pgfpathlineto{\pgfpoint{0.775\ResRight}{0.2\ResUp}}
  \pgfusepath{draw} %draw motor
    \pgftext[rotate=-\CircDirection]{\textsf{M}}
 }

% create motorto-path style

\def\motorpath#1{\TikzBipolePath{motor}{#1}}
\tikzset{motor/.style = {\circuitikzbasekey, /tikz/to path=\motorpath, l=#1}}

% end of setup



%\maketitle

%\begin{abstract}
%\end{abstract}
%\include{lection1}
%\include{lection2}
%\include{lection3}
%\include{lection3a}
%
\begin{equation}
E_{d0} = \frac{m}{\pi}\sqrt{2} E_{2\Phi} Sin\frac{\pi}{m}
\end{equation}
-- выпрямленная ЭДС неуправляемого преобразователя

\begin{equation}
E_d = E_{d0} Cos \alpha
\end{equation}
-- выпрямленная ЭДС управляемого преобразователя

\begin{equation}
U_d = E_{d} - U_0 - \frac{m}{2\pi} X_\Phi I_d - I_d \left\{
R_\Phi + \left(r_\phi + R_D\right) \left( 1 - \frac{\gamma m}{4 \pi}\right) 
\right\}
\label{three}
\end{equation}
-- Выпрямленное напряжение для угла $\alpha$, или максимальное (при $\alpha=0$).

$U_0$ -- напряжение на вентиле, $R_D$ -- дифференциальное сопротивление вентиля.
$r_\phi$ -- полное эквивалентное сопротивление фазы с учетом сопротивления сети,
приведённого ко вторичной обмотке, сопротивление проводников.

Для точности отметим член ${\displaystyle \frac{\gamma}{4\pi}}$.

\hspace{-3 cm}
\begin{tikzpicture}
  \begin{scope}[xscale=2,yscale=3]
    \newcommand{\alphaa}{26.0 * pi / 180}
    \newcommand{\gammaa}{21.0 * pi / 180}

    \draw[thin, ->] (-2.7, 0) -- (5.8,0) node[right] {$\omega t$};
    \draw[thin, ->] (0,0) -- (0,1.3) node[left] {$U$};
    
    \foreach \x/\xtext in {{-pi/3}/{-\frac{\pi}{m}}, 0,
      {pi/3}/{\frac{\pi}{m}},{pi}/\pi}
    \draw (\x,0.1) -- (\x,-0.1) node [below] {$\xtext$};

    % A,B,C
    \draw[domain=-2.7:5.8, smooth, yellow]
    plot (\x,{cos((\x) r)}); \node[above] at (0.2,1) {$e_k$};
    \draw[domain=-2.7:5.8, help lines, smooth, green]
    plot (\x,{cos((\x-2/3*pi) r)}); \node[above] at (2*pi/3+0.2,1) {$e_{k+1}$};
    \draw[domain=-2.7:5.8, help lines, smooth]
    plot (\x,{cos((\x-4/3*pi) r)}); \node[above] at (-2*pi/3+0.2,1) {$e_{k-1}$};
    % (A+B)/2 
    \draw[domain={pi/3:pi/3+pi}, loosely dotted]
    plot (\x,{cos(\x r)/2 + cos((\x - 2/3*pi) r)/2});


    
    \foreach \qq [evaluate=\qq as \qqshft using \qq*2*pi/3] in {0,...,1}
     {
      \begin{scope}[xshift=\qqshft cm,
          every path/.style={color=red}]

    % Ed' Ed''
    \draw[domain={-pi/3 + \alphaa + \gammaa}:
        {pi/3+\alphaa-0.03},loosely dotted,red]
    plot ({\x-0.03}, {cos(\x r)-0.03})
    -| ({pi/3+\alphaa-0.06},  {cos((\alphaa +pi/3-2/3*pi) r)-0.03})
    [domain={pi/3+\alphaa-0.06}:{pi+\alphaa}]
    plot ({\x-0.03}, {cos((\x-2/3*pi)r)-0.03});

    \draw[domain={-pi/3 + \alphaa + \gammaa}:
                 {pi/3+\alphaa+\gammaa+0.03},loosely dashed,red]
    plot ({\x+0.03}, {cos(\x r)+0.03})
    -| ({pi/3+\alphaa+\gammaa+0.03},
                     {cos((\alphaa+\gammaa +pi/3-2/3*pi) r)+0.03})
    [domain={pi/3+\alphaa+\gammaa+0.03}:{pi+\alphaa}]
    plot ({\x+0.03}, {cos((\x-2/3*pi)r)+0.03});
        
        %Ed
        \draw[domain={-pi/3 + \alphaa + \gammaa}:{pi/3},ultra thick]
        plot (\x, {cos(\x r)});
        \draw[domain={pi/3}:{\alphaa + pi/3},ultra thick]
        plot (\x,{cos(\x r)})
        -| (\alphaa+pi/3, {cos((\alphaa + pi/3) r)/2 + cos((\alphaa-pi/3) r)/2})
        [domain={\alphaa+pi/3}:{\alphaa+pi/3+\gammaa}]
        plot (\x,{cos(\x r)/2 + cos((\x-2/3*pi) r)/2})
        -| (\alphaa+pi/3+\gammaa, {cos((\alphaa +pi/3+ \gammaa-2*pi/3) r) })
        [domain={\alphaa+pi/3+\gammaa}:{pi/3+2*pi/3}]
          plot(\x,  {cos((\x-2/3*pi) r)})
        ;        
       \end{scope}
     }
     %hatch
     \draw[domain=pi/3 + \alphaa:pi/3 + \alphaa+\gammaa,red,
         pattern=north east lines,pattern color=red]
       ({pi/3+\alphaa},  {cos((\alphaa +pi/3-2/3*pi) r)}) --
     plot (\x,{cos(\x r)/2 + cos((\x - 2/3*pi) r)/2});
     \draw[domain=pi/3 + \alphaa:pi/3 + \alphaa+\gammaa,red,
       pattern=north east lines,pattern color=red]
     plot (\x,{cos((\x - 2/3*pi) r)})
     -| ({pi/3 + \alphaa+\gammaa},{cos((pi/3 + \alphaa+\gammaa) r)/2
       + cos((pi/3 + \alphaa+\gammaa - 2/3*pi) r)/2});
     %
     \node[above,color=red] at ({\gammaa + pi/3},1) {$e_{d\:'}$};
     \draw[thin,->,color=red,dotted]  ({\gammaa + pi/3},1) --
     ({\gammaa + pi/3},{cos((pi/3+\alphaa - 2*pi/3) r)+0.05}); 
     \node[below,color=red] at ({pi-pi/6},-0.5) {$e_{d\:''}$};
     \draw[thin,->,color=red,dashed] ({pi-pi/6},-0.5) --
     ({pi/3+\alphaa+\gammaa+0.1},{cos((pi/3+\alphaa+\gammaa) r)+0.1});
     \node[below] at (pi+0.3, -0.6) {${\displaystyle \frac{e_k + e_{k+1}}{2}}$};
     \draw[thin,->] (pi+0.2,-0.6) --
     (pi+0.3,{cos((pi+0.3) r)/2 + cos((pi-2*pi/3+0.3) r)/2 -0.1});
     \draw[thin] (pi/3, -0.35) -- (pi/3,-0.8);
     \draw[thin] (pi/3+\alphaa,-0.1) -- (pi/3+\alphaa, -0.8);
     \draw[thin,->] (pi/3, -0.6) -- (pi/3+\alphaa,-0.6);
     \node[left] at (pi/3, -0.6) {$\alpha\:'=\alpha$};
     \draw[thin] (pi/3+\alphaa+\gammaa,-0.35) --
     (pi/3+\alphaa+\gammaa,-0.8);
     \draw[thin] (pi/3-0.05,-0.65) -- (pi/3+0.05,-0.55);
     \draw[thin,->] (pi/3+\alphaa-0.25,-0.75) -- (pi/3+\alphaa,-0.75);
     \draw[thin,<-] (pi/3+\alphaa+\gammaa,-0.75) --
     (pi/3+\alphaa+\gammaa+0.3,-0.75) node[below] {$\gamma$};
     \draw[thin] (pi, 0.55) -- (pi,1.2);
     \draw[thin] (pi +\alphaa + \gammaa,1) -- (pi +\alphaa + \gammaa,1.2);
     \draw[thin,->] (pi,1.15) -- (pi +\alphaa + \gammaa,1.15)
     node[right] {$\alpha\:'\,' = \alpha+\gamma$};
     \draw[thin] (pi-0.05,1.1) -- (pi+0.05,1.2);
  \end{scope}
  \end{tikzpicture}    

Почти в каждом билете будет задачка, в которой придется рисовать.
кривую
\begin{tikzpicture}
  \draw[ultra thick,red] (0,0) -- (1,0);
  \end{tikzpicture}
можно получить как полусумму двух кривых $e_{d\:'}$
\begin{tikzpicture}
  \draw[thick,loosely dotted,red] (0,0) -- (1,0);
\end{tikzpicture}
и кривой $e_{d\:'\:'}$
\begin{tikzpicture}
  \draw[thick,loosely dashed,red] (0,0) -- (1,0);
\end{tikzpicture}.

В прошлый раз было убедительно доказано, что при соединении двух ЭДС
с равными внутренними сопротивлениями суммарная ЭДС будет
полусуммой этих ЭДС.

Из-за индуктивности ток в $k$-й фазе спадает с конечной скоростью, а в
$k+1$ -й нарастает с конечной скоростью. Разность ЭДС $e_{k+1} - e_k$
падает на сопротивлении. Площадка
\begin{tikzpicture}
  \begin{scope}[xscale=2,yscale=3]
      \newcommand{\alphaa}{26.0 * pi / 180}
    \newcommand{\gammaa}{21.0 * pi / 180}
         %hatch
         \draw[color=red,domain=pi/3 + \alphaa:pi/3 + \alphaa+\gammaa,
           pattern=north east lines,pattern color=red]
         ({pi/3+\alphaa},  {cos((\alphaa +pi/3-2/3*pi) r)}) --
         % (A+B)/2
         plot (\x,{cos(\x r)/2 + cos((\x - 2/3*pi) r)/2})
         %B
         plot (\x,{cos((\x - 2/3*pi) r)})
         -| ({pi/3 + \alphaa+\gammaa},{cos((pi/3 + \alphaa+\gammaa) r)/2
           + cos((pi/3 + \alphaa+\gammaa - 2/3*pi) r)/2});
         
  \end{scope}
\end{tikzpicture}
показывает падение напряжения из-за индуктивности фазы.
Напомню, $X_\phi = 2\pi f L_\phi$ (индуктивность это величина отношения
${\displaystyle \frac{\textcyrillic{потока}}{\textcyrillic{к току}}}$,
где магнитный поток--это магнитный поток рассеяния,
обусловленный током нагрузки, а не током намагничивания.
\begin{tikzpicture}
  \begin{scope}[xscale=1.7,yscale=2.8]
          \newcommand{\alphaa}{26.0 * pi / 180}
    \newcommand{\gammaa}{21.0 * pi / 180}
     % I
     \draw[thin, ->] (-1.0, -1.7) -- (2,-1.7) node[right] {$\omega t$};
     \draw[thin, ->] (0,-1.9) -- (0,-1.2) node[left] {$I$};

     \draw[green] ({\gammaa - pi/3},  -1.4) -- (\alphaa, -1.4);
     \draw[green] (\alphaa, -1.4) -- (\gammaa, -1.7);
     \draw[green] (\gammaa, -1.7) -- ({\alphaa + pi/3},-1.7);

     \draw[yellow] ({\gammaa - pi/3},  -1.7) -- (\alphaa, -1.7);
     \draw[yellow] (\alphaa, -1.7) -- (\gammaa, -1.4);
     \draw[yellow] (\gammaa, -1.4) -- ({\alphaa + pi/3},-1.4);

     \draw[thin] (\alphaa,-1.65) -- (\alphaa,-1.9);
     \draw[thin] (\gammaa,-1.65) -- (\gammaa,-1.9);
     \draw[thin, ->] ({\alphaa-pi/12}, -1.8) -- (\alphaa,-1.8);
     \draw[thin, <-] (\gammaa,-1.8) -- ({\gammaa+pi/6},-1.8) node[below] {$\gamma$};
     \draw[thin,<->] ({(\gammaa+\alphaa)/2+0.6},-1.7) --
     ({(\gammaa+\alphaa)/2+0.6},-1.4);
     \node[left] at  ({(\gammaa+\alphaa)/2+0.6},-1.55) {$I_d$};
  \end{scope}
\end{tikzpicture}
Делали допущение $i_d = I_d$ -- мгновенное равно среднему, пульсаций нет.
Уменьшить пульсации до нуля мы не можем, но уменьшить до уровня, когда
пульсациями можем пренебречь технически возможно.

Мгновенное значение ЭДС при угле регулирования $\alpha$ и угле коммутации
$\gamma$ представим в виду полусуммы

$$
e_{\alpha,\gamma} = \frac{e_{\begin{tikzpicture}
  \draw[thick,dotted,red] (0,0) -- (0.5,0);
\end{tikzpicture}} +e_{
\begin{tikzpicture}
  \draw[thick,dashed,red] (0,0) -- (0.5,0);
\end{tikzpicture}}}{2} =
\frac{e_{d\:'} + e_{d\:''}}{2} =
$$
Это значит, что каждая из них кривая мгновенного напржения у которого нет $\gamma$.
$$
=\frac{e_{d\:'}(\alpha\:' = \alpha, \gamma\:' =0) +
e_{d\:''}(\alpha\:'' = \alpha + \gamma, \gamma\:''=0)}{2} =
$$

\begin{tikzpicture}
  \draw[thick,dashed,red] (0,0) -- (0.5,0);
\end{tikzpicture} -- при угле $\alpha\:'' = \alpha + \gamma$

$$
E_d(\alpha,\gamma) = \frac{E_{d'}(\alpha,\gamma=0) +
  E_{d''}(\alpha'' = \alpha + \gamma, \gamma'' = 0)
}{2} 
$$

Для чего это делаем. Нужно проинтегрировать на интервале повторяемости

$$
E_d = \frac{E_{d0} cos \alpha' + E_{d0} cos \alpha''}{2} =
E_{d0}\frac{cos \alpha + cos(\alpha + \gamma)}{2}
$$

$$
  E_d = E_{d0}\frac{cos \alpha + cos(\alpha + \gamma)}{2}
$$

Проанализируем результат.
На каждом интервале повторяемости теряем эту площадку
\begin{tikzpicture}
  \begin{scope}[xscale=2,yscale=3]
      \newcommand{\alphaa}{26.0 * pi / 180}
    \newcommand{\gammaa}{21.0 * pi / 180}
          %hatch
          \draw[color=red,domain=pi/3 + \alphaa:pi/3 + \alphaa+\gammaa,
            pattern=north east lines,pattern color=red]
          ({pi/3+\alphaa},  {cos((\alphaa +pi/3-2/3*pi) r)}) --
          % (A+B)/2
          plot (\x,{cos(\x r)/2 + cos((\x - 2/3*pi) r)/2})
          %B
          plot (\x,{cos((\x - 2/3*pi) r)})
          -| ({pi/3 + \alphaa+\gammaa},{cos((pi/3 + \alphaa+\gammaa) r)/2
            + cos((pi/3 + \alphaa+\gammaa - 2/3*pi) r)/2});          
\end{scope}
\end{tikzpicture} -- ${\scriptstyle \Delta}E_d$ -- разница между $e_{d'}$
и $e_{d''}$

\begin{equation}
  E_d = E_{d0}\frac{cos \alpha + cos(\alpha + \gamma)}{2}
  \end{equation}

$$
{\scriptstyle \Delta}E_d = \underbrace{E_d(\alpha,\gamma=0)}_
{\textcyrillic{без коммутации}} - \underbrace{E_d(\alpha,\gamma\ne 0)}_
{\textcyrillic{при  реальном }\gamma} 
$$

Смотрим внимательно на уравнение (\ref{three}) где учтено ${\scriptstyle \Delta}E_d$.
Отметим $X_\phi$ -- падение на индуктивности. Кажущаяся нелепость:
Постоянный ток умножается на индуктивность. Это есть ЭДС самоиндукции
$\frac{m}{2\pi}X_\phi I_d$ -- коммутационное.
В учебниках пишут ${\scriptstyle \Delta}U$, у нас написано $E$, подчеркивая
что природа этого -- падение на самоиндукции.

\begin{equation}
{\scriptstyle \Delta}U_{d\gamma} = {\scriptstyle \Delta}E_{d\gamma} =
E_{d0}\frac{cos\alpha -cos(\alpha + \gamma)}{2} =\frac{m}{2\pi}X_\phi I_d
\label{stressed}
\end{equation}

Коэффициэнт $R_{K} ={\displaystyle \frac{m}{2\pi}X_\phi}$ мы умножаем на ток
нулевой частоты. Это фикция, но так говорят, мы упоминаем шуткаи ради.
Но это сопротивление не греется от проходящего тока, поэтому оно фиктивное.

Из предыдущего (\ref{stressed}) уравнения можем найти $\gamma$

\begin{equation}
  \gamma = arccos\left[cos \alpha -\left(\frac{\frac{\displaystyle m}
      {\displaystyle \pi}X_\phi I_d}{E_{d0}}\right)\right] - \alpha  
\end{equation}

В уравнении (\ref{stressed}) ${\scriptstyle \Delta}U_{d\gamma}$ индекс $\gamma$
подчёркивает, что это падение напряжения вследствие угла коммутации $\gamma$

\begin{equation}
  \gamma = arccos\left[cos \alpha -2\left(\frac{{\scriptstyle \Delta}U_{d\gamma}}
         {E_{d0}}\right)\right] - \alpha  
\end{equation}

\begin{equation}
  \gamma = arccos\left[cos \alpha -\left(2\frac{R_K I_d}
         {E_{d0}}\right)\right] - \alpha  
\end{equation}

где
\begin{equation}
  R_K = \frac{m}{2\pi}X_\phi
\end{equation}

Еще раз перепишем формулу (\ref{three}):
$$
U_d = E_d - U_0 - I_d\left[R_K +R_{\textcyrillic{эквивалентное}}\right]
$$
При практических расчётах приближённно $U_0$ мало, также пренебрегают
влиянием $\gamma$ на коммутационное сопротивление.
$$
\hspace{-2 cm}
U_d = E_{d0} \;cos\;\alpha - I_d\left(R_K +
\underbrace{r_\phi}_{
  \begin{array}{c}
    \textcyrillic{на стороне переменного} \\
      \textcyrillic{тока выпрямителя}
    \end{array}
} +
\underbrace{R_\Phi}_{
    \begin{array}{c}
      \textcyrillic{на стороне постоянного}\\
      \textcyrillic{тока выпрямителя}
      \end{array}
}
\right)
$$
Эту формулу применяют для практических расчётов.

$$
R_K =\frac{m}{2\pi} X_\phi -\left(r_\phi + R_D\right) \frac{\gamma}{4\pi}
$$
-- эта формула для совсем точных расчётов.
Активное сопротивление на постоянном токе уменьшается.

Уравнение (\ref{three}) -- статические характеристики. В статическом режиме
$\alpha$ и $I$ не меняются. То что присутствует $\gamma$ -- его нужно
исключить, решив уравнение относительно $\gamma$

$
U_d = F(\alpha,I_d) 
$ -- можно рассматривать уравнение как функцию двух переменных.

$U_d = f(\alpha)$ при $I_d = const$ -- {\bf регулировочные характеристики}. Почему
во множественном числе? потому что для разных $I_d$.

$U_d = f(I_d)$ при $\alpha=const$ -- {\bf внешние характеристики}.

Как их строят? Строят семейство регулировочных характеристик и семейство
внешних характеристик.

При $I_d$ малых (пренебрегаем падением на внутренних элементах). Это одна
из регулировочных характеристик, причём основная, а внешних -- много.

\subsection{регулировочные характеристики}
характеристики строятся в относительных единицах.



\begin{tikzpicture}
  \begin{scope}[xscale=3.5,yscale=4.5]
    \newcommand{\alphaa}{26.0 * pi / 180}
    \newcommand{\gammaa}{21.0 * pi / 180}

    \draw[thin, ->] (0, 0) -- (3.5,0) node[right] {$\alpha$};
    \draw[thin, ->] (0,-1.2) -- (0,1.3) node[left] {$\frac{U_d}{E_{d0}}$};
    \draw[thin,loosely dashed] (0,-1) -- (pi,-1);
    
    \foreach \x/\xtext in {{pi/6}/{\frac{\pi}{6}}, {pi/3}/{\frac{\pi}{3}},
      {pi/2}/{\frac{\pi}{2}}, {2*pi/3}/\frac{2\pi}{3},
      {5*pi/6}/\frac{5\pi}{6},
      {pi}/{\pi}}
    \draw (\x,0.1) -- (\x,-0.1) node [below] {$\xtext$};

    \foreach \y/\ytext in {-1/-1,-0.5/-0.5,0.5/0.5,1,1}
    \draw (0.1,\y) -- (-0.1,\y) node [left] {$\ytext$};

    % A
    \draw[domain=0:pi, help lines, smooth]
    plot (\x,{cos((\x) r)});
    % нижняя граница
    \draw[domain=0:{pi-(23/180)*pi}, help lines, smooth,dashed]
    plot (\x,{cos((\x) r) - 0.08});
    %
    \draw[thin,<->] ({pi/4}, {cos((pi/4) r)}) --
    ({pi/4}, {cos((pi/4) r) - 0.08});
    \draw[thin] ({pi/4}, {cos((pi/4) r) - 0.04}) --
    ({pi/4+0.3}, {cos((pi/4) r)}) node[right]
         {${\displaystyle \frac{{\scriptstyle \Delta}U_d}{E_{d0}}}$};
    %
    \node[below] at ({pi/6},0.45)
         {$\begin{array}{c}
             \textcyrillic{выпрямительный}\\
             \textcyrillic{режим}
           \end{array}$
         };
    \node[below] at ({5*pi/6},-0.3)
         {$\begin{array}{c}
             \textcyrillic{инверторный}\\
             \textcyrillic{режим}
           \end{array}$
         };

         \node[below] at (2*pi/3,-0.85) {${\scriptstyle \Delta}U_d>0$};
         \draw[thin,->] (2*pi/3,-0.85) -- (2*pi/3+0.25,-0.8);
         \draw[thin,<-] (5*pi/6-0.1,-0.8) -- (5*pi/6+0.4,-0.7)
         node[right]
         {$\begin{array}{c}
             {\displaystyle \frac{E_d}{E_{d0}} {\textcyrillic{-- всё падение}}}\\
             {\textcyrillic{напряжения}}\approx 0\\
             {\textcyrillic{пренебрегаем всем}}
             \end{array}$};
  \end{scope}
\end{tikzpicture}

$\alpha\ge 0$, потому что не сможем включить раньше. Какой теоретический
предел $\alpha$
\begin{tikzpicture}
  \begin{scope}[xscale=1.5,yscale=3]
    \newcommand{\alphaa}{26.0 * pi / 180}
    \newcommand{\gammaa}{21.0 * pi / 180}
    \draw[thin, ->] (0, 0) -- (pi+pi/6+pi/3,0) node[right] {$\omega t$};
    \draw[thin, ->] (0,0) -- (0,1.3) node[left] {$U$};
    
    \foreach \x/\xtext in {{pi/6}/{\frac{\pi}{6}},
      {pi/3}/{\frac{\pi}{3}}, {pi}/{\pi},{4*pi/3}/{\frac{4\pi}{3}}}
    \draw (\x,0.1) -- (\x,-0.1) node [below] {$\xtext$};

    % A,B,C
    \draw[domain=-0.1:{pi/6+pi+pi/3}, smooth, yellow]
    plot (\x,{cos((\x) r)}); \node[above] at (0.2,1) {$e_k$};
    \draw[domain=-0.1:{pi/6+pi+pi/3}, help lines, smooth, green]
    plot (\x,{cos((\x-2/3*pi) r)}); \node[above] at (2*pi/3+0.2,1) {$e_{k+1}$};

    \draw[thin] ({pi/3},-0.35) -- (pi/3,-1.27);
    \draw[thin] ({4*pi/3},{cos((4*pi/3) r)-0.1}) --  ({4*pi/3},-1.27);
    \draw[thin,<->] ({pi/3},-1.15) -- ({4*pi/3},-1.15);
    \node[below] at ({(pi/3+4*pi/3)/2},-1.25)
         {теоретический диапазон $\alpha=\pi$};
  \end{scope}
\end{tikzpicture}

Обычно строят графики $I_d \approx 0$ и $I_d\approx I_{\textcyrillic{номинальный}}$
(близко к номиналу). Падение
\begin{tikzpicture}
  \begin{scope}
     \draw[thin,<->] (0, 0.25) --
    (0, -0.25);
    \draw[thin,<-] (0.1, 0) --   (0.5, 0) node[right]
         {${\displaystyle \frac{{\scriptstyle \Delta}U_d}{E_{d0}}}$};
  \end{scope}
  \end{tikzpicture}
-- единицы процента, до $10\%$. $30\%$ быть не может, поскольку, 15\% падение
на реактивных и 15\% падение на активных сопротивлениях, а 15\% потерь
недопустимо для КПД выпрямителей. У выпрямителях КПД $\approx$ 95\%.

Выпрямленное напряжение становится меньше 0, как это понять? $I>0$ всегда.
Это не диод, $\alpha \ne 0$, это тиристор. $I\cdot U<0$, значит преобразователь
преобразует в обратном направлении. И КПД $\ne 95\%$, потом поговорим об этом
подробнее.

Когда $U \approx 0$, напряжение примерно 10\% от номинала, КПД уменьшается не
потому что увеличиваются потери, а потому что уменьшается мощность.

$P_d = U_dI_d <0$ -- инверторный режим. Инверторный режим преобразователя обычно имеет место когда $E_d<0$ и $\alpha>90^\circ$.

%\hspace{-3 cm}
\begin{tikzpicture}
  \begin{scope}[xscale=2,yscale=3]
    \newcommand{\alphaa}{106.0 * pi / 180}
    \newcommand{\gammaa}{21.0 * pi / 180}

    \draw[thin, ->] (-0.2, 0) -- (5.8,0) node[right] {$\omega t$};
    \draw[thin, ->] (0,0) -- (0,1.3) node[left] {$U$};
    
    \foreach \x/\xtext in {0, {pi/3}/{\frac{\pi}{m}},{pi}/\pi}
    \draw (\x,0.1) -- (\x,-0.1) node [below] {$\xtext$};

    % A,B,C
    \draw[domain=-0.2:5.8, smooth, yellow]
    plot (\x,{cos((\x) r)}); \node[above] at (0.2,1) {$e_k$};
    \draw[domain=-0.2:5.8, help lines, smooth, green]
    plot (\x,{cos((\x-2/3*pi) r)}); \node[above] at (2*pi/3+0.2,1) {$e_{k+1}$};
    \draw[domain=-0.2:5.8, help lines, smooth]
    plot (\x,{cos((\x-4/3*pi) r)}); \node[above] at (4*pi/3+0.2,1) {$e_{k-1}$};
    % (A+B)/2 
    \draw[domain={pi/3:pi/3+pi}, loosely dotted]
    plot (\x,{cos(\x r)/2 + cos((\x - 2/3*pi) r)/2});


    
    \foreach \qq [evaluate=\qq as \qqshft using \qq*2*pi/3] in {0,...,1}
     {
      \begin{scope}[xshift=\qqshft cm,
          every path/.style={color=red}]
        
        %Ed
%        \draw[domain={-pi/3 + \alphaa + \gammaa}:{pi/3},ultra thick]
%        plot (\x, {cos(\x r)});
        \draw[domain={\alphaa+\gammaa}:{\alphaa + pi/3},ultra thick]
        plot (\x,{cos(\x r)})
        -| (\alphaa+pi/3, {cos((\alphaa + pi/3) r)/2 + cos((\alphaa-pi/3) r)/2})
        [domain={\alphaa+pi/3}:{\alphaa+pi/3+\gammaa}]
        plot (\x,{cos(\x r)/2 + cos((\x-2/3*pi) r)/2})
        -| (\alphaa+pi/3+\gammaa, {cos((\alphaa +pi/3+ \gammaa-2*pi/3) r) })

        [domain={\alphaa+\gammaa-pi+2*pi/3}:{\alphaa-pi/3+2*pi/3}]
          plot(\x,  {cos((\x) r)})
        ;        
       \end{scope}
     }

     \node[below] at (pi+0.3, -0.6) {${\displaystyle \frac{e_k + e_{k+1}}{2}}$};
     \draw[thin,->] (pi+0.2,-0.6) --
     (pi+0.3,{cos((pi+0.3) r)/2 + cos((pi-2*pi/3+0.3) r)/2 -0.1});
     % alpha
     \draw[thin] (pi/3, 0.35) -- (pi/3,0.8);
     \draw[thin] (pi/3+\alphaa,0.1) -- (pi/3+\alphaa, 0.8);
     \draw[thin,->] (pi/3, 0.75) -- (pi/3+\alphaa,0.75);
     \node[left] at (pi/3, 0.75) {$\alpha$};
     % beta
     \draw[thin] (4*pi/3,-0.4) -- (4*pi/3,0.8);
     \draw[thin,<->] (pi/3+\alphaa, 0.6)--(4*pi/3, 0.6) node[right]
           {$\beta=\pi-\alpha$};
     % U_инвертора
     \draw[thin,color=red] (pi/2,-0.36) -- (5*pi/6+pi,-0.36) node[right]
     {$U_{\textcyrillic{инв}}$};

  \end{scope}
  \end{tikzpicture}    

За счет чего течёт ток?
\begin{figure}[H]
\begin{circuitikz}\draw
(0,4) to[L] (0,2)
node[left] at (0,3.5) {-}
node[left] at (0,2.5) {+}
node[left] at (0,2) {${\textcyrillic{было}}$}
node[right] at (0,3.5) {(+)}
node[right] at (0,2.5) {(-)}
node[right] at (0,2) {${\textcyrillic{стало}}$}
to[Ty,mirror] (0,0)
(2,4) to[L,*-] (2,2)
to[Ty,mirror] (2,0)
(4,4) to[L,*-] (4,2)
to[Ty,mirror] (4,0)
(0,4) to[short,-*] (2,4)
to[short,-*] (4,4)
to[short] (8,4)
(0,0) to[short,-*] (2,0)
to[short,-*] (4,0)
to[short] (8,0)
to (8,1)
to[american voltage source] (8,3)
to[short] (8,4)
node[right] at (8,2.5) {(+)}
node[right] at (8,1.5) {(-)}
node[right] at (8.2,2) {$E_{\textcyrillic{н}}{\textcyrillic{ЭДС нагрузки}}$}
node[right] at (8,0.5) {$
\begin{array}{c}
{\textcyrillic{все напряжения}}\\
{\textcyrillic{отсчитываем}}\\
{\textcyrillic{относительно}}\\
{\textcyrillic{нулевого провода}}
\end{array}
$}
;\end{circuitikz}
\end{figure}
Все напряжения отсчитываются относительно нулевого провода и поэтому
 ток течёт за счёт ЭДС нагрузки!

$\alpha$ двигаем дальше, от этого положительная часть уменьшается,
а отрицательная увеличивается. Если энергия течёт в сеть, значит
\begin{circuitikz}\draw
node[right] at (0,0) {$e_{\textcyrillic{н}}$}
node[left] at (0,0.2) {+}
node[left] at (0,-0.2) {-}
;\end{circuitikz}.

Теперь рисуем картинку при $\alpha>90^\circ$ Введем угол $\beta$:
$$
\beta = \pi - \alpha
$$

$\alpha$ -- угол запаздывания, $\beta$ -- угол опережения.
Рисуем ток.
\begin{figure}[H]
\begin{tikzpicture}
  \begin{scope}[xscale=2,yscale=3]
      \newcommand{\alphaa}{106.0 * pi / 180}
          \newcommand{\gammaa}{21.0 * pi / 180}

    \draw[thin, ->] (-0.2, 0) -- (5.8,0) node[right] {$\omega t$};

    \draw[domain=-0.2:1.8, smooth, yellow]
        plot (\x,0)
    [domain=1.8:2.2, smooth, yellow]
         plot (\x, {1/0.4*(\x-1.8)})
    [domain=2.2:3.8, smooth, yellow]
         plot (\x,1)
    [domain=3.8:4.2, smooth, yellow]
         plot (\x, {-1/0.4*(\x-4.2)})
    [domain=4.2:5.8, smooth, yellow]
         plot (\x,0);
    \draw[domain=-0.2:1.8, help lines, smooth, green]
         plot (\x,1)
    [domain=1.8:2.2, help lines, smooth, green]
        plot (\x,{-1/0.4*(\x-2.2)})
    [domain=2.2:3.8, help lines, smooth, green]
         plot (\x,0)
    [domain=3.8:4.2, help lines, smooth, green]
        plot (\x, {1/0.4*(\x-3.8)})
    [domain=4.2:5.8, help lines, smooth, green]
         plot (\x,1);
    %
    \draw[domain=1.8:3.6, smooth, dashed, yellow]
    plot (\x, {cos(1.4*(\x - 2.7) r) - cos(1.4*(1.8 - 2.7) r)});
    \draw[domain=1.8:3.6, help lines, smooth, loosely dashed,green]
    plot (\x, {-cos(1.4*(\x - 2.7) r) + cos(1.4*(1.8 - 2.7) r)+1});
    %
    \draw (2.7,0.1) -- (2.7,-0.1) node[below] {$\pi$};
\end{scope}
\end{tikzpicture}        
\end{figure}

$\beta>0$ Если не выключить $k$-й вентиль до точки пересечения, то он уже
никогда не выключится.
Если $L$ большая, коммутация не кончится. Темп изменения тока пропорционален
ЭДС. Но ЭДС, а значит и скорость нарастания тока уменьшаются, а дальше
скорость станет отрицательной, а значит, коммутация не будет продолжаться.

Коммутация вентилей должна закончиться до момента отсчета угла $\beta$.
В противном случае переключения фаз не произойдёт. $\beta$ не просто
больше 0, а $\beta>\gamma$. Предположим, что разность $\beta-\gamma$ мала.
Сравним её со временем выключения вентиля. Вентиль может самопроизвольно
включиться $I\approx n$ носителей. $I$ убывает, но рекомбинации носителей
не произошло(вентиль не восстановился).
$\beta-\gamma>{\textcyrillic{времени запирания тиристора}}$.
$\omega t_{\textcyrillic{выкл}}= \sigma$ -- угол запирания или выключения вентиля.

$\beta>\gamma+\sigma$ -- каждый раз условие увеличивается.

Практически, учитывается наличие несимметрии напряжения сети, несимметрии
углов регулирования $\alpha$ (несимметрия СИФУ), с учётом несинусоидальности
и разброса параметров тиристора необходим запас по углу для устойчивой
работы инвертора ($\psi$ -- угол запаса)

\begin{equation}
\beta \ge (\gamma+\beta+\psi) = \beta_{min}
\label{steady_invertor}
\end{equation} -- условие устойчивой работы инвертора.
отсюда, $\alpha_{max} = \pi - \beta_{min}$


\begin{tikzpicture}
  \begin{scope}[xscale=1.5,yscale=1.5]
      \newcommand{\alphaa}{106.0 * pi / 180}
      \newcommand{\gammaa}{21.0 * pi / 180}
    % A,B,C
        \draw[domain=0.1:1.4, help lines, smooth, black]
            plot (\x,{cos((\x) r)});
        \draw[domain=0.1:1.4, help lines, smooth, green]
            plot (\x,{cos((\x-2/3*pi) r)});
        \draw[ultra thick] (1.6,0.6) -- (2,0.6);
        \draw[ultra thick] (1.6,0.4) -- (2,0.4);
        \draw[ultra thick] (1.9,0.7) -- (2.1,0.5);
        \draw[ultra thick](1.9,0.3) -- (2.1,0.5);
        \draw[domain=0.1:1.4, help lines, smooth, dashed, black]
                    plot ({2+\x},{cos((\x) r)});
       \draw[domain=0.1:1.4, help lines, smooth, black]
                           plot ({1.8+\x},{cos((\x) r)});                           
        \draw[domain=0.1:1.4, help lines, smooth, green]
                    plot ({2+\x},{cos((\x-2/3*pi) r)});
                                        
\end{scope}
\end{tikzpicture}
Если сместилась фаза, $\beta$ тоже сместилась. Несимметрия СИФУ - средний угол
18,22. Там где не хватит $\beta$.
Несинусоидальность фазы - опять приводит к необходимости увеличить $\beta$.
От температуры, тока, который был, напряжение приложенное.
${\displaystyle \Delta}_\alpha + {\displaystyle \Delta}\phi$ -- несимметрия
сети + $ {\displaystyle \Delta}\phi$ --несинусоидальность +
 ${\displaystyle \Delta}$ (разброс на углы включения).
 Эмпирически $\beta_{min} = 15...30^\circ$, $\alpha_{max}=150,165^\circ$
К чему приведет невыполнение (\ref{steady_invertor})? Невыполнение (\ref{steady_invertor}) может привести к ``опрокидыванию'' инвертора: вместо инвертора получим
выпрямительный режим -- аварийный режим, связанный с переходом преобразователя
в выпрямительный режим с резким возрастанием выпрямленного тока. Выпрямленный ток
может возрастать до значений, близких к току К.З, Иногда называют током
``двойного'' К.З., т.е. к К.З. одновременно и преобразователя и источника.


\begin{figure}[H]
\begin{circuitikz}\draw
(0,4) to[L] (0,2)
node[left] at (0,3.5) {-}
node[left] at (0,3) {$U_d$}
node[left] at (0,2.5) {+}
node[left] at (0,2) {${\textcyrillic{выпрямленное}}$}
%node[right] at (0,3.5) {(+)}
%node[right] at (0,2.5) {(-)}
%node[right] at (0,2) {${\textcyrillic{стало}}$}
to[Ty,mirror] (0,0)
(1,4) to[L,*-] (1,2)
to[Ty,mirror] (1,0)
(2,4) to[L,*-] (2,2)
to[Ty,mirror] (2,0)
(0,4) to[short,-*] (1,4)
to[short,-*] (2,4)
to[short] (8,4)
(0,0) to[short,-*] (1,0)
to[short,-*] (2,0)
to[short] (8,0)
to (8,1)
to[american voltage source] (8,3)
to[short] (8,4)
%
(3,3.9) to[short,v^=$U_{\textcyrillic{инв}}$] (3,0.1)
(2.9,0.2) -- ++ (0.1,-0.1)
-- ++ (0.1,0.1)
(2.9,3.8) -- ++ (0.1,0.1)
-- ++ (0.1,-0.1)
%
node[right] at (8,2.5) {+}
node[right] at (8,1.5) {-}
node[right] at (8.2,2) {$E_{\textcyrillic{н, противо-ЭДС}}$} 
node[right] at (8,0.5) {}
% I
(6,3.8) to[short,i^>={$I_d$}] (5,3.8)
;\end{circuitikz}
\end{figure}

$$
I_d = \frac{\overbrace{E_{\textcyrillic{н}}}^{\textcyrillic{согласно с током}} -
\overbrace{\mid E_d\mid}^{\textcyrillic{само отрицательно}}
}{R_{\textcyrillic{н}} +R_{\textcyrillic{эквив}}}
$$
Была разность ЭДС, а станет сумма. $U_{\textcyrillic{инв}}$ -- аккумулятор,
выпрямитель, солнечная батарея. $R$ -- маленькая, $R_{\textcyrillic{н}} - 5\%$
и если напряжение не в плюсе а в минусе, то в 19 раз вырастет ток. Получаем
КЗ и для инвертора и для нагруки. Не двойной ток, а ``удвоение'' явления.
А ток $I \approx I_{\textcyrillic{К.З.}}/2$.

Опрокидывание инвертора приводит к аварийному отключению преобразователя. А есть и электронные средства.

Инверторный режим принципиально менее надежен чем выпрямительный режим.

Опрокидывание инвертора может происходить в случае
\begin{itemize}
\item кратковременного исчезновения или резкого уменьшения напряжения питающей сети;
\item в случае пропуска (даже одиночного) управляющего импульса.
\item в случае ложного несвоевременного срабатывания (даже одиночного) какого-либо
вентиля.
\end{itemize}

4 причины: 1) -- невыполнение условий.
пропустили импульс -- пропал контакт.
Ложное отпирание может произойти вследствие сбоя СИФУ -- ток растет лавиной.

Где используется инверторный режим.

Солнечные батареи $-\;\leftarrow;=$. Где инверторы крайне необходимы.
Гидрогенераторы мощностью 100МВт возбуждаются с помощью 10МВт.
Для управления генераторами а обмотке возбуждения гигантская магнитная энергия,
её и нужно передать в сеть. Для включения-выключения генератора требуется форсировка в 5-8-14 раз. На синхронных компенсаторах
в 10-13 раз, 1300В вместо 100в. И такая же скорость снижения должна быть.

Отключение линий.
Разгон -- выпрямительный режим, Самое экономное торможение -- рекуперация,инверторный режим.

\subsection{Прерывистый режим работы преобразователя}
Режим прерывистого выпрямленного тока. Ток нагрузки, правильный ток $I_d$. А если
он(ток) $I_d$ начинает прерываться -- это прерывистый режим.
Пульсация большая. 1й случай -- это нонсенс. А вторая ситуация, когда ток маленький
и ток сравним с пульсацией.

$$
\frac{E_d - E_{\textcyrillic{н}}}{R_\Sigma} = I_d 
$$
В этой формуле $I_d$ -- постоянный, варьируется за счет $E_d - E_{\textcyrillic{н}}$.

$$
e_d = E_d + \left(e_d\right) =const
$$
$\alpha$ не меняется.
$$
\frac{\left(e_d\right)}{{\textcyrillic{переменная}} ``Z''} = const
$$
а ток разложить в ряд Фурье.

Двигатель в холостом режиме $E_{\textcyrillic{двиг}} \approx
E_{\textcyrillic{источника}}$

\begin{tikzpicture}
  \begin{scope}[xscale=1,yscale=1]
  \newcommand{\alphaa}{106.0 * pi / 180}
  \newcommand{\gammaa}{21.0 * pi / 180}
  \draw[thin, ->] (-0.2, 0) -- (9,0) node[right] {$\omega t$};
  \draw[thin,dashed] (-0.2, 2) -- (3.5,2);
  \draw[thin] (0,2) -- (0.5,2.5) -- (1.5,1.5) -- (2.5,2.5) -- (3.5,1.5);
% A,B,C
  % =>
  \draw[ultra thick] (3.8,1.1) -- (4.5,1.1);
  \draw[ultra thick] (3.8,0.9) -- (4.5,0.9);
  \draw[ultra thick] (4.4,1.2) -- (4.6,1.0);
  \draw[ultra thick](4.4,0.8) -- (4.6,1.0);

 \draw[thin,dashed] (5, 0.2) -- (8.5,0.2);
 \draw[thin,red] (5,0.2) -- (5.5,0.7) -- (6.2,0) -- %(6.5,-0.3) -- (7.5,0.7)
 (6.8,0) -- (7.5,0.7)-- (8.2,0) -- (8.5,0);


\end{scope}
\end{tikzpicture}

Пересечь ``нуль'' не может, потому что тиристоры не проводят ток в обратном направлении. Такая
ситуация бывает в при работе двигателя режиме Х.Х. Ток плохой(маленький), но управлять двигателем
нужно и в этом режиме. Для установок,например, для того чтобы вывести двигатель в нужную точку
(позиционирование х,у), это может иметь важное значение.

\hspace{-3 cm}
\begin{tikzpicture}
  \begin{scope}[xscale=2,yscale=3]
    \newcommand{\alphaa}{106.0 * pi / 180}
    \newcommand{\gammaa}{21.0 * pi / 180}
    \newcommand{\En}{0.37}

    \draw[thin, ->] (-0.2, 0) -- (5.8,0) node[right] {$\omega t$};
    \draw[thin, ->] (0,0) -- (0,1.3) node[left] {$U$};

    \draw[thin, ->] (-0.2, -2) -- (5.8,-2) node[right] {$\omega t$};
    \draw[thin, ->] (0,-2) -- (0,-1) node[left] {$I$};


    \foreach \x/\xtext in {0, {pi/3}/{\frac{\pi}{m}},{pi}/\pi}
    \draw (\x,0.1) -- (\x,-0.1) node [below] {$\xtext$};

    % A,B,C
    \draw[domain=-0.2:5.8, smooth, yellow]
    plot (\x,{cos((\x) r)}); \node[above] at (0.2,1) {$e_{k-1}$};
    \draw[domain=-0.2:5.8, help lines, smooth, green]
    plot (\x,{cos((\x-2/3*pi) r)}); \node[above] at (2*pi/3+0.2,1) {$e_k$};
    \draw[domain=-0.2:5.8, help lines, smooth]
    plot (\x,{cos((\x-4/3*pi) r)}); \node[above] at (4*pi/3+0.2,1) {$e_{k+1}$};

    % E_н
    \draw[thin] (-0.2,\En) -- (5.8,\En);
    \draw[thin,<->] (5.6,\En)--(5.6,0)node[midway,right]
    {$E_\textcyrillic{н}$};     

    \foreach \qq [evaluate=\qq as \qqshft using \qq*2*pi/3] in {0,...,1}
     {
      \begin{scope}[xshift=\qqshft cm,
          every path/.style={color=red}]
        
        %Ed
        \draw[domain={pi/3-0.2}:{pi/3 + 0.4},ultra thick]
        plot (\x, {cos(\x r)})
        -| ({pi/3+0.4}, \En);
        \draw[domain={pi/3 + 0.4}:{pi-0.2},ultra thick]
        plot (\x, \En)
        -| ({pi-0.2},{cos((pi-0.2-2/3*pi) r)});

        %Id
        \draw[domain={pi/3-0.2}:{pi/3 + 0.4},ultra thick]
plot(\x, {-2 + (pi/0.6*cos((\x-pi/3-0.1) r) - pi/0.6*cos((0.3) r) });
%plot(\x, {-3*(\x-pi/3+0.7)*(\x-pi/3+0.2)*(\x-pi/3-0.4)-2});
        \draw[domain={pi/3 + 0.4}:{pi-0.2},ultra thick]
        plot (\x, -2);
     \end{scope}

      % подписи под Id
      \draw[thin] (pi/3-0.2,-2)-- (pi/3-0.2,-2-0.12);
       \draw[thin] (pi-0.2,-2)-- (pi-0.2,-2-0.12);
        \draw[thin,<->] (pi/3-0.2,-2.1)--(pi-0.2,-2.1)
        node[midway,below] {$2\pi/m$};
        %alpha
        \draw[thin] (pi/3, 0.6) -- (pi/3, 1);
        \draw[thin] (pi/3-0.1, 0.7) -- (pi/3+0.1, 0.9)
          node[above right] {$\alpha$};
        \draw[thin] (pi-0.2, 0.7) -- (pi-0.2, 1);
%        \draw[thin,<-] (pi-0.2, 0.9) -- (pi+0.1, 0.9) node[right] {$\alpha$};
        \draw[thin,-latex] (pi/3,0.8) -- (pi-0.2, 0.8);
        % \lamdba - угол проводимости
        \draw[thin] (pi-0.2, \En-0.1) -- (pi-0.2, -0.5);
        \draw[thin] (pi+0.4, 0.1) -- (pi+0.4, -0.5);
         \draw[thin,<->] (pi-0.2, -0.4) -- (pi+0.4, -0.4) node[right]
         {$\lambda{\textcyrillic{ -- угол проводимости}}$};
         \node[right] at (pi/3,-1.1) {$
         \begin{array}{c}         
         {\textcyrillic{производная}} \\
         {\textcyrillic{пропорциональна}} \\
         {\textcyrillic{разности напряжений}}
         \end{array}$};
         \draw[thin,->] (pi-0.3,-1.3) -- (pi-0.1,-1.8);
          \draw[thin,->] (2*pi/3,-0.9) -- (pi-0.25, 0.5);
          %
          \draw[thin,dashed] (pi+0.14,-1.75)--(pi+0.14,0.38);
          \draw[thin,<-] (pi+0.1,-2.05) -- (pi+0.1,-2.2) node[right]
          {$\textcyrillic{максимум немного левее из-за } i\cdot r_\phi$};

          %заштрихованная площадка
          \draw[thin,domain={pi-0.2}:{pi + 0.14},pattern=north west lines,
          pattern color=blue]
          (pi-0.2,\En)--
          plot (\x,{cos((\x-2/3*pi) r)});
          \draw[thin,blue] (pi,\En +0.2) --(pi+0.5,\En +1.2)
          node[above] {
$\begin{array}{c}
\textcyrillic{заштрихованная площадка }\sim
          \textcyrillic{ току}\\
          e_\phi -E\textcyrillic{н} -\xcancel{i\cdot r_\pi = L
          \frac{\partial i}{\partial t}}
          \end{array}$};
          
   }
\end{scope}
\end{tikzpicture}
Если бы не было $i\cdot r_\phi$, то максимум тока был бы где
$U=E_\textcyrillic{н}$ (производная в точке максимума равна 0).

Угол проводимости $\lambda$ -- угол, в течении которого ток больше 0.

Никакого угла коммутации не будет. Ток проводит -- пауза -- проводит
следующий вентиль.

Постоянная ЭДС нагрузки
$$
\underbrace{E_d + (e_d)}_{\textcyrillic{выпрямленное}} -
E_{\textcyrillic{н}} = \underbrace{
L \frac{\partial i_d}{\partial t} +R_\Sigma i_d}_
{\textcyrillic{и нагрузка и преобразователь}}
$$
$U_0$ включили в $E_{\textcyrillic{н}}$. Разность
$E_d + (e_d) -E_{\textcyrillic{н}}$ больше 0, чтобы протекал ток.



\begin{figure}[H]
\begin{circuitikz}\draw
(0,4) to[L] (0,2)
to[Ty,mirror] (0,0)
(1,4) to[L,*-] (1,2)
to[Ty,mirror] (1,0)
(0,4) to[short,-*] (1,4)
to[short] (4,4)
(0,0) to[short,-*] (1,0)
to[short] (4,0)
to[american voltage source] (4,1.5)
to[L] (4,2.75)
to[R] (4,4)
%
node[right] at (4.2,1.3) {-}
node[right] at (4.2,0.2) {+}
node[right] at (4.2,0.75) {$E_{\textcyrillic{н}}
{\textcyrillic{ положительное}}$}
;\end{circuitikz}
\end{figure}
В паузе будет ЭДС нагрузки, все вентили выключены.

$$
U_d = \frac{1}{2\pi/m}\left[
\int\limits_{-\frac{\pi}{m} + \alpha}^{-\frac{\pi}{m} + \alpha + \lambda}
\sqrt{2}E_{2\phi}cos(\omega t) d\omega t +
E_{\textcyrillic{н}} \left(\frac{2\pi}{m} - \lambda \right)
\right] =
$$

В вырaжении для $U_d$ выбор начала отсчета и слагаемые изображены на рисунке


\begin{tikzpicture}
  \begin{scope}[xscale=2,yscale=3]
    \newcommand{\alphaa}{106.0 * pi / 180}
    \newcommand{\gammaa}{21.0 * pi / 180}
    \newcommand{\En}{0.37}

    \draw[thin, ->] (-0.2, 0) -- (5.8,0) node[right] {$\omega t$};
    \draw[thin, ->] ({2/3*pi},0) -- ({2/3*pi},1.3) node[left] {$U$};

    % A,B,C
%    \draw[domain=-0.2:5.8, smooth, yellow]
%    plot (\x,{cos((\x) r)}); \node[above] at (0.2,1) {$e_{k-1}$};
    \draw[domain=-0.2:5.8, help lines, smooth, green]
    plot (\x,{cos((\x-2/3*pi) r)}); \node[above] at (2*pi/3+0.2,1) {$e_k$};
    \draw[domain=-0.2:5.8, help lines, smooth]
    plot (\x,{cos((\x-4/3*pi) r)}); \node[above] at (4*pi/3+0.2,1) {$e_{k+1}$};

    % E_н
    \draw[thin] (-0.2,\En) -- (5.8,\En);
    \draw[thin,<->] (5.6,\En)--(5.6,0)node[midway,right]
    {$E_\textcyrillic{н}$};

      % стрелка к -pi/2
        \draw[thin,latex-] (pi/3,0.7) -- (2*pi/3, 0.7) node[midway, below] {$\pi/2$};
      %alpha
        \draw[thin] (pi/3, 0.6) -- (pi/3, 1);
        \draw[thin] (pi/3-0.1, 0.7) -- (pi/3+0.1, 0.9)
          node[above right] {$\alpha$};
        \draw[thin] (pi-0.5, 0.7) -- (pi-0.5, 1);
        \draw[thin,-latex] (pi/3,0.8) -- (pi-0.5, 0.8);
      % \lamdba - угол проводимости
        \draw[thin] (pi-0.5, \En-0.1) -- (pi-0.5, -0.5);
        \draw[thin] (pi+0.4, 0.1) -- (pi+0.4, -0.5);
         \draw[thin,<->] (pi-0.5, -0.4) -- (pi+0.4, -0.4) node[right]
         {$\lambda{\textcyrillic{ -- угол проводимости}}$};

         %заштрихованная площадка
          \draw[thin,domain={pi-0.5}:{pi + 0.4},pattern=north west lines,
          pattern color=blue]
          (pi-0.5,0)--
          plot (\x,{cos((\x-2/3*pi) r)}) -- (pi+0.4,0);
       
      % начало следующего угла проводимости
         \newcommand{\lambdaNext}{(pi+2/3*pi-0.3)}
         \draw[thin] (pi+2/3*pi-0.5,0) -- (pi+2/3*pi-0.5, {cos(((pi+2/3*pi-0.5) -4/3*pi) r)} ) ;
   
      % штрихованная площадка
        \draw[thin,domain={pi+0.4}:{pi+2/3*pi-0.5},pattern=north east lines, pattern color=gray]
         (pi+0.4,0) -- plot (\x,\En) -- (pi+2/3*pi-0.5,0);

  \end{scope}
\end{tikzpicture}

${\displaystyle \lambda < \frac{2\pi}{m}}$ -- означает, что ток прерывистый.

${\displaystyle \lambda > \frac{2\pi}{m} => \lambda = \frac{2\pi}{m} + \gamma}$
-- режим непрерывного тока.

при ${\displaystyle \lambda = \frac{2\pi}{m}}$ -- граница, должны сливаться оба режима, сливаются на отрезке длиной $\gamma=0$.

$$
= \frac{m}{2\pi} \sqrt{2} E_{2\phi}\left[ sin\left(-\frac{\pi}{m} +
\alpha + \lambda\right) - sin\left(-\frac{\pi}{m} +\alpha\right)
\right]
+ E_{\textcyrillic{н}} \left( 1 - \frac{\lambda m}{2\pi} \right) =
$$

$$
\frac{m}{2\pi} \sqrt{2} E_{2\phi} sin \frac{\lambda}{2}
cos\left( \alpha - \frac{\pi}{m} +\frac{\lambda}{2} \right)
+ E_{\textcyrillic{н}} \left( 1 - \frac{\lambda m}{2\pi}\right) =
$$

$$
U_d = E_{d0} \frac{sin\frac{\lambda}{2}}{sin\frac{\pi}{m}}
cos\left(\alpha-\frac{\pi}{m} + \frac{\lambda}{2}\right) +
E_{\textcyrillic{н}} \left( 1 - \frac{\lambda m}{2\pi} \right)
$$

${\displaystyle \lambda = \frac{2\pi}{m}}$ -- граничный режим $\frac{\pi}{m}?$

Частный случай режима работы управляемого преобразователя
на чисто активную нагрузку.
$$
= \int\limits_{-\frac{\pi}{2} + \alpha}^{\frac{\pi}{m}} \cos\omega t\; d\omega t
$$

а если $\alpha=0$ -- тогда не будет прерывистого режима. Прерывистый режим
будет в случае когда $\alpha>0$

$$
\underbrace{\frac{\pi}{m} + \alpha}_{\begin{array}{c}
{\textcyrillic{момент включения}}\\
{\textcyrillic{следующей фазы}}
\end{array}} > \frac{\pi}{2}
$$

Должно выполнятся неравенство, иначе нет прерывистого режима. 
Прерывистый режим при чисто активной нагрузке изображен на рисунке:

\begin{tikzpicture}
  \begin{scope}[xscale=2,yscale=3]
    \newcommand{\alphaa}{106.0 * pi / 180}
    \newcommand{\gammaa}{21.0 * pi / 180}
    \newcommand{\En}{0.37}

    \draw[thin, ->] (-0.2, 0) -- (5.8,0) node[right] {$\omega t$};
    \draw[thin, ->] ({2/3*pi},0) -- ({2/3*pi},1.3) node[left] {$U$};

    % A,B,C
%    \draw[domain=-0.2:5.8, smooth, yellow]
%    plot (\x,{cos((\x) r)}); \node[above] at (0.2,1) {$e_{k-1}$};
    \draw[domain=-0.2:5.8, help lines, smooth, green]
    plot (\x,{cos((\x-2/3*pi) r)}); \node[above] at (2*pi/3+0.2,1) {$e_k$};
    \draw[domain=-0.2:5.8, help lines, smooth]
    plot (\x,{cos((\x-4/3*pi) r)}); \node[above] at (4*pi/3+0.2,1) {$e_{k+1}$};

      % стрелка к -pi/2
        \draw[thin,latex-] (pi/3,0.7) -- (2*pi/3, 0.7) node[midway, below] {$\pi/2$};

      %alpha
        \draw[thin] (pi/3, 0.6) -- (pi/3, 1);
        \draw[thin] (pi/3-0.1, 0.7) -- (pi/3+0.1, 0.9)
          node[above right] {$\alpha$};
        \draw[thin] (pi-0.5, 0.7) -- (pi-0.5, 1);
        \draw[thin,-latex] (pi/3,0.8) -- (pi-0.5, 0.8);
      % \lamdba - угол проводимости
        \draw[thin] (pi-0.5, \En-0.1) -- (pi-0.5, -0.5);
        \draw[thin] (2/3*pi + pi/2, 0.1) -- (2/3*pi + pi/2, -0.5);
         \draw[thin,<->] (pi-0.5, -0.4) -- (2/3*pi + pi/2, -0.4) node[right]
         {$\lambda{\textcyrillic{ -- угол проводимости}}$};

      % начало следующего угла проводимости
         \newcommand{\lambdaNext}{(pi+2/3*pi-0.3)}
         \draw[red] (pi+2/3*pi-0.5,0) -- (pi+2/3*pi-0.5, {cos(((pi+2/3*pi-0.5) -4/3*pi) r)} ) ;

      \draw[red] (pi-0.5, 0) -- (pi-0.5, {cos((pi-0.5-2/3*pi) r)});
       \draw[red,domain=pi-0.5:2/3*pi+ pi/2] plot (\x,{cos((\x-2/3*pi) r)});
       \draw[red] (2/3*pi+ pi/2,0) -- (pi+2/3*pi-0.5,0);
\end{scope}
\end{tikzpicture}

\begin{equation}
\frac{m}{2\pi}\sqrt{2}E_{2\phi}\left[1 - sin\left(\alpha-\frac{\pi}{m}\right)
\right] =
E_{d0} \frac{1 - sin\left(\alpha-\frac{\pi}{m}\right)}{2sin\frac{\pi}{m}}
\end{equation}

$$
\alpha>\frac{\pi}{2} -\frac{\pi}{m}
$$
\begin{equation}
\frac{\pi}{2} -\frac{\pi}{m} < \alpha < \frac{\pi}{2} +\frac{\pi}{m}
\label{7a}
\end{equation}
для разных m разные неравенства (\ref{7a})

%\begin{tikzpicture}
  \begin{scope}[xscale=3.5,yscale=4.5]
    \newcommand{\alphaa}{26.0 * pi / 180}
    \newcommand{\gammaa}{21.0 * pi / 180}
    \newcommand{\betamax}{165/180*pi}

    \draw[thin, ->] (0, 0) -- (3.5,0) node[right] {$\alpha$};
    \draw[thin, ->] (0,-1.2) -- (0,1.3) node[left]
         {${\displaystyle \frac{U_d}{E_{d0}}}$}
     node[right] {${\displaystyle \approx\frac{E_d}{E_{d0}}}$};
    \draw[thin,loosely dashed] (0,-1) -- (pi,-1);

    \draw (\betamax,-0.1) -- (\betamax,0.1) node [above] {$165^\circ$};
    
    \foreach \x/\xtext in {{pi/6}/{\frac{\pi}{6}}, {pi/3}/{\frac{\pi}{3}},
      {pi/2}/{\frac{\pi}{2}}, {2*pi/3}/\frac{2\pi}{3},
      {5*pi/6}/\frac{5\pi}{6},
      {pi}/{\pi}}
    \draw (\x,0.1) -- (\x,-0.1) node [below] {$\xtext$};

    \foreach \y/\ytext in {-1/-1,-0.5/-0.5,0.5/0.5,1,1}
    \draw (0.1,\y) -- (-0.1,\y) node [left] {$\ytext$};

    % A
    \draw[domain=0:\betamax, help lines, smooth]
    plot (\x,{cos((\x) r)});
    % нижняя граница
    \draw[domain=0:{pi-(23/180)*pi}, help lines, smooth,dashed]
    plot (\x,{cos((\x) r) - 0.08});
    %
    %m=2
    \draw[domain=0:\betamax, help lines, smooth]
    plot (\x, {(1-sin((\x-pi/2) r))/2/sin((pi/2) r)});
    %m=3
    \draw[domain={pi/2-pi/3}:{pi/2+pi/3}, help lines, smooth]
    plot (\x, {(1-sin((\x-pi/3) r))/2/sin((pi/3) r)});
    %m=6
    \draw[domain={pi/2-pi/6}:{pi/2+pi/6}, help lines, smooth]
    plot (\x, {(1-sin((\x-pi/6) r))/2/sin((pi/6) r)});
    %m=12
    \draw[domain={pi/2-pi/12}:{pi/2+pi/12}, help lines, smooth]
    plot (\x, {(1-sin((\x-pi/12) r))/2/sin((pi/12) r)});
    %
    \draw[thin] ({pi/2+0.1}, {(1-sin(((pi/2+0.1)-pi/12) r))/2/sin((pi/12) r)})
    -- (2*pi/3+0.4, 0.4) node[right] {m=12};
    \draw[thin] ({pi/2+0.05}, {(1-sin(((pi/2+0.05)-pi/6) r))/2/sin((pi/6) r)})
    -- (2*pi/3+0.3, 0.47) node[right] {m=6};
    \draw[thin] ({pi/2}, {(1-sin(((pi/2)-pi/3) r))/2/sin((pi/3) r)})
    -- (2*pi/3, 0.54) node[right] {m=3};
    \draw[thin] ({pi/2-0.05}, {(1-sin(((pi/2-0.05)-pi/2) r))/2/sin((pi/2) r)})
    -- (2*pi/3-0.3, 0.68) node[right] {m=2};
    \draw[thin] ({pi/2},0) -- ({(pi/3+pi/2)/2},-0.35) node[below] {$m=\infty$};
    %        
    %
    \node[below] at ({pi/6},0.45)
              {$\begin{array}{c}
                  \textcyrillic{выпрямительный}\\
                  \textcyrillic{режим}
                \end{array}$
              };
%    \node[below] at ({5*pi/6},-0.3)
%                   {$\begin{array}{c}
%                       \textcyrillic{инверторный}\\
%                       \textcyrillic{режим}
%                     \end{array}$
              %                   };
              
      \draw[thin,<-] (5*pi/6-0.1,-0.8) -- (5*pi/6-0.5,-0.9)
      node[left]  {$I_d\approx 0$};
      \draw[thin] (pi,-0.2) -- (pi,-0.4);
      \draw[thin] (\betamax,-0.15) -- (\betamax,-0.4);
      \draw[thin,<->] (\betamax,-0.35) -- (pi,-0.35) node[right] {$\beta_{max}$};
  \end{scope}
\end{tikzpicture}

Если ${\displaystyle \alpha<\frac{\pi}{2} - \frac{\pi}{m}}$ тогда это непрерывный
 режим. При $m=2$ прерывистый режим начинается как только $\alpha>0$.
$$
\begin{array}{cc}  
  m=2 & {\displaystyle E_d =\frac{1+cos\alpha}{2}}\\
  m=3 & {\textcyrillic{до 30}}^\circ{\textcyrillic{ справедлива синусоида, затем }}
  {\displaystyle E_d = 1-sin(\alpha)}
\end{array}
$$

\subsection{Внешние характеристики $\alpha=const$}

\hspace{-1.5cm}
\begin{tikzpicture}
  \begin{scope}[xscale=1.5,yscale=3.5]
    \newcommand{\betamin}{160/180}
    % axis x,y
    \draw[thin, ->] (0, 0) -- (8,0) node[right]
         {${\displaystyle \frac{I_d}{I_{d0}}}$};
    \draw[thin, ->] (0, -1) -- (0,1.2) node[left]
         {${\displaystyle \frac{U_d}{E_{d0}} }$};
         \draw[thin, loosely dashed] (0,-1) -- (8,-1);
    % ylabel
    \foreach \y/\ytext in {-1/-1,{-\betamin}/\beta_{min},0/0,1/1}
    \draw (0.1,\y) -- (-0.1,\y) node[left] {$\ytext$};
    % \beta_min
    \draw[thin,loosely dashed] (0, {-\betamin}) -- (8, {-\betamin+0.1});
    \node at (0.3,-0.4) {$'1-0'$};
    \draw[thin,<-] ({sqrt(0.49*(1-0.8*0.8))} ,-0.8) -- (-0.5,-0.7) node[left]
         {$\begin{array}{c}\textcyrillic{режим}\\'1-1'\end{array}$};
     \draw[thin,<-] (5.5,{3*sqrt(1-5.5*5.5/36)-1}) -- (7,0.4) node[right]
         {$\begin{array}{c}\textcyrillic{режим}\\'2-2'\end{array}$};
     \node at (7,-0.4) {$\begin{array}{c}\textcyrillic{режим}\\'3-2'\end{array}$};
    \node at (1.3,-0.6) {$\begin{array}{c}\textcyrillic{режим}\\'2-1'\end{array}$};
    % eclipse (1-1)
    \draw[domain=0:0.7, help lines,dotted, smooth]
    plot (\x,{sqrt(1-\x*\x/0.49)});
    \draw[domain=0:0.7, help lines,dotted, smooth]
    plot (\x,{-sqrt(1-\x*\x/0.49)});
    %ellipse (2-2)
    \draw[domain=4.75:6, help lines,dotted, smooth]
    plot (\x,{3*sqrt(1-\x*\x/36)-1});
    %
% наклон    \draw[thin] (0,1) -- (6,0.8);
    \draw[thin] (0,1) -- (4.76,1-1/30*4.76);
    % 0.2/6*180/pi
    \draw[domain=4.76:6.79, help lines, smooth]
    plot (\x, {-0.4*\x*\x +2*0.4*4.76*\x + 1 -1/30*4.76 -0.4*4.76*4.76
    });
    
    
    \node[rotate=-4.45] at (3,1) {$\alpha=0^\circ$};    
    \draw[thin] ({sqrt(0.49*(1-0.67*0.67))},{0.67-1/30}) -- (5.2, 0.67-1/30*5.2);
    % рисуем параболу ax^2 + bx + c
    % выбираем a=1
    % из f'(x)=0 = 2ax+b => b=-2 a x0
    % a x0^2 + b x0 + c = y0
    % a x0^2 -2 a x0^2 +c = y0
    % c = y0+a x0^2
    \draw[domain=0:{sqrt(0.49*(1-0.67*0.67))}, help lines, smooth]
    plot (\x, {\x*\x -2*sqrt(0.49*(1-0.67*0.67))*\x + 0.67 +
      0.49*(1-0.67*0.67 ) -0.0335
     });
    \draw[domain=5.2:6.67, help lines, smooth]
    plot (\x, {-0.6*\x*\x +2*0.6*5.2*\x + 0.67 -1/30*5.2 -0.6*5.2*5.2
    });
    
    
    \node[rotate=-4.45] at (3.1,0.67) {$\alpha=30^\circ$};
    \draw[thin] ({sqrt(0.49*(1-0.33*0.33))},{0.33-1/30}) -- (5.54, 0.33-1/30*5.54);
    \draw[domain=0:{sqrt(0.49*(1-0.33*0.33))}, help lines, smooth]
    plot (\x, {1.2*\x*\x -2*1.2*sqrt(0.49*(1-0.33*0.33))*\x + 0.33 +
      1.2*0.49*(1-0.33*0.33 ) -0.0335
    });
    \draw[domain=5.54:6.57, help lines, smooth]
    plot (\x, {-0.9*\x*\x +2*0.9*5.54*\x + 0.33 -1/30*5.54 -0.9*5.54*5.54
    });
    
       
    \node[rotate=-4.45] at (3.2,0.33) {$\alpha=60^\circ$};
    \draw[thin] ( 0.7, {0-1/30})  -- (5.8, -1/30*5.8);
    \draw[domain=0:{sqrt(0.49*(1-0*0))}, help lines, smooth]
    plot (\x, {1.4*\x*\x -2*1.4*sqrt(0.49*(1-0.*0.))*\x + 0. +
      1.4*0.49*(1-0.*0. ) -0.0335
    });
    \draw[domain=5.8:6.46, help lines, smooth]
    plot (\x, {-1.4*\x*\x +2*1.4*5.8*\x  -1/30*5.8 -1.4*5.8*5.8
                      });
    
    \node[rotate=-4.45] at (5.2,-0.1) {$\alpha=90^\circ$};
    \draw[thin] ({sqrt(0.49*(1-0.33*0.33))},{-0.33-1/30}) -- (5.92, -0.33-1/30*5.92);
    \draw[domain=0:{sqrt(0.49*(1-0.33*0.33))}, help lines, smooth]
    plot (\x, {1.8*\x*\x -2*1.8*sqrt(0.49*(1-0.33*0.33))*\x -0.33 +
      1.8*0.49*(1-0.33*0.33 ) -0.0335
          });
    \draw[domain=5.92:6.32, help lines, smooth]
    plot (\x, {-1.8*\x*\x +2*1.8*5.92*\x  -0.33-1/30*5.92 -1.8*5.92*5.92
                });
    \node[rotate=-4.45] at (3.4,-0.33) {$\alpha=120^\circ$};

    
    \draw[thin] ({sqrt(0.49*(1-0.66*0.66))},{-0.66-1/30}) -- (5,  {-0.66-0.16});
    \draw[domain=0:{sqrt(0.49*(1-0.67*0.67))}, help lines, smooth]
    plot (\x, {3*\x*\x -2*3*sqrt(0.49*(1-0.67*0.67))*\x -0.67 +
      3*0.49*(1-0.67*0.67 ) -0.028
    });
    

    \node[rotate=-4.45] at (3.5,-0.66) {$\alpha=150^\circ$};

    \node[rotate=2] at (3.5,-0.9)
         {\textcyrillic{граница устойчивости инверторного режима}};
  \end{scope}
\end{tikzpicture}

$$
\beta_{min} = \gamma + \delta + \psi
$$
Если тока нет, то нет $\gamma$ и $\delta$, остается одна $\psi$.
С ростом тока $I$ растет $\gamma$ и $\delta$.
Максимальные пульсации $U$ при $\alpha=0$. На границе прерывистого режима
среднее напряжение примерно равно пульсациям.
${\displaystyle
  \lambda = \frac{2\pi}{m}}$
-- граничный режим. Справа справедливо уравнение (3)
Кривые близки к прямой линии до прерывистого, граничного режима.
Режим ``1-0'' ${\displaystyle 0<\lambda < \frac{2\pi}{m}}$ -- прерывистый режим,
  проводит один вентиль ``1'', затем он выключается ``-0'' -- никто из
  вентилей не проводит.

$$
  \begin{array}{ccl}
    ``1-1'' & {\displaystyle \lambda = \frac{2\pi}{m}} &
{\textcyrillic{-- граничный режим}} \\ 
   ``1-2'' &{\displaystyle \frac{4\pi}{m}>\lambda>\frac{2\pi}{m}} &{\textcyrillic{-- двухвентильная коммутация}}
  \end{array}
$$

Рассмотрим общий случай многофазного $m\ge 6$ преобразователя.
Основной режим -- режим двухвентильной коммутации.
  При больших токах и больших $L$

  ``2-2'' -- ${\displaystyle \lambda = \frac{4\pi}{m}}$;

 ``2-3'' -- ${\displaystyle \frac{4\pi}{m} <\lambda < \frac{6\pi}{m}}$;

  Это были характеристики для нулевых схем. В мостовых схемах происходит
  коммутация:  одна половина моста взаимодействует с другой половиной
  моста.

  Иногда режим ``2-2'' создают специально. Представим К.З.

\begin{tikzpicture}
  \begin{scope}[xscale=1,yscale=1.2]
    \draw[thick,red] (0,2) -- (7,0.1);
    \draw[thin, ->] (0, 0) -- (8,0) node[right] {$I$};
    \draw[thin, ->] (0, -3) -- (8,-3) node[right] {$I$};
    \draw[thin,<-] (7,-0.05) -- (6,-0.2) node[below]
    {${\textcyrillic{это величина тока К.З.}}$};
    \draw[thick,red] (0,-1) -- (5,-2);
    \draw[thick,red] (5, -2) -- (5.5,-2.9);
    \draw[thin,<-] (5.5,-3.05) -- (5.5,-3.2) node[below]
    {${\textcyrillic{облегчённое выключение тока К.З.}}$};
  \end{scope}
\end{tikzpicture}

\subsection{реверсивные преобразователи}
Ток только справа, отрицательного тока быть не может.
$$
%\textcyrillic{отрицательная } \underbrace{U\cdot I}_
\begin{array}{ccc}
  & {U\cdot I} & \omega\cdot M\\
  \textcyrillic{отрицательная} & \textcyrillic{электрическая} &
  \textcyrillic{механическая } \\
  & \textcyrillic{мощность} & \textcyrillic{мощность}
\end{array}
$$

Двигатель постоянного тока:

$$
E_\phi = \omega C_\phi
$$
$$
M_\textcyrillic{эм} = C_\phi I
$$
$$
C_\phi = \frac{pN}{2\pi a}
$$
здесь $p$ -- количество полюсов, $N$ -- число активных...,
$a$ -- число параллельных ветвей.

Положительный ток, отрицательный момент.
Не дай бог двигатель постоянного тока потеряет возбуждение
$\Phi\downarrow$ $\omega\uparrow$
Если $\Phi\downarrow$ 10-кратное форсирование в 10 раз дороже.
Применялось при ртутных вентилях, с 20В падением в дуге, и охлаждением.
С реверсом поля якоря тогда пытались <сделать> реверсивный поток.
Лучше сделаем отрицательный ток.

\subsection{классификация реверсивных тиристорных преобразователей}

\begin{figure}[H]
  \begin{tikzpicture}
    \begin{scope}
      \draw (0,0) -- (4,0) -- (4,2) -- (0,2) -- (0,0);
      \node at (2,1) {$\begin{array}{c}
          \textcyrillic{с одной}\\
          \textcyrillic{группой вентилей}
        \end{array}$};
      \draw (6,0) -- (12,0) -- (12,2) -- (6,2) -- (6,0);
      \node at (9,1) {$\begin{array}{c}
          \textcyrillic{с двумя}\\
          \textcyrillic{группами вентилей}
        \end{array}$};
      \draw[thin,->] (1,0) -- (0,-1);
      \node at (0,-1.5) {$\begin{array}{c}
        \textcyrillic{с контактным}\\
        \textcyrillic{реверсором}
        \end{array}$};
       \draw[thin,->] (3,0) -- (3.2,-0.7);
      \node at (3.2,-1.8) {$\begin{array}{c}
          \textcyrillic{с помощью ключей}\\
          \textcyrillic{с бесконтактным}\\
          \textcyrillic{полупроводниковым}\\
          \textcyrillic{реверсором}\\
          \textcyrillic{переключает концы}
        \end{array}$};
      \node at (9,-0.4) {$\textcyrillic{\underline{по типу силовой схемы}}$};
      \draw[thin,->] (8,-0.6) -- (7,-1);
      \draw[thin,->] (9,-0.6) -- (9,-1.8);
      \draw[thin,->] (10,-0.6) -- (11,-0.8);
      \node at (7,-1.3) {$\textcyrillic{перекрёстная}$};
      \node at (11,-1.3) {$\begin{array}{c}
          \textcyrillic{встречно-}\\
          \textcyrillic{паралельная}
        \end{array}$};
      \node at (9,-2) {$\textcyrillic{Н-схема}$};

      \node at (9,-3.8){$\begin{array}{c}
          \textcyrillic{\underline{по способу управления}}\\
          \\
          - \textcyrillic{c совместным управлением комплектами вентилей}\\
          - \textcyrillic{с раздельнам управлением комплектами вентилей}
      \end{array}$};
      \end{scope}
    \end{tikzpicture}
\end{figure}    


\subsection{Контактный реверсор}
\begin{figure}[H]
\begin{tikzpicture}
  \begin{scope}\draw
    (0,0) to[short] (1.5,0)
    (2,0) circle (0.5cm)
    node at(2,0) {M}
    (2.5,0) to[short] (4,0)
    to[short,-*] (4,1)
    to[ospst,mirror,l_=$K_4$] (2,1)
    to[cspst,-*,l=$K_3$] (0,1)
    to[short] (0,0)
    (4,1) to[short] (4,2)
    to[cspst,l=$K_2$] (2,2)
    to[ospst,mirror,l_=$K_1$] (0,2)
    to[short,-*] (0,1)
    (1,3) -- (3,3) -- (3,4) -- (1,4) -- (1,3)
    (2.5,3.5) to[Ty] (1.5,3.5)
    (1.5,3) to[short,-*] (1.5,2)
    (2.5,3) to[short,-*] (2.5,1)
    % сеть
    (2,4) to[short,-*] (2,5)
    (1.9,4.3) -- ++ (0.2,0.2)
    (1.9,4.4) -- ++ (0.2,0.2)
    (1.9,4.5) -- ++ (0.2,0.2)
    (1,5) to (3,5)
    (1.3,4.9) -- ++ (0.2,0.2)
    (1.4,4.9) -- ++ (0.2,0.2)
    (1.5,4.9)  -- ++ (0.2,0.2)
  ;\end{scope}
\end{tikzpicture}
\end{figure}
Если включены $K_2$ и $K_3$, ток течет через двигатель справа налево $\Longleftarrow$,
Если включены $K_1$ и $K_4$, ток течет через двигатель слева направо $\Longrightarrow$.

Как разорвать, возникает дуга. На переменном токе энергия $\uparrow\downarrow$. Дуга затухает, когда переменный ток перейдет через 0.

Для преобразователя уменьшим средний ток до 0, и в этот момент перебросим контакты.

Переключение инверторов
\begin{figure}[H]
  \begin{tikzpicture}
    \draw[thin,->] (0,0) -- (7,0) node[right] {$\omega t$};
    \draw[thin,->] (0,-1.8) -- (0,1.7) node[left] {$U,I$};
    \draw[thick,blue] (0,1.5) -- (2,1.5) -- (2,-1.5) -- (4,-1.5) -- (4,1.5) -- (6,1.5);
    \draw[thick,red] (0,1) -- (2,1) -- (3,0) -- (4,0) -- (5,-1) -- (6,-1);
    \node[above] at (1.5, 1.5) {$U$};
    \node[below] at (1,1) {$i_d$};
    \node[below] at (0.5,0.1) {
      $\begin{array}{c}
        \textcyrillic{выпрямительный}\\
        \textcyrillic{режим}
      \end{array}$};
    \node[right] at (4.5,1) {$\textcyrillic{переключение в инверторный режим}$};
    \draw[thin,->] (4.4,0.9) -- (3.4,0);
  \end{tikzpicture}
\end{figure}  

\subsection{бесконтактный ключ}
\begin{figure}[H]
  \begin{tikzpicture}
    \begin{scope}\draw
      (0,0) to[short] (1.5,0)
      (2,0) circle (0.5cm)
      node at(2,0) {M}
      (2.5,0) to[short] (4,0)
      to[short,-*] (4,1)
      to[Do,l^=$VT_4$] (2,1)
      to[Do,-*,l^=$VT_3$] (0,1)
      to[short] (0,0)
      (4,1) to[short] (4,2)
      to[Do,l^=$VT_2$] (2,2)
      to[Do,l^=$VT_1$] (0,2)
      to[short,-*] (0,1)
      (1,3) -- (3,3) -- (3,4) -- (1,4) -- (1,3)
      (2.5,3.5) to[Ty] (1.5,3.5)
      (1.5,3) to[short,-*] (1.5,2)
      (2.5,3) to[short,-*] (2.5,1)
      % сеть
      (2,4) to[short,-*] (2,5)
      (1.9,4.3) -- ++ (0.2,0.2)
      (1.9,4.4) -- ++ (0.2,0.2)
      (1.9,4.5) -- ++ (0.2,0.2)
      (1,5) to (3,5)
      (1.3,4.9) -- ++ (0.2,0.2)
      (1.4,4.9) -- ++ (0.2,0.2)
      (1.5,4.9)  -- ++ (0.2,0.2);
      %
      \draw[thin,<-,red] (0.5,0.5) -- (0,-0.5) node[below,red] {$\textcyrillic{К.З.}$};
      \draw[loosely dashed,red] (0.9, 1.4) ellipse (0.6cm and 1.1cm)   
      ;\end{scope}
  \end{tikzpicture}
\end{figure}
Схема включения ключей аналогична контактному реверсору. $VT_1$--$VT_4$ и
$VT_2$--$VT_3$.
Необходимо сделать паузу чтобы не были включены одновременно $VT_1$ и $VT_3$
иначе произойдет короткое замыкание.
Недостатки -- сэкономил 1 тиристор, а купил 4. Схема применяется для малой мощности,
для возбудителей, для гальваники. В гальванике технологически требуется сныть-нанести покрытие. Тренировка аккумуляторов.
В линиях постоянного тока происходит реверс мощности а не реверс тока.
 
%Под реверсией понимают изменение чего-либо на противопожное.
Если иметь ввиду двигатель, то реверс означает вращение в другую сторону.
Напряжение реверсируется даже в нереверсивном преобразователе.
Для того чтобы ток был реверсивным нужно как минимум два комплекта вентилей.
В классификации реверсивных преобразователей в первом классе были
преобразователи с одной группой вентилей. Переключение тока происходило
посредством переключения полюсов нагрузки с полюсами питания.
В истинно реверсивных преобразователях существуют две группы вентилей.

\subsection{перекрестрая схема}
исторически самая первая.

\begin{figure}[H]
\begin{circuitikz}\draw
  (0,3.5) to[L] ++ (0,-1)
  to[Ty,l_=$\begin{array}{c}
      \textcyrillic{1я группа}\\
      \textcyrillic{вентилей}\end{array}$] ++ (0,-1.5)
  -- ++ (1,0)
  (0,3.5) -- ++ (1,0)
  to[L] ++ (0,-1)
  to[Ty,-*] ++ (0,-1.5)
  -- ++ (1,0)
  (1,3.5) -- ++ (1,0)
  to[L] ++ (0,-1)
  to[Ty,-*] ++ (0,-1.5)
  % уравнительный реактор Ур1
  -- ++ (1.5,0)
  to[L,l={$\textcyrillic{Ур1}$}] ++ (0,1)
  -- ++ (1.5,1.5)
  -- ++ (1,0)
  % 2я группа вентилей
  to[L] ++ (0,-1)
  to[Ty,-*] ++ (0,-1.5)
  -- ++ (1,0)
  (6,3.5) -- ++ (1,0)
  to[L] ++ (0,-1)
  to[Ty,-*] ++ (0,-1.5)
  -- ++ (1,0)
  (7,3.5) -- ++ (1,0)
  to[L] ++ (0,-1)
  to[Ty,l=$\begin{array}{c}
      \textcyrillic{2я группа}\\
      \textcyrillic{вентилей}\end{array}$] ++ (0,-1.5)
  % уравнительный реактор Ур2
  (6,1) -- ++ (-1.5,0)
  to[L,l_={\textcyrillic{Ур2}}] ++ (0,1)
  -- ++ (-1.5,1.5)
  -- ++ (-1.5,0)
  % мотор
  (1,1) -- ++ (0,-1.5)
  -- ++ (2.5,0)
  (4,-0.5) circle (0.5 cm)
  (4,-0.5) node {M}
  (4.5,-0.5) -- ++ (2.5,0)
  -- ++ (0, 1.5)
  % первичная сторона
  (3, 5) to[L] ++ (0,-1)
  -- ++ (1,0)
  (4, 5) to[L] ++ (0,-1)
  -- ++ (1,0)
  (5, 5) to[L] ++ (0,-1)
  % core
  [thick] (0,3.74) rectangle (8,3.77)
  ;\end{circuitikz}
\caption{перекрестная схема реверсивного преобразователя}
\end{figure}

$\textcyrillic{Ур1}$ и $\textcyrillic{Ур1}$ -- уравнительные реакторы.
Силовые индуктивности рисуются как
\begin{circuitikz}\draw
  (0,0.8) -- ++ (0,-0.8)
  (0.5,0) arc(0:270:0.5)
  (0.5,0) -- ++ (-0.5,0)
  (0,-0.5) -- ++ (0,-0.3)
  ;\end{circuitikz}
 или как  
\begin{circuitikz}\draw
(0,0.8) --++ (0,-0.3)
(-0.5,0) arc(-180:90:0.5)
(-0.5,0)--++(0.5,0)
(0,0)--++(0,-0.8)  
  ;\end{circuitikz}

\begin{figure}[H]
  \begin{circuitikz}\draw
    (0,1) to[L,-*] ++(2,0)
    to[L] ++(2,0)
    --++(0,1)
    to[Ty,*-] ++(-1.3,0)
    --++(-1.4,0)
    to[Ty,*-*] ++(-1.3,0)
    --++(0,-1)
    (4,2)--++(0,1)
    to[Ty,*-*]++(-1.3,0)
    --++(-1.4,0)
    to[Ty,-*]++(-1.3,0)
    --++(0,-1)
    (4,3)--++(0,1)
    to[Ty]++(-1.3,0)
    to[short,-*]++(-0.7,0)
    --++(-0.7,0)
    to[Ty]++(-1.3,0)
    --++(0,-1)
    %LLL
    (1.3,2)--++(0,2)
    to[L]++(0,1.3) %La'
    --++(-0.4,0)
    --++(0,1.7)
    --++(1.8,0)
    (2,4) to[L]++(0,1.3) %Lb
    --++(-0.4,0)
    --++(0,1.5)
    --++(-0.3,0)
    (1.3,5.5) to[L]++(0,1.3) %La''

    (1.3,5.5)--++(0.7,0)
    to[L,*-]++(0,1.3) %Lb''
    (2.7,3)--++(0,1)
    to[L]++(0,1.3) %Lc'
    --++(-0.4,0)
    --++(0,1.5)
    --++(-0.3,0)
    (2,5.5)--++(0.7,0)
    to[L,*-]++(0,1.3) %Lc''
    --++(0,0.2)
    %мотор
    (2,1)--++(0,-0.5)
    (2,0) circle (0.5cm)
    (2,0) node {M}
    (2,-0.5)--++(0,-0.5)
    --++(2.3,0)
    --++(0,6.5)
    --++(-1.6,0)
    %core
    [thick] (0.9,7.2) rectangle (3.1,7.22)
    % первичная сторона
    (1.3,7.4) to[L]++(0,1.3)
    (2,7.4) to[L]++(0,1.3)
    (2.7,7.4) to[L]++(0,1.3)
    (1.3,7.4) -- (2.7,7.4)
    ;
    % стрелки
%    \draw[<-] (0.2,1.5) -- (0.8,1.5) node[right] {i};
    \draw[<-] (0.2,2.5) -- (0.8,2.5)  node[right] {\it{i}};
    \draw[->] (0.8,3.5) -- (0.2,3.5);
    \draw[->] (0.1,4.4) arc (90:180:0.4) --++(0,-2.9) arc (180:270:0.4) --++(0.3,0)
    node at (-0.5,2.5) {\it{i}};
    \draw[dashed,->] (3.5,0.8) -- (3.9,0.8) arc (270:360:0.3) --++(0,1.1)
    arc (0:90:0.3) --++ (-0.5,0) node[left] {\it{i}};
    % подписи
    \draw (-1.4,3.9) node {$\begin{array}{c}\textcyrillic{1я группа}\\
        \textcyrillic{вентилей}\end{array}$}
    (5.3,3.9) node {$\begin{array}{c}\textcyrillic{2я группа}\\
                \textcyrillic{вентилей}\end{array}$}
    ;\end{circuitikz}
\caption{встречно-параллельная схема} 
\end{figure}


\begin{figure}[H]
  \begin{circuitikz}\draw
    (0,1) to[Ty] ++(1.3,0)
    --++(1.6,0)
    (4,1) to[Ty,-*]++(-1.3,0)
    %
    (0,1) --++(0,1)
    to[Ty,*-] ++(1.3,0)
    to[short,-*] ++ (0.7,0)
    --++(0.7,0)
    (4,1) --++ (0,1)
    to[Ty,*-] ++(-1.3,0)
    %
    (0,2) --++(0,1)
    to[Ty,*-*] ++(1.3,0)
    --++(1.6,0)
    (4,2) --++(0,1)
    to[Ty,*-] ++(-1.3,0)
    % LLL
    (1.3,-0.3) to[L] ++(0,1.3)
    --++(0,2)
    (2,-0.3) to[L] ++(0,1.3)
    --++(0,1)
    (2.7,-0.3) to[L] ++(0,1.3)
    % мотор
    (0,3) to[short,-*]++(0,1)
    --++(1.5,0)
    (2,4) circle (0.5cm)
    (2,4) node {M}
    (4,3) to[short,-*]++(0,1)
    --++(-1.5,0)
    %
    (0,4)--++(0,3)
    (4,4)--++(0,3)
    (1.3,5) to[Ty,-*]++(-1.3,0)
    (2.7,5) to[Ty,*-*]++(1.3,0)
    (1.3,5)--++(1.4,0)--++(0,2)
    %
    (1.3,6) to[Ty,-*]++(-1.3,0)
    (2.7,6) to[Ty,-*]++(1.3,0)
    (1.3,6) to[short,-*]++(0.7,0)--++(0.7,0)
    (2,6)--++(0,1)
    %
    (1.3,7) to[Ty,*-*]++(-1.3,0)
    (2.7,7) to[Ty,-*]++(1.3,0)
    (1.3,7)--++(1.4,0)
    % LLL
    (1.3,7) to[L]++(0,1.3)
    (2,7)   to[L]++(0,1.3)
    (2.7,7) to[L]++(0,1.3)
    % Lурав
    (1.3,-0.3)--++(3.7,0)
    --++(0,3.8)
    to[L,l_={$L_\textcyrillic{урав}$}]++(0,1)
    --++(0,3.8)
    --++(-3.7,0)
    % core
    [thick] (1,8.4) rectangle (3,8.43)
    % первичная сторона
    (1.3,8.53)--++(1.4,0)
    (1.3,8.53) to[L]++(0,1.3)
    (2,8.53) to[L]++(0,1.3)
    (2.7,8.53) to[L]++(0,1.3)
    ;
    %стрелки
    \draw[->] (0.1,7.4) arc (90:180:0.4) --++ (0,-3) arc (180:270:0.4)
    --++ (0.7,0) node[right] {\it{i}};
    \draw[->] (2.8,3.7)--++(1.1,0) arc(90:0:0.4)--++(0,-1.4)arc(0:-90:0.4)
    --++(-0.4,0);
    \draw[dashed,<-] (2.8,4.4)--++(1.1,0) arc(279:360:0.4)--++(0,2.2)
    arc (0:80:0.4);
    \draw[dashed,->] (-0.3,3)--++(0,-2) arc(180:270:0.4)--++(0.4,0)
    ; 
  \end{circuitikz}
\caption{H-схема}
\end{figure}  

Все схемы изображены для случая трёхфазной нулевой схемы.
На примере одной схемы разберём как различаются схемы по признакам
классификации.
Перекрестная схема состоит из двух групп вентилей, каждая из которых
питается от 3х индивидуальных обмоток трехфазного трансформатора.
Вторая схема. Если управлять вентилями с одинаковым $\alpha$ то получим
К.З. на индуктивность. Первая группа вентилей проводит сверху вниз,
другая группа вентилей проводит снизу вверх. Питается от одного комплекта
вентильных обмоток.

Всякий раз когда появляется нулевая схема вентилей трансформаторные обмотки
соединяются в зигзаг во избежание вынужденного намагничивания трансформатора.

Возвращаясь к первой схеме реверс это реверс тока, не обязательно реверс скорости,
но обязательно реверс момента.
Если нужно реверсировать момент, то нужно применять реверсивный преобразователь.
Прокатные станы производят металический лист для автомобилей. Нужно регулировать скорость, нужно регулируемое торможение.
Ток идет по первой группе вентилей, затем по мотору, затем через Ур2.
Вторая обмотка не работает! В трансформаторе ток течет в одну сторону,
подмагничивает сердечник постоянным током. Если есть 3-х фазный трансформатор с
3мя штыревыми сердечниками на каждом сердечнике ток течёт в одну сторону.
Как преодолеть подмагничивание трансформатора? Рассмотрим схему 2. Ток дотёк
до трансформатора и разделяется на 2 обмотки. Ток протекает по обмоткам
создавая МДС в обе стороны
\begin{tikzpicture}
  \begin{scope}[xscale=1,yscale=1]
  %A;
  \draw[very thick,red,->] (0,0) --++(0,1);
  \draw (0,1) --++ ({-cos(240+90)},{-sin(240+90)});
  %B
  \draw (0,0) --++({cos(120+90)},{sin(120+90)});
  \draw[very thick,red,->] ({cos(120+90)},{sin(120+90)}) --++ (0,-1);
  %C
  \draw (0,0) --++({cos(240+90)},{sin(240+90)});
  \draw ({cos(240+90)},{sin(240+90)}) --++ ({-cos(120+90)},{-sin(120+90)});
\end{scope}  
\end{tikzpicture}
2 обмотки зигзага работают в разные стороны. Для преобразователя малой мощности
можно использовать схемы звезда или треугольник. Для преобразователя большой
мощности используют зигзаг. Есть альтернатива для того чтобы избежать
подмагничивания: можно увеличить количество железа в трансформаторе, а
можно увеличить количество меди соединив обмотки в зигзаг.
\begin{tikzpicture}
  \draw[->] (0,0)--(0,1);
  \draw[->] (0,1)--++({-cos(240+90)},{-sin(240+90)});
  \draw[dashed,->] (0,0) --({-cos(240+90)},{1-sin(240+90)});
\end{tikzpicture}
Векторное сложение МДС даст $\sqrt{3}$ при соединении обмоток в зигзаг,
а если соединить последовательно получилось бы увеличение в 2 раза.
\begin{tikzpicture}
  \draw[->] (0,0)--(0,1);
  \draw[->] (0,1)--++(0,1);
\end{tikzpicture}
В зигзаге получил примерно 15\% снижение возможного напряжения. Меди намотал на
200вольт, а получил 173вольта.
Одним из условий борьбы с вынужденным намагничиванием, чтобы ток протекал
в обе стороны. Из-за этого МДС будет направлена в разные стороны.

3-я схема. Ток протекает как нарисовано на схеме. Обмотки можно перевернуть
как в зигзаге
\begin{circuitikz}
\begin{scope}[xscale=0.8,yscale=0.8]\draw
  (0,0) to[L]++(0,1) node[right=2.4mm,below=0.6mm] {$\bullet$}
  (0.8,0) to[L]++(0,1) node[right=2.4mm,below=0.6mm] {$\bullet$}
  (1.6,0) to[L]++(0,1) node[right=2.4mm,below=0.6mm] {$\bullet$}
  %
  (0,1.2) to[L]++(0,1)
  (0,1.2) node[right=2.4mm,above=0mm] {$\bullet$}
  (0.8,1.2) to[L]++(0,1)
  (0.8,1.2) node[right=2.4mm,above=0mm] {$\bullet$}
  (1.6,1.2) to[L]++(0,1)
  (1.6,1.2) node[right=2.4mm,above=0mm] {$\bullet$}  
  ;\end{scope}\end{circuitikz}
можно не переворачивать
\begin{circuitikz}
  \begin{scope}[xscale=0.8,yscale=0.8]\draw
    (0,0) to[L]++(0,1) node[right=2.4mm,below=0.6mm] {$\bullet$}
    (0.8,0) to[L]++(0,1) node[right=2.4mm,below=0.6mm] {$\bullet$}
    (1.6,0) to[L]++(0,1) node[right=2.4mm,below=0.6mm] {$\bullet$}
    %
    (0,1.2) to[L]++(0,1) node[right=2.4mm,below=0.6mm] {$\bullet$}
    (0.8,1.2) to[L]++(0,1) node[right=2.4mm,below=0.6mm] {$\bullet$}
    (1.6,1.2) to[L]++(0,1) node[right=2.4mm,below=0.6mm] {$\bullet$}
    ;\end{scope}\end{circuitikz}.
Реактор здесь один -- $L$. Если ток течёт слева направо то половина вентилей
не работает. Вентильная группа $\textcyrillic{ВГ}_2$ обеспечивает протекание
тока в другую сторону. Ток по обмоткам трансформатора будет протекать в том же
направлении. В обратную сторону протекает ток обозначенный штриховкой.

1я схема) 2 набора обмоток и две вентильные группы. Каждый набор обмоток на
свою вентильную группу.

2я схема) Один набор обмоток на две группы вентилей.

3я схема)

\begin{tikzpicture}\draw
  (0,0.5) node {1й набор обмоток}
  (0,0) node {2й набор обмоток}
  (4.5,0.5) node {1я группа вентилей}
  (4.5,0) node {2я группа вентилей}  
  ;
  \draw[->] (1.7,0) -- (2.6,0);
  \draw[->] (1.7,0) -- (2.6,0.5);
  \draw[->] (1.7,0.5)  -- (2.6,0);
  \draw[->] (1.7,0.5) -- (2.6,0.5);
\end{tikzpicture}

Первая схема применялась с ионными вентилями. В то время применялись схемы с общим
катодом. Два бака катодов не применялось.
С тиристорными вентилями применялась вторая схема. Один комплект обмоток лучше чем
два.

Если считать что в каждом направлении машина работает 50мин - 1 час, то
считается что машина работает долго. Габариты удваиваются
$\displaystyle{\frac{S_1 + S_2}{2}}$. В длительном режиме суммарная мощность
трансформатора увеличивается в полтора раза. Сощность обмотки определяет расход
меди. Первичная обмотка всегда входит с коэффициэнтом 100\%.

$$
S_\textcyrillic{трансформатора} = \frac{S_1 + S_2}{2}
$$
Если в каждом направлении длительно работать, то
$S \approx 1.5 S_\textcyrillic{нагрузки}$, но так не говорят для постоянного тока,
говорят $=1.5  P_\textcyrillic{нагрузки}$

$[S] = V\!A(kV\!A)$, а мощность нагрузки киловатты
$S_\textcyrillic{трансформатора} = 1.5k P_\textcyrillic{нагрузки}$,
где $k$ зависит от коэффициэнта мощности и от ... (коэффициэнта формы?)

Если включение кратковременное м 50\% течёт и 50\% не течёт, то
среднеквадратичное включение $\displaystyle{\frac{1}{\sqrt{2}}}$
\begin{tikzpicture}
  \draw[thin,->] (0,0) -- (6,0);
  \draw[thick] (0,0)--(1,0)--(1,1)--(2,1)--(2,0)--(3,0)--(3,1)--(4,1)--(4,0)--(5,0);
  \end{tikzpicture}

$$
S_\textcyrillic{трансформатора} = \frac{1+2{\displaystyle \frac{1}{\sqrt{2}}}}{2}
\approx 1.205 k P_\textcyrillic{нагрузки}
$$

В любом случае в первой схеме больше потерь чем второй. Это главный недостаток
перекрёстной схемы. Третья схема: сделать выводы из трансформатора --
конструкция получается белее сложная.

Классификация по способу управления -- совместное управление группами вентилей
и раздельное управление группами вентилей.
Рассмотрим схему два. Ток не будет протекать справа -- раздельное управление.
По реализации оно сложнее. Включается инверторный режим, возникает опрокидывание.

Если не отключать вентили, а регулировать
$$
cos \alpha_1 + cos \alpha_2?
$$

\subsection{Эквивалентная схема замещения реверсивного тиристорного преобразователя}
\begin{circuitikz}\draw
  (0,5.6)--++ (0,-0.3)
  (0,4.8) circle (0.5cm)
  (0,4.8) node {$e_{d1}$}
  (0.5, 4.8) node[right] {$f(\alpha_1)$}
  (0,4.3)--++(0,-0.3)
  (0,1) to[battery1,l=$U_0$]++(0,1)
  to[L,l=$L_1$]++(0,1)
  to[R,l=$R_1$]++(0,1)
  (0,1) to[Do]++(0,-1)
%  --++(0,-0.5)
  %
  --++(3,0)
%  --++(0,0.5)
  to[Do,l=''VD'']++(0,1)
  to[battery1,l=$U_0$]++(0,1)
  to[L,l=$L_2$]++(0,1)
  to[R,l=$R_2$]++(0,1)
  --++(0,0.3)
  (3,4.8) circle (0.5cm)
  (3,4.8) node {$e_{d2}$}
  (3.5, 4.8) node[right] {$f(\alpha_2)$}
  (3,5.3)--++(0,0.3)
  %
  (3,0)--++(3,0)
  --++(0,0.5)
  (6,1) circle(0.5)
  (6,1) node {$E_\textcyrillic{н}$}
  (6,1.5)--++(0,0.5)
  to[L,l_=$L_\textcyrillic{н}$]++(0,1.5)
  to[R,l_=$R_\textcyrillic{н}$]++(0,2.1)
  --++(-6,0);
  %
  \draw[<-](-0.4,0.5)--++(-1.2,0) node[left] {идеальный диод};
  \draw[<-](2.4,0.2)--++(-1.5,-1) node[below] {все его параметры вошли в элементы выше выше}
  ;\end{circuitikz}

Вторая вентильная группа $U_0,L_2,R_2$

Предполагаем
$$
L_1=L_2=L
$$
$$
R_1=R_2=R
$$

$$
\underbrace{\displaystyle{}e_{d1}}_{\displaystyle \textcyrillic{мгновенное значение}}=
\overbrace{E_{d0}\;cos\alpha_1}^{
  \begin{array}{c}\textcyrillic{среднее значение}\\
    \textcyrillic{выпрямленного ЭДС}\end{array}} +
\underbrace{e_{d1\sim}}_{\displaystyle \textcyrillic{добавили пульсации}}
$$

$$
e_{d2} = E_{d0} \; cos \alpha_2 + e_{d2\sim}
$$

Чему равны $E_{d01}$ и $E_{d02}$ в первой схеме? Мы полагаем, что если $\alpha$ неодинаковые,
то пульсации неодинаковые.

Если раздельное управление?

А почему $\alpha$ разные.

\begin{circuitikz}\draw
  (0,1) to[Do] (0,0)
  (0,0) -- (1.4,0)
  (1.4,0) to[Do]++(0,1);
  \draw[->] (0.3,2)--(0.3,0.6) arc(180:360:0.4)--++(0,1.4);
  \draw (0,1.7) node {\it{i}};
\end{circuitikz}
Ток может протекать минуя нагрузку.


\begin{tikzpicture}[scale=10]
  \draw[thin,domain={-pi/4}:{pi/4},samples=100]
  plot (canvas polar cs:angle=\x r,radius=  {5*sqrt(2*cos(2*\x r))});
  \draw[thin,domain={pi-pi/4}:{pi-pi/4+0.15},samples=100]
  plot (canvas polar cs:angle=\x r,radius=  {5*sqrt(2*cos((-2*\x) r))});
  \draw[thin,domain={pi-pi/4+0.2}:{pi+pi/4},samples=100]
    plot (canvas polar cs:angle=\x r,radius=  {5*sqrt(2*cos((-2*\x) r))});
  \draw[thin,domain={pi/4}:{-pi/4},samples=100]
  plot (canvas polar cs:angle=\x r,radius=  {5*sqrt(2*cos(2*\x r))});
  \draw[thin,domain={pi+pi/4}:{pi},samples=100]
  plot (canvas polar cs:angle=\x r,radius=  {5*sqrt(2*cos((-2*\x) r))});
  % стрелка
  \draw[domain={pi-pi/4+0.1}:{pi-pi/4+0.15},samples=100,very thick,->]
  plot (canvas polar cs:angle=\x r,radius=  {0.02+5*sqrt(2*cos((-2*\x) r))})
  node[above right] {\it{i}};
  \end{tikzpicture}
В первой схеме протекает уравнительный ток, нежелательный, может быть неприемлимо
большой, аварийно-опасный.

%\begin{tikzpicture}[scale=2]
%  \draw[->] (-1,0) -- (1,0);
%  \draw[->] (0,-1) -- (0,1);
%  \draw node [red] at (-1,.25) {\scriptsize{Kardioida $r=5-5\sin \theta$}};
%  \draw[color=red,domain=0:6.28,samples=200,smooth]
%  plot (canvas polar cs:angle=\x r,radius={5-5*sin(\x r)});  %r = angle en radian
%  \end{tikzpicture} 

\begin{circuitikz}\draw
  (0,5.6)--++ (0,-0.3)
  (0,4.8) circle (0.5cm)
  (0,4.8) node {$e_{d1}$}
  (-0.3,5.3) node[left] {-}
  (-0.3,4.3) node[left] {+}
  (0.5, 4.8) node[right] {$\alpha=0$}
  (0,4.3)--++(0,-0.3)
  (0,1) to[battery1,l=$U_0$]++(0,1)
  to[L,l=$L_1$]++(0,1)
  to[R,l=$R_1$]++(0,1)
  (0,1) to[Do]++(0,-1)
  ;\end{circuitikz}
Условие отсутствия уравнительного тока $e_{d1}+e_{d2}\le 0!$ при пренебрежении $U_0$.

$e_{d1}+e_{d2}\le 2U_0$

$$
E_{d0} (cos \alpha_1 + cos \alpha_2) + e_\textcyrillic{уравнительное} \le 2U_0
$$
где $e_\textcyrillic{уравнительное} = e_{d1\sim} + e_{d2\sim}$

Можно записать отдельно условия отсутствия уравнительного тока:

Переменная составляющая уравнительного тока должна быть $\le 2U_0$ и
постоянная составляющая уравнительного тока должна быть $\le 2U_0$.

Для постоянной составляющей
$$
E_{d0} (cos \alpha_1 + cos \alpha_2)\le 2U_0 \Rightarrow cos \alpha_1 + cos \alpha_2
\le \frac{2U_0}{E_{d0}}
$$

Если $\displaystyle \frac{2U_0}{E_{d0}}\approx 0$ тогда пишем для случая равенства 0:

$$
  \bcancel{2}cos\frac{\alpha_1 +\alpha_2}{2}\cdot cos \frac{\alpha_1 -\alpha_2}{2} \le 0
$$

$$
  \begin{array}{c}
    0<\alpha_1<\pi-\xcancel{\beta_{min}}\\
    0<\alpha_2<\pi-\xcancel{\beta_{min}}
    \end{array}
$$
  
но $\displaystyle cos \frac{\alpha_1-\alpha_2}{2}\ne 0$ никогда не равно нулю на всем
  диапазоне.

  \begin{equation}
    cos \frac{\alpha_1+\alpha_2}{2}\le 0
  \end{equation}

  $$
  E_{d0}(cos \alpha_1+ cos\alpha_2)\le2U_0
  $$
  \begin{equation}\left\{\begin{array}{ll}
      \alpha_1+\alpha_2\ge\pi&
        {\displaystyle\textcyrillic{ ( при} \frac{U_0}{E_{d0}}=0)}\\
      {\displaystyle cos\; \alpha_1+ cos\;\alpha_2 \le \frac{2U_0}{E_{d0}}}&
    \end{array}\right.
\label{codrive}
  \end{equation}
  Условия  (\ref{codrive}) являются условиями совместного управления.
  $\alpha_1$ меняется, чтобы вторая группа не мешала $\alpha_1=\pi-\alpha_2$

  \begin{tikzpicture}
    \begin{scope}[scale=1]
      \draw[thin,->] (0,0) -- (6,0) node[right] {$\alpha_1$};
      \draw[thin,->] (0,0) -- (0,6) node[left] {$\alpha_2$};
    \draw[domain=0:5]
    plot (\x,5-\x);
    \draw[dashed,domain=0:4.9]
        plot (\x,5-\x-0.1);
    \draw[domain=0:4.28]
    plot (\x, {acos( (0.1-cos((\x*pi/5) r)) )/180*5 });
    % 2U_0/E_d0 = 0.5/5
    \draw[loosely dotted,domain=0:4.5]
        plot (\x,5-\x-0.5);
    \draw[dotted,domain=0:3.33]
    plot (\x, {acos( (0.5-cos((\x*pi/5) r)) )/180*5 });
    \draw[thin,<-] (0.3,{5-0.3}) -- (1.5,5.9) node[right] {$\alpha_2 = \pi - \alpha_1$};
    \draw[<-] (0.5,{5-0.5-0.1}) -- (2.5,4.7) node[right]
         {$\displaystyle{\frac{2U_0}{E_{d0}}}$};
    \draw[thin,<-] (3.4,4.9) -- (4.5,5.1) node[right] {2B};
    \draw[thin,<-] (3.4,4.4) -- (4.5,4.2) node[right] {100B};
    \end{scope}
  \end{tikzpicture}  

  Регулируем напряжение на нагрузке.

  \begin{tikzpicture}
    \begin{scope}[xscale=2,yscale=3]
      \draw[thin,->] (-1.2,0) -- (1.3,0) node[right] {$\frac{U_d}{E_{d0}}$};
      \draw[thin,->] (0,0) -- (0,1.2) node[left] {?};
      \draw[thin] (-1.1,1) -- (1.1,1) node[right] {$\pi$};
      \draw[thin,dashed] (-1.1,0.9) -- (1.2,0.9) node[right] {$\pi-\beta_{min}$};
      \draw[domain=-1:1,samples=200]
      plot(\x, {acos(\x)/180});
      \draw[domain=-1:1,samples=200]
      plot(\x, {asin(\x)/180+0.5});
      %
      \draw[thin] (0,0.5) -- (-0.3,0.5) node[left] {$\frac{\pi}{2}$};
      \draw[thin] (0.75,0.25) -- (1,0.3) node[right] {$arccos,arcsin$};
      \draw[thin,<-] (-0.05,0.45) -- (-0.25,-0.25) node[right] {$\alpha_1= \alpha_2$};
    \end{scope}
  \end{tikzpicture}

  Получили соотношение для постоянных составляющих при совместном управлении.

  $$
  e_{d1\sim} + e_{d2\sim} = U_\textcyrillic{ур} \le 0
  $$
  Корректно ли ставить такую задачу? переменная составляющая на самом деле --
  знакопеременная. Уравнительный ток останется. Если ток переменный, то его можно
  ограничить индуктивностью до заранее выбранного значения.

  \begin{tikzpicture}[scale=3]
    \draw[thin,domain={-pi/4}:{pi/4},samples=100]
    plot (canvas polar cs:angle=\x r,radius=  {5*sqrt(2*cos(2*\x r))});
    \draw[thin,domain={pi-pi/4}:{pi-pi/4+0.15},samples=100]
    plot (canvas polar cs:angle=\x r,radius=  {5*sqrt(2*cos((-2*\x) r))});
    \draw[thin,domain={pi-pi/4+0.2}:{pi+pi/4},samples=100]
    plot (canvas polar cs:angle=\x r,radius=  {5*sqrt(2*cos((-2*\x) r))});
    \draw[thin,domain={pi/4}:{-pi/4},samples=100]
    plot (canvas polar cs:angle=\x r,radius=  {5*sqrt(2*cos(2*\x r))});
    \draw[thin,domain={pi+pi/4}:{pi},samples=100]
    plot (canvas polar cs:angle=\x r,radius=  {5*sqrt(2*cos((-2*\x) r))});
    % стрелка
    \draw[domain={pi-pi/4+0.1}:{pi-pi/4+0.15},samples=100,very thick,->]
    plot (canvas polar cs:angle=\x r,radius=  {0.02+5*sqrt(2*cos((-2*\x) r))})
    node[above right] {\it{i}};
  \end{tikzpicture} -- переменная составляющая 50-70\%. Что можно сделать?

  Иногда делают навстречу постоянное напряжение
  \begin{tikzpicture}
    \begin{scope}[xscale=1.5,yscale=2]
      \draw[thin,->] (-1.2,0) -- (1.3,0) node[right] {$\frac{U_d}{E_{d0}}$};
      \draw[thin,->] (0,0) -- (0,1.2) node[left] {?};
      \draw[thin] (-1.1,1) -- (1.1,1) node[right] {$\pi$};
%      \draw[thin,dashed] (-1.1,0.9) -- (1.2,0.9) node[right] {$\pi-\beta_{min}$};
      \draw[domain=-1:1,samples=200]
      plot(\x, {0.3+acos(\x)/180});
      \draw[domain=-1:1,samples=200]
      plot(\x, {0.3+asin(\x)/180+0.5});
    \end{scope}
  \end{tikzpicture}

  Для ограничения уравнительного тока обусловленного переменной составляющей
  уравнительного тока используются уравнительные реакторы.

  Схему иногда называют ''восьмеркой''. Как расчитать реакторы? Существуют графики,
  можно определить зависимость от одного угла, либо от выпрямленного напряжения.
  Ток переменный или постоянный. Пойдёт ток как в однополупериодной схеме.
  
  \begin{circuitikz}\draw
    (0,1) to[Do] (0,0)
    (0,0) -- (1.4,0)
    (1.4,0) to[Do]++(0,1);
  \end{circuitikz} -- однополупериодная схема.

  Если ток течёт без перерыва, то падение напряжения $U=0$.
  $I_\textcyrillic{малый ток}\cdot r_\textcyrillic{проводов}$ -- величина второго
  порядка малости если обеспечен $I_\textcyrillic{малый ток}$.

  Зависимость $U_\textcyrillic{выпрямленного}$ от $\alpha$.

  Задавшись максимальным уравнительным током $\Rightarrow$ зададимся индуктивностью.
  Ток загружает вентили, загружает трансформатор.
  Используемые уравнительные реакторы выбираются по максимальному значению
  $U_\textcyrillic{ур.}$, зависящему от соотношения углов $\alpha$. Обычно
  величина $I_\textcyrillic{ур.}$ ограничивается на 10\% меньше от номинального
  $I_\textcyrillic{выпрямленного}$

  Как протекает ток в схеме 2
  \begin{circuitikz}\draw
    (1,3) to[L,l=$A$] (1,2)--(0,2)--(0,0)
    to[L] (1,0) --(2,0) to[L](3,0)--(3,2)--(2,2)--(2,3) to[L,l=C] (2,4)
;    \end{circuitikz}
  Ток из одной фазы переходит в другую.
  В тертьей схеме
  \begin{circuitikz}\draw
    (1,5)--(1,4)to[short,l=A](0,4)--(0,2)to[short,l=B](1,2)--(1,0)--(3,0)to[L,l=$L$](3,2)
    (0,2)--(0,1)to[short,l=C](1,1)
    ;\end{circuitikz}
  существует два контура уравнительного тока, а реактор один для двух контуров тока.
  H-схема -- средняя. Вторая схема самая хорошая. Фирмы-производители не часто
  использовали H-схему.

  Для мостовой схемы зигзаг не нужен.
  
  \begin{circuitikz}\begin{scope}
    \draw
  (1,1) to[Ty] (2,1) -- (4,1) to[Ty,*-] (5,1)--(5,2)
  (1,1)--
  (1,2) to[Ty] (2,2) -- (3,2)--(4,2) to[Ty] (5,2)--(5,3)
  (1,2)--
  (1,3) to[Ty,-*] (2,3) -- (4,3) to[Ty] (5,3)
  %
  (5,4) to[Ty,-*] (4,4)--(2,4) to[Ty] (1,4)--(1,5)
  (5,4)--
  (5,5) to[Ty] (4,5)--(2,5) to[Ty] (1,5)--(1,6)
  (5,5)--
  (5,6) to[Ty] (4,6)--(2,6) to[Ty,*-] (1,6)
  %
  (2,3)--(2,8)
  (3,2)to[short,*-*](3,5)--(3,8)
  (4,1)--(4,8)
  %
  (2,8)to[L,l=A](2,9.3)--(2.4,9.3)--(2.4,8)--(3,8)
  (3,8)to[L,l=B](3,9.3)--(3.4,9.3)--(3.4,8)--(4,8)
    
    %уравнительные реакторы
    (1,2)--(0,2)--(0,2.5) arc(90:-180:0.5)--(-1,2)
    (-0.4,2.1) node[above] {$L_2$}
    (1,5)--(0,5)--(0,5.5) arc(90:-180:0.5)--(-1,5)
    (-0.4,5.1) node[above] {$L_1$}
    %
    (5,2)--(6,2)--(6,2.5) arc(90:360:0.5)--(7,2)
    (6.4,2.1) node[above] {$L_4$}
    (5,5)--(6,5)--(6,5.5) arc(90:360:0.5)--(7,5)
    (6.4,5.1) node[above] {$L_3$}
    %к мотору
    (-1,5)to[short,-*](-1,2)--(-1,0)--(2.5,0)
    (7,5)to[short,-*](7,2)--(7,0)--(3.5,0)
    %мотор
    (3,0) circle (0.5cm)
    (3,0) node {M}
    %core трансформатора
    (2,9.49) rectangle (4,9.51)
    %LLL
    (2,9.7)to[L](2,11)
    (3,9.7)to[L](3,11)
    (4,9.7)to[L](4,11)
    (2,9.7)--(4,9.7)
    
    %рисуем ток
    (1.4,6.5) node {$\leftarrow A$}
    (2.4,2.3) node {$B\rightarrow$}
    ;
    % рисуем ток КЗ при коммутации
    \draw[dotted,color=red,->] (1.9,6.1)--(1,6.1)arc(90:180:0.1)--(0.9,5.5)
    arc(0:-90:0.4)--(-0.3,5.1)arc(90:180:0.6)--(-0.9,2.7)arc(180:270:0.6)--
    (0.5,2.1)arc(270:360:0.4)--(0.9,3)arc(180:90:0.1)--(1.4,3.1)arc(270:360:0.5)
    --(1.9,5.8);
    \draw[dotted,color=red,<-] (-1,3.6)--(-1.5,-1) node[right]
         {часть периода при коммутации будет течь ток К.З.};
   %
  \draw[dotted] (4.1,8.1)to[L,l=C](4.1,9.4)--(4.5,9.4)--(4.5,8.1)--(5.1,8.1);
  \draw[thin,<-] (4.5,8.7) -- (6,9) node[right]
       {$\begin{array}{c}\textcyrillic{чтобы подчеркнуть,}\\
           \textcyrillic{что средняя точка не нужна}\\
\textcyrillic{и обмотка C не нужна!}
         \end{array}$};
    \end{scope}
\end{circuitikz}  

  Куда включить реакторы? Сначала включим по самой сложной/полной схеме. Потом будем
  выбрасывать. $\cancel{L_2}\cancel{L_3}$ -- можно убрать два реактора, только
  симметрично. Могут использоваться 2 реактора или все четыре. Наличие двух
  контуров уравнительного тока с большой составляющей переменного напряжения
  является главным недостатком встречно-параллельной схемы.

  \begin{circuitikz}\draw
    (4,2)to[Ty](3,2)--(1,2)to[Ty](0,2)
    (4,3)to[Ty](3,3)--(1,3)to[Ty](0,3)
    (4,4)to[Ty](3,4)--(1,4)to[Ty](0,4)
    %
    (10,2)to[Ty](9,2)--(7,2)to[Ty](6,2)
    (10,3)to[Ty](9,3)--(7,3)to[Ty](6,3)
    (10,4)to[Ty](9,4)--(7,4)to[Ty](6,4)
    %
    (0,4)--(0,0)--(1,0)to[L,l=$\textcyrillic{Ур}_1$](2,0)--(5,0)--(8,0)
    to[L,l=$\textcyrillic{Ур}_2$](9,0)--(10,0)--(10,4)
    % средняя точка моста
    (4,2)--(4,4)
    (6,2)--(6,4)
    (4,3)--(6,3)
    %LLL LLL
    (1,4)to[short,*-]++(0,1)to[L](1,6.3)
    (2,3)to[short,*-]++(0,2)to[L](2,6.3)
    (3,2)to[short,*-]++(0,3)to[L](3,6.3)--++(-2,0)   
    %
    (7,4)to[short,*-]++(0,1)to[L](7,6.3)
    (8,3)to[short,*-]++(0,2)to[L](8,6.3)
    (9,2)to[short,*-]++(0,3)to[L](9,6.3)--++(-2,0)

    %мотор
    (5,0)to[short,*-](5,1)
    (5,2)to[short,-*](5,3)
    (5,1.5) circle (0.5cm)
    (5,1.5) node {M}
    
    %core
    (1,6.49) rectangle (9,6.51)
    % LLL первичная сторона
    (4,6.7)to[L]++(0,1.3)
    (5,6.7)to[L]++(0,1.3)
    (6,6.7)to[L]++(0,1.3)
    (4,6.7)--(6,6.7)
    ;\end{circuitikz}
  
  Выбор уравнительного напряжения -- по амплитуде.
  6-ти пульсная схема + 6-ти пульсная = пульсации меньше. А в предыдущей схеме 3-х
  пульсная (4 реактора,мощность на большую амплитуду и низкую частоту) H-схема
  имеет преимущество, потому что один реактор.

  При раздельном управлении
  импулься поступают на одну ... Необходимость использования уравнительных реакторов
  отсутствует. По этому признаку можем отличить раздельное управление.

  По сравнению с предыдущей схемой
  
  \begin{circuitikz}\draw
    % средняя точка моста
    (4,2)--(4,4)
    (6,2)--(6,4)
    (4,3)--(6,3)
    
    %мотор
    (5,0)to[L,*-](5,1)
    (5,2)to[L,-*](5,3)
    (5,1.5) circle (0.5cm)
    (5,1.5) node {M}
    %
    (2,0)--(8,0)
    ;
    \draw[thin,<-] (5.4,0.5)--(6.4,2) node[right]
         {$\begin{array}{c}\textcyrillic{сглаживающие реакторы}\\
             \textcyrillic{ставим здесь или здесь}\end{array}$};
    \draw[thin,<-] (5.4,2.5)--(6.4,2);
  \end{circuitikz}
  
  Отсутствие уравнительных токов и отсутствие реакторов является главным достоинством
  раздельного способа управления.

  Если один выпрямитель а другой инвертор сумма углов 180.
  $$
  \begin{array}{cc}
    \alpha=60 & \alpha=120\\
    cos = \frac{1}{2} &  cos = -\frac{1}{2}
    \end{array}
  $$

  В схеме замещения:
 \begin{circuitikz}\draw
    (0,1) to[Ty] (0,0) -- (2,0) to[Ty] (2,1)
    (0,1.5) node{+}
    (0,2) node{-}
    (2,1.5) node{+}
    (2,2) node{-}
    (0,-0.6)node{напряжение $\frac{1}{2}$ и $\frac{1}{2}$}
    ;\end{circuitikz}

 А как протекает ток? агрузка выбирает куда течь току. Возникает полная аналогия
 с системой Леонардо(генератор+двигатель)
 
 \begin{circuitikz}\draw
   (0,0)circle(0.5)
   (0,0)node{M}
   (0,-0.7)node{противо ЭДС}
   (0,2)circle(0.5)
   (0,2)node{G}
   (-0.5,0)--(-1,0)--(-1,2)--(-0.5,2)
   (0.5,0)--(1,0)--(1,2)--(0.5,2)
   ;\end{circuitikz}   
 
 Двигатель выбирает направление тока.

 Обмоткой возбуждения можно заставить ...

 Недостаток раздельного управление ... ток не протекает ... но из-за управл.

 Плохо, что всегда есть инвертор. потому что принципиально инвертор ненадежен.
 Инверторный двигатель не в состоянии выбрать направление.

 Второй недостаток схемы совместного?? управления -- всегда имеет инверторный режим.
 Достоинство = полная идентичность с системой двигатель+генератор
 

 А при раздельном управлении нужно соответствующее управление.

 Как реализовать раздельное управление?

 \subsection{Принципы построения раздельной системы управления}

 \begin{circuitikz}\draw
   (0,5.6)--++ (0,-0.3)
   (0,4.8) circle (0.5cm)
   (0,4.8) node {$e_{d1}$}
   (-0.5, 4.8) node[left] {$\alpha_1\rightarrow$}
   (0,4.3)--++(0,-0.3)
   (0,1) to[battery1,l=$U_0$]++(0,1)
   to[L,l=$L_1$]++(0,1)
   to[R,l=$R_1$]++(0,1)
   (0,1) to[Do]++(0,-1)
   %  --++(0,-0.5)
   %
   --++(3,0)
   %  --++(0,0.5)
   to[Do]++(0,1)
   to[battery1,l=$U_0$]++(0,1)
   to[L,l=$L_2$]++(0,1)
   to[R,l=$R_2$]++(0,1)
   --++(0,0.3)
   (3,4.8) circle (0.5cm)
   (3,4.8) node {$e_{d2}$}
   (1.5, 4.8) node[right] {$\alpha_2\rightarrow$}
   (3,5.3)--++(0,0.3)
   %
   (3,0)--++(3,0)
   --++(0,0.5)
   (6,1) circle(0.5)
   (6,1) node {$E_\textcyrillic{н}$}
   (6,1.5)--++(0,0.5)
   to[L,l_=$L_\textcyrillic{н}$]++(0,1.5)
   to[R,l_=$R_\textcyrillic{н}$]++(0,2.1)
   --++(-6,0);
   %
   ;\end{circuitikz}
 
 При переходе к раздельному управлению технологи не торопились снимать уравнительные реакторы.
 Инерционность реакторов помогала защите.

 При переключении вентильных групп
 \begin{enumerate}
 \item вначале выключаются импульсы одной вентильной группы, затем включаются на другой;
 \item отключать можно в тот момент когда ток равен нулю. Потому что пока есть ток
   есть инверторный режим нельзя включать. Включение при нулевом токе в нагрузке.
 \item после того как ток спал до нуля, но включать только после паузы превышающей
   время выключения вентилей
 \end{enumerate}

 \begin{enumerate}
 \item вначале отключили, потом включили
 \itemотключение после того как ток $I=0$
 \item выключение спустя $\scriptstyle{\Delta}t >t_\textcyrillic{выключение тиристора}$.
   Есть дополнительное условие: нельзя отключать импульсы в момент формирования импульсов
 \item тиристор открылся а датчик не смог ... определить что ток есть. Запрет отключения
   в момент когда импульс сформировался.
 \item (замена второго пункта, вариант) Вместо контроля за током используется контроль
   состояния всех тиристоров(вентилей)   
 \end{enumerate}

 $10kA$, чтобы гарантировать что тиристоры выключились, $I<I_\textcyrillic{удержания}$,
 100-200mA. Покажет ток утечки тиристоров, а этот ток может быть большим, потому что
 используется несколько тиристоров.
 Ток утечки 18 закрытых тиристоров может быть больше чем ток удержания одного тиристора.
 Определив ток, не знаем что эо за ток: ток удержания одного тиристора или ток утечки всех.

 Датчики ДЗВ (запирания вентилей). Будем контролировать
 $\scriptstyle{\Delta}U_\textcyrillic{тиристора}$. 15 вольт -- гарантия что тиристор заперт.
 На каждом больше 15 вольт -- точно заперт
\begin{figure}[H]
 \begin{tikzpicture}
   \draw[thin,->] (-pi/2,0) -- (3*pi/2+pi/2,0) node[right] {$\omega t$};
   %
   \draw[domain=-pi/2:pi/2]
   plot (\x,{cos(\x r)});
   \draw[domain={0+pi/6}:{pi+pi/6}]
   plot (\x,{cos((\x-pi/2-pi/6) r)});
   \draw[domain={pi/2+pi/3}:{3*pi/2+pi/3}]
   plot (\x,{cos((\x-pi-pi/3) r)});
   % скобки
   \draw[red,thin] (pi/6-0.15,-0.2)--(pi/6-0.15,0.2) (pi/6+0.15,-0.2)--(pi/6+0.15,0.2);
   \draw[thin,<-] (pi/6,0) -- (pi/6,-0.6) node[below] {зоны ложного запрета};
   % отрезки на кривых
   \draw[red,domain={pi/2-0.2}:{pi/2+0.2},very thick]
   plot (\x,{cos(\x r)+0.05});
   \draw[red,domain={pi/2+pi/3-0.2}:{pi/2+pi/3+0.2},very thick]
   plot (\x,{cos((\x-pi-pi/3) r)+0.05});
   \end{tikzpicture}
\end{figure}
Тиристор открыт -- на нем нуловое напряжение. Лучше ложный запрет чем ложное разрешение.

НПЧ -- непосредственный преобразователь частоты.
 

%Завершаем рассмотрение реверсивных преобразователей.
\subsection{Функциональная схема раздельного управления}
\begin{circuitikz}
  \draw[thick,dashed,double,->](0,0)--(2,0);
  \draw[thick,dashed,double,->](0,0)--(0,2)
        node[midway,left]{$U_\textcyrillic{ЗНТ}$};

  \draw
  (2,-0.5) rectangle (3.5,0.5) (2.75,0)node{БРУ}
  (-1,2) rectangle (1,3) (0,2.5)node {САР} (0,3.5)node{(ACP)}
  (1.5,1)rectangle(2.5,5) (2,3)node{ВУ}
  (4,1.5)rectangle(5.5,2.5) (4.75,2)node{СУ2}
  (4,4)rectangle(5.5,5) (4.75,4.5)node{СУ1};
%  \draw[->] (4.75,1)--(4.75,1.5);
%  \draw[->] (4.75,3.5)--(4.75,4);
  \draw[->] (2.5,2)--(4,2)node[midway,above]{$U_\textcyrillic{упр2}$};
  \draw[->] (2.5,4.5)--(4,4.5)node[midway,above]{$U_\textcyrillic{упр1}$};
  \draw[->] (3.5,0)--(3.75,0)--(3.75,3.25)--
  (4.75,3.25)node[midway,above]{$u_1$}--(4.75,4);
  \draw[->] (3.5,0)--(3.75,0)--(4.75,0)node[midway,above]{$u_2$}--(4.75,1.5);
  \draw[->,dashed] (5.5,4.5)--(7,4.5)node[midway,above]{$\alpha_1$}; 
  \draw[->,dashed] (5.5,2)--(9.5,2)--(9.5,4.5)--(11,4.5)node[midway,above]{$\alpha_2$};
  \draw[->] (1,2.5)--(1.5 ,2.5);
  % выпрямитель-инвертор
  \draw
  (7,4)rectangle(9,5)
  (7,5)node[above]{ВГ1}
  (8.5,4.5)to[Do](7.5,4.5)--(8.5,4.5)
  (7.87,4.34)--(7.70,4.17)
  (11,4)rectangle(13,5)
  (13,5)node[above]{ВГ2}
  (12.5,4.5)to[Do](11.5,4.5)--(12.5,4.5)
  (11.87,4.34)--(11.70,4.17)
  %соединения от выпрямителя-инвертора
  
  (7.5,4)--(7.5,1.5)to[L](7.5,0)
  to[american controlled current source,l=$\textcyrillic{ДТ}_1$](10,0)
  to[american controlled current source,l=$\textcyrillic{ДТ}_2$](12.5,0)  
  (8.5,4)--(8.5,3.5)--(11.5,3.5)--(11.5,4)
  (12.5,4)--(12.5,1.5)to[L](12.5,0)
  (8,5)--(8,5.5) -- (10-0.433*21/12,6.65-21/12*0.25)
  (8,5.8)node[above]{m фаз}
  (12,5)--(12,5.5)--(10+0.433*21/12,6.65-21/12*0.25)
  (12,5.8)node[above]{m фаз}
  (10,6.65+0.82)to[ospst,l_=QF](10,8.7)--++(-3,0) % к трехфазной сети
  (8,8.5)--(8.4,8.9)
  (7.8,8.5)--(8.2,8.9)
  (7.6,8.5)--(8.0,8.9)
  %мотор  (10,3.5)--(10,1.5)to[short](10,0) (10,1.25)node[component]{M}
  (10,3.5)to[short,*-](10,1.5)to[motor,-*](10,0);
  \draw[dotted](7.5,0.75)circle(0.5); % выбрасываем L уравнительный
  \draw[thin](7,0.25)--(8,1.25);
  \draw[dotted](12.5,0.75)circle(0.5); % выбрасываем L уравнительный
  \draw[thin](12,0.25)--(13,1.25);
  \draw[->] (8.75,-0.3)--(8.75,-0.75)--(3,-0.75)--(3,-0.5);
  \draw[->] (11.25,-0.3)--(11.25,-1)--(2.5,-1)--(2.5,-0.5);
  %трансформатор (10, 6.65) (-0.433,-0.25)
  \draw({10-0.433*2/3},{6.65-0.25*2/3}) circle(0.5)
  ({10+0.433*2/3},{6.65-0.25*2/3}) circle(0.5)
  (10,{6.65+0.5*2/3}) circle(0.5) 
  ;\end{circuitikz}

Трансформатор имеет особенность преобразовывать число фаз. У Каждой вентильной группы ({\it ВГ}) своя
система управления СИФУ ({\it СУ}) или система управления вентилями ({\it СУВ}).
Каждая ${\textcyrillic{\it СУ}}_N$ генерирует
сигналы с углом $\alpha_N$.
Импульсов должно быть {\it m}. Уравнительные реакторы используются только в случае совместного управления.
На схеме уравнительные реакторы обведены кружками, чтобы подчеркнуть, что в раздельном управлении они не
обязательны. При совместном управлении должно выполнятся условие $\alpha_1+\alpha_2>\pi$, в противном
случае протекает большой уравнительный ток.

\begin{circuitikz}
  \draw (4,1.5)rectangle(5.5,2.5) (4.75,2)node{СУ2};
  \draw[red,<-] (4.75,1.5)--(4.75,1) node[right,color=black]{-- вход для выключения СИФУ.};
\end{circuitikz}

\begin{circuitikz}
  \draw
  (2,-0.5) rectangle (3.5,0.5) (2.75,0)node{БРУ};
  \draw[<-] (2.2,-0.5)--(2.2,-1) node[below]{$\textcyrillic{ДТ}_1$};
  \draw[<-] (3.3,-0.5)--(3.3,-1) node[below]{$\textcyrillic{ДТ}_2$};
  \draw[dashed,double,->] (0,0)--(2,0);
\end{circuitikz}

    {\it БРУ} -- блок раздельного управления. Токи управления от датчиков тока
    $\textcyrillic{ДТ}_1$ и  $\textcyrillic{ДТ}_2$ (10-100mA, мах 10-15V).
    {\it БРУ} $\equiv$ {\it ЛПУ} -- он же логическое переключающее устройство.
  Стрелкой \begin{circuitikz}
    \draw[dashed,double,->] (0,0)--(1,0);\end{circuitikz} обозначен сигнал инициирующий работу
  (вначале отключить, затем включить после паузы).

\begin{figure}[H]  
  \begin{tikzpicture}[scale=1]
  \draw[thin,->,domain={pi/3}:{2*pi},samples=100,red]
  plot (canvas polar cs:angle=\x r,radius= {30*20/sqrt((30*cos(\x r))^2 +(20*sin(\x r))^2)})
  node[above right] {$I_1$};  
  \draw(-0.6,1.2)node {$\textcyrillic{ВГ}_1$};
  \draw(0.5,-1.1) to[motor] (0.5,-0.1);
  \end{tikzpicture}
  \hspace{2cm}
  \begin{tikzpicture}
  \draw[thin,->,domain={pi/2}:{2*pi-pi/4},samples=100,red]
  (4,0) plot (canvas polar cs:angle=\x r,radius= {30*20/sqrt((30*cos(\x r))^2 +(20*sin(\x r))^2)})
  node[above right] {$I_2$};
  \draw(-0.5,-1.1) to[motor] (-0.5,-0.1);
   \end{tikzpicture}
  \caption{направление тока} 
\end{figure}

Одновременно токи $I_1$ и $I_2$ не могут существовать, должна быть пауза, чтобы тиристоры
выключились.

Для измерения переменного тока существует трансформатор тока, изолированный от силовой цепи.
Существуют ``трансформатор постоянного тока'' построенный на принципе подмагничивания.
Чаще всего используется токоизменительный шунт.
%\begin{tikzpicture}
%\end{tikzpicture}
Исторически шунты расчитывались на протекание тока $45mV$, в настоящее время
расчитываются на $75mV$. С помощью шунта превратили сигнал тока в напряжение. Чтобы не было
ошибок в динамике нужно чтобы шунт был безиндуктивный. Представим, что через
преобразователь проходит ток $10kA$, или 1000А, тогда даже миливольты превращаются
в большее ватты.

Заострим проблему: Сумма токов утечек например 10 включенных параллельно вентилей
может превысить ток удержания. Выключаются вентили последовательно, и наконец остается
один последний вентиль в котором
$$
\begin{array}{ccc}
  10kA &-& 75mV\\
  10mA&-&
\end{array}
$$
величина может превысить порог чувствительности для удержания одного.

({\it АСР}) -- автоматическая система регулирования по ГОСТу. ({\it САР}) --
система автоматического регулирования.

\begin{circuitikz}
  \draw[->,dashed,double] (0,0)--(0,1) node[midway,right]{$U_\textcyrillic{знт}$}; 
\end{circuitikz} -- сигнал заданного направления тока, тоже логический (0 или 1)
Отключать можно тогда, когда $I=0$.
Увеличиваем $\alpha$, ток $I$ падает до нуля, в этот момент {\it БРУ} отключает импульсы,
и это же побудительный момент для перемены направления тока. Не показаны выдержки
времени. Кроме того есть запрет на переключения в момент формирования импульсов

\begin{circuitikz}
  \draw[<-,dashed](0,0)--(1,0)--(1,1) node{$\alpha_1$};
  \draw[<-,dashed](0,-0.3)--(2,-0.3)--(2,0.5)node{$\alpha_2$};
  \draw[<-,thin] (2.1,-0.1)--(3,-0.1)node[right]
       {$\begin{array}{c}
           \textcyrillic{сам импульс }\alpha_2\\
           \textcyrillic{запрещает себя выключать}
        \end{array}$};
  \end{circuitikz}

функция необязательная но часто используеммая. Ненадежность токовой логики
шунта с усилителем. Возникает проблема: ток перегрузки двигателя 250\% номинала.
Контролируемый ток 1/1000 доля номинала. Вместо датчика тока используются датчики
напряжения 1-2...4В -- падение напряжения в открытом состоянии.
Вместо датчика тока используются датчик запертого состояния тиристора.
Бывает, что датчик говорить ``0'', но это может оказаться переменный ток проходящий
через нуль. Схемы управления бывают аналоговыми, цифроаналоговыми
Когда система управления реализуется аппаратно нужны ли две систмемы управления.
У трансформатора не работает. Убрали уравнительные реакторы, для этого и была нужна
система раздельного управления, чтобы оптимизировать.
Зачем два СИФУ? Аппаратно и програмно алгоритм может быть выполнен с одной
{\it СУ}. Тогда вместо двух систем импульсов(включения и выключения) используется
переключатель между двумя СИФУ. Если СИФУ одно, $\alpha_1\downarrow$
$\alpha_2\uparrow$ (переключение и на выходе и на входе). Обязательно должен
предусмотреть переключение на входе.

\subsection{Внешние характеристики реверсивных преобразователей}
До этого рисовали внешние характеристики в 2х квадрантах, сейчас нарисуем в
4х квадрантах

\hspace{-2cm}
\begin{tikzpicture}
  \begin{scope}[xscale=1.3,yscale=3]
    \newcommand{\betamin}{160/180}
    % axis x,y
    \draw[thin, ->] (0, 0) -- (6,0) node[right]
         {${\displaystyle \frac{I_d}{I_{d0}}}$};
    \draw[thin, ->] (0, -1) -- (0,1.2) node[left]
         {${\displaystyle \frac{U_d}{E_{d0}} }$};
         \draw[thin, loosely dashed] (0,-1) -- (6,-1);
    % ylabel
    \foreach \y/\ytext in {-1/-1,{-\betamin}/\beta_{min},0/0,1/1}
    \draw (0.1,\y) -- (-0.1,\y) node[left] {$\ytext$};
    % \beta_min
    \draw[thin,loosely dashed] (0, {-\betamin}) -- (6, {-\betamin+0.1});
    \node at (0.3,-0.4) {$'1-0'$};

    % eclipse (1-1)
    \draw[domain=0:0.7, help lines,dotted, smooth]
    plot (\x,{sqrt(1-\x*\x/0.49)});
    \draw[domain=0:0.7, help lines,dotted, smooth]
    plot (\x,{-sqrt(1-\x*\x/0.49)});
    %ellipse слева
    \draw[domain=-0.7:0, help lines,dotted, smooth]
    plot (\x,{sqrt(1-\x*\x/0.49)});
    \draw[domain=-0.7:0, help lines,dotted, smooth]
        plot (\x,{-sqrt(1-\x*\x/0.49)});
% наклон    \draw[thin] (0,1) -- (6,0.8);
    \draw[thin] (0,1) -- (4.76,1-1/30*4.76);
    % 0.2/6*180/pi
%    \draw[domain=4.76:6.79, help lines, smooth]
%    plot (\x, {-0.4*\x*\x +2*0.4*4.76*\x + 1 -1/30*4.76 -0.4*4.76*4.76
%    });


    \node[rotate=-4.45] at (3,1) {$\alpha=0^\circ$};
    \draw[thin] ({sqrt(0.49*(1-0.67*0.67))},{0.67-1/30}) -- (5.2, 0.67-1/30*5.2);
    % рисуем параболу ax^2 + bx + c
    % выбираем a=1
    % из f'(x)=0 = 2ax+b => b=-2 a x0
    % a x0^2 + b x0 + c = y0
    % a x0^2 -2 a x0^2 +c = y0
    % c = y0+a x0^2
    \draw[domain=0:{sqrt(0.49*(1-0.67*0.67))}, help lines, smooth]
    plot (\x, {\x*\x -2*sqrt(0.49*(1-0.67*0.67))*\x + 0.67 +
      0.49*(1-0.67*0.67 ) -0.0335
     });

    \draw[thin] ({-sqrt(0.49*(1-0.67*0.67))},{0.67+1/30}) -- (-5.2, 0.67+1/30*5.2);
%    \draw[domain=0:{sqrt(0.49*(1-0.67*0.67))}, help lines, smooth]
%    plot (\x, {\x*\x -2*sqrt(0.49*(1-0.67*0.67))*\x + 0.67 +
%      0.49*(1-0.67*0.67 ) -0.0335
%    });
    


    \node[rotate=-4.45] at (3.1,0.67) {$\alpha=30^\circ$};
    \draw[thin] ({sqrt(0.49*(1-0.33*0.33))},{0.33-1/30}) -- (5.54, 0.33-1/30*5.54);
    \draw[domain=0:{sqrt(0.49*(1-0.33*0.33))}, help lines, smooth]
    plot (\x, {1.2*\x*\x -2*1.2*sqrt(0.49*(1-0.33*0.33))*\x + 0.33 +
      1.2*0.49*(1-0.33*0.33 ) -0.0335
    });
    \draw[thin] ({-sqrt(0.49*(1-0.33*0.33))},{0.33+1/30}) -- (-5.54, 0.33+1/30*5.54);
%    \draw[domain=0:{sqrt(0.49*(1-0.33*0.33))}, help lines, smooth]
%    plot (\x, {1.2*\x*\x -2*1.2*sqrt(0.49*(1-0.33*0.33))*\x + 0.33 +
%      1.2*0.49*(1-0.33*0.33 ) -0.0335
%    });
    


    \node[rotate=-4.45] at (3.2,0.33) {$\alpha=60^\circ$};
    \draw[thin] ( 0.7, {0-1/30})  -- (5.8, -1/30*5.8);
    \draw[domain=0:{sqrt(0.49*(1-0*0))}, help lines, smooth]
    plot (\x, {1.4*\x*\x -2*1.4*sqrt(0.49*(1-0.*0.))*\x + 0. +
      1.4*0.49*(1-0.*0. ) -0.0335
    });
    \draw[thin] ( -0.7, {0+1/30})  -- (-5.8, 1/30*5.8);
    
    \node[rotate=-4.45] at (5.2,-0.1) {$\alpha=90^\circ$};
    \draw[thin] ({sqrt(0.49*(1-0.33*0.33))},{-0.33-1/30}) -- (5.92, -0.33-1/30*5.92);
    \draw[domain=0:{sqrt(0.49*(1-0.33*0.33))}, help lines, smooth]
    plot (\x, {1.8*\x*\x -2*1.8*sqrt(0.49*(1-0.33*0.33))*\x -0.33 +
      1.8*0.49*(1-0.33*0.33 ) -0.0335
          });
   \draw[thin] ({-sqrt(0.49*(1-0.33*0.33))},{-0.33+1/30}) -- (-5.92, -0.33+1/30*5.92);
    
    \node[rotate=-4.45] at (3.4,-0.33) {$\alpha=120^\circ$};


    \draw[thin] ({sqrt(0.49*(1-0.66*0.66))},{-0.66-1/30}) -- (5,  {-0.66-0.16});
    \draw[domain=0:{sqrt(0.49*(1-0.67*0.67))}, help lines, smooth]
    plot (\x, {3*\x*\x -2*3*sqrt(0.49*(1-0.67*0.67))*\x -0.67 +
      3*0.49*(1-0.67*0.67 ) -0.028
    });
    \draw[thin] ({-sqrt(0.49*(1-0.66*0.66))},{-0.66+1/30}) -- (-5,  {-0.66+0.16});

    \node[rotate=-4.45] at (3.5,-0.66) {$\alpha=150^\circ$};

    \node[rotate=2] at (3.5,-0.9)
         {\textcyrillic{граница устойчивости инверторного режима}};
  \end{scope}
\end{tikzpicture}

Говорили, что область внутри эллипса это область прерывистого тока.
По умолчанию все углы в радианах. $\beta_{min}(\gamma,\delta,\psi)$,
$\alpha_{max} = 150^\circ-160^\circ$.
Забыли про правую полуплоскостью На левой--другой преобразователь.
По отношению к нагрузке этот преобразователь включён ``наоборот''?

Симметрично относительно начала координат внешние характеристики
реверсивных преобразователей с раздельным управлением.

А что с совместным управлением? $\alpha_1+\alpha_2>180^\circ$.
Не учёл падение от активных сопротивлений и $U_0$ и не рисуем область
многовенлильной коммутации.

При совместном управлении различают два способа управления:
\begin{itemize}
\item совместное согласованное  $\alpha_1+\alpha_2\approx \pi$
  с максимально доступной точностью.
  \item совместное несогласованное, когда  $\alpha_1+\alpha_2>\pi$
  \end{itemize}

В инженерном плане равенство $=\pi$ невозможно точно измерить и оно
легко может нарушиться из-за нестабильности. Если же заведомо не стремиться
к равенству нулю, тогда будет заведомо большой уравнительный ток.
<Если импульсы не отключаются то правая ЭДС присутствтвует слева?>

\begin{circuitikz}\draw
  (0,0)to[L](1,0)to[european resistor](2,0)--(2.2,0)(2.5,0)circle(0.3)
  (2.8,0)--(3,0)to[ammeter](4,0)--(4,1.5)
  (2.5,0)node{$E_m$}
  (4,1.5)--
  (4,3)to[ammeter,mirror](3,3)--(2.8,3)(2.5,3)circle(0.3)(2.2,3)--(2,3)
  to[Ty](1,3)to[L](0,3)--(0,0)
  (2.5,3)node{$E_{d1}$}
  (0,1.5)to[L,*-](1,1.5)to[Ty](2,1.5)--(2.2,1.5)
  (2.5,1.5)circle(0.3)(2.8,1.5)--(3,1.5)to[ammeter,-*](4,1.5)
  (2.5,1.5)node{$E_{d2}$};
  \draw[red,->](3,2.7)--(0.8,2.7)arc(90:270:0.45)--(3,1.8);
  \draw[thin,<-](0.3,2.25)--(-0.4,2.25) node[left]
       {$\begin{array}{c}\textcyrillic{уравнительный ток}\\
         \textcyrillic{всегда в одну сторону}\end{array}$}
;\end{circuitikz}

Уравнительный ток всегда идет минуя нагрузку. Рассмотрим пример
$$
\begin{array}{c}
  60A\leftarrow\\
  80A\rightarrow\\
  20A\leftarrow
\end{array}
$$
Какой уравнительный ток? Уравнительный ток равен $60A$
$$
\begin{array}{c}
  350A\leftarrow\\
  90A\rightarrow\\
  260A\rightarrow
\end{array}
$$

Меньший из двух вентильных токов -- уравнительный. А разность есть ток нагрузки.
Увеличили угол значит падение напряжения на вентидлях уменьшились?
Уравнительный ток большой, значит углы раздвигаются.

Совместное управление не миф, а может быть реализовано достаточно точно.

На графике изображен ток в нагрузке, но есть уравнительный ток.
Для реверсивного преобразователя
\begin{tikzpicture}
  \draw[domain=0:1]
  plot(\x,{(\x-1)^2+1});
\end{tikzpicture}
Внутри эллипса ток собственно в вентильной группе.

Поэтому характеристики спрямляются

\begin{tikzpicture}
  \begin{scope}[xscale=1.5,yscale=2]
  \draw[thin,->] (-2,0)--(2,0) node[right]{$\omega t$};
  \draw[thin,->] (0,-0.2)--(0,1);
  \draw[domain=-1.5:-0.25]
  plot(\x,-0.2*\x+0.5);
  \draw[domain=-0.25:0.25]
  plot(\x,-0.4*\x+0.45);
  \draw[domain=0.25:1.5]
  plot(\x,-0.2*\x+0.4);
  \end{scope}
\end{tikzpicture}

Дадим оценку тому что прошли: совместное управление мы его обругали,
но есть спрямление характеристик в области малых токов.
В теории автоматического управления электроприводами.  
Нужен измеритель  уравнительного тока. В одной вентильной группе сумма
уравнительного тока плюс нагрузка, в другом только уравнительный ток.
Спрямление характеристик -- достоинство совместного управления.

\subsection{Пульсации выпрямленного напряжения и тока}
Выпрямленное с дефектами, пульсациями. Нужно количественно оценить амплитуду
пульсаций. Как определить амплитуду переменной составляющей?

\begin{tikzpicture}
  \draw[red](-pi,0.65)--(pi,0.65);
  \draw[domain=-pi:pi,help lines,smooth]
  plot(\x, {cos(\x r)})
  plot(\x, {cos((\x-pi/3) r)})
  plot(\x, {cos((\x+pi/3) r)})
  plot(\x, {cos((\x-2*pi/3) r)})
  plot(\x, {cos((\x+2*pi/3) r)})
  plot(\x, {cos((\x+pi) r)});
  \draw[domain=-pi+0.3:-2*pi/3+0.3-pi/6,red,pattern=north east lines,
  pattern color=red]
  (-pi+0.3, 0.65) --
  plot(\x, {cos((\x+pi) r)});
  \draw[domain=-pi+0.3+pi/6:-2*pi/3+0.3,red,pattern=north east lines,
  pattern color=red]
  plot(\x, {cos((\x+pi) r)})
  -|({-2*pi/3+0.3}, 0.65);
  
  \draw[domain=-2*pi/3+0.3:-pi/3+0.3-pi/6,red,pattern=north east lines,
  pattern color=red]
  (-2*pi/3+0.3, 0.65) --
  plot(\x, {cos((\x+2*pi/3) r)});
  \draw[domain=-2*pi/3+0.3+pi/6:-pi/3+0.3,red,pattern=north east lines,
  pattern color=red]
  plot(\x, {cos((\x+2*pi/3) r)})
    -|({-pi/3+0.3}, 0.65);
  
  \draw[domain=-pi/3+0.3:0*pi/3+0.3-pi/6,red,pattern=north east lines,
  pattern color=red]
  (-pi/3+0.3, 0.65) --
  plot(\x, {cos((\x+pi/3) r)});
  \draw[domain=-pi/3+0.3+pi/6:0*pi/3+0.3,red,pattern=north east lines,
  pattern color=red]
  plot(\x, {cos((\x+pi/3) r)})
  -|({0*pi/3+0.3}, 0.65);
  

  \draw[domain=0*pi/3+0.3:pi/3+0.3-pi/6,red,pattern=north east lines,
  pattern color=red]
  (0*pi/3+0.3, 0.65) --
  plot(\x, {cos((\x) r)});
  \draw[domain=0*pi/3+0.3+pi/6:pi/3+0.3,red,pattern=north east lines,
  pattern color=red]
  plot(\x, {cos((\x) r)})
  -|({pi/3+0.3}, 0.65);
    

  \draw[domain=pi/3+0.3:2*pi/3+0.3-pi/6,red,pattern=north east lines,
  pattern color=red] 
  (pi/3+0.3,0.65) --
  plot(\x, {cos((\x-pi/3) r)});
  \draw[domain=pi/3+0.3+pi/6 : 2*pi/3+0.3,red,
      pattern=north east lines,pattern color=red]
  plot(\x, {cos((\x-pi/3) r)})
  -|({2*pi/3+0.3}, 0.65);
  \draw[red]
  ({-pi+5*pi/3+0.3}, 0.65) -- ({-pi+5*pi/3+0.3}, {cos((-pi+3*pi/3+0.3) r)});  
  ;

  % http://tex.stackexchange.com/questions/54464/hatch-a-rectangle-in-tikz
  %  \draw [thick,pattern=north west lines, pattern color=red] (1,0)--(1,1) to [bend left] (4,4) -- (4,0) --cycle;
  \end{tikzpicture}

Можно представить рядом Фурье m-пульсаций.
$$
f_\textcyrillic{гармоники} = k(mf_c)
$$
где $k=1...k...$, $f_C$ -- частота сети

Найдем $U_k$ гармоники.

При $m=6$, $f_c=50\textcyrillic{Гц}$
$$
\begin{array}{ccc}
  f_\textcyrillic{гармоники}&=&300\\
  &&600\\
  &&900\\
  &&1200
\end{array}
$$

Допустим, мы знаем как выбрать фильтр? Нам нужно фильтровать гармоники тока,
а не гармоники $U$. Зная $U_k$ найти $I_k$, затем сложить. Неблагодарная
задача. Обычно берут первую гармонику, учитывают с коэффициэнтом запаса.

$$
E_{dkm} = \frac{\sqrt{2}E_{d0}}{(km)^2-1}
\sqrt{cos^2\alpha + (km \:sin\alpha)^2}
$$
Чем больше $\alpha$ гармоника возрастает. Максимальная величина гармоники
при $\displaystyle \alpha=\frac{\pi}{2}$, Это понятно из графика и из
формулы

$$
(E_{dkm})_{max} = \frac{km}{(km)^2-1} \sqrt{2} E_{d0}
$$
С ростом $m$ гармоники убывают.

Можно считать, что гармоники обратно пропорционально частоте.

$$
E_{\sim} = \frac{m}{(m)^2-1} \sqrt{2} E_{d0}
$$
$$
f = mf_c
$$
С коэффициэнтом запаса 10-20\%.
Если $\alpha$ не доходит до $90^\circ$, тогда считают для максимального $\alpha$.

Ток возбуждения $I_{min}$, нулём никогда не бывает. По заданному току

$$
I_\sim = \frac{E_\sim}{\omega L_\Sigma}=
$$
   $L_\Sigma$ по всей цепи и постоянного и переменного тока

$$
=\frac{\cancel{m}\sqrt{2}E_{d0}}{(m^2-1)2\pi \cancel{m}F_cL_\Sigma}
(m\textcyrillic{ можно сократить})
$$

$$
L_\Sigma = L_\phi + L_\textcyrillic{н} + L_\Phi
$$
Возникает задача. Если нужно ограничить до заданной величины.

Рассмотрим другой способ решения, нравится больше, но тоже приближенный.

Рассмотрим небольшую задачу:

\begin{circuitikz}\draw
  (0,1)to[battery1,*-*](0,2)
  (0,1)--(1,1)to[voltmeter,l_=$V_1$](1,2)--(0,2)
  (0,1)--(0,0)--(2,0)
  (3,0)to[L](2,0)
  (2,-0.5)--(2,-1.5)--(2.24,-1.5)
  (2,-0.5)to[L](3,-0.5)--(3,-1.5)--(2.76,-1.5)
  (2.5,-1.5)circle(0.26)
  (2.5,-1.5)node{G}
  (2.5,-1.76)node[below]{100V}
  (2,-1.5)node[left]{$f=100\textcyrillic{Гц}$}
  (2,-0.24)rectangle(3,-0.26)
  (0,2)--(0,3)--(3,3)
  (3,0)to[voltmeter,l=$V_2$,*-*](3,3)
  %
  (3,3)to[L](4,3)
  (3,3.24)rectangle(4,3.26)
  (4,4)--(4,3.5)to[L](3,3.5)--(3,4)
  (2.9,3.9)--(3,4)--(3.1,3.9)
  (3.9,3.9)--(4,4)--(4.1,3.9)
  (2.9,4)--(3,4.1)--(3.1,4)
  (3.9,4)--(4,4.1)--(4.1,4) node[right]{$50\textcyrillic{Гц}$}
  (3,4.1)--(3,4.5) node[above right]{$\sim 100V$}
  (4,4.1)--(4,4.5)
  %
  (3,0)--(4,0)
  (5,0)to[L,-*](4,0)
  (4,-0.5)--(4,-1.5)--(4.24,-1.5)
  (4,-0.5)to[L](5,-0.5)--(5,-1.5)--(4.76,-1.5)
  (4,-0.24)rectangle(5,-0.26)
  (4.5,-1.5)circle(0.26)
  (4.5,-1.5)node{G}
  (4.5,-1.76)node[below]{100V}
  (5,-1.5)node[right]{$f=200\textcyrillic{Гц}$}
  (4,0)to[voltmeter,l_=$V_3$,*-*](4,3)
  %
  (5,0)--(7,0)
  (4,3)--(7,3)
  (7,0)to[voltmeter](7,3)
  ;
  \draw[thin,<-] (7.26,1.5)--(8,1.5)
  node[right]{$\begin{array}{c}\textcyrillic{действующее значение}\\
      \textcyrillic{не магнитоэлектрический}\end{array}$};
\end{circuitikz}

Если частоты не одинаковы, то действующее значение равно квадратному корню из
суммы квадратов.

Оценим
$$
E_d=E_{d0}\;cos \alpha
$$
Не учтен угол коммутации в формуле. При больших $\alpha$ угол $\gamma$ уменьшается.
$\gamma$ вызван индуктивностью, для высших гармоник это сопротивление -- гармоники будут
уменьшаться. Должен учитывать худший случай. Поэтому сейчас угол коммутации игнорируем.
Найдем среднеквадратичную составляющую:
$$
\sqrt{2} E_{2\phi} = \frac{E_{d0}}{\frac{m}{\pi} sin \frac{\pi}{m}}
$$
из формулы (1)

$$
E_{d\textcyrillic{средне квадратичное}} = \sqrt{
  \frac{1}{2\pi/m}\int\begin{array}{c}\textcyrillic{от квадрата}\\
  \textcyrillic{мгновенного значения}\end{array}} =
$$

$E_{d\textcyrillic{средне квадратичное}}$ -- это на активную нагрузку(физический смысл)

$$
=\sqrt{
  \frac{1}{2\pi/m}\int\limits_{-\frac{\pi}{m}+\alpha}^{\frac{\pi}{m}+\alpha}
  \frac{E_{d0}}{\frac{m}{\pi} sin \frac{\pi}{m}} cos \omega t}
=
\frac{E_{d0}}{\sqrt{2}\frac{m}{\pi} sin \frac{\pi}{m}}
\sqrt{1+\frac{m}{2\pi}sin\frac{2\pi}{m}cos2\alpha}
$$

$$
E_{d\nu} = \sqrt{E^2_{d\textcyrillic{ср.кв.}} - E_d^2} =
$$

$E_{d\nu}$ -- действующее значение всех гармоник на стороне постоянного тока. $E_d$ -- $E_{d\textcyrillic{среднее}}$

$$
= E_{2\phi}\sqrt{1-\frac{m}{\pi}sin\frac{\pi}{m}
  \left[2\frac{m}{\pi}sin\frac{\pi}{m}\left(cos\alpha\right)^2 -
    cos\frac{\pi}{m}cos2\alpha\right]}
$$

Вместо $E_{2\phi}$ можно подставить
$\displaystyle \frac{E_{d0}}{\sqrt{2}\frac{m}{\pi}sin\frac{\pi}{m}}$, можно преобразовать,
в разной форме можно записать, но эта форма понятнее.

Напряжение может интересовать только при чисто активной нагрузке.

Наибольшие пульсации имеет граничный режим, который можно понимать как случай когда
постоянная составляющая равна пульсациям
\begin{tikzpicture}
  \begin{scope}[scale=0.5]
  \draw[domain=-2*pi:2*pi]
  plot(\x, {sin(\x r)+1.5});
  \draw[thin,loosely dotted](-2*pi,1.5)--(2*pi,1.5)node[right]{};
  \draw[thin,->](-2*pi,0)--(2*pi,0)node[right]{$\omega t$};
  \draw[thin,->](-2*pi+0.2,1.5)--(-2*pi+0.4,0.1) node[below]
       {$\textcyrillic{если среднее будет уменьшаться}$};
\end{scope}
\end{tikzpicture}

Переменная составляющая не влияет на постоянную. Зависит от $\alpha$, но зависит еще от
ЭДС нагрузки. если среднее будет уменьшаться то это никак не отразится на переменной
составляющей.

Прерывистый режим начнётся, когда переменная составляющая коснётся 0. Это и будет
граничный режим.

\begin{tikzpicture}
  \begin{scope}[scale=0.5]
    \draw[domain=-2*pi:2*pi]
    plot(\x, {sin(\x r)+1});
    \draw[thin,loosely dotted](-2*pi,1)--(2*pi,1)node[right]{};
    \draw[thin,->](-2*pi,0)--(2*pi,0)node[right]{$\omega t$};
  \end{scope}
\end{tikzpicture}

6-ти пульсная кривая.

\begin{tikzpicture}
  \begin{scope}[scale=1.5]
    \draw[very thin,->] (-pi-0.1,0) -- (pi+0.1,0) node[right] {$\omega t$};
    \draw[thin,->] (-2*pi/3,-0.1)--(-2*pi/3,1.2) node[left] {$U$};
    
  %  \draw[red](-pi,0.65)--(pi,0.65);
  \draw[domain=-pi/2-0.1:pi/2+0.1,help lines,smooth]
  plot(\x, {cos(\x r)});
  \draw[domain=-pi/2+pi/3-0.1:pi/2+pi/3+0.1,help lines,smooth]
  plot(\x, {cos((\x-pi/3) r)});
  \draw[domain=-pi/2-pi/3-0.1:pi/2-pi/3+0.1,help lines,smooth]
  plot(\x, {cos((\x+pi/3) r)});
  \draw[domain=-pi/2+2*pi/3-0.1:pi-pi/4+0.1,help lines,smooth]  
  plot(\x, {cos((\x-2*pi/3) r)});
  \draw[domain=-pi-0.1:-2*pi+pi/2+2*pi/3+0.1,help lines,smooth]
  plot(\x, {cos((\x-2*pi/3) r)});
  \draw[domain=-pi-0.1:pi/2-2*pi/3+0.1,help lines,smooth]  
  plot(\x, {cos((\x+2*pi/3) r)});
%    \draw[domain=2*pi-pi/2-2*pi/3-0.1: pi-pi/4+0.1,help lines,smooth]
%    plot(\x, {cos((\x+2*pi/3) r)});
  \draw[domain=-pi-0.1:-pi/2+0.1,help lines,smooth]  
  plot(\x, {cos((\x+pi) r)});
  \draw[domain=pi/2-0.1:pi-pi/4+0.1,help lines,smooth]
    plot(\x, {cos((\x+pi) r)});
  
  \draw[red] (-pi+0.3-pi/24,0.65)-- (-pi+0.3+pi/24,0.65);
  
  \draw[domain=-pi+0.3+pi/24:-2*pi/3+0.3-pi/6,red,pattern=north east lines,
    pattern color=red]
  (-pi+0.3+pi/24, 0.65) --
  plot(\x, {cos((\x+pi) r)});
  \draw[domain=-pi+0.3+pi/6:-2*pi/3+0.3-pi/24,red,pattern=north east lines,
    pattern color=red]
  plot(\x, {cos((\x+pi) r)})
  -|({-2*pi/3+0.3-pi/24}, 0.65);

  \draw[red] (-2*pi/3+0.3-pi/24,0.65)-- (-2*pi/3+0.3+pi/24,0.65); 

  \draw[domain=-2*pi/3+0.3+pi/24:-pi/3+0.3-pi/6,red,pattern=north east lines,
    pattern color=red]
  (-2*pi/3+0.3+pi/24, 0.65) --
  plot(\x, {cos((\x+2*pi/3) r)});
  \draw[domain=-2*pi/3+0.3+pi/6:-pi/3+0.3-pi/24,red,pattern=north east lines,
    pattern color=red]
  plot(\x, {cos((\x+2*pi/3) r)})
  -|({-pi/3+0.3-pi/24}, 0.65);

 \draw[red] (-pi/3+0.3-pi/24,0.65)-- (-pi/3+0.3+pi/24,0.65);  
  
  \draw[domain=-pi/3+0.3+pi/24:0*pi/3+0.3-pi/6,red,pattern=north east lines,
    pattern color=red]
  (-pi/3+0.3+pi/24, 0.65) --
  plot(\x, {cos((\x+pi/3) r)});
  \draw[domain=-pi/3+0.3+pi/6:0*pi/3+0.3-pi/24,red,pattern=north east lines,
    pattern color=red]
  plot(\x, {cos((\x+pi/3) r)})
  -|({0*pi/3+0.3-pi/24}, 0.65);
  
 \draw[red] (0*pi/3+0.3-pi/24, 0.65)--(0*pi/3+0.3+pi/24, 0.65);
  
  \draw[domain=0*pi/3+0.3+pi/24:pi/3+0.3-pi/6,red,pattern=north east lines,
    pattern color=red]
  (0*pi/3+0.3+pi/24, 0.65) --
  plot(\x, {cos((\x) r)});
  \draw[domain=0*pi/3+0.3+pi/6:pi/3+0.3-pi/24,red,pattern=north east lines,
    pattern color=red]
  plot(\x, {cos((\x) r)})
  -|({pi/3+0.3-pi/24}, 0.65);

 \draw[red] (pi/3+0.3-pi/24,0.65) -- (pi/3+0.3+pi/24,0.65);
  
  \draw[domain=pi/3+0.3+pi/24:2*pi/3+0.3-pi/6,red,pattern=north east lines,
    pattern color=red]
  (pi/3+0.3+pi/24, 0.65) --
  plot(\x, {cos((\x-pi/3) r)});
  \draw[domain=pi/3+0.3+pi/6 : 2*pi/3+0.3-pi/24,red,
    pattern=north east lines,pattern color=red]
  plot(\x, {cos((\x-pi/3) r)})
  -|({2*pi/3+0.3-pi/24}, 0.65);

  % ток
  \draw[thin,->] (-pi-0.1,-1)--(pi+0.1,-1) node[right] {$\omega t$};
  \draw[thin,->] (-2*pi/3,-1.1)--(-2*pi/3,-0.2) node[left] {$I$};
  \draw[red,domain=-pi/3+0.3+pi/24: 0*pi/3+0.3-pi/24]
  plot(\x, {-1-1.5*(\x-(-pi/3+0.3+pi/24))*(\x-(0*pi/3+0.3-pi/24))*(\x-(-pi/2+0.1))});
  \draw[red,domain=0*pi/3+0.3+pi/24: 1*pi/3+0.3-pi/24]
  plot(\x, {-1-1.5*(\x-(0*pi/3+0.3+pi/24))*(\x-(1*pi/3+0.3-pi/24))*(\x-(pi/3-pi/2+0.1))});

  \draw[red,domain=-2*pi/3+0.3+pi/24: -pi/3+0.3-pi/24]
  plot(\x, {-1-1.5*(\x-(-2*pi/3+0.3+pi/24))*(\x-(-pi/3+0.3-pi/24))*(\x-(-pi/3-pi/2+0.1))});

  \draw[red] (-pi/3+0.3-pi/24,-1)--(-pi/3+0.3+pi/24,-1);
  \draw[red] (0*pi/3+0.3-pi/24,-1)--(0*pi/3+0.3+pi/24,-1);
  
  % линии напряжение-ток-\lambda
  \draw[thin] (-pi/3+0.3+pi/24,0.6)-- (-pi/3+0.3+pi/24,-0.95);
  \draw[thin](-pi/3+0.3+pi/24,-1.05)--(-pi/3+0.3+pi/24,-1.4);
  \draw[thin] (0*pi/3+0.3-pi/24,0.3)-- (0*pi/3+0.3-pi/24,-0.95);
  \draw[thin](0*pi/3+0.3-pi/24,-1.05)--(0*pi/3+0.3-pi/24,-1.4);
  \draw[thin,dashed] (-pi/6+0.3+0.03,0.6)--(-pi/6+0.3+0.03,-0.65);
  % lambda
  \draw[thin,<->] (-pi/3+0.3+pi/24,-1.3) -- (0*pi/3+0.3-pi/24,-1.3) node[midway,below]{$\lambda$};  
  \end{scope}  
\end{tikzpicture}  

Знаю, что ток маленький $R_\phi$, $R_D$, $R_\Phi$. $I_\textcyrillic{малое} R_\Sigma$,
много меньше ЭДС синусоиды

$$
\begin{array}{ccc}
  \textcyrillic{ЭДС нагрузки}& + & \xcancel{\textcyrillic{синусоидa}}\\
  \uparrow&&\\
  \textcyrillic{непренебрежимо}
\end{array}
$$

Есть ЭДС нагрузки плюс индуктивность.

$$
e_d -E_\textcyrillic{н} = L_\Sigma \frac{\partial i_d}{\partial t} + \xcancel{i_dR_\Sigma}
$$

Член $i_dR_\Sigma$ демонстративно зачеркнут, потому что второго порядка малости. Даже
при номинальном токе составляет несколько процентов.

$i_d$ это и есть $i_\textcyrillic{фазы}$.

Проинтегрировав эту разность $i_d = f(\omega t)$, а потом возьмём интеграл

$$
I_{d \textcyrillic{среднее}} = \int i_d\; d(\omega t) =
$$

$\displaystyle \lambda = \frac{2\pi}{m}$ -- граничный режим.

$$
\frac{1}{2\pi/m} \int\limits_{-\frac{\pi}{m}+\alpha}^{\frac{\pi}{m}+\alpha}
i_d\;d(\omega t)
$$
-- среднее значение граничного тока.

$$
I_{d\;\textcyrillic{гр}} = f(m,\alpha) = I(\sim)
$$
-- амплитуда пульсаций выпрямленного тока.

Полагая, что $I_\textcyrillic{среднее} =
I_\textcyrillic{амплитуда пульсаций выпрямленного тока}$, получим зависимоть
максимальных пульсаций с учётом гармонического состава. Этот способ учитывает реальную
кривую без разложения её в ряд Фурье.

Чему равна амплитуда тока?

%\begin{tikzpicture}
  \draw[thin,->] (-pi/2-0.1,0)--(5*pi/2+0.1,0) node[right] {$\omega t$};
  \draw[thin,dashed] (-pi/2-0.1,0.5)--(5*pi/2+0.1,0.5);
  \draw[domain=-pi/2:pi/2]
  plot (\x, {cos(\x r)});
  \draw[domain=pi/2:3*pi/2]
  plot(\x, {cos((\x-pi) r)});
  \draw[domain=3*pi/2:5*pi/2]
  plot (\x, {cos(\x r)});
  %  \draw[domain=3*pi/2:2*pi]
  %  plot(\x, {-cos(\x r)});
  \draw[thin,<->] (0,0.5)--(0,1);
  \draw[thin,<-] (0,0.75)--(1,1.3) node[right]{амлитуда положительной полуволны};
  \draw[thin,<->] (pi/2,0)--(pi/2,0.5);
  \draw[thin,<-] (pi/2,0.25)--(pi/2+1,-0.3) node[right]{амлитуда отрицательной полуволны};
\end{tikzpicture}

Можно ли считать полную амплитуду удвоением амлитуды нижней полуволны. Точного
расчёта пульсаций тока не существует.

Размах колебаний $(A_+ + A_-)$.

Зачем ограничивать пульсации тока. Каким образом выбирать ограничения. Ограничивают,
потому что потери на $R$ чем меньше тем лучше. Но ноль не может быть. Граница прерывистого
режима плохо. Чем меньше граница тем лучше.

Хороший критерий был -- электропривод постоянного тока. Но коллектор -- слабое место.
Коллектор--пластины--искрение, меняется магнитный поток. Нужно постараться, чтобы
тока в этот момент не было. ЭДС движения... Сумма ЭДС должна быть равна 0.
$I_\textcyrillic{якоря максимум}$ -- ограничивают.

$n_\textcyrillic{оборотов}$ -- $\frac{\partial i}{\partial t}$ тем больше искрение.
Переменная составляющая  тока. Добавил пульсации -- возникло искрение
$L\frac{\partial i}{\partial t}$ -- от переменной составляющей. А от частоты зависит
переменная составляющая? Принято считать, что не зависит, 2\% от номинального
для крупных машин. Станина -- литая сталь, поковка, литьё. Наборные пластины -- дорого.
Если поток добавочных полюсов должен компенсировать $I_\textcyrillic{якоря}$
Якорь шихтованный -- нет намагничивающих токов и поток не запаздывает, а в станине
эти эффекты ограничивают 2\%
Станина тоже из стали: из шихтованной стали 20\% пульсации

$
{\scriptstyle \Delta} I = I^2_\textcyrillic{ном} R
$
несколько процентов 3-2\%. КПД 96.5 - 3,4\% -- потери стали,меди,механические.

Добавилась пульсация 5\%
$$
I = \sqrt{I^2_\textcyrillic{пост} + I^2_\textcyrillic{перем}} R
$$

Выражение граничного тока:

$$
I_{d\textcyrillic{гран}} = \frac{E_{d0}}{\omega L}\left[1-\frac{\pi}{m}\,ctg \frac{\pi}{m}
\right]
\,sin\,\alpha
$$
где, $\omega$ -- питающей сети. $\omega_\textcyrillic{сети} =2\pi f_\textcyrillic{сети}$
Максимум при  $\alpha=90^\circ$, очевидно, зависит от числа фаз.

Это выражение даст эллипс

\begin{tikzpicture}
  %ellipse elipse
  \draw[domain=0:0.7, help lines,dotted, smooth]
  plot (\x,{sqrt(1-\x*\x/0.49)});
  \draw[domain=0:0.7, help lines,dotted, smooth]
  plot (\x,{-sqrt(1-\x*\x/0.49)});
  \draw[thin] (0.2,0) -- (0.2,{sqrt(1-0.2*0.2/0.49)}) --++ (2,-0.3);
  \draw[thin] (0.5,0) -- (0.5,{sqrt(1-0.5*0.5/0.49)}) --++ (2,-0.3);
\end{tikzpicture}

\subsection{Энергетические характеристики тиристорных преобразователей}

КПД $\eta = f(U_d)$, при $I_d=const$. Зависимость от напряжения и тока, $\alpha$ -- управления.

Электрические машины:
\begin{itemize}
\item постоянного тока(генераторы,моторы))
\item синхронные
\item асинхронные
\item трансформаторы 
\end{itemize}

Напряжение сети $=const$, от $cos\,\phi$

В электрических машинах $\displaystyle \eta = f(\frac{I}{I_\textcyrillic{ном}})$ или
$\displaystyle f(\frac{S}{S_\textcyrillic{ном}})$ или
$\displaystyle f(\frac{P}{P_\textcyrillic{ном}})$, $cos\,\phi$

У синхронных генераторов зависимость от $\phi$ нагрузки.

Когда я меняю напряжение интересует что будет с $cos\,\phi$

КМ($\lambda$) --коэффициент мощности, есть отношение активной мощности к полной.

$$
\textcyrillic{КПД} = \frac{P_\textcyrillic{вых}}{P_\textcyrillic{входа}}
$$

$$
\lambda = \frac{P}{S}
$$
где, $P$ -- активная мощность, $S$ -- полная мощность.

$
\textcyrillic{Коэффициэнт мощности} = cos\,\phi
$ -- неверно в общем случае. Так можно говорить, если напряжение
симметрично, синусоидально, и одной частоты. В нашем случае разные частоты.

Допущение: сеть симметрична и синусоидальна.

$$
\lambda = \frac{S_{(1)}}{S}\cdot\frac{P}{S_{(1)}}
$$

$S$ -- полная мощность $U\cdot I$ формальное произведение.

$S_{(1)} = U_{(1)}\cdot I_{(1)}$ -- если однофазная система.
$$
S =\underbrace{U_\textcyrillic{средне квадратичное}}_
{\textcyrillic{по допущению }U_1}\cdot I_\textcyrillic{средне квадратичное}
$$
А ток -- это наш ток и мы знаем, что он несинусоидальный. Если включу
на активное сопротивление последовательно 100V и 100V, получу 200V.
А если включить на активно-индуктивную нагрузку.
У нас было 50Гц, А в токе разные частоты. Другие частоты в токе несут ли
активную и реактивну мощность? Несут токи одинаковой частоты
$U(\omega)\cdot I(\omega)$ -- одинаковой частоты.
\begin{tikzpicture}
  \draw[domain=0:3*pi,samples=100,red]
  plot(\x, {sin(\x r)});
  \draw[domain=0:3*pi,samples=100,blue]
  plot(\x,{sin((2*\x) r)});
\end{tikzpicture}

Есть колебательная мощность. Это реактивная? нет

\begin{tikzpicture}
  \draw[thin,->] (0,0)--(2*pi+pi/2+0.2,0) node[right] {$\omega t$};
  \draw[domain=0:2*pi+pi/2]
  plot(\x,{sin(\x r)});
  \draw[domain=0:2*pi+pi/2,red]
  plot(\x,{sin((\x-pi/2) r)});  
  \end{tikzpicture}

Знаки разные(в этом смысле похоже на предыдущий случай), но не совсем. <могу поставить <C> и
скомпенсировать>

$$
\begin{array}{cccc}
  P&[\textcyrillic{вт}]&=&UI_\textcyrillic{актив}\\
  Q&[\textcyrillic{вар}]&=&UI_\textcyrillic{реактив}\\
  S&[\textcyrillic{ВА}]&=&{\displaystyle S=\sqrt{P^2+Q^2}}
  \end{array}
$$
Эти соотношения для одинаковых частот.

Если частоты разные, то это мощность искажения
$$
\lambda = \frac{S_1}{S}\frac{P}{S_1} = \nu\,cos\,\phi
$$,
где $S_1 = U_1\cdot I_1$

Нелинейные нагрузки искажают форму тока, портят стандарты, требуют принятия мер для качетва напряжения.
Несинусоидальность не больше стольки-то \%.

Пренебрегаем влиянием внешних.

$$
\eta = \frac{P_\textcyrillic{вых}}{P_\textcyrillic{вх}}
$$
$$
\lambda = \frac{S_1}{S}\frac{P}{S_1} = \nu\,cos\,\phi
$$

Наша специальность самая энернгетическая.

$$
Р\textcyrillic{вых} = U_dI_d
$$

$$
Р\textcyrillic{вх} = Р\textcyrillic{вых} + {\scriptstyle \Delta}P_\Sigma
$$,
где $P_\Sigma$ -- потери, то что преобразовалось в тепло, электромагнитную волну, звук, свет.

Представим, что мы регулируем напряжение.

$$
\eta = \textcyrillic{КПД} = \frac{U_d\,I_d}{U_d\,I_d + {\scriptstyle \Delta}P_\Sigma}
$$

${\scriptstyle \Delta}P_\Sigma $ на активных элементах:
  $I_d^2\,r_{\Sigma\textcyrillic{эквив}} +...+I_d\,k_0U_0 + {\scriptstyle \Delta}P_{xx}$,
  где $k_0U_0$ может быть несколько вентилей. Ток в вентилях меняется пропорционально $I_d$,
  $I_\phi$ -- в обмотке трансформатора. $I_\phi\div I_d$ (ток в обмотке трансформатора пропорционален
  $I_d$, в два раза увеличилось $I_\phi$, значит в два раза увеличилось $I_d$, т.е. есть
  составлающая пропорциональная $I_d$)

  А если выключено, потери не зависят от $I_d$ (система управления, лампочки)

  $$
  = aI^2_d + bI^1_d + cI^0_d
  $$

  $$
  \eta = (\textcyrillic{добавил и вычел } \pm {\scriptstyle \Delta}P_\Sigma) =
  1 - \frac{{\scriptstyle \Delta}P_\Sigma}{U_dI_d +{\scriptstyle \Delta}P_\Sigma} =
\label{eta}
  $$
Выражение(\ref{eta}) меньше единицы когда и $U_d$ и $I_d$ одного знака. $I_d>0$ всегда больше нуля.
значит $0<U_d<0$ (может быть и больше и меньше нуля)

$$
= 1 - \frac{\displaystyle \frac{{\scriptstyle \Delta}P_\Sigma}{U_d\,I_d}}
{\displaystyle 1\;+\;\frac{{\scriptstyle \Delta}P_\Sigma}{U_d\,I_d}}
$$

Сейчас будем рассматривать $I_d=const$ при бесконечном увеличении $U_d$ $\eta\rightarrow1$
Но $U_d$ не больше чем $U_{d\;max}$

\begin{tikzpicture}
  \begin{scope}[scale=5]
    \draw[thin,->] (-1.2,0)--(1.2,0) node[right]{$\frac{U}{E_{d0}}$};
    \draw[thin,->] (0,-0.1)--(0,1.2) node[left]{$\eta$};
    \draw[thin,dotted] (-1,0)--(-1,1)--(1,1)--(1,0);
    \draw[thin,dotted] (-1.2,1) -- (-1,1);
    \draw[thin,dotted] (1,1) -- (1.3,1);
  \draw[domain=0.001:0.8]
  plot(\x, {1-(1/\x)/(10+(1/\x))});
  \draw[domain=0.8:1.3,dashed]
  plot(\x, {1-(1/\x)/(10+(1/\x))});
  \draw[domain=-0.8:-0.1]
  plot(\x, {1+(1/\x)/10});
  \draw[domain=-1.2:-0.8,dashed]
  plot(\x, {1+(1/\x)/10});
  %
  \draw[thin,dashed]  (0.8, {1-(1/0.8)/(10+(1/0.8))})-- (0.8,0) node[below]
       {$\frac{U_{d\,max}}{E_{d0}}$};
       \draw[thin,dashed]  (-0.8, {1+(1/(-0.8))/10}) -- (-0.8,0) node[below]
       {$\frac{U_{d\,max}}{E_{d0}}$};
  
  %если ток вырос
  \draw[domain=0.001:0.8,dashed]
  plot(\x, {1-(1/\x)/(7+(1/\x))});
  \draw[domain=-0.8:-0.125,dashed]
  plot(\x, {1+(1/\x)/8});
  %
  \draw[thin,<-] (-0.1,0)--(0.3,-0.3) node[below]{$U_d=-\frac{{\scriptstyle \Delta}P_\Sigma}{I_d}$};  
  \draw[thin,<-] (-0.125,0) -- (-0.4,-0.3) node[below]{если ток вырос};

  \draw[thin,<-] (0.7,{1-(1/0.7)/(10+(1/0.7))}) -- (0.9,0.7) node[right] {$I_d=const$};
  \draw[thin,<-] (-0.7,{1+(1/(-0.7))/10}) -- (-0.9,0.7) node[left] {$I_d=const$};
  \draw[thin,<-] (0.6, {1-(1/0.6)/(7+(1/0.7))}) -- (0.6,0.5) node[right] {$I_{d2}>I_{d1}$};
  %  \draw[thin,<-]]
  \end{scope}
  \end{tikzpicture}


$P_\textcyrillic{вых}$. При отрицательном $\eta$ поменялись местами выход и вход.
То, что было $\eta_\textcyrillic{выпрямителя}$ при $(U_d>0)$ стало
при $U_d<0$

$$
\eta_\textcyrillic{инвертора} = \frac{U_d\,I_d+{\scriptstyle \Delta}P_\Sigma}
    {U_d\,I_d} = 1 +\hspace{-0.8cm}
    \underbrace{\frac{{\scriptstyle \Delta}P_\Sigma}{U_d\,I_d}}_
               {\begin{array}{c}\textcyrillic{отрицательно}\\
                   \textcyrillic{эта дробь}\\
                   \textcyrillic{станет равна -1}\end{array}}
               $$
при $\displaystyle u_d = -\frac{{\scriptstyle \Delta}P_\Sigma}{I_d}$
-- КПД упадёт до 0 и обратно пропорционально току.

$$
{\scriptstyle \Delta}P_\Sigma = \frac{a\,I_d^2}{I_d} 
$$

$\eta_\textcyrillic{граничное} =0!$ при $U_d>0$

$\eta_\textcyrillic{граничное}$ при $\displaystyle U_d< U_{d_\textcyrillic{граничное}} =
-\frac{{\scriptstyle \Delta}P_\Sigma}{I_d}$

Осталось понять чему равен $\eta$ при $U_{d_\textcyrillic{граничное}}<U_d<0$:

\underline{$\eta$ по определению} равен нулю потому что $P_\textcyrillic{вых}$ =0.

\subsection{Энергетические реки:}

\begin{tikzpicture}
  \draw[thick,help lines] (0,3)--(5,3);
  \draw[thick,help lines] (0,0)--(2,0)arc(90:0:1);
  \draw[thick,help lines] (2,0.8)arc(90:0:1.8);
  \draw[thick,help lines] (2,0.8)--(5,0.8);
  \draw (-0.8,1.3) node[left] {сеть};
  \draw[<-,double] (0.8,1.7)--(0.3,1.7) node[left]{$P_\textcyrillic{вх}$};
  \draw[->,double] (4.2,1.7)--(4.7,1.7) node[right]{$P_\textcyrillic{вых}$};
  \draw[->,double] (3.15,-0.3) arc(40:0:1.1) node[below]
       {${\scriptstyle \Delta} P_\Sigma$};
   \draw (5.8,1.3) node[right] {нагрузка};
\end{tikzpicture}


\begin{tikzpicture}
  \draw[thick,help lines] (0,3)--(5,3);
  \draw[thick,help lines] (5,0)--(3,0)arc(90:180:1);
  \draw[thick,help lines] (3,0.8)arc(90:180:1.8);
  \draw[thick,help lines] (3,0.8)--(0,0.8);
  \draw (-0.8,1.3) node[left] {сеть};
  \draw (5.8,1.3) node[right] {нагрузка};
  \draw[->,double] (4,0.4)--(3,0.4)arc(90:180:1.4) node[below]
  {${\scriptstyle \Delta} P_\Sigma$};
  \draw[<->,thin] (0.2,3)--(0.2,0.8) node[midway,right]
       {$P_\textcyrillic{сети}=P_\textcyrillic{вых}$};
  \draw[<->,thin] (4.8,3)--(4.8,0) node[midway,right]
       {$P_\textcyrillic{вх}$};
  \draw[thick,help lines] (4.6,1)--(3.6,1);
  \draw[thin,<-] (4.4,1)--(4.4,-0.6) node[below]
       {$\begin{array}{c}\textcyrillic{сокращается }U_d\,I_d,\\
         \textcyrillic{а потери неизменны}\end{array}$};
\end{tikzpicture}  


\begin{tikzpicture}
    \draw[thin,->] (-1.2,0)--(1.2,0);% node[right]{$\frac{U}{E_{d0}}$};
    \draw[thin,->] (0,-0.1)--(0,1.1);
    \draw[domain=-0.8:-0.1]
    plot(\x, {1+(1/\x)/10});
    \draw[thin,<-] (-0.1,-0.05)-- (-0.4,-0.5) node[below]
    {Нагрузка едва покрывает потери};
\end{tikzpicture}    
%--Нагрузка едва покрывает потери.

Нагрузка отдает, а до сети ничего не доходит.

\begin{tikzpicture}
  \draw[thick,help lines] (0,0.8)--(5,0.8);
  \draw[thick,help lines] (0,0)--(1,0)arc(90:0:1);
  \draw[thick,help lines] (5,0)--(4,0)arc(90:180:1);
  \draw[->,double] (0.2,0.35)--(0.95,0.35)arc(90:0:1.35)
  node[below] {${\scriptstyle \Delta}P_\Sigma\textcyrillic{ потери}$};
  \draw[->,double] (4.8,0.35)--(4.05,0.35)arc(90:180:1.35);
  
\end{tikzpicture}
Нагрузка не выпрямительный режим, но и не инверторный.
Получают только потери.

А если $I_d$ изменится: $I_d$ вырос в два раза, числитель вырастет в два раза,
но и ${\scriptstyle \Delta}P_\Sigma$ тоже вырастет. Всё зависит от
коэффициента(-тов).

Реально, КПД упадт.

слева, при большом токе вырастет $\gamma$

\subsection{Коэффициент мощности}
$\nu$ -- коэффициэнт искажения $cos\,\phi$ -- коэффиниент угла сдвига тока,
численно равен коэф мощности, когда напряжение и ток синусоидальны.

$$
cos\,\phi = \frac{P}{S}
$$

$S$ всегда положительна, $P$ бывает и положительна и отрицательна.
$Q>0$ -- индуктивность, так принято. Преобладают нагрузки, которые
потребляют положительную реактивную мощность.
В воздушной линии $C\sim\frac{S}{d}$
В кабельных линиях -- емкости.

$C$ генерирует в сеть реактивную мощность.

$\phi$ положительный, когда ток отстает от напряжения.

$sin\,\phi<0 \Rightarrow Q<0$

Cos может быть и положительным и отрицательным.

$$
\frac{P}{S} = |I|\cdot|U|\cdot cos\,\phi
$$

$cos\,\phi=-0.5$ Инвертор работает с углом $\sim 60^\circ$.

Реактивная мощность не передаёт балластную мощность. Из-за реактивных токов
приходится увеличивать мощность <приборов,трансформаторов>

Ток, который создаёт магнитное поле

$$
L=\frac{\Phi}{i}
$$

У синхронных машин магнитное поле создаётся обмоткой возбуждения.
У аснхронных двигателей создаётся реактивным током.

$\alpha$ сдвинул на $90^\circ$ -- я принудительно сдвинул начало генерации тока.

\begin{tikzpicture}
  % оси
  \draw[thin,->] (-pi/2,0)--(2*pi,0) node[right]{$\omega t$};
  \draw[thin,->] (-pi/2,-0.1)--(-pi/2,1.2) node[left] {$U$};
  \draw[thin,->] (-pi/2,-1.5)--(2*pi,-1.5) node[right]{$\omega t$};
  \draw[thin,->] (-pi/2,-1.6)--(-pi/2,-0.4) node[left] {$I$};
  %
  \draw[domain=-pi/2:pi/2,help lines,smooth]
  plot(\x, {cos(\x r)});
  \draw[domain=pi/6:7*pi/6,help lines,smooth]
  plot(\x, {cos((\x-2*pi/3) r)});
  \draw[domain=5*pi/6:11*pi/6,help lines,smooth]
  plot(\x, {cos((\x-4*pi/3) r)});
  % 
  \draw[thin,red,domain=-pi/3+0.3:-pi/3+2*pi/3+0.3]
  plot(\x, {cos(\x r)+0.05})
  |- (pi/3+0.3, {cos((pi/3+0.3-2*pi/3) r)+0.05});
  \draw[thin,red,domain=-pi/3+2*pi/3+0.3:-pi/3+4*pi/3+0.3]
  plot(\x, {cos((\x-2*pi/3) r)+0.05})
  |- (-pi/3+4*pi/3+0.3, {cos((-pi/3+4*pi/3+0.3-4*pi/3) r)+0.05});
  \draw[thin,red,domain=-pi/3+4*pi/3+0.3:-pi/3+6*pi/3+0.3]
  plot(\x, {cos((\x-4*pi/3) r)+0.05});
  %
  \draw[help lines,smooth] (-pi/2,-1.3)--(-pi/3,-1.3)--(-pi/3,-0.8)--(pi/3,-0.8)
  --(pi/3,-1.3)--(pi/2,-1.3);
  \draw[help lines,smooth] (pi/6,-0.6) node[right]{$I(\alpha=0)$};
  \draw[help lines,smooth,red](-pi/2+0.3,-1.4)--(-pi/3+0.3,-1.4)--
  (-pi/3+0.3,-0.9)--(pi/3+0.3,-0.9)--(pi/3+0.3,-1.4)--(pi/2+0.3,-1.4);
  \draw[red] (pi/2,-1.2)node[right]{$I(\alpha>0)$};
  %
  \draw[thin] (0,-0.8) -- (0,-2);
  \draw[thin] (0.3,-0.9) -- (0.3,-2);
  \draw[thin,->] (-0.3,-1.9)--(-0,-1.9);
  \draw[thin,<-] (0.3,-1.9)--(0.8,-1.9) node[right]
       {$\phi\approx\alpha\textcyrillic{ в первом приближении}$};
  \draw[thin,<-] (-0,-2.1)--(-0.3,-2.1) node[left]{$\alpha$};
  \draw[thin,<-] (0.3,-2.1)--(0.5,-2.1);
\end{tikzpicture}

$\alpha=0$ -- фаза тока совпадает с фaзой напряжения. $\phi = \alpha$ в
первом приближении.

Насколько сдвинулись гармоники тока? Но это никак не связано с энергией!

$I_d=const$ будем рассматривать для всех энергетических характеристик.

$$
P=U_d\,I_d\;\; \xcancel{(+{\scriptstyle \Delta}P_\Sigma)}
$$
равенство буду считать хотя бы примерно(пока не буду считать потери)

Поддерживая $I_d=const$ $S$ будет оставаться неизменной.
$S_1 \cong const$, потому что $I_d \cong const$

\begin{tikzpicture}
  \draw (0,1) node {$\displaystyle cos\,\phi = \frac{P}{S_{(1)}} \approx U_d \approx Cos\,\alpha$};
  \draw (-0.5,0.35) node[rotate=80] {=};
  \draw (-0.5, 0) node{$const$};
  \draw (0.8,0.4) node[rotate=110] {=};
  \draw (0.8,0) node[right]
   {$\displaystyle U_d=E_{d0}\frac{cos\,\alpha + cos(\alpha+\gamma)}{2}$};
\end{tikzpicture}

При $\alpha=0\;\gamma=0$ $U_d=E_{d0}$

Чисто активная мощность, значит ...

При $\alpha=0\;\gamma=0$ $U_d=E_{d0}$, $P=S_1$

$P\Bigl|_{\scriptstyle \begin{array}{c}\scriptstyle{\alpha=0}\\
    \scriptstyle{\gamma=0}\end{array}} = S_1 = E_{d0}I_d$

при $P$, которое я регулирую

$$
\frac{P}{S} = \frac{E_d\,I_d \frac{cos\,\alpha + cos(\alpha+\gamma)}{2}}{E_d\,Id}
$$

отсюда
$$
cos\,\phi = \frac{cos\,\alpha + cos(\alpha + \gamma)}{2}
$$
достаточно точно, угол $\phi$ пропорционален $\alpha$.

%Мы рассматривали энергетические характеристики тиристорных преобразователей, зависимых
от сети. Рассматривали КПД.

Напомню коэффициэнт мощности, который определялся как
$$
\lambda = \frac{P}{S}
$$
при допущении: Напряжение <на первичной стороне> синусоидальное.

$U = U_{(1)}$ -- напряжение равно напряжению первой гармоники.

Если сеть трехфазная, то 
$S = 3U_\phi I$. Формула написана для фазного напряжения, если рассматривать линейное, то
добавим $\sqrt{3}$. $I$ синусоидальным считать нельзя.

$$
I = \sqrt{\sum\limits_{k=0}^\infty I_k^2}
$$

$$
k = \frac{f_{\textcyrillic{гармоники}}}{f_{\textcyrillic{сети}}} = m,m \pm 1
$$

грубо ${\displaystyle \frac{I_{(k)}}{I_{(1)}}} \simeq \frac{1}{k}$ -- так можно считать для
малых $k$.

при $m=6$ $k=5,7,11,13,17,19$ `Это соотношение выполняется для первых $m$. Выполняется
тем более точно, чем меньше угол коммутации $\gamma=0$, А угол $\alpha=0$. Эти условия
выполняются в том случае если ток мал и индуктивность мала, или и то и другое.

Если подставить значение $S$  и $P$.

Активная мощность передается током и напряжением одинаковых частот.

$P = U\cdot I_{(1)}Cos\phi_{\cancel{(1)} \textcyrillic{ -- но могу и не писать}}$

Напряжение и токи разных частот не передают ни активной ни реактивной мощности.
Активную и реактивную мощность передают $U$ и $I$ одинаковых частот.

Это допущение. На самом деле сеть не бесконечной мощности, ток будет падать на сопротивлениях
и напряжение тоже будет иметь гармоники, но по допущению гармоник напряжения нет.

$$
Q =U\cdot I_{(1)}\; Sin\:\phi = 3U_\phi I_{(1)}Sin\phi
$$

$$
\lambda = U\cdot I{(1)}\; Cos\phi = \frac{ 3U_\phi I_{(1)}Cos\phi}{3U_\phi I}=
\frac{ I_{(1)}}{\sqrt{\sum\limits_{k=1}^\infty I_k^2}}Cos\phi = \nu\cdot Cos\phi
$$

\begin{equation}
  \lambda = \nu\cdot Cos\phi
  \end{equation}

Что такое $\nu$? Это коэффициэнт несинусоидальностиЮ лучше назвать коэффициэнт искажения тока.
$\nu < 1$, А насколько.

Пренебрегая углом коммутации
$$
\nu \cong \frac{3}{\pi} = 0.955
$$

С ростом гармоник амплитуда $I_{(k)}$ падает. Первые гармоники вносят наибольший вклад в
$I =\sqrt{\sum I^2_{{k}}}$.

Продолжим равенство
$$
=  \frac{I_{1}}{\sqrt{I_{(k)}^2 + \sum\limits_{k>1} I^2}} = \frac{I_{(1)}}
{\sqrt{I_1^2+I_\nu^2}}
$$
Все $I$ в формуле -- действующие значение, и $\nu= \sum\limits_{k>1} I^2$

Коэффициэнт $\lambda$ равен $cos\phi$ только в случае если есть только первая гармоника,
а если нет, то
напомним
$$
cos \phi = \frac{cos\alpha + cos(\alpha+\gamma)}{2} 
$$

Если $\gamma$ мало, то $cos\phi = cos\alpha$. Часто используют приближение
$cos\phi = cos(\phi + \frac{\gamma}{2})$. Если знаете $\alpha$ и $\gamma$, то зачем
пользоваться приближенным, если есть точное.

Если $cos\phi$ мал, то это плохо, потому что $Q \sim sin\phi$. Так что надо повышать,
говорят что до 0.95.

Если я возьму катушку, это плохо, но важна еще мощность, сила тока. Если $cos\phi=0.1$ это
не плохо, если полная мощность 10ватт.

У трансформаторов на холостом ходу очень низкий $cos\phi$, но при холостом ходе
течёт малы ток.

Когда говорят о коэффициэнте мощности тиристорного преобразователя, то ...

$$
P = 3U_\phi I_{(1)} cos\phi_{(1)} \simeq U_d I_d
$$

$$
Q=3U_\phi I_{(1)} sin\phi_{(1)} \simeq \sqrt{1-\left(
\frac{P}{E_dI_d}
  \right)^2} 
$$
Почему так. Вернулись к $S$ полной мощности.

$$
S = 3U_\phi \sqrt{I_{(1)}^2 + I_{(\nu)}^2} =
\sqrt{\left(3U_\phi I_{(1)}\right)^2 + \left(3U_\phi I_\nu\right)^2} =
  \sqrt{S_{(1)}^2+S_\nu^2}
$$

  $S_{(1)}$ -- так обозначили по аналогии с током. $S_{(\nu)}$ -- полная мощность остальных
  гармоник, мощность искажения по аналогии с током искажения.

  Для первой гармоники справедливо

  $$
  S_{(1)} =\sqrt{P^2+Q^2}
  $$
  \begin{equation}
    S=\sqrt{P^2+Q^2+S_\nu^2}
  \end{equation}

$$  
  S_{(1)} = 3U_\phi I_{(1)} \simeq const
  $$

  Рассматриваем, как меняется $S$, меняя $\alpha$ при $I_d=const$

  \begin{tikzpicture}
    \draw[->,thin] (-0.2,0)--(2*pi+0.4,0);
    \draw[domain=-0.2:2*pi+0.2,help lines,smooth]
    plot (\x, {sin(\x r)});
    \draw[domain=-0.2:pi/6+0.1,help lines,smooth]
     plot (\x, {sin((\x+2*pi/3) r)});
   %
     \draw[domain=pi/3:pi/3+0.04]
     plot(\x, {10*(\x-pi/3)});
     \draw[domain=5*pi/6-0.3:5*pi/6-0.26]
     plot(\x, {-10*(\x-5*pi/6+0.26)});
     \draw[domain=pi/3+0.04:5*pi/6-0.3]
     plot(\x, 0.41);
     %
     \draw[thin] (pi/3,-0.05)--(pi/3,-0.55) (5*pi/6-0.26,-0.05)--(5*pi/6-0.26,-0.55);
     \draw[thin,<->] (pi/3,-0.5)--(5*pi/6-0.3,-0.5) node[midway,below]{$\lambda$};
     \draw[thin,<-]  (pi/3-0.05,0.15) -- (pi/6,-1)
     node[below] {$\textcyrillic{наклон--из-за влияния }\gamma$};

\draw[thin,<-] (5*pi/6-0.2,0.41)--(pi+0.2,0.41) node[right]{ток $I_d$ имеет постоянную амплитуду};
   \draw[thin] (pi/6,0.6) -- (pi/6,1.3) (pi/3+0.04,0.6)--(pi/3+0.04,1.3);
   \draw[thin,->] (pi/6,1.2)--(pi/3+0.04,1.2) node[midway,above] {$\alpha$};
    \draw[thin] (pi/6-0.08,1.2-0.08) -- (pi/6+0.08,1.2+0.08)
;\end{tikzpicture}

Ток $I_d$ имеет постоянную амплитуду, $\lambda$ примерно одной ширины.
Влияние $\gamma$ мало. Значит действующее значение не меняется.

При постоянстве выпрямленного тока $I_d$ и изменении угла $\alpha$ величина
1-й гармоники сетевого тока сохраняется практически неизменной с точностью
до долей процента!

\begin{tikzpicture}
\draw[thin,->](-5.2,0)--(7,0) node[below]
{$\displaystyle \frac{U_d}{E_{d0}} = \frac{P}{E_{d0}I_d} =\overline{p}$};
\draw[thin,->] (0,-0.5)--(0,6) node[left]
{$\displaystyle \overline{q}=\frac{Q}{E_{d0}I_d}$}
node[right] {   $\overline{q}$ -- с чертой относительная величина}
;
\draw[domain=-5:5,samples=400]
plot(\x, {sqrt(25-\x*\x)});
\draw[dashed,thin](1.5,0)--(1.5,{sqrt(25-1.5*1.5)})--(0,{sqrt(25-1.5*1.5)});
\draw[thin,->] (0,0)--(1.5,{sqrt(25-1.5*1.5)}) node[midway]
{$\overline{s_1}_{\textcyrillic{относит.}}$};
\draw (1.5,0) node[above right]{$p_\textcyrillic{отн.}$}
(0, {sqrt(25-1.5*1.5)}) node[below left]{$q_\textcyrillic{отн.}$}
(-5,0)node[below]{-1} (5,0) node[below] {1};
%\begin{axis}[stack plots=y,area style,smooth,enlarge x limits=false]
%\addplot coordinates{(0,1.2) (1,1) (2,2) (3,2)}\closedcycle;
%\addplot coordinates{(0,1) (1,1.5) (2,0) (3,0)}\closedcycle;
%\addplot coordinates{(0,1) (1,1) (2,2) (3,2)}\closedcycle;\end{axis}
\draw [thick,pattern=north east lines, pattern color=red]
(4.5,0)--(5,0)arc(0:25:5)--(4.5,0);
\draw[thin,dashed] (4.5,1.5)--(4.5,5) node[midway]{$I_d=const$};
\draw[thin,dashed] (-1,-1)--(4.5,4.5) node[right] {$cos\:\phi$};
%\draw[decoration={text along path,text={$cos\:\phi$},
%text align={center}},decorate] (-1,-1)--(4.5,4.5);
%\draw[thin,dashed] (-1,-1)--(4.5,4.5) node[at end]{$cos\:\phi$};

\draw[thin] (4.7,1)--(5.5,1.3) node[right]
{\begin{tabular}{c}
коммутационное\\
падение
\end{tabular}
};
\draw[thin,<-] (4.45,-0.05) -- (2.8,-2.2) node[below]
{$U_{d\textcyrillic{относит.}max}$};
\draw[thin,<-] (-4.45,-0.05) -- (-3.2,-2.2) node[below]
{\begin{tabular}{c}$U_{d\textcyrillic{относит.}min}$\\дальше идет\\
опрокидывание инвертора\end{tabular}};
\draw[thin,<-] (2.5,-0.2) -- (2,-0.8) node[below]{выпрямитель};
\draw[thin,<-] (-2.5,-0.2) -- (-2,-0.8) node[below]{инвертор};
\draw[thin] (-4.5,0) -- (-4.5,{sqrt(25-4.5*4.5)});
\draw[thin] (4.5,-0.1)--(4.5,-1.3) (5,-0.6)--(5,-1.3);
\draw (4.75,-1.3) node[below]
{$\displaystyle \frac{\scriptstyle{\Delta}U_\gamma}{E_{d0}}$};
\draw[<->] (0.8,0)arc(0:72:0.8) node[midway,above right] {$\phi$};
\end{tikzpicture}

\begin{equation}
S_1 \textcyrillic{ (в относительных единицах) } = \frac{S_{1}}{E_{d0}I_d}
\approx 1
\end{equation}
Академически не точно, но с точностью доли процента.

При $\alpha=0$, при отсутствии $\gamma$, $cos\phi = 1$, значит,
полная мощность равна активной мощности.

\begin{equation}
S_{1max} \approx E_{d0}I_d
\end{equation}
значит $\alpha=0$, $\gamma=0$. $S_1$ с достаточно большой точностью
можно считать равным $E_{d0}I_d$.

$\left(\overline{q}\right)^2 + \left(\overline{p}\right)^2
= \left(\overline{s_{1}}\right)^2 = 1 
$ в относительных единицах.

Провели окружность, но пользоваться с осторожностью. Например, если
$I_d$ номинал, два номинала.
Коммутационное $U_{d\textcyrillic{коммутационное}}$ зависит от $I$

$\overline{p} = \textcyrillic{численно равно} = cos\phi$

Круговая диаграмма более информативна.

Представим $cos\phi =0$, но если ток номинальный $Q=100\%$.

Низкое значение(большое значение) $cos\phi$ при глубоком регулировании
выпрямленного напряжения, или, точнее, большое значение реактивной мощности
является главным недостатком тиристорных преобразователей.

Второй главный недостаток -- это сравнительно большая величина высших
гармоник, потребляемых из сети тока.

Достоинство -- высокий КПД

<Провели линию $cos\phi$> Что такое отрицательный $cos\phi$? Это...

$m=6$, $k=5,7,11,13,17,19,23,25$

$$
\nu \approx \frac{3}{\pi} = 0.955 = \frac{I_\nu}{I}
$$

$$
I = \sqrt{I_1^2 + I_\nu^2}
$$

Хочу найти в абсолютных единицах чему равно $I_\nu$.
$I_{\nu max} \approx 0.3I$ -- высшие гармоники 30\%. Это доказывает, что
высших гармоник достаточно много.

\subsection{Непосредственные преобразователи частоты}
правильное наименование -- с непосредствственной связью, без звена
постоянного тока.

НПЧ представляет собой реверсивный тиристорный преобразователь, выпрямленное
значение которого изменяется с заданной частотой. Выпрямленное напряжение
изменяется по величине и знаку.

\begin{tikzpicture}\draw
(0,0)to[short,o-](0,1.5)--(0.5,1.5)
(0.5,0.5)rectangle(3.5,2.5)
(1,1)to[Ty](3,1)--(3,2)to[Ty](1,2)--(1,1)
(3.5,1.5)--(4,1.5)to[short,-o](4,0)
(2,2.5)--(2,3)
(2,3.5)circle(0.5)
(2.5,3.2) node[right] {m фаз}
(2,4.3)circle(0.5) 
(2,4.8)--(2,5.3)--(1.7,5.8)
(2,5.5)node[right]{Q}
(2,5.8)to[short,-*](2,6.3)
(0,6.3)--(4,6.3)
(1,6.1)--(1.4,6.5)
(0.85,6.1)--(1.25,6.5)
(0.7,6.1)--(1.1,6.5)
;\end{tikzpicture}

\hspace{-4cm}
\begin{tikzpicture}
 \begin{scope}[xscale=1.3,yscale=2.8]
    \newcommand{\alphaa}{68 * pi / 180}
    \newcommand{\betaa}{200 * pi / 180}
    \newcommand{\gammaa}{82 * pi / 180}

    \draw[thin, ->] (-2*pi/3, 0) -- (4*pi+0.3,0) node[right] {$\omega t$};
    \draw[thin, ->] (0,0) -- (0,1.3) node[left] {$U$};

    \foreach \x/\xtext in {{-pi/6}/{-\frac{\pi}{m}}, 0,{pi}/\pi,
      {2*pi}/{2\pi}, {3*pi}/{3\pi}}
    \draw (\x,0.1) -- (\x,-0.1) node [below] {$\xtext$};

   % Vs
    \draw[domain=-2*pi/3:4*pi, help lines, smooth,samples=200]
    plot (\x,{sin((\x + pi/6) r)});

    \draw[domain=-2*pi/3:4*pi, help lines, smooth,samples=200]
    plot (\x,{sin((\x + pi / 3 + pi/6) r)});

    \draw[domain=-2*pi/3:4*pi, help lines, smooth]
    plot (\x,{sin((\x + 2 * pi / 3 + pi/6) r)});

    \draw[domain=-2*pi/3:4*pi, help lines, smooth,samples=200]
    plot (\x,{sin((\x + 3*pi / 3 + pi/6) r)});

    \draw[domain=-2*pi/3:4*pi, help lines, smooth,samples=200]
    plot (\x,{sin((\x + 4 * pi / 3 + pi/6) r)});

    \draw[domain=-2*pi/3:4*pi, help lines, smooth]
    plot (\x,{sin((\x + 5 * pi / 3 + pi/6) r)});

  %red
\draw[domain=-2*pi/3:-pi/6,red]
 plot (\x,{sin((\x + pi/6 + 2*pi/3) r)+0.02});
\draw[domain=-pi/6:pi/6+pi/6,red]
 plot (\x,{sin((\x + pi/6 + pi/3) r)+0.02})
-| (pi/3, {sin((pi/3 + pi/6) r)+0.03});
\draw[domain=pi/3:pi - pi/6,red]
plot (\x,{sin((\x + pi/6) r)+0.03})
-| (pi - pi/6, {sin((pi-pi/3) r)+0.03});
\draw[domain=pi - pi/6:pi+pi/3,red]
plot (\x,{sin((\x - pi/6) r)+0.03})
-| (pi+pi/3, {sin((pi-pi/6) r)+0.03});
\draw[domain=pi+pi/3:pi+pi/3+pi/6 + pi/3,red]
plot (\x,{sin((\x - pi/3 - pi/6) r)+0.03})
-| (2*pi-pi/6,{sin((pi) r)+0.03});
\draw[domain=2*pi-pi/6:2*pi-pi/6+pi/6 + pi/3,red]
plot (\x,{sin((\x - 2*pi/3 - pi/6) r)-0.03})
-| (2*pi+ pi/3,{sin((pi+ pi/6) r)-0.02});
\draw[domain=2*pi+ pi/3:2*pi+ pi/3+pi/6 + pi/3,red]
plot (\x,{sin((\x - pi - pi/6) r)-0.02});
\draw[domain=3*pi-pi/6:3*pi-pi/6+pi/6 + pi/3,red]
plot (\x,{sin((\x + pi/3 + pi/6) r)})
-| (3*pi+ pi/3,{sin((3*pi+ pi/3 + pi/6) r)});
\draw[domain=3*pi+ pi/3:3*pi+ pi/3+pi/6 + pi/3,red]
plot (\x,{sin((\x + pi/6) r)})
-| (4*pi-pi/6,{sin((4*pi-pi/6-pi/6) r)});
\draw[domain=4*pi-pi/6:4*pi,red]
plot (\x,{sin((\x - pi/6) r)});

% blue
\draw[domain=-pi/6:pi/6+pi/6+pi/6,blue]
 plot (\x,{sin((\x + pi/6 + pi/3) r)-0.02})
-| (pi/2,{sin((pi/2 + pi/6) r)-0.03});
\draw[domain=pi/6+pi/6+pi/6:pi/2+2*pi/3,blue]
plot (\x,{sin((\x + pi/6) r)-0.03})
-| (pi/2+2*pi/3, {sin((pi/2+2*pi/3 - pi/6) r)-0.02});
\draw[domain=pi/2+2*pi/3:pi/2+4*pi/3+0.03,blue]
plot (\x,{sin((\x - pi/6) r)-0.02});
\draw[domain=pi/2+4*pi/3:pi/2+6*pi/3,blue]
plot (\x,{sin((\x - pi/3 - pi/6) r)+0.02});


    % Vo and Io
%    \foreach \qq [evaluate=\qq as \qqshft using \qq*pi/3] in {-1,...,2}
%     {
%      \begin{scope}[xshift=\qqshft cm,
%          every path/.style={ultra thick, color=red}]
        %Vo
%        \draw[domain={pi/6+\gammaa-pi/2}:\alphaa]
%        plot (\x,{cos(\x r)})
%        -| (\alphaa,{cos((\alphaa - pi/3) r)/2 + cos(\alphaa r)/2})
%        [domain=\alphaa:\gammaa]
%        plot (\x,{cos((\x - pi/3) r)/2 + cos(\x r)/2 })
%        -| (\gammaa, {cos((\gammaa - pi/3) r) })
%        ;
%       \end{scope}
%     }
\end{scope}
\end{tikzpicture}

$\alpha$ изменяется периодически $\alpha=f(t)$.

Было
\begin{equation}
E_d = E_{d0}Cos\alpha
\label{lec8_1}
\end{equation}
, $\alpha=f(t)$

\begin{equation}
E_d = E_m Cos\left(\phi_\textcyrillic{нач.} + w_\textcyrillic{вых.}t\right)
\label{lec8_2}
\end{equation}
где, $w_\textcyrillic{вых.} = 2\pi f_\textcyrillic{вых.}$

Приравняв \ref{lec8_1} и \ref{lec8_2} находим:

$$
Cos\alpha = \frac{E_m}{E_{d0}} Cos\left(\phi_\textcyrillic{нач.} + w_\textcyrillic{вых.}t\right)
$$

\begin{equation}
\alpha = arccos\left[\frac{E_m}{E_{d0}} Cos\left(\phi_\textcyrillic{нач.} + 
w_\textcyrillic{вых.}t\right)\right]
\end{equation}

Частный случай, при $E_m=E_{d0}$

\begin{equation}
\alpha = \phi_\textcyrillic{нач.} + w_\textcyrillic{вых.}t
\end{equation}

Это объясняет, почему были сделаны равные приращения на рисунке выше.

Принцип работы при любом числе фаз. 

Было написано $U$, а использую $E$, потому что не учитываем падение на вентилях.

На выходе <было рассмотрено> однофазное напряжение. Но нужно трехфазное. Поэтому 
такие НПЧ содержат три реверсивных преобразователя.

%Коротко о том,что прошли.

Рассматривали ИППН, классификацию ИППН, классификация в основном по квадрантам. Рассматривали
одно-квадрантные,двух-квадрантные, четырех-квадрантные.

\begin{circuitikz}\draw
%  (0,2)to[short,o-](1,2)
  (1.5,2) node[nigbt,rotate=90](nigbt1){}
  (1.5,2) node[above] {VT}
  (0,2)to[short,o-] (nigbt1.C)
  (nigbt1.E)-- (3,2)to[L,-o](5,2)
(0,0)to[short,o-o](5,0)
(3,0)to[Do,l_=$VD$,*-*](3,2)
  ;\end{circuitikz}

Замечание: может стоять IGBT-транзистор, может стоять мосфет,
\begin{circuitikz}\draw
  (0,0)node[nigfete](nigfete){}
  (nigfete.B) node[anchor=west] { подложка}
  (nigfete.G) node[anchor=east] {затвор }
  (nigfete.D) node[anchor=south] {сток}
  (nigfete.S) node[anchor=north] {исток}
;\end{circuitikz}

а может стоять обычный тиристор с углом искуственной коммутации:  

\begin{circuitikz}\draw
  (0,5)to[short,o-*](1,5)--(1,3)to[C,v_>=$ $](2.5,3)
  (1,3)to[C,v^<=$ $](2.5,3)--(3,3)to[Do,*-*](3,5)
  (1,5)to[Ty](3,5)--(4.5,5)to[L](4.5,4)to[Do](4.5,3)--(3,3)
  (4.5,5)--(6,5)
  (0,0)to[short,o-*](6,0)to[Do,*-*](6,5)--(8,5)to[L,l=$L_\textcyrillic{н}$](8,3.5)
  to[R,l=$R_\textcyrillic{н}$](8,2)
  (8,1)circle(0.35)
  (8.35,1)node[right]{$E_\textcyrillic{н}$}
  (6,0)--(8,0)--(8,0.65)
  (8,1.35)--(8,2);
  \draw[thin,<-] (1.75,2.6) -- (1.75,2) node[below]{должен зарядиться так};
  \draw[thin,<-] (1.75,3.4) --(1.2,3.6) node[above]{исходный заряд};
  \draw[thin,<-] (6.3,2.5) -- (6.7,2.5) node[below,rotate=90] {оставим этот диод};
  \draw[dashed] (1.2,5)-- (1.2,6);
  \draw (1.2,6)to[short,-o](1.2,6.1)node[left] {\large{-}};
  \draw[dashed] (2.8,5)-- (2.8,6);
  \draw (2.8,6)to[short,-o](2.8,6.1)node[right] {\large{+}};
;\end{circuitikz}

Искусственная коммутация $\cong$ принудительная коммутация $\cong$ ёмкостная коммутация.
Искусственная коммутация и  принудительная коммутация -- синонимы.
\begin{circuitikz}
\begin{scope}[scale=0.75]
  \draw[dashed] (1.2,5)-- (1.2,6);
  \draw (1.2,6)to[short,-o](1.2,6.1)node[left] {\large{-}};
  \draw[dashed] (2.8,5)-- (2.8,6);
  \draw (2.8,6)to[short,-o](2.8,6.1)node[right] {\large{+}};
  \end{scope}
  ;\end{circuitikz} -- Кратковременно подключить, искусственно включить, принудительно
включить источник. Чаще всего таким источником является заряженный конденсатор. 

\begin{circuitikz}
\begin{scope}[scale=0.75]
  \draw (1.2,5)-- (1.2,6);
  \draw (1.2,6)to[short,-o](1.2,6.1)node[left] {\large{-}};
  \draw (2.8,6)to[ospst] (2.8,5);
  \draw (2.8,6)to[short,-o](2.8,6.1)node[right] {\large{+}};
  \draw[thin,red,->] (0,5.2) -- (0.4,5.2) arc(-90:0:0.2) -- (0.6,6)  
  arc(180:90:0.6)  --(2.8,6.6) node[midway,above] {ток}
  arc(90:0:0.6)-- (3.4,5.4) arc(180:270:0.2) -- (3.9,5.2);
  \end{scope}
;\end{circuitikz} -- источник перехватывает \underline{ток} нагрузки.
Но главная задача -- отключить нагрузку.

Это делается в два этапа:
\begin{itemize}
\item запереть тиристор
\item отключить нагрузку
  \end{itemize}

\begin{circuitikz}
\begin{scope}[scale=0.75] 
  \draw (0,5)-- (1.2,5)-- (1.2,6) to[C,l_=$C$](2.8,6)--(2.8,5) -- (4,5);
  \draw[thin,red,->] (0,5.2) -- (0.4,5.2) arc(-90:0:0.2) -- (0.6,6)  
  arc(180:90:0.6)  --(2.8,6.6) node[midway,above] {ток}
  arc(90:0:0.6)-- (3.4,5.4) arc(180:270:0.2) -- (3.9,5.2);
  \end{scope}
;\end{circuitikz} -- конденсатор идеально подходит для обоих этапов. Конденсатор перезаряжается
и перехватывает энергию. Ёмкостная коммутация -- частный случай искуссственной коммутации,
когда источником является конденсатор.

Можем использовать импульсный трансформатор. 

%\section{Непосредственные преобразователи частоты (без промежуточного звена постоянного тока)}

НПЧ -- реверсивный тиристорный преобразователь, с любым числом фаз, с любым способом управления (совместимым с ...)

Общий принцип -- угол $\alpha$ непрерывно изменяется обеспечивая .... изменение постоянного напряжения выходной составляющей.

$f_{\textcyrillic{выхода}} \le f_{\textcyrillic{сети}}, f_{\textcyrillic{входа}} = 50 {\textcyrillic{Гц.}} $

Это главная особенность и один из главных недостатков. $f_{\textcyrillic{выхода}} = 50 {\textcyrillic{Гц.}} $ может быть достигнуто
при большом $m \rightarrow \infty$

\begin{circuitikz}
\newcommand{\PI}{3.149265}
\draw[thin] (-2,0) -- ({\PI/2+0.5}, 0);
\draw[domain=-2:-1.5,help lines,smooth]
    plot (\x, {cos((\x+1.5) r)});
\draw[domain=-1.5:-1,help lines,smooth]
    plot (\x, {cos((\x+1) r)});
\draw[domain=-1:-0.5,help lines,smooth]
    plot (\x, {cos((\x+0.5) r)});
\draw[domain=-0.5:pi/2+0.5,help lines,smooth]
    plot (\x, {cos(\x r)});
\draw[thin,-{Stealth[scale=1.5]}] (-0,-0.8) node[below] {быстрее снижаться не может}  -- ({\PI/4},{0.5});

\draw[domain=-0.5+2*pi:pi/4+2*pi,help lines,smooth]
    plot (\x, {cos(\x r)});
\draw[domain=pi/4+2*pi:pi/3+2*pi + 0.5,help lines,smooth]
    plot (\x, {cos((\x+0.7) r)});
\draw[thin,dashed] ({\PI*(1/4.+2)},0.1) -- ({\PI*(1/4.+2)}, {sqrt(2)/2});
\draw[thin,-{Stealth[scale=1.5]}] ({\PI*(1/4.+2)-0.6},-0.6) node[below] 
{$\begin{array}{c}
\textcyrillic{такое невозможно --}\\
\textcyrillic{связано с невозможностью}\\ 
\textcyrillic{запирания тиристора}\end{array}$} -- ({\PI*(1/4.+2)-0.1},-0.1);
\end{circuitikz}

Так считалось пока не было запираемых тиристоров. Но нужно менять принцип управления -- нужно включать предыдущую фазу $(k-1)$ а не последующую.
Но с энергетическими процессами. Запасенная в индуктивности фазы связана с большими потерями. Чем меньшее число фаз, я должен набрать 
новую синусоиду из кусочков синусоид 50Гц. Из четырех кусочков, из трех, из двух! Очень большие искажения если из трех.
Полуволну 25Гц набрать из 3х кусочков синусоид.
\begin{circuitikz}
\newcommand{\PI}{3.149265}
\draw[thin] (0,0) -- ({\PI}, 0);
\draw[domain=0:pi, help lines,smooth] 
plot (\x, {sin(\x r)});
\draw[domain=0.2:pi/3] 
  plot (\x, {sin((2*\x+ \PI/2) r)});
\draw[domain=pi/3:2*pi/3]
  plot (\x, {sin((2*\x- \PI/3) r)});
\draw[domain=2*pi/3:pi-0.2]
  plot (\x, {sin((2*\x+ 4*\PI/3) r)});
\end{circuitikz}

Можно сделать вывод: при $m \approx 6$ -- самое распространенное число фаз ${\displaystyle f_{\textcyrillic{выхода max}} \approx \frac{f_{\textcyrillic{сети}}}{2}}$, оптимисты говорят $\frac{2}{3}$

При $f_{\textcyrillic{сети}} = 50 {\textcyrillic{Гц.}}$ $f_{\textcyrillic{выхода max}} = 25...33$

\section{Силовые схемы НПЧ}

\begin{tabular}{c|cl}
\toprule
$m_{\textcyrillic{вх}}$ & $m_{\textcyrillic{вых}}$ \\
\midrule
1(2) & 1(2), 3 & одно-двухполупериодные\\
3 & 1 \\
3 & 3 \\
3 & >3 \\
\bottomrule
\end{tabular}

Где применять? Для электропривода нужны 3 фазы. Для мощной нагрузки $3\rightarrow 3$. А каждая фаза -- два нереверсивных преобразователя,
значит нужны 6 комплектов вентилей. Питание либо от 3-х фазного трансформатора, либо от четырехобмоточного.


\begin{tikzpicture}\draw 
(1.3,1.0)rectangle(2.7,2.3) (1.5,1.3)to[Ty](2.5,1.3)--(2.5,2)to[Ty](1.5,2)--(1.5,1.3) % преобразователь
(2.4,1)to[short,-*] (2.4,0.7)
(1.6,1)--(1.6,0.5) node[below] {B}
(-0.7,1.0)rectangle(0.7,2.3) (-0.5,1.3)to[Ty](0.5,1.3)--(0.5,2)to[Ty](-0.5,2)--(-0.5,1.3) % преобразователь
(0.4,1)to[short,-*] (0.4,0.7)
(-0.4,1)--(-0.4,0.5) node[below] {A}
(3.3,1.0)rectangle(4.7,2.3) (3.5,1.3)to[Ty](4.5,1.3)--(4.5,2)to[Ty](3.5,2)--(3.5,1.3) % преобразователь
(4.4,1)to[short,-*] (4.4,0.7)
(3.6,1)--(3.6,0.5) node[below] {C}
%
(2,2.3)--(2,3) (1.9,2.55) -- (2,2.65) node[right] {m} -- (2.1,2.75) % соединитель /m
(2,3.5)circle(0.5) % нижяя обмотка 
({2-0.8*cos(30)}, {4.3 - 0.8*sin(30)}) circle(0.5) % левая обмотка 
({2-1.3*cos(30)}, {4.3 - 1.3*sin(30)})--({2-2.3*cos(30)}, {4.3 - 2.3*sin(30)})--({2-2.3*cos(30)}, 2.3) 
 (-0.1,2.55) -- (0,2.65) node[right] {m}-- (0.1,2.75) % соединитель  / m
%
({2+0.8*cos(30)}, {4.3 - 0.8*sin(30)}) circle(0.5) % правая обмотка
({2+1.3*cos(30)}, {4.3 - 1.3*sin(30)})--({2+2.3*cos(30)}, {4.3 - 2.3*sin(30)})--({2+2.3*cos(30)}, 2.3)
(3.9,2.55) -- (4,2.65) node[right] {m} -- (4.1,2.75) % соединитель /m
%
(2,4.3)circle(0.5)  %  верхняя обмотка , расстояние между центрами 1.3
(2,4.8)--(2,5.3)--(1.7,5.8)
(2,5.5)node[right]{Q}
(2,5.8)to[short,-*](2,6.3)
(0,6.3)--(4,6.3) % верхний провод
(1.1,6.2)--(1.3,6.4) (0.95,6.2)--(1.15,6.4) (0.8,6.2)--(1.0,6.4) % ///
(0.4,0.7) -- (4.8,0.7) % нижний провод

(7,3.5)circle(0.5) % нижяя обмотка
(7,4.3)circle(0.5)  %  верхняя обмотка
(7,4.8) -- (7,5.8)
(6.9,5.15) -- (7.0,5.2) -- (7.1,5.25) % соединитель ///
(6.9,5.25) -- (7.0,5.3) -- (7.1,5.35) % соединитель ///
(6.9,5.35) -- (7.0,5.4) -- (7.1,5.45) % соединитель ///

(9,3.5)circle(0.5) % нижяя обмотка
(9,4.3)circle(0.5)  %  верхняя обмотка
(9,4.8) -- (9,5.8)
(8.9,5.15) -- (9.0,5.2) -- (9.1,5.25) % соединитель ///
(8.9,5.25) -- (9.0,5.3) -- (9.1,5.35) % соединитель ///
(8.9,5.35) -- (9.0,5.4) -- (9.1,5.45) % соединитель ///

(11,3.5)circle(0.5) % нижяя обмотка
(11,4.3)circle(0.5)  %  верхняя обмотка
(11,4.8) -- (11,5.8)
(10.9,5.15) -- (11.0,5.2) -- (11.1,5.25) % соединитель ///
(10.9,5.25) -- (11.0,5.3) -- (11.1,5.35) % соединитель ///
(10.9,5.35) -- (11.0,5.4) -- (11.1,5.45) % соединитель ///

(9,2.8) node {вместо одного четырехобмоточного 3 двухобмоточных}

;\end{tikzpicture}

Если $m=3$, значит 2 моста по 6 вентилей, 36 тиристоров.

Для упрощения схема НПЧ используется без трансформатора. Трансформаторы обычно используются для преобразования напряжения, числа фаз, ограничения тока К.З.,
для гальванической развязки. Для функции ограничения тока К.З. используются реакторы.

\begin{tikzpicture}
\draw 
(0,4.2) to[Ty] (0,3)
(0.5,4.2) to[Ty] (0.5,3)
(1,4.2) to[Ty] (1,3)

(2,3) to[Ty] (2,4.2)
(2.5,3) to[Ty] (2.5,4.2)
(3,3) to[Ty] (3,4.2)
% соединяем тиристоры сверху
(1,4.2) -- (2,4.2)
(0.5,4.2) -- (0.5,4.3) -- (2.5,4.3) -- (2.5,4.2)
(0,4.2) -- (0,4.4) -- (3,4.4) -- (3,4.2)
% соединяем тиристоры снизу
(0,3) -- (1,3) (2,3) -- (3,3) 


(0,3) -- (0,2)                                           (3,2) -- (3,3)
           (0,2) to[L,l={Ур$_1$}] (1.5,2) to[L,l={Ур$_2$}] (3,2)
(1.5,6) to[L] (1.5,4.4) to[short,-*] (1.5,4.3) 
(1.1,6) to[L,-*] (1.1,4.4)
(1.9,6) to[L] (1.9,4.4) to[short,-*] (1.9,4.2)
;\end{tikzpicture}

\section{Автономные инверторы}

Есть три основных типа инверторов: АИН -- инвертор напряжения, этот инвертор больше всего будет нас интересовать; АИТ -- инвертор тока; АИР -- резонансный (наименне интересный)
В чем различие АИН, АИТ, АИР. Предварительно обратим внимание на различие свойств.
АИН формирует на выходе форму кривой напряжения. Форма тока может отличаться. Если нагрузка реактивная $L(\omega)\nearrow, C(\omega)\searrow$ с ростом частоты. 

Классификация АИН:
\begin{enumerate}
\item однофазные, трехфазные (регулируемой частоты, постоянной частоты). Имеем ввиду управление трехфазным приводом. Для двигателя нужно создать 
вращающееся поле. Чаще всего 3 фазы, бывает 2 фазы, бывает много. Изучив однофащные и трехфазные, остальные получим простым увеличением числа фаз.

\item схемы нулевые, мостовые (в выпрямителях концевые).\\
Тип СПП: преобразователт из постоянного в переменное $=\rightarrow\approx$, причем автономные, т.е. независимые не ведомые сетью. Включить СПП я могу, а выключить не
получиться. Обязательно нужны запираемые приборы: запираемые тиристоры, силовые транзисторы.
1й тип будет похлж на ИППН. Наработки, полученные в ИППН будут использоваться в АИ.
\item трансформатороные, безтрансформаторные
\end{enumerate}
24в $\rightarrow$ в переменное, затем через трансформатор увеличить.\\

Начнем с однофазного АИН

%https://tex.stackovernet.com/ru/q/124245/%D0%BA%D0%B0%D0%BA-%D0%BD%D0%B0%D1%80%D0%B8%D1%81%D0%BE%D0%B2%D0%B0%D1%82%D1%8C-%D1%82%D1%80%D0%B0%D0%BD%D1%81%D1%84%D0%BE%D1%80%D0%BC%D0%B0%D1%82%D0%BE%D1%80-%D1%82%D0%BE%D0%BA%D0%B0-%D0%B2-tikzcircuitikz
%https://tex.stackexchange.com/questions/355717/circuittikz-4-pinfet-symbol/355802#35580
\begin{wrapfigure}{l}{0.4\linewidth}
\begin{circuitikz} 
\draw (1.5,6) node[Lnigbt, bodydiode, rotate=90, xscale=-1] (t1) {};
\draw (0,6) to[short, o-] (t1.D);
\draw (t1.S) -- (3,6) -- (3,4.5); \draw (0,4.5) -- (0.5,4.5)  to[R] (1.75,4.5) to[L] (3,4.5); \draw (0,4.2) to[short,o-o] (0,4.8);
\draw (1.5,3) node[Lnigbt, bodydiode, rotate=90, xscale=1] (t2) {};
\draw (0,3) to[short,o-] (t2.D);
\draw (t2.S) -- (3,3) -- (3,4.5);
\draw (-0.2, {4.5 + 1.5/2}) node() {$U_{T/2}$};
\draw (-0.3, 5.9) node() {$+$} (0.3,4.9) node() {$-$} (0.8,4.8) node() {$-$} (2.7,4.8) node() {$+$};
\draw (0.3,4.1) node() {$+$};
\draw (-0.3, 3)  node() {$-$};
\end{circuitikz}
\end{wrapfigure}
Нулевая схема получится, если включил верхний транзистор. Вторую полуволну получу, если подключу нижний. Знак на транзисторе означает, что транзистор управляется напряжением.
На сомом деле это два транзистора. Иногда из-за ``нулевой'' точки схема называется нулевой.

Обратим внимание. Когда нагрузка индуктивная, то включать верхний транзистор -- будет перенапряжение. Поэтому всегдя в схеме транзисторы шунтируются обратными
диодами. При подаче напряжение $+$ $-$ -- запираем диод. Как диоды проводят? При выключении нижний диод поддержит ток при выключении верхнего транзистора. %диода.

%${\displaystyle \frac{\partial i}{\partial t}}$ громадное, но установлены диоды, но кроме $L_\text{н}$ может быть индуктивность питания, то беды не миновать.

\begin{wrapfigure}{l}{0.3\linewidth}
\begin{circuitikz} 
\draw (1.5,6) node[Lnigbt, bodydiode, rotate=90, xscale=-1] (t1) {};
\draw (0,6) to[short,o-] (t1.D);
\draw (t1.S) -- (3,6) -- (3,4.5); \draw (0,4.5) -- (0.5,4.5)  to[R] (1.75,4.5) to[L] (3,4.5); \draw (0,4.2) to[short,o-o] (0,4.8);
\draw[help lines,smooth] (0,3) to[C,o-o] (0,4.2);
\draw[help lines, smooth] (0,6) to[C,o-o] (0,4.8);
\end{circuitikz}
\end{wrapfigure}
${\displaystyle \frac{\partial i}{\partial t}}$ громадное, но установлены диоды, но кроме $L_\text{н}$ может быть индуктивность питания, то беды не миновать.

Чтобы этого не произошло, устанавливают конденсатор, Большая (полярная) емкость. Параллельно ставят с маленькой паразитной емкостью (внутренней индуктивностью).
Элетролитический конденсатор, громадная емкость 6000$\mu$F.\\

\begin{minipage}{\textwidth}
%Параллельно включают маленький несколько pF. В первые пикосекунды ток идет через маленький конденсатор. Он предназначен,
%чтобы поглотить начальный бросок.
  \begin{wrapfigure}{r}{0.2\textwidth}
  \begin{circuitikz}
    \draw (0,2) to[elko] (0,0);
    \draw (0,0.4) -- (1,0.4) to[C] (1,1.6) -- (0,1.6);
    \end{circuitikz}
\end{wrapfigure}
Параллельно включают маленький конденсатор несколько pF. В первые пикосекунды ток идет через маленький конденсатор. Он предназначен,
чтобы поглотить начальный бросок.
\end{minipage}\\ \\ \\

\begin{minipage}{\textwidth}
\begin{wrapfigure}{l}{0.2\linewidth}
  \begin{circuitikz}
    \draw (0,3) -- (1.5,3) to[C] (1.5,1.5) to[C] (1.5,0) -- (0,0);
    \draw (0.65,3) to[R] (0.65,1.5) to[R] (0.65,0);
    \draw (0.65,1.5) to[short,*-*] (1.5,1.5); 
  \end{circuitikz}  
\end{wrapfigure}
Конденсаторы образуют емкостной делитель, который не работает на постоянном токе. Конденсатор нужно шунтировать баллсатным сопротивлением.
Сопротивление нужно выбрать меньше гарантированного сопротивления утечки конденсатора. Если сопротивление утечки конденсатора 5мОм, то сопротивление
балластных резисторов выберем в 100 раз меньше сопротивления утечки. Если не ставить сопротивления, то напряжение поделится пропорционально нестабильному
сопротивлению изоляции C.
\end{minipage}\\ \\

%\begin{wrapfigure}{l}{0.4\linewidth}
\begin{circuitikz} \begin{scope}[scale=1.1]
    \draw (1.5,6) node[Lnigbt, bodydiode, rotate=90, xscale=1] (t1) {};
    \draw (2.3,5.5) node {VT$_1$} (2.4,6.4) node {VD$_1$};
    \draw (-1.5,6) node[left] {$+$} to[short,o-] (t1.D);
    \draw (2.3,3.5) node {VT$_2$} (2.4,2.6) node {VD$_2$};
\draw (t1.S) -- (3,6) -- (3,4.5); \draw (0,4.5) -- (0.5,4.5)  to[R] (1.65,4.5) to[L]  (2.75,4.5) -- (3,4.5); 
\draw (1.5,3) node[Lnigbt, bodydiode, rotate=90, xscale=-1] (t2) {};
\draw (-1.5,3) node[left] {$-$} to[short,o-] (t2.D);
\draw (t2.S) -- (3,3) -- (3,4.5);
\draw (0,6) to[R,-*] (0,4.5) to[R] (0,3); 
\draw (-.8,6) to[C,-*] (-.8,4.5) to[C] (-.8,3);
\draw (0,4.5) -- (-.8,4.5) node[left] {точка (o)\ \ \ };
\draw[help lines, smooth] (0.5,4.25) rectangle (2.8,4.75);
\draw[thin,<-,>=stealth'] (-0.2,3.6) -- (-0.8,2.2) node[below] {50 kV};  
\draw[thin,<->,>=stealth'] (0,2.3) -- node[midway, below] {$U_\text{н}\text{ маленкое}$} (3,2.3); 
\draw[thin,<-,>=stealth'] (0.4,3.3) -- (1.1,3.9) -- (3.5,3.9) node[right] {Эта цепь должна быть очень короткая};
\end{scope}
\end{circuitikz}
%\end{wrapfigure}

\begin{circuitikz}
\usetikzlibrary{calc}
\usetikzlibrary{decorations.pathreplacing,intersections,calc}
  \draw[thin,->,>=stealth'] (0,-3) -- (0,3.5) node[left] {$U$};
  \draw[thin,->,>=stealth'] (0,0) coordinate (x_1) -- (8,0) coordinate (x_2) node[above right] {$t$};
  \draw (0,-2) -- (1.5,-2) -- (1.5,2) -- (3,2) -- (3,-2) -- (4.5,-2) -- (4.5,2) -- (6,2) -- (6,-2) -- (7.5,-2) -- (7.5,2);
  \draw (0,-2) node[left] {$\displaystyle -\frac{U_V}{2}$};
  \draw[thin,loosely dashed] (1.5,2) -- (0,2) node[left] {$\displaystyle \frac{U_V}{2}$};
  \draw [domain=0:1.5] plot (\x, {-2.6+4*exp(-.8*\x)});        \draw[domain=1.5:6, thin, loosely dashed] plot (\x, {-2.6+4*exp(-.8*\x)});
  \draw [domain=1.5:3] plot (\x, {2.6 - 4*exp(-.8*(\x-1.5))}); \draw[domain=3:7.5, thin, loosely dashed] plot (\x, {2.6 - 4*exp(-.8*(\x-1.5))});
  \draw [name path=aa, domain=3:4.5] plot (\x, {-2.6+4*exp(-.8*(\x-3))});    \draw[domain=4.5:7.5, thin, loosely dashed] plot (\x, {-2.6+4*exp(-.8*(\x-3))});
  \draw [name path=cc, domain=4.5:6] plot (\x, {2.6 - 4*exp(-.8*(\x-4.5))}); \draw[domain=5:7.5, thin, loosely dashed] plot (\x, {2.6 - 4*exp(-.8*(\x-4.5))});
  \draw [domain=6:7.5] plot (\x, {-2.6+4*exp(-.8*(\x-6))});
  \draw [very thin, loosely dotted] (0,2.6) -- (7.3,2.6);
  \draw [thin] (1.5,2) -- (1.5,3.2) (3,2) -- (3,3.2) (4.5,2) -- (4.5,3);
  \draw [very thin,<->,>=stealth'] (1.5,3) -- (3,3) node[midway,above] {$T/2$};\draw[very thin,<->,>=stealth'] (3,2.8) -- (4.5,2.8) node[midway,above] {$T/2$};
  \draw [thin] (3,-2) -- (3,-3.2) node (a1) {};
  \path [name path=bb] (0,0) -- (7,0);
  \path [name intersections={of=aa and bb,by=E}];
  \draw [thin] (E) -- ++ (0,-3.2) node (a2) {};
  \draw [thin] (4.5, -2) -- (4.5, -3.2) node (a3) {};
  \path [name intersections={of=cc and bb,by=C}];
  \draw [thin] (C) -- ++ (0,-3.2) node (a4) {};
  \draw [thin] (6, -2) -- (6, -3.2) node (a5) {};
  \draw [thin,<->,>=stealth'] (3,-2.9) -- (6,-2.9) node[right] {$\displaystyle T=\frac{1}{f}$};
  \path (a1.center) -- (a2.center) node[midway, below] {1};
  \path (a2.center) -- (a3.center) node[midway, below] {2};
  \path (a3.center) -- (a4.center) node[midway, below] {3};
  \path (a4.center) -- (a5.center) node[midway, below] {4};
  \path [name path=dd] (0,.6) -- (7,.6);
  \path [name intersections={of=aa and dd,by=D}];
  \draw[thin,->,>=stealth'] (2.5,-3.9) node[right] {нижний транзистор еще не включен} -- (3,-2.7);
  \draw[thin,->,>=stealth'] (1.6,-4.4) node[right] {ток течет по диоду} -- (D);
\end{circuitikz}  

% https://tex.stackexchange.com/questions/241959/aligned-circle-around-letter-tikz
\newcommand*\circled[1]{\tikz[baseline=(char.base)]{
        \node[shape=circle,draw,minimum size=4mm, inner sep=0pt] (char)
        {\rule[-3pt]{0pt}{\dimexpr2ex+2pt}#1};}}



\begin{tabular}{l}
На интервале \circled1  ток протекает через $VD_2$ --- $i_{VD2}$.\\
В момент \circled2 транзистор $VT_2$ должен быть открыт. \circled2 --- $i_{VT2}$\\
\circled3 --- $i_{VD1}$\\
\circled4 --- $i_{VT1}$
\end{tabular}

На интервалах \circled2 и \circled4 конденсаторы разряжаются, на интервалах \circled1 и \circled3 -- заряжаются.

На \circled1 ток шел в нижний конденсатор $C_2$, на \circled3 --- в верхний конденсатор $C_1$.

Обратим внимание на маленький момент:

Если
\begin{circuitikz}[scale=0.3]
  \draw  (1.5,0) -- (1.5,2) -- (3,2) -- (3,-2) -- (4.5,-2) -- (4.5,0);
  \path[pattern=north east lines,fill opacity=5,opacity=0.2] (1.5,0) rectangle (3,2);
  \path[pattern=north east lines,fill opacity=5,opacity=0.2] (3,-2) rectangle (4.5,0);
\end{circuitikz}
то в нагрузке постоянная составляющая $U$, значит будет ток $I$.

Если
\begin{circuitikz}[scale=0.3]
  \draw [domain=2.04:3] plot (\x, {2.6 - 4*exp(-.8*(\x-1.5))}); 
  \draw [name path=aa, domain=3:4.5] plot (\x, {-2.6+4*exp(-.8*(\x-3))});
  \draw [name path=cc, domain=4.5:5.04] plot (\x, {2.6 - 4*exp(-.8*(\x-4.5))});
  %  \draw [domain=6:7.5] plot (\x, {-2.6+4*exp(-.8*(\x-6))});
%  \draw [name path=bb] (2.04,0) -- (5.04,0);  
\end{circuitikz}
--- тогда нет постоянного тока. Если нагрузка двигатель, то постоянная составляющая очень опасна.

Отличаются на 1\%. Делитель не обеспечивает деление, или равенство полуволн необязательно. $400V - 1\% - 4V$. Сопротивление миллиОмы, то
$$
I = \frac{4V}{10\mu\Omega} = \frac{4}{10\cdot 10^{-3}} = 400A
$$
Пусть будет $40A$. Что такое $40A$ по сравнению со $100A$? Ток намагничивания $1-2\%$ у трансформатора $500A$ ток намагничивания $10A$.

Во всех инверторах выходное напряжение не должно содержать постоянной составляющей. Постоянная составляющая, даже небольшая, может создать в обмотках питаемого двигателя
постоянную составляющую тока, которая способна насытить магнитную цепь двигателя. А это, в свою очередь, вызовет резкое увеличение намагничивающего тока.
Процесс развивается лавинообразно.

У мощного трансформатора, двигателя $X \gg R$. Если $Z$ мало, при номинальном токе $5-7\%$
$$
I_\text{ном}\cdot Z_\text{обмоток} = 5\%
$$

$X \gg R$ то 1\% медь 1\% сталь
\begin{tabular}{l}
  $I^2R - 98\%$ \\
  $IR = 1\%$
\end{tabular}

Напряжение около $1\%$, то ток номинальный

\begin{tabular}{lll}
  у двигателя & &\\
  у трансформатора & 3-4-5\% & ток Х.Х. - 5\% \\
  асинхронный & 20-30\% даже 40
\end{tabular}  

А если $6,7\%$ -- уже трансформатор насытится! Точность нужна почти метрологическая. $500A$. Резисторы $50K$.

Самосинхронизация

Для \begin{circuitikz}[scale=0.4]
\draw (0,0) to[C] (0,1.3) to[C] (0,2.6);
\end{circuitikz} постоянного тока -- большое сопротивление.
Если появляется постоянная составляюшая на конденсаторе -- вствечное падение напряжения. Конденсатор подзарядится -- полуволны выровняются. В других схемах может быть придется позаботится.

{\large Недостаток схемы}

$U_\text{питания} = 100\text{ вольт}$, если нижний заперт, то к СПП --- $100\text{ вольт}$, а к нагрузке $\displaystyle \frac{U_\text{п}}{2} = 50\text{ вольт}$, а на транзисторах и диодах
полное $U_\text{п}$

Нужна ``синусоида'' напряжения. Разложим в ряд Фурье:

\begin{circuitikz}
  \draw (0,0) node[right] {$\displaystyle U_{(0)} = \frac{1}{2\pi}\int\limits_0^t U\;dt\;\cos\frac{\pi}{2\pi f} t$};
  \draw [<-,>=stealth'] (0.4,-0.3) -- (0, -1) node[below] {$\begin{array}{c}\text{основной}\\ \text{гармоники}\end{array}$};
\end{circuitikz}

Если выбрать начало координат так, чтобы оставалась только косинусная составляющая

\begin{circuitikz}
  \newcommand{\pp}{3.149265}
  \draw[thin,->,>=stealth'] (-2,0) -- (4,0);
  \draw[thin,->,>=stealth'] (0,-1.5) -- (0,1.7);
  \draw (-0.5,0) -- (-0.5,1) -- (0.5,1) -- (0.5,-1) -- (1.5,-1) -- (1.5,1) -- (2.5,1) -- (2.5,-1) -- (3.5,-1) -- (3.5,0);
  \draw[domain=-0.5:3.5, help lines,samples=200]
    plot (\x, {2*0.45*cos(\pp*\x r)});
\end{circuitikz}

$$
U_{(\text{п})} = 0.45 U_\text{п}
$$

Гармоники $\displaystyle \frac{U_{(\text{п})}}{U_{(0)}}$, а основную частоту задает система упрввления.

Амплитуда гармоники 0.635 $(0.45\sqrt{2})$. Действующее значение $U_{(\text{п})} = $ в $n$ раз меньше, где $n$ -- весь ряд нечетных чисел, четных гармоник нет. Четные гармоники по разному искажают полуволны.

\begin{wrapfigure}{l}{0.7\linewidth}
\begin{circuitikz}[yscale=1.3,samples=200]
  \draw[thin,->,>=stealth'] (-3.5,0) -- (3.5,0);
  \draw[domain=-3.14:3.14, help lines,smooth]  plot (\x,{-sin(\x r)});
  \draw[domain=-3.14:3.14, help lines, smooth] plot (\x,{.3*sin(3*\x r)});
  \draw[domain=-3.14:3.14] plot (\x,{-sin(\x r) + .3*sin(3*\x r)});
\end{circuitikz}
\end{wrapfigure} форма обеих полуволн искажается одинаково.
\vspace{1.5cm}
\begin{wrapfigure}{l}{0.7\linewidth}
  \begin{circuitikz}[yscale=1.3,samples=200]
  \draw[thin,->,>=stealth'] (-1.7,0) -- (5,0);  
  \draw[domain=-1.57:4.71, help lines,smooth] plot (\x, {cos(\x r)});   
  \draw[domain=-1.57:4.71, help lines, smooth] plot (\x,{.5*cos(2*\x r)});
  \draw[domain=-1.57:4.71] plot (\x,{cos(\x r) + .5*cos(2*\x r)});
\end{circuitikz}  
\end{wrapfigure} по площади одинаково, но выглядит неодинаково
\vspace{1cm}
На нашей кривой обе полуволны одинаковы, но появится постоянная составляющая

%\include{lection4}
\end{document}
