\section{Непосредственные преобразователи частоты (без промежуточного звена постоянного тока)}

НПЧ -- реверсивный тиристорный преобразователь, с любым числом фаз, с любым способом управления (совместимым с ...)

Общий принцип -- угол $\alpha$ непрерывно изменяется обеспечивая .... изменение постоянного напряжения выходной составляющей.

$f_{\textcyrillic{выхода}} \le f_{\textcyrillic{сети}}, f_{\textcyrillic{входа}} = 50 {\textcyrillic{Гц.}} $

Это главная особенность и один из главных недостатков. $f_{\textcyrillic{выхода}} = 50 {\textcyrillic{Гц.}} $ может быть достигнуто
при большом $m \rightarrow \infty$

\begin{circuitikz}
\newcommand{\PI}{3.149265}
\draw[thin] (-2,0) -- ({\PI/2+0.5}, 0);
\draw[domain=-2:-1.5,help lines,smooth]
    plot (\x, {cos((\x+1.5) r)});
\draw[domain=-1.5:-1,help lines,smooth]
    plot (\x, {cos((\x+1) r)});
\draw[domain=-1:-0.5,help lines,smooth]
    plot (\x, {cos((\x+0.5) r)});
\draw[domain=-0.5:pi/2+0.5,help lines,smooth]
    plot (\x, {cos(\x r)});
\draw[thin,-{Stealth[scale=1.5]}] (-0,-0.8) node[below] {быстрее снижаться не может}  -- ({\PI/4},{0.5});

\draw[domain=-0.5+2*pi:pi/4+2*pi,help lines,smooth]
    plot (\x, {cos(\x r)});
\draw[domain=pi/4+2*pi:pi/3+2*pi + 0.5,help lines,smooth]
    plot (\x, {cos((\x+0.7) r)});
\draw[thin,dashed] ({\PI*(1/4.+2)},0.1) -- ({\PI*(1/4.+2)}, {sqrt(2)/2});
\draw[thin,-{Stealth[scale=1.5]}] ({\PI*(1/4.+2)-0.6},-0.6) node[below] 
{$\begin{array}{c}
\textcyrillic{такое невозможно --}\\
\textcyrillic{связано с невозможностью}\\ 
\textcyrillic{запирания тиристора}\end{array}$} -- ({\PI*(1/4.+2)-0.1},-0.1);
\end{circuitikz}

Так считалось пока не было запираемых тиристоров. Но нужно менять принцип управления -- нужно включать предыдущую фазу $(k-1)$ а не последующую.
Но с энергетическими процессами. Запасенная в индуктивности фазы связана с большими потерями. Чем меньшее число фаз, я должен набрать 
новую синусоиду из кусочков синусоид 50Гц. Из четырех кусочков, из трех, из двух! Очень большие искажения если из трех.
Полуволну 25Гц набрать из 3х кусочков синусоид.
\begin{circuitikz}
\newcommand{\PI}{3.149265}
\draw[thin] (0,0) -- ({\PI}, 0);
\draw[domain=0:pi, help lines,smooth] 
plot (\x, {sin(\x r)});
\draw[domain=0.2:pi/3] 
  plot (\x, {sin((2*\x+ \PI/2) r)});
\draw[domain=pi/3:2*pi/3]
  plot (\x, {sin((2*\x- \PI/3) r)});
\draw[domain=2*pi/3:pi-0.2]
  plot (\x, {sin((2*\x+ 4*\PI/3) r)});
\end{circuitikz}

Можно сделать вывод: при $m \approx 6$ -- самое распространенное число фаз ${\displaystyle f_{\textcyrillic{выхода max}} \approx \frac{f_{\textcyrillic{сети}}}{2}}$, оптимисты говорят $\frac{2}{3}$

При $f_{\textcyrillic{сети}} = 50 {\textcyrillic{Гц.}}$ $f_{\textcyrillic{выхода max}} = 25...33$

\section{Силовые схемы НПЧ}

\begin{tabular}{c|cl}
\toprule
$m_{\textcyrillic{вх}}$ & $m_{\textcyrillic{вых}}$ \\
\midrule
1(2) & 1(2), 3 & одно-двухполупериодные\\
3 & 1 \\
3 & 3 \\
3 & >3 \\
\bottomrule
\end{tabular}

Где применять? Для электропривода нужны 3 фазы. Для мощной нагрузки $3\rightarrow 3$. А каждая фаза -- два непеверсивных преобразователя,
значит нужны 6 комплектов вентилей. Питание либо от 3-х фазного трансформатора, либо от четырехобмоточного.

