\section{Непосредственные преобразователи частоты (без промежуточного звена постоянного тока)}

НПЧ -- реверсивный тиристорный преобразователь, с любым числом фаз, с любым способом управления (совместимым с ...)

Общий принцип -- угол $\alpha$ непрерывно изменяется обеспечивая .... изменение постоянного напряжения выходной составляющей.

$f_{\textcyrillic{выхода}} \le f_{\textcyrillic{сети}}, f_{\textcyrillic{входа}} = 50 {\textcyrillic{Гц.}} $

Это главная особенность и один из главных недостатков. $f_{\textcyrillic{выхода}} = 50 {\textcyrillic{Гц.}} $ может быть достигнуто
при большом $m \rightarrow \infty$

\begin{circuitikz}
\newcommand{\PI}{3.149265}
\draw[thin] (-2,0) -- ({\PI/2+0.5}, 0);
\draw[domain=-2:-1.5,help lines,smooth]
    plot (\x, {cos((\x+1.5) r)});
\draw[domain=-1.5:-1,help lines,smooth]
    plot (\x, {cos((\x+1) r)});
\draw[domain=-1:-0.5,help lines,smooth]
    plot (\x, {cos((\x+0.5) r)});
\draw[domain=-0.5:pi/2+0.5,help lines,smooth]
    plot (\x, {cos(\x r)});
\draw[thin,-{Stealth[scale=1.5]}] (-0,-0.8) node[below] {быстрее снижаться не может}  -- ({\PI/4},{0.5});

\draw[domain=-0.5+2*pi:pi/4+2*pi,help lines,smooth]
    plot (\x, {cos(\x r)});
\draw[domain=pi/4+2*pi:pi/3+2*pi + 0.5,help lines,smooth]
    plot (\x, {cos((\x+0.7) r)});
\draw[thin,dashed] ({\PI*(1/4.+2)},0.1) -- ({\PI*(1/4.+2)}, {sqrt(2)/2});
\draw[thin,-{Stealth[scale=1.5]}] ({\PI*(1/4.+2)-0.6},-0.6) node[below] 
{$\begin{array}{c}
\textcyrillic{такое невозможно --}\\
\textcyrillic{связано с невозможностью}\\ 
\textcyrillic{запирания тиристора}\end{array}$} -- ({\PI*(1/4.+2)-0.1},-0.1);
\end{circuitikz}

Так считалось пока не было запираемых тиристоров. Но нужно менять принцип управления -- нужно включать предыдущую фазу $(k-1)$ а не последующую.
Но с энергетическими процессами. Запасенная в индуктивности фазы связана с большими потерями. Чем меньшее число фаз, я должен набрать 
новую синусоиду из кусочков синусоид 50Гц. Из четырех кусочков, из трех, из двух! Очень большие искажения если из трех.
Полуволну 25Гц набрать из 3х кусочков синусоид.
\begin{circuitikz}
\newcommand{\PI}{3.149265}
\draw[thin] (0,0) -- ({\PI}, 0);
\draw[domain=0:pi, help lines,smooth] 
plot (\x, {sin(\x r)});
\draw[domain=0.2:pi/3] 
  plot (\x, {sin((2*\x+ \PI/2) r)});
\draw[domain=pi/3:2*pi/3]
  plot (\x, {sin((2*\x- \PI/3) r)});
\draw[domain=2*pi/3:pi-0.2]
  plot (\x, {sin((2*\x+ 4*\PI/3) r)});
\end{circuitikz}

Можно сделать вывод: при $m \approx 6$ -- самое распространенное число фаз ${\displaystyle f_{\textcyrillic{выхода max}} \approx \frac{f_{\textcyrillic{сети}}}{2}}$, оптимисты говорят $\frac{2}{3}$

При $f_{\textcyrillic{сети}} = 50 {\textcyrillic{Гц.}}$ $f_{\textcyrillic{выхода max}} = 25...33$

\section{Силовые схемы НПЧ}

\begin{tabular}{c|cl}
\toprule
$m_{\textcyrillic{вх}}$ & $m_{\textcyrillic{вых}}$ \\
\midrule
1(2) & 1(2), 3 & одно-двухполупериодные\\
3 & 1 \\
3 & 3 \\
3 & >3 \\
\bottomrule
\end{tabular}

Где применять? Для электропривода нужны 3 фазы. Для мощной нагрузки $3\rightarrow 3$. А каждая фаза -- два нереверсивных преобразователя,
значит нужны 6 комплектов вентилей. Питание либо от 3-х фазного трансформатора, либо от четырехобмоточного.


\begin{tikzpicture}\draw 
(1.3,1.0)rectangle(2.7,2.3) (1.5,1.3)to[Ty-](2.5,1.3)--(2.5,2)to[Ty-](1.5,2)--(1.5,1.3) % преобразователь
(2.4,1)to[short] (2.4,0.7)
(1.6,1)--(1.6,0.5) node[below] {B}
(-0.7,1.0)rectangle(0.7,2.3) (-0.5,1.3)to[Ty-](0.5,1.3)--(0.5,2)to[Ty-](-0.5,2)--(-0.5,1.3) % преобразователь
(0.4,1)to[short] (0.4,0.7)
(-0.4,1)--(-0.4,0.5) node[below] {A}
(3.3,1.0)rectangle(4.7,2.3) (3.5,1.3)to[Ty-](4.5,1.3)--(4.5,2)to[Ty-](3.5,2)--(3.5,1.3) % преобразователь
(4.4,1)to[short] (4.4,0.7)
(3.6,1)--(3.6,0.5) node[below] {C}
%
(2,2.3)--(2,3.3) (1.9,2.55) -- (2,2.65) node[right] {m} -- (2.1,2.75) % соединитель /m
(2,3.8)circle(0.5) % нижяя обмотка 
({2-0.5*cos(30)}, {4.3 - 0.5*sin(30)}) circle(0.5) % левая обмотка 
({2-1.0*cos(30)}, {4.3 - 1.0*sin(30)})--({2-2.3*cos(30)}, {4.3 - 2.3*sin(30)})--({2-2.3*cos(30)}, 2.3) 
 (-0.1,2.55) -- (0,2.65) node[right] {m}-- (0.1,2.75) % соединитель  / m
%
({2+0.5*cos(30)}, {4.3 - 0.5*sin(30)}) circle(0.5) % правая обмотка
({2+1.0*cos(30)}, {4.3 - 1.0*sin(30)})--({2+2.3*cos(30)}, {4.3 - 2.3*sin(30)})--({2+2.3*cos(30)}, 2.3)
(3.9,2.55) -- (4,2.65) node[right] {m} -- (4.1,2.75) % соединитель /m
%
(2,4.3)circle(0.5)  %  верхняя обмотка , расстояние между центрами 1.3
(2,4.8)--(2,5.3)--(1.7,5.8)
(2,5.5)node[right]{Q}
(2,5.8)to[short](2,6.3)
(0,6.3)--(4,6.3) % верхний провод
(1.1,6.2)--(1.3,6.4) (0.95,6.2)--(1.15,6.4) (0.8,6.2)--(1.0,6.4) % ///
(0.4,0.7) -- (4.8,0.7) % нижний провод

(7,3.8)circle(0.5) % нижяя обмотка
(7,4.3)circle(0.5)  %  верхняя обмотка
(7,4.8) -- (7,5.8)
(6.9,5.15) -- (7.0,5.2) -- (7.1,5.25) % соединитель ///
(6.9,5.25) -- (7.0,5.3) -- (7.1,5.35) % соединитель ///
(6.9,5.35) -- (7.0,5.4) -- (7.1,5.45) % соединитель ///

(9,3.8)circle(0.5) % нижяя обмотка
(9,4.3)circle(0.5)  %  верхняя обмотка
(9,4.8) -- (9,5.8)
(8.9,5.15) -- (9.0,5.2) -- (9.1,5.25) % соединитель ///
(8.9,5.25) -- (9.0,5.3) -- (9.1,5.35) % соединитель ///
(8.9,5.35) -- (9.0,5.4) -- (9.1,5.45) % соединитель ///

(11,3.8)circle(0.5) % нижяя обмотка
(11,4.3)circle(0.5)  %  верхняя обмотка
(11,4.8) -- (11,5.8)
(10.9,5.15) -- (11.0,5.2) -- (11.1,5.25) % соединитель ///
(10.9,5.25) -- (11.0,5.3) -- (11.1,5.35) % соединитель ///
(10.9,5.35) -- (11.0,5.4) -- (11.1,5.45) % соединитель ///

	(9,2.8) node {\begin{tabular}{c}вместо одного четырехобмоточного\\ 3 двухобмоточных\end{tabular}}

;\end{tikzpicture}

Если $m=3$, значит 2 моста по 6 вентилей, 36 тиристоров.

Для упрощения схема НПЧ используется без трансформатора. Трансформаторы обычно используются для преобразования напряжения, числа фаз, ограничения тока К.З.,
для гальванической развязки. Для функции ограничения тока К.З. используются реакторы.

\begin{tikzpicture}[american inductors]

\draw 
(0,4.2) to[Ty-] (0,3)
(0.5,4.2) to[Ty-] (0.5,3)
(1,4.2) to[Ty-] (1,3)

(2,3) to[D-] (2,4.2)
(2.5,3) to[D-] (2.5,4.2)
(3,3) to[D-] (3,4.2)
% соединяем тиристоры сверху
(1,4.2) -- (2,4.2)
(0.5,4.2) -- (0.5,4.4) -- (2.5,4.4) -- (2.5,4.2)
(0,4.2) -- (0,4.6) -- (3,4.6) -- (3,4.2)
% соединяем тиристоры снизу
(0,3) -- (1,3) (2,3) -- (3,3) 
% уравнительные реакторы
(0,3) -- (0,2.2)                                           (3,2.2) -- (3,3)
           (0,2.2) to[L,l={Ур$_1$}] (1.5,2.2) to[L,l={Ур$_2$}] (3,2.2)
% дроссели со стороны сети	   
(1.5,6.4) --(1.5,6) to[L] (1.5,4.6) --(1.5,4.4) to[short] (1.5,4.4) 
(1.1,6.8) --(1.1,6) to[L] (1.1,4.6)
(1.9,6) to[L] (1.9,4.6) -- (1.9,4.4) to[short] (1.9,4.2)

(1.5,2.2) --(1.5,2.0) to[R] (1.5,0.7) (1.6,1.6) node[right] {a} % нагрузка фазы A
             (4.5,2.0) to[R] (4.5,0.7) (4.6,1.6) node[right] {b} % нагрузка фазы B
             (6.2,2.0) to[R] (6.2,0.7) (6.3,1.6) node[right] {c} % нагрузка фазы C
(1.5,0.7) -- (6.2,0.7)	     
(4.2,4.8) -- (4.2, 6.8)  (4.5,4.8) -- (4.5, 6.4) (4.8,4.8) -- (4.8, 6)	     
(5.9, 4.8) -- (5.9, 6.8)  (6.2,4.8) -- (6.2, 6.4) (6.5,4.8) -- (6.5, 6);	     
%
\draw[dashed] (-0.3,2) rectangle (3.3,4.8)
	(4,2) rectangle (5,4.8) (5.7, 2) rectangle (6.7, 4.8) 
	(0,5.7) -- (7.5,5.7) node[right] {0 сети}
	-- (7.5,0.7) node[right] {соединены в звезду}-- (6.2,0.7); % сеть
% сеть
\draw (0,6) node[left] {C} -- (7,6) 
	(0,6.4) node[left] {B} -- (7,6.4)
	(0,6.8) node[left] {A} -- (7,6.8)

;\end{tikzpicture}

Но можно не соединять с $(0)$ если нагрузка симметрична. Нагрузка может быть соединена в треугольник. УР$_1$ УР$_2$ нужны если совмесстное управление. 
Отсутсвие -- свидетельство раздельного управления. 12 тиристоров, 18 вентилей. 3 фазы мало для получения 25герц. Количество полохое из-за пульсаций.
Применяются 6-фазные мостовые схемы. Трансформатор -- дорогая вещь.

Рассмотрим бестрансформаторный мостовой НПЧ. О-ой провод, если схема мостовая тоже не нужен, его даже некуда подключать.

\begin{tikzpicture}
\draw (0,0) to[D-] (1,0) (0, 0.5) to[D-] (1,0.5) (0,1) to[D-] (1,1)  (1,0) -- (2,0) (1,0.5) -- (2,0.5) (1,1) -- (2,1) 
(2,0) to[D-] (3,0) (2,0.5) to[D-] (3,0.5) (2,1) to[D-] (3,1)
(1,2) to[D-] (0,2) (1,2.5) to[D-] (0,2.5) (1,3) to[D-] (0,3) (1,2) -- (2,2) (1,2.5) -- (2,2.5) (1,3) -- (2,3)
(3,2) to[D-] (2,2) (3,2.5) to[D-] (2,2.5) (3,3) to[D-] (2,3)
(1.1,1) to[short] (1.1,3) (1.5,0.5) to[short] (1.5,2.5) (1.9,0) to[short] (1.9,2)
(0,0) -- (0,3) (3,0) -- (3,3)
;\end{tikzpicture}
