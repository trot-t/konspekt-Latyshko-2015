\chapter{Схемы выпрямителей}

Мы занимаемся статическими преобразователями в противовес электромашинным. Статические преобразователи управляются ключами.
Диод, вообще говоря, имеет управление по силовой цепи: в тот момент когда ток спадает до 0, диод выключается.
Диод управляется полярностю сети.
Для тиристоров лучше употреблять термин "запираемый" или "незапираемый", а не управляемый.

Кроме "классических" приборов существуют модули, в которых несколько приборов интегрированы в одном корпусе.
\begin{figure}[H]
%  \includegraphics[scale=1]{l4a.eps}
\centering
\begin{circuitikz}\draw
  (0,2.5) to[Ty] (0,0.5)
  (1.7,0.5) to[Tr] (1.7,2.5)
  node[left] at (2.9,1.5) {$\approx$}
  (4,0) to[short,-*] (4,0.5)
  -- (4.5,0.5)
  to[Ty,mirror] (4.5, 2.5)
  -- (3.5,2.5)
  to[Ty,mirror] (3.5,0.5)
  -- (4,0.5)
  (4,2.5) to[short,*-] (4,3)  
  ;
  \draw[thin, ->] (3.1, 0) -- (3.1, 0.9);
  \draw[thin, ->] (4.9, 0) -- (4.9, 1.5);
  \draw node[below] at (4,-0.1) {это разные цепи управления};
  \draw node[below] at (4,-0.5) {и между ними могут быть киловольты};
\end{circuitikz}
\caption{Семистор}
\end{figure}

\begin{figure}[H]
\centering
%  \includegraphics[scale=1]{l4_1.eps}
\begin{circuitikz}\draw
(1,3) node[npn,rotate=90] (npn1) {}
(npn1) node[anchor=south] {VT1}
(1,0) to[R] (npn1.B)
(3,3) node[npn,yscale=-1,rotate=-90] (npn2) {}
(npn2) node[anchor=south] {VT2}
(npn2.C) to[short] (4,3)
to[short,l_=b] (4.5,3)
(1,0) to[short] (3,0)
      to[R] (npn2.B)
(npn1.E) to[short] (npn2.E)
(2,3) to[short,*-*] (2,4)
(4,3) to[short,*-] (4,4)
(2,4) to[Do, l=VD2] (4,4)
(2,4) to[short] (4,4) % ГОСТ
(0,3) to[short] (0,4)
(2,4) to[Do, l_=VD1] (0,4)
(2,4) to[short] (0,4) % ГОСТ
(0,3) to[short] (npn1.C)
(-0.5,3) to[short,l_=a] (0,3) % put label
(2,3) to[short,*-] (2,-0.3)
;\end{circuitikz}

  \caption{аналог запираемого тиристора}
\end{figure}


Диоды $VD1$ и $VD2$ шунтируют транзисторы. Если к концу $a$ приложен положительный потенциал, а к концу $b$ приложен отрицательный,
то ток идет по цепи $VT1$ -- $VD2$. При противоположной полярности ток идет по цепи $VT2$ -- $VD1$

%\begin{figure}[H]
%  \includegraphics[scale=1]{l4_2.eps}
%  \caption{}
%  \end{figure}

\begin{figure}[H]
\centering
\begin{circuitikz}\draw
(0,0) node[npn] (npn1) {}
(npn1.B) node[anchor=east] {+} %{$\begin{array}{c}VT1\\+\end{array}$}
%(npn1) node[anchor=west] {VT1}
(1,2) -- (0,2) to[Do] (npn1.C)
(0,2) to[short] (npn1.C) % ГОСТ
(1,0) node[pnp,xscale=-1,yscale=-1] (pnp1) {} 
(pnp1.B) node[anchor=west] {-}
(pnp1.C) to[Do] (1,2)
(pnp1.C) to[short] (1,2) % ГОСТ
(npn1.E) to[short] (pnp1.E)
;\end{circuitikz}
\caption{}
\end{figure}


Диоды параллельны. Чтобы открыть транзисторы нужно подать разнополярные сигналы. Но можно схитрить:

\begin{figure}[H]
%  \includegraphics[scale=1]{l4_3.eps} 
\centering
\begin{circuitikz}\draw
(2,0) node[npn] (npn1) {}
(0,0) to[short] (npn1.B)
(0,-2) to[short,o-] (0,0)
(0,-2) node[anchor=north] {+}
(1,-2) to[R] (1,0)
(3,2) -- (2,2) to[Do] (npn1.C)
(2,2) to[short] (npn1.C) % ГОСТ
(3,0) node[pnp,xscale=-1,yscale=-1] (pnp1) {} 
(pnp1.B) to[short] (5,0)
(4,-2) to[R] (4,0)
%(pnp1.B) node[anchor=west] {-} 
(pnp1.C) to[Do] (3,2)
(pnp1.C) to[short] (3,2) % ГОСТ
(npn1.E) to[short] (pnp1.E)
(5,-2) to[short, o-] (5,0)
(5,-2) node[anchor=north] {-}
(2.5,-2) to[short,*-*] (2.5,-0.75)
(1,-2) to[short] (4,-2)
;\end{circuitikz}
\caption{}
\end{figure}

В этой схеме оба транзистора открыты. При противоположной полярности -- закрыты.

\begin{figure}[H]
\centering
%  \includegraphics[scale=1]{l4_4.eps}
%  \caption{}
%  \end{figure}

\begin{circuitikz}[american voltages]\draw
(0,0) node[npn] (npn1) {}
(npn1.C)  to[short, i<=$i$] (0,2)
(0,2) to[short] (1,2)
(1,2) to[short] (2.25, 0.75)
(2.25, 0.75) to[Do] (1,2)
(2.25, 3.25) to[Do, i=$i$] (1,2)
(2.25, 3.25) to[short] (1,2)
(3.5, 2) to[Do,i=$i$] (2.25, 0.75)
(2.25, 0.75) to[short] (3.5, 2)
(3.5, 2) to[Do] (2.25, 3.25)
(2.25, 3.25) to[short] (3.5, 2)
(3.5,2) to[short] (4,2)
        to[short,i<=$i$] (4,-0.75)
        to[short] (0,-0.75)
(2.25, 0.75) to[short, -o] (2.25, 0)
(2.25, 0) node[anchor=north] {$\sim$}
(2.25, 3.25) to[short,-o] (2.25, 4)
(2.25, 4) node[anchor=south] {$\sim$}
;\end{circuitikz}
  \caption{}
  \end{figure}

На этой схеме используется биполярный транзистор, но можно использовать MOSFET, IGBT, pnp. 
Недостаток: ток всегдa будет протекать по трем приборам. В предыдущей схеме ток протекал по двум приборах.
А в случае двух встечных запираемых тиристорах -- один (правда пятислойный).

Самым распространённым типом преобразовательного прибора является выпрямитель.
Затем по распространённости следут НПЧ, тиристорный регулятор переменного
напряжения.

Силовую структуру выпрямителя мы уже знаем. Дальше идет информационная электроника.

Фильтр - как правило, используется индуктивный фильтр и комбинации $LC$, $CL$, $CLC$, резонансные фильтры.

Трансформаторы - схемы включения \rotatebox[origin=c]{180}{$Y$},$\Delta$,$Z$ и обратные $Y$, \rotatebox[origin=c]{180}{$\Delta$}, $Z$, в
которых соединение концов обмоток поменяно местами по сравнению с
прямой конфигурацией. Такие же схемы соединения обмоток используются
и на первичной стороне.

Трансформатор, вентильная группа, фильтр. Фильтр, как правило,
индуктивный фильтр. Конденсаторы не любят быстроменяющегося
напряжения. Емкость эффективна при холостом ходе и малых токах.
Конденсаторы используются в звене постоянного тока.
В инверторах напряжения используется конденсатор,
если инвертор тока -- индуктивность. Кроме $C$-фильтра могут
быть $CL$ и $CLC$ фильтры.

Основные схемы -- "нулевые". Если "нуль" не выведен, то можно
восстановить "искусственный нуль".

\begin{figure}[H]
  \centering
  \begin{circuitikz}[ american voltages]\draw
    (0,4) to[short] (1,4)
    (0,4) node[anchor=east] {A}
    (1,0) to[R, l=Z] (1,2)
    (1,2) to[short] (1,4)
    (0,3) to[short] (2,3)
    (0,3) node[anchor=east] {B} 
    (2,0) to[R, l=Z] (2,2)
    (2,2) to[short] (2,3)
    (0,2) to[short] (3,2)
    (0,2) node[anchor=east] {C}
    (3,0) to[R, l=Z] (3,2)
    (4.5,0) to[lamp] (4.5,2)
    (4.5,2) to[short, -*] (3,2)
    (1,0) to[short,-*] (2,0)
    (2,0) to[short,-*] (3,0)
    (3,0) to[short] (4.5,0)
    ;\end{circuitikz}
  \caption{схема восстановления искусственного нуля}
\end{figure}
При линейном напряжении между фазами 380V получаем напряжение
на лампе 220V. Активное сопротивление $Z$ должно быть на порядок,
или хотя бы в несколько раз меньше сопротивления нагрузки
$R_{\textcyrillic{н}}$
Использовать $R$ вместо $L$ или $С$ - глупость.

\section{нулевые схемы выпрямителей}
\subsection{Однофазная нулевая схема}
\begin{figure}[H]
  \centering
  \begin{circuitikz}[ american voltages]\draw
    (0,2) to[Ty, o-,v^=$ $] (2,2)
    (0,0) to[short, o-] (2,0)
    ;\end{circuitikz}
\caption{однофазная нулевая схема}
\end{figure}

Подавать на тиристор управляющий сигнал, когда на тиристоре
обратное напряжение крайне нежелательно.

При $U_{VS}<0$ и $|I_S| \approx I_{\textcyrillic{упр}} $ возникает
так называемый транзисторный эффект тиристора, который приводит к
нагреву тиристора. Эти токи имеют порядок $10^{-2}...10^{-1}A$
при температуре около $40^\circ$, но могут достигать $1-2A$
при низких температурах $\approx -40^\circ$. Потому
тиристорные(транзисторные) преобразователи требуют подогрева
при низких температурах.
Максимальный управляющий ток при нормальных условиях достигает
$600 mA$. К чему приводит $I_{\textcyrillic{упр}} = 10^{-2}$,
а $U=1000V$? Тиристор рассеивает сотни ватт, а к ним добавится
полкиловатта. А если в схеме 6-12 тиристоров? Поэтому при
отрицательном напряжении на тиристоре управляющий сигнал не подают.

Электрохимическая нагрузка эквивалентна аккумулятору--встречной ЭДС.
В нагрузке обязательно бывает $L$. $e_{\textcyrillic{н}}$, полярность
может быть согласна, а может быть противоположна приложенному
напряжению.
\begin{circuitikz}[ american voltages]\draw
  (0,0) to[R,l=R] (2,0)
  to[L,l=L]  (4,0)
  to[voltage source, l=e] (6,0)
;\end{circuitikz}  
В частном случае роль $E$ может выполнять конденсатор $C$.
Но один $C$ быть не может, он обязательно должен быть в паре c $R$.
\begin{circuitikz}[ american voltages]\draw
  (0,0) to[R,l=R] (0,2)
        to[short] (1,2)
        to[C,l=C] (1,0)
        to[short] (0,0)  
        ;\end{circuitikz}


\subsection{симметричная двухфазная схема}
% http://tex.stackexchange.com/questions/207770/transformers-in-circuitikz
\begin{figure}[H]
    \centering
\begin{circuitikz}[american]
\draw (0,0) node [transformer core](T){}  % reminded by @PaulGessler, thanks.
      (T.A1) node[above] {+}%A1}
      (T.A2) node[below] {-}%A2}
      (T.B1) node[above] {+}%B1} 
      (T.B2) node[below] {-}%B2}
%(T.base) node{K};
(T.B1) to[Ty] ++ (2,0)
to[short] ++ (1,0)
to[R, l=''$Z_{\textcyrillic{н}}$''] ($(T.B1)!0.515!(T.B2)+(3.0,0)$)
(T.B2) to[Ty] ++ (2,0)
to[short, -*] ++ (0,2.1)
;
% 2 new lines for neutral line on the secondary side.
\draw[thick] ($(T.B1)!0.515!(T.B2)-(0.7,0)$)--node[pos=0.5,above,inner sep=0pt](n){$ $}++ (3.7,0);
;\end{circuitikz}
\end{figure}
\subsection{Двухполупериодная однофазная нулевая схема}
\begin{figure}[H]
      \centering
\begin{circuitikz}[american]
  \draw
  (0,0) to[short,o-*] (1,0)
  to[C,-*] (1,1)
  to[C,-*] (1,2)
  to[short,-o] (0,2)
  (1,0) to[Do] (4,0)
  (1,0) to[short] (4,0)
  to[short] (4,2)
  to[short] (1,2)
  to[Do] (4,2)
  (1,1) to[short,-o] (2,1)
  (2,1) node[anchor=south] {-}
  (4,1) to[short,-o] (3,1)
  (3,1) node[anchor=south] {+}
  ;\end{circuitikz}
\end{figure}
Потечет ли ток? Ток не потечёт, поскольку конденсатор не пропускает постоянный ток.
По ``нулевому'' проводу должен протекать постоянный ток.
Создадим искусственный ток:

\begin{figure}[H]
  \centering
  \begin{circuitikz}[american]
    \draw
    (0,0) to[short,o-*] (1,0)
    to[C,-*] (1,1)
    to[C,-*] (1,2)
    to[short,-o] (0,2)
    (1,0) to[short] ++ (2,0)
    (3,0)  to[Do] ++ (3,0)
    (3,0) to[short] ++ (3,0)
    to[short] ++ (0,2)
    to[short] (1,2)
    (3,2) to[Do] ++ (3,0)
    (1,1) to[short,-o] ++ (3,0)
    node[anchor=south] {-}
    (6,1) to[short,-o] ++ (-1,0)
    node[anchor=south] {+}
    (2,0) to[R,*-*,color=red] (2,1)
    (2,1) to[R,-*,color=red] (2,2)
    (3,0) to[L,*-*,color=red] (3,1)
    (3,1) to[L,-*,color=red] (3,2)
    ;\end{circuitikz}
\end{figure}

Для постоянного тока $Z$ не может быть емкостным. ``$Z$'' должен проводить
постоянный ток.

В ``нулевых'' схемах число вентилей равно числу фаз.
\begin{figure}[H]
\caption{$m=3$}
\centering
\begin{circuitikz}[american]
  \draw
  (0,1) to[short] ++ (2,0)
  (0,3) to[Ty,*-*] ++ (0,-2)
  (1,3) to[Ty,*-*] ++ (0,-2)
  (2,3) to[Ty,*-*] ++ (0,-2)
  (1,1) -- ++ (0,-1)
  -- ++ (1,0)
  to[R,l=$Z_{\textcyrillic{н}}$] ++ (2,0)
  -- ++ (0,3)
  -- ++ (-4,0)
  ;\end{circuitikz}
\end{figure}
В $Z_{\textcyrillic{н}}$ должны быть учтены параметры провода. ``Искусственный ноль''
должен пропускать постоянный ток и $Z_{\textcyrillic{н}}$ должен быть на порядок
больше $Z$ от ``искусственного ноля''

%\begin{figure}[H]
  \begin{circuitikz}[american]\draw
    (1,2) to[Ty,mirror,-*] (1,0)
    (2,2) to[Ty,mirror,-*] (2,0)
    (3,2) to[Ty,mirror,-*] (3,0)
    to[short,-o] (0,0)
    node[anchor=east] {-} 
    (0.7,0.6) to[short,-o] (0, 0.6)
    node[anchor=east] {+}
    ;\end{circuitikz}
%\end{figure}

$U_{\textcyrillic{управления}} \approx 10V$, может быть 5-8-10,max 15
  
  \begin{circuitikz}[american]\draw
    (1,1) to[Ty] ++ (0,2)
    node[anchor=south] {A}
    (1,1) to[short] ++ (2,0)
    (2,1) to[short,*-] ++ (0,-1)
    to[R,l=$Z_{\textcyrillic{н}}$] ++ (2,0)
%    (0,2.4) to[short] ++ (0.7,0)
    (0.7,2.4) to[short,-o] (0,2.4)
    node[anchor=east] {+}
    (1,2.7) to[short,*-o] ++ (-1,0)
    node[anchor=east] {-}
    ;\end{circuitikz}
  В этой схеме сигнал управления подается относительно фазы $A$.
  Это означает, что сигналы управления нужно развязывать, например, посредством
  импульсного трансформатора.

  6-фазная схема $m=6$, забирать будем с таких обмоток
Y, \rotatebox[origin=c]{180}{$Y$}
  , рисовать не буду

мостовая схема -- это объединение двух нулевых. Ток будет протекать в два раза дольше. Если это трансформатор, то обмотка лучше используется.
Для того чтобы идти дальше, вернемся к частному случаю. Нулевая
однополупериодная схема.
\begin{figure}[H]
  \begin{circuitikz}[american]\draw
    (0,0) to[short,l_=$\textcyrillic{как бы ``ноль''}$] (4,0)
    (0,2) to[short,o-] ++ (1,0)
    to[Ty] ++ (2,0)
    to[C,*-*] ++ (0,-2)
    (4,0) to[R,*-,color=red] (4,2)
    -- ++ (-1,0)
    (5,0) to[short] ++ (1,0)
    to[C,*-] ++ (0,3)
    -- (3,3)
    to[Ty] ++ (-2,0)
    to[short,-*] ++ (0,-1)
    (4.25,0) node[circ] {}
    (4.5,0) node[circ] {}
    (4.75,0) node[circ] {}
    (6,3) node[anchor=south] {-}
    (4,2) node[anchor=south] {+}
    ;\end{circuitikz}
\end{figure}
Напряжение между $(+)$ и $(-)$ равно удвоенному напряжению на конденсаторе.
$u_{d1}$ поставлена маленькими, поскольку это функция времени.
Сумма $u_{d1} + u_{d2}$. По среднему значению будет в два раза больше чем
$U$. Это схема удвоения, $2U$ -- это напряжения на двух конденсаторах.
Эта схема Латура. Однофазная мостовая -- это схема Греса.
Трехфазная мостовая -- Ларионова. Имена возвращаются.

Рассмотрим еще одну схему удвоения:

%\begin{circuitikz}[ american voltages]
%\draw (0,0) to[R, l=$R_1$, i=$i_1$, v_>=$U_1$, o-*] (2,0);
%\end{circuitikz}
%
%\begin{circuitikz} 
%\draw 
%(0,0) node[npn] (npn2) {} 
%(2,0) node[npn,xscale=-1] (npn1) {}
%(npn2.C) to[short] (npn1.C) 
%(npn2.E) to[short] (npn1.E) 
%;\end{circuitikz}
%
%\begin{figure}[ht]
%\centering
%\begin{circuitikz}\draw
%(0,0) node[npn] (npn1) {} 
%%(npn1.B) node[anchor=east] {$\begin{array}{c}VT1\\+\end{array}$}
%%(npn1) node[anchor=west] {VT1}
%(1,2) -- (0,2) to[D*] (npn1.C)
%(1,0) node[pnp,xscale=-1,yscale=-1] (pnp1) {+}
%(pnp1.B) node[anchor=west] {-}
%(pnp1.C) to[D*] (1,2)
%(npn1.E) to[short] (pnp1.E)
%;\end{circuitikz}
%\caption{}
%\end{figure}
%
%\begin{circuitikz}
%    \draw
%    (0,0)
%        to[V, l=$V_s$] ++(0,2.5)
%;
%\end{circuitikz}
  %
%\begin{figure}[H]
%\begin{circuitikz}\draw
%(0,0) to[C, l=$\SI {10}{\micro\farad}$] (0,2) -- (0,3)
%      to[R, l=2.2] (4,3) -- (4,2)
%      to[L, l=12, i=$i_1$] (4,0) -- (0,0)
%(4,2) to[D*, *-*, color=red] (2,0) 
%(0,2) to[R, l=1, *-] (2,2)
%      to[cV, v=$\SI {.3}{\kilo \ohm} i_1$] (4,2)
%(2,0) to[I, i=1] (2,2)
%;\end{circuitikz}
%\end{figure}
