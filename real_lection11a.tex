\chapter{}

Рассматривали ИППН, классификацию ИППН, классификация в основном по квадрантам. Рассматривали
одно-квадрантные,двух-квадрантные, четырех-квадрантные.

\begin{circuitikz}\draw
%  (0,2)to[short,o-](1,2)
  (1.5,2) node[nigbt,rotate=90](nigbt1){}
  (1.5,2) node[above] {VT}
  (0,2)to[short,o-] (nigbt1.C)
  (nigbt1.E)-- (3,2)to[L,-o](5,2)
(0,0)to[short,o-o](5,0)
(3,0)to[Do,l_=$VD$,*-*](3,2)
  ;\end{circuitikz}

Замечание: может стоять IGBT-транзистор, может стоять мосфет,
\begin{circuitikz}\draw
  (0,0)node[nigfete](nigfete){}
  (nigfete.B) node[anchor=west] { подложка}
  (nigfete.G) node[anchor=east] {затвор }
  (nigfete.D) node[anchor=south] {сток}
  (nigfete.S) node[anchor=north] {исток}
;\end{circuitikz}

а может стоять обычный тиристор с углом искуственной коммутации:  

\begin{circuitikz}\draw
  (0,5)to[short,o-*](1,5)--(1,3)to[C,v_>=$ $](2.5,3)
  (1,3)to[C,v^<=$ $](2.5,3)--(3,3)to[Do,*-*](3,5)
  (1,5)to[Ty](3,5)--(4.5,5)to[L](4.5,4)to[Do](4.5,3)--(3,3)
  (4.5,5)--(6,5)
  (0,0)to[short,o-*](6,0)to[Do,*-*](6,5)--(8,5)to[L,l=$L_\textcyrillic{н}$](8,3.5)
  to[R,l=$R_\textcyrillic{н}$](8,2)
  (8,1)circle(0.35)
  (8.35,1)node[right]{$E_\textcyrillic{н}$}
  (6,0)--(8,0)--(8,0.65)
  (8,1.35)--(8,2);
  \draw[thin,<-] (1.75,2.6) -- (1.75,2) node[below]{должен зарядиться так};
  \draw[thin,<-] (1.75,3.4) --(1.2,3.6) node[above]{исходный заряд};
  \draw[thin,<-] (6.3,2.5) -- (6.7,2.5) node[below,rotate=90] {оставим этот диод};
  \draw[dashed] (1.2,5)-- (1.2,6);
  \draw (1.2,6)to[short,-o](1.2,6.1)node[left] {\large{-}};
  \draw[dashed] (2.8,5)-- (2.8,6);
  \draw (2.8,6)to[short,-o](2.8,6.1)node[right] {\large{+}};
;\end{circuitikz}

Искусственная коммутация $\cong$ принудительная коммутация $\cong$ ёмкостная коммутация.
Искусственная коммутация и  принудительная коммутация -- синонимы.
\begin{circuitikz}
\begin{scope}[scale=0.75]
  \draw[dashed] (1.2,5)-- (1.2,6);
  \draw (1.2,6)to[short,-o](1.2,6.1)node[left] {\large{-}};
  \draw[dashed] (2.8,5)-- (2.8,6);
  \draw (2.8,6)to[short,-o](2.8,6.1)node[right] {\large{+}};
  \end{scope}
  ;\end{circuitikz} -- Кратковременно подключить, искусственно включить, принудительно
включить источник. Чаще всего таким источником является заряженный конденсатор. 

\begin{circuitikz}
\begin{scope}[scale=0.75]
  \draw (1.2,5)-- (1.2,6);
  \draw (1.2,6)to[short,-o](1.2,6.1)node[left] {\large{-}};
  \draw (2.8,6)to[ospst] (2.8,5);
  \draw (2.8,6)to[short,-o](2.8,6.1)node[right] {\large{+}};
  \draw[thin,red,->] (0,5.2) -- (0.4,5.2) arc(-90:0:0.2) -- (0.6,6)  
  arc(180:90:0.6)  --(2.8,6.6) node[midway,above] {ток}
  arc(90:0:0.6)-- (3.4,5.4) arc(180:270:0.2) -- (3.9,5.2);
  \end{scope}
;\end{circuitikz} -- источник перехватывает \underline{ток} нагрузки.
Но главная задача -- отключить нагрузку.

Это делается в два этапа:
\begin{itemize}
\item запереть тиристор
\item отключить нагрузку
  \end{itemize}

\begin{circuitikz}
\begin{scope}[scale=0.75] 
  \draw (0,5)-- (1.2,5)-- (1.2,6) to[C,l_=$C$](2.8,6)--(2.8,5) -- (4,5);
  \draw[thin,red,->] (0,5.2) -- (0.4,5.2) arc(-90:0:0.2) -- (0.6,6)  
  arc(180:90:0.6)  --(2.8,6.6) node[midway,above] {ток}
  arc(90:0:0.6)-- (3.4,5.4) arc(180:270:0.2) -- (3.9,5.2);
  \end{scope}
;\end{circuitikz} -- конденсатор идеально подходит для обоих этапов. Конденсатор перезаряжается
и перехватывает энергию. Ёмкостная коммутация -- частный случай искуссственной коммутации,
когда источником является конденсатор.

Можем использовать импульсный трансформатор. 

Как отключить тиристор:

\begin{circuitikz}
\draw (0,4) to[Ty-] (0,0);
	\draw (0,3.5) -- (2,3.5) to[C] (2,2) to [Ty-] (2,0.5) -- (0,0.5);
	\draw[<-] (2.25, 1.25) -- (3.25,1.25) node[right] {\begin{minipage}[t]{0.5\textwidth}вспомогательный тиристор, 
	он включается, после включения заряжается конденсатор, когда конденсатор зарядится, тиристор отключится\end{minipage}}; 
\end{circuitikz}


Латышко в 1970 году защищал кандидатскую диссертацию по автоматическому регулированию реактивной мощности. В спимке литературы
было 270 работ, часть из которых обзоры. 96 патентов
В 1970 году человечество было сконцентрировано на искусственной коммутации. Количество статей,
посвященных искуственной коммутации измерялось четырехзначными цифрами.

Пример работы схемы:

\begin{circuitikz}\draw
  (0,5)--(1,5)--(1,3)to[C,v_>=$ $](3,3)to[Ty,l=$V_{SK}$,mirror](3,5)--(4,5)
  (0,5) node[left] {+} (0,1.2) node[left] {$-$} (4,5)node[right]{+} (4,1.2)node[right]{$-$};
%  (0,1.2)--(4,1.2); зачем это было нужно?
  \draw[thin,<-] (2,2.6) -- (1.6,2.3) node[below] {\begin{tabular}{c}по щучьему веленью\\
      конденсатор заряжен так\end{tabular}};
  
  \draw[->, thin] (4.2,4.5) -- (4.2,1.5); % от плюса к минусу
  
	% рисунок справа
	\draw[thin,->,>=latex] (5.5,3) -- (10,3);
	\draw[thin,->,>=latex] (5.5,3) -- (5.5,5.5);
	\draw (6.5,3) -- (6.5,4) -- (6.8,4) -- (6.8,3); 
	\draw[dashed] (6.5,4) -- (5.5,4) node[left] {$U_\textcyrillic{п}$};
        \draw (6.8,4) -- (6.8,5);
	\draw[thin,<->] (6.5,5.2) -- (6.8,5.2) node[midway, above] {$\tau$}; 
	\draw[domain=6.8:7.5] 
	 plot ({\x},{3.98*(-1+exp(-(\x-6.8)))+5});
        \draw[<-,thin,>=latex] (6.9,4.7) --++ (2,1) node[right] {напряжение на конденсаторе};
	% 4 = 3.98*(-1+exp(-(\x-6.8)))+5 
	% => exp(-(\x-6.8)) = 1 - 1/3.98   =>   -(\x-6.8) = ln(1 - 1/3.98)   =>   \x = 6.8 - ln(1 - 1/3.98) = 6.8 + 0.29 = 7.09
	\newcommand{\x}{7.09}
	\draw[thin] (\x, {3.98*(-1+exp(-(\x-6.8)))+5}) -- (\x,3);
	\draw[thin, <-,>=latex] (\x-0.03,3-0.03) --++ (-0.5,-0.25) node[below] {$t$ отпирания};
	%3 = 3.98*(-1+exp(-(\x-6.8)))+5
	% => exp(-(\x-6.8)) = 2 - 1/3.98 => \x = 6.8 - ln(2 - 1/3.98) = 7.36
	\renewcommand{\x}{7.36}
	\draw[thin, <-,>=latex] (\x+0.15,3-0.03) --++ (0.3,-0.7) node[below right] {$t$ выключения};
	\draw[thin, <->] (6.6, 4) -- (6.6,5);
	\draw[thin, <-]  (6.6+0.03, 4.5) --++ (1,0) node[right] {$U_C$ начальное};

	\draw (5,1.5) node[below right] {\begin{minipage}[t]{0.6\textwidth} $\delta t > t_\textcyrillic{выключения}$ 
	-- условие успешного завершения емкостной коммутации\end{minipage}};

	\draw (0,0.5) -- (1,0.5) -- (1,-0.5) to[C] (3,-0.5) -- (3,0.5) -- (4,0.5);
	% стрелка
	\draw[red,->,>=latex] (0,0) -- (0.25,0) arc(90:0:0.25) -- (0.5,-1) arc (180:270:0.25) -- (1,-1.25) -- (3.25,-1.25) 
	arc (-90:0:0.25) -- (3.5,-0.25) arc(180:90:0.25) -- (4,0);
  
	\draw(5,-0.5) node[right] {${\displaystyle C = \frac{\Delta q}{\Delta U} = \frac{I\Delta t}{\Delta U}}$};
	\draw(5,-1.5) node[right] {${\displaystyle \Delta U  = \frac{I}{C}\Delta t}$};


	\draw (0,-2.2) node[right] {стало так};
	\draw (0,-2.5) -- (1,-2.5) -- (1,-3.5) to[C] (3,-3.5) -- (3,-2.5) -- (4,-2.5);
	\draw (1.5,-3.2) node {$+$}  (2.5,-3.2) node {$-$} ;
	% стрелка
        \draw[red,->,>=latex] (0,-3) -- (0.25,-3) arc(90:0:0.25) -- (0.5,-4) arc (180:270:0.25) -- (1,-4.25) -- (3.25,-4.25)
        arc (-90:0:0.25) -- (3.5,-3.25) arc(180:90:0.25) -- (4,-3);

	% большая стрелка y=-3.2
	\draw (4.5,-3) -- (5.5,-3) -- (5.5,-2.8) -- (5.9,-3.2) -- (5.5,-3.6) -- (5.5,-3.4) -- (4.5,-3.4);


	\draw (6.3,-2.2) -- (8.3,-2.2);
	\draw ({6.8}, -4.2) to[Ty-] ({6.8}, -2.2);
	\draw (7.8,-2.2) to[L, l={$L_k$}] (7.8,-3.2) to[D-,l={VD2}] (7.8,-4.2) node[below]
	{\begin{minipage}[t]{0.45\textwidth}отпирается диод -- и в этот момент отключаетя сам\end{minipage}};

\end{circuitikz}

Как по щучьему веленью. В первый раз открываем $VS_{k}$. В момент, когда я отпираю $V_S$:

\begin{tikzpicture} 
	\draw (0,1) node[above] {тиристор включил два контура};
	\draw (0,0) ellipse (2 and 1);
	\draw (-0.5,1) to[Ty-] (0.5,1); \draw  (-0.5,-1) to[C] (0.5,-1);
	\draw (1.3,0.8) to[L] (1.8,0.2) (1.8,-0.2) to[D-] (1.3,-0.8); 

	\draw (5,0) ellipse (1.4 and 1);
	\draw (6.35,0.4) to[L,l={${\displaystyle \frac{LI^2}{2}}$}] (6.35,-0.4);
	\draw (4.5,-0.9) to[C] (5.5,-0.9);
	\draw (7.5,0.1) node[right] {\begin{minipage}[t]{0.27\textwidth} -- Энергия поля. \\ в этом контуре ток перезаряжает\end{minipage}};
	\path[->] (6.5,-0.4) edge[bend left=30] node[below right] {энергия вернулась в конденсатор} (5.2,-1.2);
\end{tikzpicture}

Диод $VD2$ прекращает колебания. При включении включается $V_{SK}$. При этом конденсатор $C$ проводит ток
от $U_\textcyrillic{П}$ в нагрузку и заряжается полярностью в  кружке 
\begin{tikzpicture}
	\draw (0,0) circle(0.2) node {$+$}  (1.3,0) circle(0.2) node {$-$};
\end{tikzpicture}. 
Процесс заряда завершается, когда $U_C$ сравняется с $U_\textcyrillic{П}$ и откроется диод $V_{D1}$.
В дальнейшемм при отпирании $V_S$ и приложении в нагрузке $U_\textcyrillic{П}$ начинает протекать ток
по контуру колебательной цепи $V_S - L_K - V_{D2} - C$. Конденсатор разряжается и перезаряжается за одну
полуволну резонансной частоты контура. Конденсатор оказывается заряжен нужной для $U_K$ полярностью 
и величиной напряжения близкой к $U_\textcyrillic{П}$.
\begin{tikzpicture}[scale=0.5]
\draw (6.8,3) -- (6.8,5);
       \draw[domain=6.8:7.5]
         plot ({\x},{3.98*(-1+exp(-(\x-6.8)))+5});
\end{tikzpicture} -- иголка примерно равна $U_\textcyrillic{П}$.

Чем хороша  -- позволяет использовать обычные тиристоры. Плоха $\delta t > t_\textcyrillic{выкл}$. Тиристоры
нужны быстродействующие.

\begin{tikzpicture}
\draw (0,0) node[right] {$\displaystyle \delta t = U_C \sim \frac{U_\textcyrillic{П}}{I_{max}}$}; 
\draw[<-,>=latex] (2.6,0.3) --++ (0.5,0.25) node[right] {(меньше из-за КПД 95,98 \%) $<U_\textcyrillic{П}$}; 
\draw[<-,>=latex] (2.8,-0.3) --++ (0.5,-0.25) node[right] {условие должно выполнятся для тока перегрузки}; 
\end{tikzpicture}

Условие $< U_\textcyrillic{П}$ -- ерунда. $I_{max}$ -- это критично.

\begin{tikzpicture}
\draw (0,0) node[right] {$\displaystyle C \geqslant K\frac{t_\textcyrillic{выкл}I_{max}}{U_C < U_\textcyrillic{П}}$}; 
\draw[<-,>=latex] (1,-0.2) --++ (0,-0.5) node[below] {K -- коэффициент запаса $>1 (1.2, 1.5,2)$}; 
\end{tikzpicture}

Выбирали на максимальный ток, а работаю при минимальной, ток Х.Х. Наклон $\sim$ току 
\begin{tikzpicture}[yscale=0.2,xscale=3]
\draw (6.8,3) -- (6.8,5);
       \draw[domain=6.8:7.5]
         plot ({\x},{3.98*(-1+exp(-(\x-6.8)))+5});
\end{tikzpicture}

Схема может не работать при малых токах.

\begin{tikzpicture}
\draw (0,0) to[C] (1.5,0) to[L] (1.5,-1.5) -- (0,-1.5) to[D-] (0,0);
\draw[<-,>=latex] (1.7,-0.75) --++(1,0) node[right] {ускоряет перезагрузку конденсатора};
\end{tikzpicture}

Надежность схемы снижается.

Последнее уточнение к схеме ... похожи на выпрямители. По аналигии с выпрямителями ИППН могут быть многофазными.
Делаются для уменьшения пульсаций. Подключаются к разным 
\begin{tikzpicture}
	\draw (0,0) to[C,l_={$U_\textcyrillic{П}$}] (0,-1);
\end{tikzpicture}

подключаются с разными фазами.

\begin{tikzpicture}
	\draw (0,4.5) to[short,o-*] (0.5,4.5) to[short,-*] (0.5,3.5) -- (0.5,1.5) to[Ty-,l_={$V_{sm}$}] (2,1.5) -- (3,1.5) node[midway, above] {$U_m$} -- (3.5,1.5) to[L,l={$L_m$}] (6,1.5);
	\draw (0.5,4.5) to[Ty-,l_={$V_{S1}$}] (2,4.5) -- (3,4.5) node[midway, above] {$U_1$} --  (4,4.5) to[L,l_={$L_1$}] (6,4.5) -- (6,1.5) 
	to[european resistor,*-] (6,0) to[short,-o] (0,0);  
	\draw (0.5,3.5) to[Ty-,l_={$V_{S2}$}] (2,3.5) -- (3,3.5) node[midway, above] {$U_2$} -- (4,3.5) to[L,l_={$L_2$},-*] (6,3.5); 
	\draw (1,2.5) node[right] {$\cdot\cdot\cdot$};
	\draw (3.2,0) to[short,*-] (3.2,3.5) to[D-,l_={$V_{D1}$},-*] (3.2,4.5);
	\draw (3.5,0) to[short,*-] (3.5,2.5) to[D-,l_={$V_{D2}$},-*] (3.5,3.5);
	\draw (3.8,0) to[D-,l_={$V_{Dm}$},*-*] (3.8,1.5);

\draw[<->,>=latex, thin] (0,0.1) -- (0,4.4) node[midway,left] {$U_\textcyrillic{П}$};
\draw[<->,>=latex, thin] (6.5,0.1) -- (6.5,1.4) node[midway,right] {$U_\textcyrillic{вых}$};
\end{tikzpicture} 

К нагрузке подключены $m$ одинаковых преобразователя. Отличаются сдвигом по времени

\begin{tikzpicture}
\draw[thin,->,>=latex] (0,0) -- (0,0.9) node[midway,left] {$U_m$};
	\draw[thin,->,>=latex] (0,0) -- (5.5,0) node[right] {$t$};
	\draw (3.5,0) -- (3.5,0.8) -- (3.8,0.8) -- (3.8,0); 
	\newcommand{\x}{2}
	\newcommand{\T}{3}
	\draw[thin,->,>=latex] (0,{0+\x}) -- (0,{0.9+\x}) node[midway,left] {$U_2$};
	\draw[thin,->,>=latex] (0,{0+\x}) -- (5.5,{0+\x}) node[right] {$t$};
	\draw[thin] (1,{\x}) -- (1,{\x-0.5}) (1.8,{\x}) -- (1.8,{\x-0.5});
	\draw[thin,<->,>=latex] (1,{\x-.2}) -- (1.8,{\x-.2}) node[midway,below] {$\frac{T}{m}$};
        \draw (1.8,{\x}) -- (1.8,{\x+0.8}) -- (2.1,{\x+0.8}) -- (2.1,{\x});
	\draw ({1.8+\T},{\x}) -- ({1.8+\T},{\x+0.8}) -- ({2.1+\T},{\x+0.8}) -- ({2.1+\T},{\x});

	\renewcommand{\x}{3.5}
	\draw[thin,->,>=latex] (0,{0+\x}) -- (0,{0.9+\x}) node[midway,left] {$U_1$};
	\draw[thin,->,>=latex] (0,{0+\x}) -- (5.5,{0+\x}) node[right] {$t$};
	\draw[thin] (1,{\x}) -- (1,{\x-0.4}); 
	\draw[thin] (4,{\x}) -- (4,{\x-0.4}); 
	\draw[thin,<->,>=latex] (1,{\x-0.2}) -- (4,{\x-0.2}) node[midway,below] {$T$};
	\draw (1,{\x}) -- (1,{\x+0.8}) -- (1.3,{\x+0.8}) -- (1.3,{\x});
	\draw ({1+\T},{\x}) -- ({1+\T},{\x+0.8}) -- ({1.3+\T},{\x+0.8}) -- ({1.3+\T},{\x});

	\draw (2.2,1.1) node[right] {$\cdot\cdot\cdot$};
\end{tikzpicture}

Напряжение на нагрузке $\displaystyle U_\textcyrillic{н} = \frac{\sum U_k}{m}$ -- среднее арифметическое (доказывать не буду).
Делается так чтобы уменьшить амплитуду пульсяций. Многофазный ИППН -- частота пульсаций в нагрузке увеличивается в $m$ раз, амплитуда пульсаций уменьшается в $m$ раз, амплитуда $I^2$ в $m^2$ раз. 

Другим достоинством многофазного ИППН является простота получения максимальной мощности а также уменьшение потерь в СПП.

Во всех приборах есть частотные потери. Ключевой режим неидеален. В 9 раз увеличили частоту f. Это не нужно. Токи делятся -- это определяется индуктивностью.

ИППН широко использовались на заре. Для большей мощности.

Вместо 
\begin{tikzpicture}%[shift={(0,0)},rotate=90]
	\draw (0,0) node[nigbt,rotate=90] {};  \draw (1,0) node[right] {включения}; \draw (1,-0.4)  node[right] {параллель транзисторов};
\end{tikzpicture}
включаются многофазные ИППН.

\section{ИППН с регулированием напряжения выше чем $ U_\text{питания} $ }

Принципиально многофазные ИППН могут использоваться для всех типов ИППН (2х,4х квадрантных), и таких как приведенный выше.

\begin{tikzpicture} 
	\draw (0,2) to[Ty-,l={Vs},o-] (2,2) to[L] (2,0) to[short,-o] (0,0);
	\draw (2,2) to[short,l={$V_D$}] (3.5,2) to[C] (3.5,0) -- (2,0);
	\draw (3.5,2) to[D-] (2,2); 
	\draw (3.5,2) -- (5,2) to[european resistor, l={$R_\textcyrillic{нагр}$}] (5,0) -- (3.5,0);
	%\draw (5,3) -- (7,3) to[R, l={$R_\textcyrillic{нагр}$}] (7,0) -- (5,0);
	\draw[<->] (6.3,0) -- (6.3,2) node[midway, right] 
	{\begin{minipage}[t]{0.4\textwidth} и выше и ниже \\ $U_\textcyrillic{нагр}$ = [0...$U_\textcyrillic{п}$] \\ 
	$U_\textcyrillic{н} < U_\textcyrillic{пит}$ \\ $U_\textcyrillic{н} > U_\textcyrillic{пит}$ \end{minipage}}; 
\end{tikzpicture}

\begin{tikzpicture}
\draw (0,2) to[L,o-] (2,2) to[Ty-,l_={Vs}] (2,0) to[short,-o] (0,0);
\draw (2,2) to[D-,l={$V_D$}] (3.5,2) to[C] (3.5,0) -- (2,0);
\draw (3.5,2) -- (5,2) to[european resistor, l={$R_\textcyrillic{нагр}$}] (5,0) -- (3.5,0);
	\draw[<->] (6.3,0) -- (6.3,2) node[midway, right]  {$U_\textcyrillic{н} > U_\textcyrillic{пит}$};
\end{tikzpicture}

2) не буду выключать $Vs$

1) \begin{tikzpicture} 
\draw(0,0) to[D-] (1,0); 
	\draw[->,>=latex] (0,-0.25) -- (0.75,-0.25) arc(90:0:0.25) -- (1,-0.75) arc(0:-90:0.25) -- (0,-1); 
\end{tikzpicture}
включил $V_s$ ток в индуктивности 
\begin{tikzpicture}
	\draw (0,0) to[L] (0,-0.9) arc(-180:-90:0.1) -- (0.9,-1) arc(-90:0:0.1) to[C] (1,-0.1) arc(0:90:0.1) to[D-] (0.2,0);
	\draw (1.2,-0.8) node[right] {$+$};
	\draw (1.2,-0.2) node[right] {$-$};
\end{tikzpicture}

На конденсаторе \begin{tikzpicture}\draw (0,0) circle(0.17) node {$+$};\end{tikzpicture} -- внизу.

2) Включил $V_S$ К.З. ток нарастает может быть гигантский. 

\begin{tikzpicture}
	\draw (0,2) to[L,o-] (2,2) to[Ty-] (2,0) to[short,-o] (0,0);
	\draw (2,2) -- (4,2) to[C] (4,0) -- (2,0);
	\draw[->,>=latex] (0,1.75) --(1.5,1.75) arc(90:0:0.25) -- (1.75,0.5) arc(0:-90:0.25) -- (0,0.25); 
	\draw[->,>=latex] (4.25, 2 ) arc(90:-90:1);
	\draw[<-,>=latex] (5.35,1) --++ (1.25,0) node[right] {Разряд конденсатора на нагрузку};
\end{tikzpicture}

Предположим, что в нагрузке $L$, $C$ - сам сглаживает. Прикладиваются \underline{импульсы тока!} 
(До этого были импульсы напряжения). У импульса тока всегда кончная величина. 

Сначала строим кривую индуктивности. Что подчеркиваем: Постоянная составляющая на индуктикности равна нулю $=0$.

\begin{tikzpicture}
\draw[thin,->,>=latex] (0,-2.7) -- (0,2.5) node[left] {U};
\draw[thin,->,>=latex] (0,0) -- (7,0);
	\draw (1,0) -- (1,1) -- (2,1) -- (2,-2.5) -- (2.4,-2.5) -- (2.4,1) -- (3.4,1) -- (3.4,-2.5) -- (3.8,-2.5) 
	-- (3.8,1) -- (4.8,1); 

\draw[pattern=north west lines, pattern color=red] (2.4,0) rectangle (3.4,1);
\draw[pattern=north west lines, pattern color=blue] (3.4,-2.5) rectangle (3.8,0);
\draw[<-,>=latex] (1,-0.05) --++ (-0.2,-0.5) node[below left] {включил $V_S$};
\draw[<-,>=latex] (2,1.05) --++ (-0.3,0.8) node[above] {выключил};

	\draw[<-,>=latex] (2.9,0.5) -- (4.5,-0.8);
	\draw[<-,>=latex] (3.6,-1.25) -- (4.5,-0.8) node[right] {площади равны};

	\draw[thin] (0,1) -- (-0.1,1) node[left] {$U_\textcyrillic{п}$};
	\draw[thin] (0,-2.5) -- (-0.1,-2.5) node[left] {$U_C =U_\textcyrillic{нагр}$};

	\draw[thin,<->,>=latex] (2,1.4) -- (3.4,1.4) node[midway, above] {$\tau_k$};

	\draw(8,-1) -- (9.5,-1) to[L] (9.5,-2.5) -- (8,-2.5);
	\draw (9.2,-1.3) node {$+$};
	\draw (9.2,-2.2) node {$-$};

	\draw[->,>=latex] ({10.5+cos(-150)},{-1.5+sin(-150)}) arc(-150:0:1); 

	\draw (9.8,-2.15) rectangle (10.15,-1.85);
	\draw (9.98,-2.0) node {$+$};
	\draw (9.75,-1.4) rectangle (10.1,-1.05);
	\draw (9.92,-1.25) node {$-$};

	\draw[thin,<-, >=latex] (9.92,-0.9) --++ (0.5,0.5) node[above] {когда выключено};
\end{tikzpicture}

$$
U_\textcyrillic{пит} \tau = U_\textcyrillic{н}(T_k - \tau) \Rightarrow
$$

$$
U_\textcyrillic{н} = \frac{\tau}{T_k - \tau} U_\textcyrillic{пит}
$$

Токи. Если считать что емкостная величина сопротивления мала, то $\frac{L}{R_\textcyrillic{диода}}$

$$
\tau \ll \tau_\textcyrillic{эл} = \frac{L}{\sum R}
$$
В цепи источника $R_\textcyrillic{индук}, R_\textcyrillic{источн.}, R_\textcyrillic{тиристора}$

\begin{tikzpicture}
\draw[thin,->,>=latex] (0,-2.7) -- (0,2.5) node[left] {i};
\draw[thin,->,>=latex] (0,0) -- (7,0);
	\draw[thin] (1,0) -- (1,1) -- (2,1) -- (2,-2.5) -- (2.4,-2.5) -- (2.4,1) -- (3.4,1) -- (3.4,-2.5) -- (3.8,-2.5)
        -- (3.8,1) -- (4.8,1);

	\draw (1,2) -- (2,3) -- (2.4,2) -- (3.4,3) -- (3.8,2) -- (4.8,3) -- (5.2,2);
	\draw (7,2) node[right] {напряжение на конденсаторе постоянно};
	\draw[<-] (4.3, 0.5) --++ (0.5,-2) node[below, right] {ток источника питания};
	\draw[thin, dashed] (4.8,0) -- (4.8,3);
	\draw[thin, dashed] (5.2,0) -- (5.2,2);
	\draw[thin, dashed] (3.8,1) -- (3.8,2);
	\draw[<-] (5,0.5) --++ (-0.9,-3) node[below, right] {ток нагрузки $=$ ток конденсатора(напряжение не меняется)};

	\draw[<-] (1.5,2.6) --++ (-0.4,0.8) node[above] 
	{${\displaystyle i_L: L\frac{di}{dt} = U_\textcyrillic{пит.}}$};

	\draw[<-] (2.2,2.6) --++ (0.8,1.0) node[above, right] 
	{${\displaystyle i_L: L\frac{di}{dt} = U_\textcyrillic{нагр.}}$ -- противоположной полярности};

	\draw (2.9,2.5) -- (3.6,2.5);
	\draw[<-] (3.25,2.4) --++ (0,-0.3) node[below] {$I_L$};
\end{tikzpicture}

$$
\underbrace{I_\textcyrillic{пит.}}_\textcyrillic{среднее} = I_L \cdot \tau f_k
$$

$$
I_\textcyrillic{н} = I_L (T_k - \tau) f_k
$$

$$
f= \frac{1}{T_k}
$$

2-я схема

\begin{tikzpicture}
\draw (0,1.5) to[L] (1.5,1.5) to[Ty-] (1.5,0) -- (0,0);
	\draw[->,>=latex] (0,1.25) -- (1,1.25) arc(90:0:0.25) -- (1.25,0.5) arc(0:-90:0.25) -- (0,0.25);
\end{tikzpicture}


\begin{tikzpicture}
\draw[thin,->,>=latex] (0,-2.7) -- (0,2.5) node[left] {i};
\draw[thin,->,>=latex] (0,0) -- (7,0);
        \draw[thin] (1,0) -- (1,1) -- (2,1) -- (2,-2.5) -- (2.4,-2.5) -- (2.4,1) -- (3.4,1) -- (3.4,-2.5) -- (3.8,-2.5)
        -- (3.8,1) -- (4.8,1);
	\draw[<->] (1,0.5) -- (2,0.5) node[midway,below] {$\tau$};
	\draw[<->] (2.7,0) -- (2.7,-2.5);
	\draw[<-] (2.75,-1.25) --++ (1.4,-1) node[right] {разность $(U_\textcyrillic{н} - U_\textcyrillic{пит.})$};
	\draw[<-] (3.85,-1) --++(1,0) node[right] {$U_L$ (маленькая) напряжение на индуктивности}; 
\end{tikzpicture}

$$
\tau \cdot U_\textcyrillic{пит.} = (U_\textcyrillic{н} - U_\textcyrillic{пит.}) (T_k - \tau)
$$


токи $i_L = i_\textcyrillic{п}$, $I_L = I_\textcyrillic{п}$

$$
i_\textcyrillic{н} = i_L \frac{T_k - \tau}{T_k} (\textcyrillic{или умножить на } f_k) 
$$

$$
I_\textcyrillic{н} = I_L\left(T_k - \tau\right) f_k
$$
