\section{Автономные инверторы}

Есть три основных типа инверторов: АИН -- инвертор напряжения, этот инвертор больше всего будет нас интересовать; АИТ -- инвертор тока; АИР -- резонансный (наименне интересный)
В чем различие АИН, АИТ, АИР. Предварительно обратим внимание на различие свойств.
АИН формирует на выходе форму кривой напряжения. Форма тока может отличаться. Если нагрузка реактивная $L(\omega)\nearrow, C(\omega)\searrow$ с ростом частоты. 

Классификация АИН:
\begin{enumerate}
\item однофазные, трехфазные (регулируемой частоты, постоянной частоты). Имеем ввиду управление трехфазным приводом. Для двигателя нужно создать 
вращающееся поле. Чаще всего 3 фазы, бывает 2 фазы, бывает много. Изучив однофащные и трехфазные, остальные получим простым увеличением числа фаз.

\item схемы нулевые, мостовые (в выпрямителях концевые).\\
Тип СПП: преобразователт из постоянного в переменное $=\rightarrow\approx$, причем автономные, т.е. независимые не ведомые сетью. Включить СПП я могу, а выключить не
получиться. Обязательно нужны запираемые приборы: запираемые тиристоры, силовые транзисторы.
1й тип будет похлж на ИППН. Наработки, полученные в ИППН будут использоваться в АИ.
\item трансформатороные, безтрансформаторные
\end{enumerate}
24в $\rightarrow$ в переменное, затем через трансформатор увеличить.\\

Начнем с однофазного АИН

%https://tex.stackovernet.com/ru/q/124245/%D0%BA%D0%B0%D0%BA-%D0%BD%D0%B0%D1%80%D0%B8%D1%81%D0%BE%D0%B2%D0%B0%D1%82%D1%8C-%D1%82%D1%80%D0%B0%D0%BD%D1%81%D1%84%D0%BE%D1%80%D0%BC%D0%B0%D1%82%D0%BE%D1%80-%D1%82%D0%BE%D0%BA%D0%B0-%D0%B2-tikzcircuitikz
%https://tex.stackexchange.com/questions/355717/circuittikz-4-pinfet-symbol/355802#35580
\begin{wrapfigure}{l}{0.4\linewidth}
\begin{circuitikz} 
\draw (1.5,6) node[Lnigbt, bodydiode, rotate=90, xscale=-1] (t1) {};
\draw (0,6) to[short, o-] (t1.D);
\draw (t1.S) -- (3,6) -- (3,4.5); \draw (0,4.5) -- (0.5,4.5)  to[R] (1.75,4.5) to[L] (3,4.5); \draw (0,4.2) to[short,o-o] (0,4.8);
\draw (1.5,3) node[Lnigbt, bodydiode, rotate=90, xscale=1] (t2) {};
\draw (0,3) to[short,o-] (t2.D);
\draw (t2.S) -- (3,3) -- (3,4.5);
\draw (-0.2, {4.5 + 1.5/2}) node() {$U_{T/2}$};
\draw (-0.3, 5.9) node() {$+$} (0.3,4.9) node() {$-$} (0.8,4.8) node() {$-$} (2.7,4.8) node() {$+$};
\draw (0.3,4.1) node() {$+$};
\draw (-0.3, 3)  node() {$-$};
\end{circuitikz}
\end{wrapfigure}
Нулевая схема получится, если включил верхний транзистор. Вторую полуволну получу, если подключу нижний. Знак на транзисторе означает, что транзистор управляется напряжением.
На сомом деле это два транзистора. Иногда из-за ``нулевой'' точки схема называется нулевой.

Обратим внимание. Когда нагрузка индуктивная, то включать верхний транзистор -- будет перенапряжение. Поэтому всегдя в схеме транзисторы шунтируются обратными
диодами. При подаче напряжение $+$ $-$ -- запираем диод. Как диоды проводят? При выключении нижний диод поддержит ток при выключении верхнего транзистора. %диода.

%${\displaystyle \frac{\partial i}{\partial t}}$ громадное, но установлены диоды, но кроме $L_\text{н}$ может быть индуктивность питания, то беды не миновать.

\begin{wrapfigure}{l}{0.3\linewidth}
\begin{circuitikz} 
\draw (1.5,6) node[Lnigbt, bodydiode, rotate=90, xscale=-1] (t1) {};
\draw (0,6) to[short,o-] (t1.D);
\draw (t1.S) -- (3,6) -- (3,4.5); \draw (0,4.5) -- (0.5,4.5)  to[R] (1.75,4.5) to[L] (3,4.5); \draw (0,4.2) to[short,o-o] (0,4.8);
\draw[help lines,smooth] (0,3) to[C,o-o] (0,4.2);
\draw[help lines, smooth] (0,6) to[C,o-o] (0,4.8);
\end{circuitikz}
\end{wrapfigure}
${\displaystyle \frac{\partial i}{\partial t}}$ громадное, но установлены диоды, но кроме $L_\text{н}$ может быть индуктивность питания, то беды не миновать.

Чтобы этого не произошло, устанавливают конденсатор, Большая (полярная) емкость. Параллельно ставят с маленькой паразитной емкостью (внутренней индуктивностью).
Элетролитический конденсатор, громадная емкость 6000$\mu$F.\\

\begin{minipage}{\textwidth}
%Параллельно включают маленький несколько pF. В первые пикосекунды ток идет через маленький конденсатор. Он предназначен,
%чтобы поглотить начальный бросок.
  \begin{wrapfigure}{r}{0.2\textwidth}
  \begin{circuitikz}
    \draw (0,2) to[elko] (0,0);
    \draw (0,0.4) -- (1,0.4) to[C] (1,1.6) -- (0,1.6);
    \end{circuitikz}
\end{wrapfigure}
Параллельно включают маленький конденсатор несколько pF. В первые пикосекунды ток идет через маленький конденсатор. Он предназначен,
чтобы поглотить начальный бросок.
\end{minipage}\\ \\ \\

\begin{minipage}{\textwidth}
\begin{wrapfigure}{l}{0.2\linewidth}
  \begin{circuitikz}
    \draw (0,3) -- (1.5,3) to[C] (1.5,1.5) to[C] (1.5,0) -- (0,0);
    \draw (0.65,3) to[R] (0.65,1.5) to[R] (0.65,0);
    \draw (0.65,1.5) to[short,*-*] (1.5,1.5); 
  \end{circuitikz}  
\end{wrapfigure}
Конденсаторы образуют емкостной делитель, который не работает на постоянном токе. Конденсатор нужно шунтировать баллсатным сопротивлением.
Сопротивление нужно выбрать меньше гарантированного сопротивления утечки конденсатора. Если сопротивление утечки конденсатора 5мОм, то сопротивление
балластных резисторов выберем в 100 раз меньше сопротивления утечки. Если не ставить сопротивления, то напряжение поделится пропорционально нестабильному
сопротивлению изоляции C.
\end{minipage}\\ \\

%\begin{wrapfigure}{l}{0.4\linewidth}
\begin{circuitikz} \begin{scope}[scale=1.1]
    \draw (1.5,6) node[Lnigbt, bodydiode, rotate=90, xscale=1] (t1) {};
    \draw (2.3,5.5) node {VT$_1$} (2.4,6.4) node {VD$_1$};
    \draw (-1.5,6) node[left] {$+$} to[short,o-] (t1.D);
    \draw (2.3,3.5) node {VT$_2$} (2.4,2.6) node {VD$_2$};
\draw (t1.S) -- (3,6) -- (3,4.5); \draw (0,4.5) -- (0.5,4.5)  to[R] (1.65,4.5) to[L]  (2.75,4.5) -- (3,4.5); 
\draw (1.5,3) node[Lnigbt, bodydiode, rotate=90, xscale=-1] (t2) {};
\draw (-1.5,3) node[left] {$-$} to[short,o-] (t2.D);
\draw (t2.S) -- (3,3) -- (3,4.5);
\draw (0,6) to[R,-*] (0,4.5) to[R] (0,3); 
\draw (-.8,6) to[C,-*] (-.8,4.5) to[C] (-.8,3);
\draw (0,4.5) -- (-.8,4.5) node[left] {точка (o)\ \ \ };
\draw[help lines, smooth] (0.5,4.25) rectangle (2.8,4.75);
\draw[thin,<-,>=stealth'] (-0.2,3.6) -- (-0.8,2.2) node[below] {50 kV};  
\draw[thin,<->,>=stealth'] (0,2.3) -- node[midway, below] {$U_\text{н}\text{ маленкое}$} (3,2.3); 
\draw[thin,<-,>=stealth'] (0.4,3.3) -- (1.1,3.9) -- (3.5,3.9) node[right] {Эта цепь должна быть очень короткая};
\end{scope}
\end{circuitikz}
%\end{wrapfigure}

\begin{circuitikz}
\usetikzlibrary{calc}
\usetikzlibrary{decorations.pathreplacing,intersections,calc}
  \draw[thin,->,>=stealth'] (0,-3) -- (0,3.5) node[left] {$U$};
  \draw[thin,->,>=stealth'] (0,0) coordinate (x_1) -- (8,0) coordinate (x_2) node[above right] {$t$};
  \draw (0,-2) -- (1.5,-2) -- (1.5,2) -- (3,2) -- (3,-2) -- (4.5,-2) -- (4.5,2) -- (6,2) -- (6,-2) -- (7.5,-2) -- (7.5,2);
  \draw (0,-2) node[left] {$\displaystyle -\frac{U_V}{2}$};
  \draw[thin,loosely dashed] (1.5,2) -- (0,2) node[left] {$\displaystyle \frac{U_V}{2}$};
  \draw [domain=0:1.5] plot (\x, {-2.6+4*exp(-.8*\x)});        \draw[domain=1.5:6, thin, loosely dashed] plot (\x, {-2.6+4*exp(-.8*\x)});
  \draw [domain=1.5:3] plot (\x, {2.6 - 4*exp(-.8*(\x-1.5))}); \draw[domain=3:7.5, thin, loosely dashed] plot (\x, {2.6 - 4*exp(-.8*(\x-1.5))});
  \draw [name path=aa, domain=3:4.5] plot (\x, {-2.6+4*exp(-.8*(\x-3))});    \draw[domain=4.5:7.5, thin, loosely dashed] plot (\x, {-2.6+4*exp(-.8*(\x-3))});
  \draw [name path=cc, domain=4.5:6] plot (\x, {2.6 - 4*exp(-.8*(\x-4.5))}); \draw[domain=5:7.5, thin, loosely dashed] plot (\x, {2.6 - 4*exp(-.8*(\x-4.5))});
  \draw [domain=6:7.5] plot (\x, {-2.6+4*exp(-.8*(\x-6))});
  \draw [very thin, loosely dotted] (0,2.6) -- (7.3,2.6);
  \draw [thin] (1.5,2) -- (1.5,3.2) (3,2) -- (3,3.2) (4.5,2) -- (4.5,3);
  \draw [very thin,<->,>=stealth'] (1.5,3) -- (3,3) node[midway,above] {$T/2$};\draw[very thin,<->,>=stealth'] (3,2.8) -- (4.5,2.8) node[midway,above] {$T/2$};
  \draw [thin] (3,-2) -- (3,-3.2) node (a1) {};
  \path [name path=bb] (0,0) -- (7,0);
  \path [name intersections={of=aa and bb,by=E}];
  \draw [thin] (E) -- ++ (0,-3.2) node (a2) {};
  \draw [thin] (4.5, -2) -- (4.5, -3.2) node (a3) {};
  \path [name intersections={of=cc and bb,by=C}];
  \draw [thin] (C) -- ++ (0,-3.2) node (a4) {};
  \draw [thin] (6, -2) -- (6, -3.2) node (a5) {};
  \draw [thin,<->,>=stealth'] (3,-2.9) -- (6,-2.9) node[right] {$\displaystyle T=\frac{1}{f}$};
  \path (a1.center) -- (a2.center) node[midway, below] {1};
  \path (a2.center) -- (a3.center) node[midway, below] {2};
  \path (a3.center) -- (a4.center) node[midway, below] {3};
  \path (a4.center) -- (a5.center) node[midway, below] {4};
  \path [name path=dd] (0,.6) -- (7,.6);
  \path [name intersections={of=aa and dd,by=D}];
  \draw[thin,->,>=stealth'] (2.5,-3.9) node[right] {нижний транзистор еще не включен} -- (3,-2.7);
  \draw[thin,->,>=stealth'] (1.6,-4.4) node[right] {ток течет по диоду} -- (D);
\end{circuitikz}  

% https://tex.stackexchange.com/questions/241959/aligned-circle-around-letter-tikz
\newcommand*\circled[1]{\tikz[baseline=(char.base)]{
        \node[shape=circle,draw,minimum size=4mm, inner sep=0pt] (char)
        {\rule[-3pt]{0pt}{\dimexpr2ex+2pt}#1};}}



\begin{tabular}{l}
На интервале \circled1  ток протекает через $VD_2$ --- $i_{VD2}$.\\
В момент \circled2 транзистор $VT_2$ должен быть открыт. \circled2 --- $i_{VT2}$\\
\circled3 --- $i_{VD1}$\\
\circled4 --- $i_{VT1}$
\end{tabular}

На интервалах \circled2 и \circled4 конденсаторы разряжаются, на интервалах \circled1 и \circled3 -- заряжаются.

На \circled1 ток шел в нижний конденсатор $C_2$, на \circled3 --- в верхний конденсатор $C_1$.

Обратим внимание на маленький момент:

Если
\begin{circuitikz}[scale=0.3]
  \draw  (1.5,0) -- (1.5,2) -- (3,2) -- (3,-2) -- (4.5,-2) -- (4.5,0);
  \path[pattern=north east lines,fill opacity=5,opacity=0.2] (1.5,0) rectangle (3,2);
  \path[pattern=north east lines,fill opacity=5,opacity=0.2] (3,-2) rectangle (4.5,0);
\end{circuitikz}
то в нагрузке постоянная составляющая $U$, значит будет ток $I$.

Если
\begin{circuitikz}[scale=0.3]
  \draw [domain=2.04:3] plot (\x, {2.6 - 4*exp(-.8*(\x-1.5))}); 
  \draw [name path=aa, domain=3:4.5] plot (\x, {-2.6+4*exp(-.8*(\x-3))});
  \draw [name path=cc, domain=4.5:5.04] plot (\x, {2.6 - 4*exp(-.8*(\x-4.5))});
  %  \draw [domain=6:7.5] plot (\x, {-2.6+4*exp(-.8*(\x-6))});
%  \draw [name path=bb] (2.04,0) -- (5.04,0);  
\end{circuitikz}
--- тогда нет постоянного тока. Если нагрузка двигатель, то постоянная составляющая очень опасна.

Отличаются на 1\%. Делитель не обеспечивает деление, или равенство полуволн необязательно. $400V - 1\% - 4V$. Сопротивление миллиОмы, то
$$
I = \frac{4V}{10\mu\Omega} = \frac{4}{10\cdot 10^{-3}} = 400A
$$
Пусть будет $40A$. Что такое $40A$ по сравнению со $100A$? Ток намагничивания $1-2\%$ у трансформатора $500A$ ток намагничивания $10A$.

Во всех инверторах выходное напряжение не должно содержать постоянной составляющей. Постоянная составляющая, даже небольшая, может создать в обмотках питаемого двигателя
постоянную составляющую тока, которая способна насытить магнитную цепь двигателя. А это, в свою очередь, вызовет резкое увеличение намагничивающего тока.
Процесс развивается лавинообразно.

У мощного трансформатора, двигателя $X \gg R$. Если $Z$ мало, при номинальном токе $5-7\%$
$$
I_\text{ном}\cdot Z_\text{обмоток} = 5\%
$$

$X \gg R$ то 1\% медь 1\% сталь
\begin{tabular}{l}
  $I^2R - 98\%$ \\
  $IR = 1\%$
\end{tabular}

Напряжение около $1\%$, то ток номинальный

\begin{tabular}{lll}
  у двигателя & &\\
  у трансформатора & 3-4-5\% & ток Х.Х. - 5\% \\
  асинхронный & 20-30\% даже 40
\end{tabular}  

А если $6,7\%$ -- уже трансформатор насытится! Точность нужна почти метрологическая. $500A$. Резисторы $50K$.

Самосинхронизация

Для \begin{circuitikz}[scale=0.4]
\draw (0,0) to[C] (0,1.3) to[C] (0,2.6);
\end{circuitikz} постоянного тока -- большое сопротивление.
Если появляется постоянная составляюшая на конденсаторе -- вствечное падение напряжения. Конденсатор подзарядится -- полуволны выровняются. В других схемах может быть придется позаботится.

{\large Недостаток схемы}

$U_\text{питания} = 100\text{ вольт}$, если нижний заперт, то к СПП --- $100\text{ вольт}$, а к нагрузке $\displaystyle \frac{U_\text{п}}{2} = 50\text{ вольт}$, а на транзисторах и диодах
полное $U_\text{п}$

Нужна ``синусоида'' напряжения. Разложим в ряд Фурье:

%\begin{figure}[!ht]
\begin{circuitikz}
  \draw (0,0) node[right] {$\displaystyle U_{(0)} = \frac{1}{2\pi}\int\limits_0^t U\;dt\;\cos\frac{\pi}{2\pi f} t$};
  \draw [<-,>=stealth'] (0.4,-0.3) -- (0, -1) node[below] {$\begin{array}{c}\text{основной}\\ \text{гармоники}\end{array}$};
\end{circuitikz}
%	\caption{ряд Фурье}
%\end{figure}

Если выбрать начало координат так, чтобы оставалась только косинусная составляющая


\begin{figure}[!ht]
\begin{circuitikz}
  \newcommand{\pp}{3.149265}
  \draw[thin,->,>=stealth'] (-2,0) -- (4,0);
  \draw[thin,->,>=stealth'] (0,-1.5) -- (0,1.7);
  \draw (-0.5,0) -- (-0.5,1) -- (0.5,1) -- (0.5,-1) -- (1.5,-1) -- (1.5,1) -- (2.5,1) -- (2.5,-1) -- (3.5,-1) -- (3.5,0);
  \draw[domain=-0.5:3.5, help lines,samples=200]
    plot (\x, {2*0.45*cos(\pp*\x r)});
\end{circuitikz}
\end{figure}

$$
U_{(\text{п})} = 0.45 U_\text{п}
$$

Гармоники $\displaystyle \frac{U_{(\text{п})}}{U_{(0)}}$, а основную частоту задает система упрввления.

Амплитуда гармоники 0.635 $(0.45\sqrt{2})$. Действующее значение $U_{(\text{п})} = $ в $n$ раз меньше, где $n$ -- весь ряд нечетных чисел, четных гармоник нет. Четные гармоники по разному искажают полуволны.

\begin{figure}[!ht]
%\begin{wrapfigure}{l}{0.7\linewidth}
\begin{circuitikz}[yscale=1.3,samples=200]
  \draw[thin,->,>=stealth'] (-3.5,0) -- (3.5,0);
  \draw[domain=-3.14:3.14, help lines,smooth]  plot (\x,{-sin(\x r)});
  \draw[domain=-3.14:3.14, help lines, smooth] plot (\x,{.3*sin(3*\x r)});
  \draw[domain=-3.14:3.14] plot (\x,{-sin(\x r) + .3*sin(3*\x r)});
\end{circuitikz}
%\end{wrapfigure} 
%\caption{форма обеих полуволн искажается одинаково.}
\parформа обеих полуволн искажается одинаково.
\end{figure}

%\vspace{1.5cm}
\begin{figure}[!ht]
%\begin{wrapfigure}{l}{0.7\linewidth}
  \begin{circuitikz}[yscale=1.3,samples=200]
  \draw[thin,->,>=stealth'] (-1.7,0) -- (5,0);  
  \draw[domain=-1.57:4.71, help lines,smooth] plot (\x, {cos(\x r)});   
  \draw[domain=-1.57:4.71, help lines, smooth] plot (\x,{.5*cos(2*\x r)});
  \draw[domain=-1.57:4.71] plot (\x,{cos(\x r) + .5*cos(2*\x r)});
\end{circuitikz}  
%\end{wrapfigure} 
\par	по площади одинаково, но выглядит неодинаково
%\vspace{1cm}
\end{figure}

На нашей кривой обе полуволны одинаковы, но появится постоянная составляющая. Но постоянная составляющая автоматически уничтожается. Ток заряда C будет идти до тех пор, пока не исчезнет постоянная составляющая.

\begin{figure}[!ht]
%\begin{figure}[!ht]
\begin{circuitikz} 
\draw[domain=0:3.14,red, samples=200] plot (\x, {sin(\x r) + .3*sin(3*\x r)});
\end{circuitikz}
%\end{figure}
\par 3-я гармоника, на полуволне - 3 гармоники
\end{figure}

\begin{tabular}{cc}
3-я гармоника	 & 33\% \\
	5-я гармоника & 20 \% \\
\end{tabular}	
