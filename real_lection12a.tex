\section{Автономные инверторы}

Есть три основных типа инверторов: АИН -- инвертор напряжения, этот инвертор больше всего будет нас интересовать; АИТ -- инвертор тока; АИР -- резонансный (наименне интересный)
В чем различие АИН, АИТ, АИР. Предварительно обратим внимание на различие свойств.
АИН формирует на выходе форму кривой напряжения. Форма тока может отличаться. Если нагрузка реактивная $L(\omega)\nearrow, C(\omega)\searrow$ с ростом частоты. 

Классификация АИН:
\begin{enumerate}
\item однофазные, трехфазные (регулируемой частоты, постоянной частоты). Имеем ввиду управление трехфазным приводом. Для двигателя нужно создать 
вращающееся поле. Чаще всего 3 фазы, бывает 2 фазы, бывает много. Изучив однофащные и трехфазные, остальные получим простым увеличением числа фаз.

\item схемы нулевые, мостовые (в выпрямителях концевые).\\
Тип СПП: преобразователт из постоянного в переменное $=\rightarrow\approx$, причем автономные, т.е. независимые не ведомые сетью. Включить СПП я могу, а выключить не
получиться. Обязательно нужны запираемые приборы: запираемые тиристоры, силовые транзисторы.
1й тип будет похлж на ИППН. Наработки, полученные в ИППН будут использоваться в АИ.
\item трансформатороные, безтрансформаторные
\end{enumerate}
24в $\rightarrow$ в переменное, затем через трансформатор увеличить.

Начнем с однофазного АИН

%https://tex.stackovernet.com/ru/q/124245/%D0%BA%D0%B0%D0%BA-%D0%BD%D0%B0%D1%80%D0%B8%D1%81%D0%BE%D0%B2%D0%B0%D1%82%D1%8C-%D1%82%D1%80%D0%B0%D0%BD%D1%81%D1%84%D0%BE%D1%80%D0%BC%D0%B0%D1%82%D0%BE%D1%80-%D1%82%D0%BE%D0%BA%D0%B0-%D0%B2-tikzcircuitikz
%https://tex.stackexchange.com/questions/355717/circuittikz-4-pinfet-symbol/355802#355802
\begin{circuitikz} 
\draw (1.5,6) node[Lnigbt, bodydiode, rotate=90, xscale=-1] (t1) {};
\draw (0,6) to[short, o-] (t1.D);
\draw (t1.S) -- (3,6) -- (3,4.5); \draw (0,4.5) -- (0.5,4.5)  to[R] (1.75,4.5) to[L] (3,4.5); \draw (0,4.2) to[short,o-o] (0,4.8);
\draw (1.5,3) node[Lnigbt, bodydiode, rotate=90, xscale=1] (t2) {};
\draw (0,3) to[short,o-] (t2.D);
\draw (t2.S) -- (3,3) -- (3,4.5);
\draw (-0.2, {4.5 + 1.5/2}) node() {$U_{T/2}$};
\draw (-0.3, 5.9) node() {$+$} (0.3,4.9) node() {$-$} (0.8,4.8) node() {$-$} (2.7,4.8) node() {$+$};
\draw (0.3,4.1) node() {$+$};
\draw (-0.3, 3)  node() {$-$};
\end{circuitikz} 

Нулевая схема получится, если включил верхний транзистор. Вторую полуволну получу, если подключу нижний. Знак на транзисторе означает, что транзистор управляется напряжением.
На сомом деле это два транзистора. Иногда из-за ``нулевой'' точки схема называется нулевой.

Обратим внимание. Когда нагрузка индуктивная, то включать верхний транзистор -- будет перенапряжение. Поэтому всегдя в схеме транзисторы шунтируются обратными
диодами. При подаче напряжение $+$ $-$ -- запираем диод. Как диоды проводят? При выключении нижний диод поддержит ток при выключении верхнего транзистора. %диода.

${\displaystyle \frac{\partial i}{\partial t}}$ громадное, но установлены диоды, но кроме $L_\text{н}$ может быть индуктивность питания, то беды не миновать.
Чтобы этого не произошло, устанавливают конденсатор, Большая (полярная) емкость. Параллельно ставят с маленькой паразитной емкостью (внутренней индуктивностью)

