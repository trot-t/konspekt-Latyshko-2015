Дисциплина называется Силовая Электроника. Также есть синонимы
Силовая Преобразовательная техника, Энергетическая электроника.

Кафедра РАПС -- первая в стране кафедра электропривода.
Дисциплина начиналась с Промышленной электроники. Электронные приборы и
устройства. Есть наука о приборах и устройствах, нелинейных.

Промышленная электроника -- это продолжение ТОЭ в область нелинейных
устройств.
Может быть недостаточно для современного специалиста. Может оказаться
недостаточно

\begin{tabular}{c}
  приборы\\
  электрические цепи\\
  электронные устройства\\
  управление, хранение, обработка информации
\end{tabular}

Было 200 часов, на силовую электронику 280 часов. Сейчас даже курсового
проекта нет. Методички по курсовому проекту будут полезны для расчетов.

Дисциплина относится к силовым устройствам
\begin{itemize}
\item Генераторы Г
\item Двигатели М
\item Трансформаторы Т
\item UZ -- преобразователи электрической энергии из одного вида в другой.
\end{itemize}

Какие параметры электрической энергии известны?:
\begin{itemize}
\item $U_{mo}$ в розетке
\item $U$
\item $f$
\item число фаз
\item ``Форма''  
\end{itemize}  
  
$G$ преобразует механическую энергию в электрическую.
Аккумулятор не зарядишь переменным током. Электролиз -- аллюминий из бокситов,
требует постоянного тока.
Управление скоростью двигателей в производстве.

Чего в мире больше: $G$, $M$, $T$, $UZ$?
Лифт не круглосуточный, предприятия ночью не работают. $G$ не загружены
полностью. $T$ -- по мощности больше всего. В среднем от одного $G$ до
потребителя 3.5 $T$ трансформатора. 6,10,35kV.

Электропривод, управляющий механизм. Электролиз, гальваника.

Задан вопрос на который у самого нет ответа.

\begin{tikzpicture}
  \draw (2,0) rectangle (5,2);
  \draw[->] (0,1)--(1.8,1) node[midway, above] {$U^\prime, f^\prime$}
  node[midway, below] {$m^\prime,(\phi^\prime)$};
  \draw[->] (5,1)--(7,1) node[midway, above] {$U^{\prime\prime}, f^{\prime\prime}$}
  node[midway, below] {$m^{\prime\prime},(\phi^{\prime\prime})$};
  \draw[<-] (3.5,-0.1) -- (3.5,-0.4) node[below]{$U_\textcyrillic{упр}$};
  \draw[<-,thin] (4,-0.6) -- (4.5,-0.6) node[right]{информационный вход};
  \draw (3.5,2)node[above] {$UZ$};  
  \draw[<-,thin] (6.7,1.3)--(7.5,1.5) node[right]
       {\begin{tabular}{l}вот этим хочу\\управлять, если не\\нужно, то нерегулируемый\end{tabular}};
\end{tikzpicture}

\section{Силовые полупроводниковые приборы}
Нарисую УГО прибора:

\begin{circuitikz}
  \draw
  (0,0) node[nfet] (nfet) {}
  (nfet.drain) node[anchor=south] {сток}
  (nfet.source) node[anchor=north] {исток}
  (nfet.gate) node[anchor=east] {затвор}
  (nfet.base) node[anchor=west] {подложка}
  (2,0) node[right] {\begin{tabular}{l}транзистор полевой\\
      с изолированным каналом\\
      канал встроенный\\
  чаще контакты соединяют\end{tabular}}
  (8,0) node[nigfete] (nigfete) {};
\end{circuitikz}

\begin{circuitikz}
  \draw
  (1,0) node[nigbt] (nigbt) {}
  (3,0) node[right] {IGBT с индуцированным затвором}
  ;\end{circuitikz}


Я знаю, что читали приборы. Приборы надо знать, надо знать их характеристики,
как ими пользоваться.

Чем силовые полупроводниковые приборы отличаются от несиловых: мощность,
габариты.

\begin{circuitikz}
  \draw
  (0,1) to[Ty,n=ty1] (0,0)
  (ty1.gate) node[anchor=east] {|} node [anchor=west] {управляющий электрод}
  (0,-0.5) node[right] {запираемый тиристор}
  ;\end{circuitikz}

\begin{tikzpicture}
  \draw (0,0)rectangle(0.5,0.5)
  (0,0.5)rectangle(0.5,1)
  (0,1)rectangle(0.5,1.5)
  (0,1.5)rectangle(0.5,2)
  (0.25,0.25) node {n}
  (0.25,0.75) node {p}
  (0.25,1.25) node {n}
  (0.25,1.75) node {p}
  (0.25,-0.5)--(0.25,0) node[midway,below right] {Катод}
  (0.25,2)--(0.25,2.5) node[midway,above right] {Анод}
  (-0.5,0.75)--(0,0.75) node[midway,below left] {управляющий электрод}
  ;\end{tikzpicture}

\begin{tikzpicture}
  \draw
  (0,0) node[nmos] {};
  \draw[<-] (-0.5,0.7) -- (0,1.2) node[above right] {канал типа n};
\end{tikzpicture}  

В течении двух недель изучить по любому учебнику по СПП
(силовые полупроводниковые приборы). По программе ну отведено ни одного
часа.

Силовые полупроводниковые приборы начали применятся с середины XX века.
До этого были ионные приборы, электровакуумные.
Чем они были хуже будем оценивать с точки зрения силовой электроники.
Допустим, что нет никаких приборов, ни ионных, ни электровакуумных на планете.

Были механические преобразователи

\begin{tikzpicture}
  \draw (0,0) circle(0.5)
%  (-0.375,-0.35)--(-0.75,-0.35)--(-0.75,0.35)--(-0.357,0.35)
%  (0.357,-0.35)--(0.75,-0.35)--(0.75,0.35)--(0.357,0.35)
  (0,0) node {M}
  (-1,-1) circle(0.5)
  (-1,-1) node{Г};
  \draw[thick](-0.643,-0.65)--(-0.2,-0.2);
  \draw[->] (-0.5,0)--(-0.75,0)--(-0.75,0.75)--(-0.25,0.75)--(-0.25,1.25);
  \draw[->] (0.5,0)--(0.75,0)--(0.75,0.75)--(0.25,0.75)--(0.25,1.25);
  \draw[->] (-0.8,-1.1) -- ++(0.8,-0.4)
;\end{tikzpicture}

И получу на выходе, что что мне нужно:
$\textcyrillic{Э}\rightarrow \textcyrillic{М} \rightarrow \textcyrillic{Э}$.
У силовых преобразователей КПД самый главный параметр. В компьютере всё
по-другому.

Что такое управляемый прибор?
\begin{circuitikz}\draw
  (0,1)to[Do,v^=$ $](0,0)
  ;\end{circuitikz}

\begin{circuitikz}\draw
  (0,0)node[nmos](nmos) {}
  [thin](nmos.gate) -- ++ (-0.8,0)
  (-2,-0.6) -- (0,-0.6) node[at start, above] {управление}
;\end{circuitikz}

Сопротивление меняется, можно плавно регулировать, значит $R$ меняется,
значит не могу силовой прибор использовать.

Чем отличается ЭДС от напряжения? В розетке ЭДС или напряжение.
Напряжением будем считать ЭДС + плюс падение на внутреннем сопротивлении.

\begin{circuitikz}
  \draw
  (0,2)to[short,o-](1,2)to[R,l^=$R_\textcyrillic{нагр}$](2,2)
  (2,2)--(2,1.4)
  (2,1)circle(0.4)
  (2,0.6)--(2,0)to[short,-o](0,0)
  (2,1) node {M}
  (0,1) node {E};
  \draw[->] (1.2,1)--(1.55,1);
  \draw(2.5,1) node[right] {\begin{tabular}{l}
      $ER$ -- максимальный ток\\
      если в 2 раза уменьшить\\
      то $R$ возрастет.\\
      Не годится управлять через $R$
  \end{tabular}}
;\end{circuitikz}

{\it Все полупроводниковые приборы должны работать в ключевом режиме --
  железное правило}

Как во всяком правиле есть исключения: если мощность не очень большая.

Неразумно отводить мощность от полупроводникового прибора(ПП), поэтому
используется ключевой режим.

Импульсный режим 0,100,0,100 -- 50\%

\begin{tikzpicture}\draw
  (0,0)--(0.25,0)--(0.25,1)--(0.5,1)--(0.5,0)--(0.75,0)--(0.75,1)--(1,1)
  --(1,0)--(1.25,0)--(1.25,1)--(1.5,1)--(1.5,0)--(1.75,0)
  (0,-0.5) node[right] {500 Герц}
;\end{tikzpicture}

Таким фильтром обычно является сама нагрузка. Если этого не хватает,
то добавим фильтр.

\section{Классификация силовых полупроводниковых преобразователей
  электрической энергии}

Существует много классификаций, в основу кладется тот иил иной
классификационный признак. Мы сделаем классификацию на основе противопоставления
переменного и постоянного тока. Выпрямитель напряжения или тока?
Немного нефизично. $I$ -- производить основную работу(ампер-часы).
Но $U$ является средством получения тока. Так что можно говорить и то и другое.

\hspace{-3cm}
\begin{tikzpicture}\draw
  (0,10.3)rectangle(16.2,11.1)
  (8,10.7) node {\large Преобразователь Параметров Электрической Энергии}
  (1.9, 9.4) node {\large выпрямитель}
  (0,8.4)rectangle(3.8,9.2);
  \draw[->] (1.5,7.6) -- (0.8,6.8);
  \draw[->] (1.9,7.6) -- (2.6,6.8);
  \draw(1.9, 8.8) node {\larger[3] $\sim$ $\rightarrow$ $=$}
  (5.9, 9.4) node {\large инвертор}
  (4,8.4)rectangle(7.8,9.2)
  (5.9, 8.8) node {\larger[3] $=$ $\rightarrow$ $\sim$}
  (8,8.4)rectangle(12.2,9.2)
  (9.9,9.7) node{\larger[1] \begin{tabular}{c}Преобр. переменного\\
      напряжения\end{tabular}}
  (9.9, 8.8) node {\larger[3] $\sim$ $\rightarrow$ $\sim$}
  (14.3,9.7) node{\larger[1] \begin{tabular}{c}Преобр. импульсов\\
  постоянного напряжения\end{tabular}}
  (12.4, 8.4)rectangle(16.2,9.2)
  (14.1, 8.8) node {\larger[3] $=$ $\rightarrow$ $=$}
  (4,8) node {\begin{tabular}{c}может быть импульсное управление\\
      может быть фазное. фазное--удобнее\end{tabular}}
  (10,8) node{\begin{tabular}{c}могут быть\\фазные методы\end{tabular}}
  (14.3,7.8) node{\begin{tabular}{c}широтно-импульсная\\
      фазово-импульсный\\частотно-импульсная\end{tabular}}

  (3,7.15) node[right] {\begin{tabular}{c}в сеть,на 
      гото-\\вую синусоиду\end{tabular}};
  \draw[thin,->] (4.6,7.4)--(5,8.35);
  \draw
  (4,5.7)rectangle(6,6.7)
  (5,6.2) node{\begin{tabular}{c}Зависимый\\инвертор\end{tabular}} 
  (4.8,4.4) node{\begin{tabular}{c}солнечная\\батарея\\
      в розетке 50Гц\\я должен дать 50\\
      по форме,частоте,\\амплитуде\end{tabular}}  
  (6.2,5.7)rectangle(8.5,6.7)
  (7.3,6.2) node{\begin{tabular}{c}Автономный\\инвертор\end{tabular}}
  (7.3,5.2) node{\begin{tabular}{c}паяльник\\я сам хозяин\end{tabular}};
  \draw[->] (7.1,4.8) -- (6.1, 3.1);
  \draw[->] (7.3,4.8) -- (7.3, 3.1);
  \draw[->] (7.5,4.8) -- (8.5, 3.1);
  \draw (5.6,2.2)rectangle(6.6,3.0)
  (6.1,2.6) node {АИН}
  (6.8,2.2)rectangle(7.8,3.0)
  (7.3,2.6) node {АИТ}
  (8,2.2)rectangle(9,3)
  (8.5,2.6) node {АИР}
  (-1.2,5.7)rectangle(1.4,6.7)
  (0.1,6.2) node{\begin{tabular}{c}неуправляемый\\выпрямитель\end{tabular}}
  (1.6,5.7)rectangle(3.8,6.7)
  (0.1,5.0) node{\begin{tabular}{c}диод,управля-\\емый силовой\\
      цепью\end{tabular}}
  (2.7,6.2) node{\begin{tabular}{c}управляемый\\выпрямитель\end{tabular}};
  \draw[->] (2.5,5.6)--(0.2,3.1);
  \draw[->] (2.7,5.6)--(2.6,3.1);
  \draw(-1.2,2.2)rectangle(1.2,3.0)  
  (0,2.6) node {нереверсивные}
  (1.4,2.2)rectangle(3.8,3.0)
  (2.6,2.6) node {реверсивные}
  (2.6,1.4) node {\begin{tabular}{c}2 тиристора\\в обоих\\напралениях
  \end{tabular}}
  ;  
  \draw[->](6.1,7.5)--(5.7,6.75);
  \draw[->](6.3,7.5)--(7.3,6.75);

  \draw (10.4,6.2) node {\begin{tabular}{c}частоту,число фаз\\
      если только\\уровень, то\\
      регулятор переменного\\
      напряжения
  \end{tabular}}
  (9.0,3.7)rectangle(10.0,4.5)
  (9.5,4.1) node {РПН}
  (10.8,3.7)rectangle(11.8,4.5)
  (11.3,4.1) node {НПЧ}
  (11.2,2.8) node {\begin{tabular}{c}непосредст-\\венный\\преобразова-\\
      тель частоты\end{tabular}};
  \draw[->] (10.2,5.2)--(9.5,4.6);
  \draw[->] (10.6,5.2)--(11.3,4.6);
  \draw (14.2,6.2) node {\begin{tabular}{c}только уровень\\
      частота ``0''\\формы нет\end{tabular}}
  (12.4,4)rectangle(13.8,4.8)
  (13.1,4.4) node{\begin{tabular}{c}ревер-\\сивный\end{tabular}}
  (14,4)rectangle(15.8,4.8)
  (14.9,4.4) node{\begin{tabular}{c}неревер-\\сивный\end{tabular}};
  \draw[->](14,5.6)--(13.1,4.9);
  \draw[->] (14.4,5.6)--(14.9,4.9)
;\end{tikzpicture}

АИН --формирует на выходе кривую напряжения.
АИТ -- формирует на выходе кривую тока.
Помним, что все приборы работают в ключевом режиме.

\begin{circuitikz}\draw
  (0,0)--(1.5,0)to[short,-o](1.5,0.75)to[short,o-*](1.5,1.5)
  to[short,*-o](1.5,2.25)--(1.5,3)--(0,3)--(3,3)
  to[short,-o](3,2.25)to[short,o-*](3,1.5)to[short,*-o](3,0.75)
  to[short,o-](3,0)--(1.5,0)
  (0,3) node[left] {{\large +}}
  (0,0) node[left] {{\large --}}
  (1.5,1.5)to[european resistor](3,1.5)
  (1.5,2.25) node[above left] {1}
  (1.5,0.75) node[below left] {2}
  (3,2.25) node[above right] {3}
  (3,0.75) node[below right] {4};
  \draw[->] (1,2.25)--(1.35,2.25);
  \draw[->] (1,0.75)--(1.35,0.75);
  \draw[->] (2.5,2.25)--(2.85,2.25);
  \draw[->] (2.5,0.75)--(2.85,0.75);
  \draw[<-] (3.45,2.4)--(4,3) node[right]{заменяем рубильниками};
  \draw (4.5,2.5)--(5.5,2.5)--(5.5,1.8)
  to[european resistor](7,1.8)--(7,1)--(8,1);
  \draw[<-] (6.5,2.1)--(7.5,2.3) node[right]
       {ток течет слева направо};
  \draw (4.5,-1.2)--(5.5,-1.2)--(5.5,-0.4)
  to[european resistor](7,-0.4)--(7,0.4)--(8,0.4);
  \draw (-2,1)--(-1,1)
  (-2,-1)--(-1,-1)
  (-1.5,-1)to[C,*-*](-1.5,1)
  (-1.5,1.2) node[above] {включаем};
  ;\end{circuitikz}

\begin{tikzpicture}
  \draw[thin,->] (0,0)--(5,0) node[right,below] {t};
  \draw[thin,->] (0,-1.2)--(0,1.3) node[left] {$I$};
  \draw[thin,->] (0,3)--(5,3) node[right,below] {t};
  \draw[thin,->] (0,1.8)--(0,4.3) node[left] {$U$};
  \draw (0,0)--(0.5,1)--(1.5,-1)--(2.5,1)--(3.5,-1)--(4.5,1)
  (0,4)--(0.5,4)--(0.5,2)--(1.5,2)--(1.5,4)--(2.5,4)--(2.5,2)--
  (3.5,2)--(3.5,4)--(4.5,4)--(4.5,2);
  \draw[<-] (5,3.5)--(6,3.5) node[right] {Это напряжение};
  \draw[<-] (5,2.2)--(6,1.8) node[right]
       {\begin{tabular}{c}а также ток\\при активном сопротивлении
       \end{tabular}};
       \draw[<-] (5,-0.5)--(6,-1) node[right] {индуктивное};
 \draw (-0.8,0) node[left] {${\displaystyle L\frac{di}{dt}=U}$}
 ;\end{tikzpicture}

Конденсатор ставить в качестве нагрузки нельзя

${\displaystyle i_c = C\frac{du_c}{dt}}\;\;\;$
${\displaystyle \frac{du}{dt}=\infty}$

Если нельзя, но очень хочется то можно

\begin{tikzpicture}\draw
  (0,0)--(2.5,0)--(2.5,0.75)to[short,o-o](2.5,2.25)--(2.5,3)
  (0,3)--(1,3)to[L](2.5,3)
  (1,0)to[C,*-*](1,3)
  (2.5,1.5)to[european resistor,*-](3.5,1.5)to[C](4.5,1.5)to[L,-*](5.5,1.5)
  (2.5,0)--(5.5,0)--(5.5,0.75)to[short,o-o](5.5,2.25)--(5.5,3)--(2.5,3);
  \draw[<-] (1.85,3.3)--(3,3.6) node[right] {добавим};
  ;\end{tikzpicture}

Включаем в нагрузку. Ток в этой индуктивности

\begin{tikzpicture}
  \draw[thin,->](0,-1)--(0,1.3) node[left] {$I$};
  \draw[thin,->] (0,0)--(5,0) node[right, below] {$t$}; 
  \draw(0,1)--(0.5,1)--(0.5,-1)--(1.5,-1)--(1.5,1)--(2.5,1)--(2.5,-1)--
  (3.5,-1)--(3.5,1)--(4.5,1)--(4.5,-1);  
\end{tikzpicture}

Если вставили индуктивность на входе, то теперь индуктивность в нагрузку нельзя.
Если на входе индуктивность, то индуктивность в нагрузку нельзя.

