Коротко о том,что прошли.

Рассматривали ИППН, классификацию ИППН, классификация в основном по квадрантам. Рассматривали
одно-квадрантные,двух-квадрантные, четырех-квадрантные.

\begin{circuitikz}\draw
%  (0,2)to[short,o-](1,2)
  (1.5,2) node[nigbt,rotate=90](nigbt1){}
  (1.5,2) node[above] {VT}
  (0,2)to[short,o-] (nigbt1.C)
  (nigbt1.E)-- (3,2)to[L,-o](5,2)
(0,0)to[short,o-o](5,0)
(3,0)to[Do,l_=$VD$,*-*](3,2)
  ;\end{circuitikz}

Замечание: может стоять IGBT-транзистор, может стоять мосфет,
\begin{circuitikz}\draw
  (0,0)node[nigfete](nigfete){}
  (nigfete.B) node[anchor=west] { подложка}
  (nigfete.G) node[anchor=east] {затвор }
  (nigfete.D) node[anchor=south] {сток}
  (nigfete.S) node[anchor=north] {исток}
;\end{circuitikz}

а может стоять обычный тиристор с углом искуственной коммутации:  

\begin{circuitikz}\draw
  (0,5)to[short,o-*](1,5)--(1,3)to[C,v_>=$ $](2.5,3)
  (1,3)to[C,v^<=$ $](2.5,3)--(3,3)to[Do,*-*](3,5)
  (1,5)to[Ty](3,5)--(4.5,5)to[L](4.5,4)to[Do](4.5,3)--(3,3)
  (4.5,5)--(6,5)
  (0,0)to[short,o-*](6,0)to[Do,*-*](6,5)--(8,5)to[L,l=$L_\textcyrillic{н}$](8,3.5)
  to[R,l=$R_\textcyrillic{н}$](8,2)
  (8,1)circle(0.35)
  (8.35,1)node[right]{$E_\textcyrillic{н}$}
  (6,0)--(8,0)--(8,0.65)
  (8,1.35)--(8,2);
  \draw[thin,<-] (1.75,2.6) -- (1.75,2) node[below]{должен зарядиться так};
  \draw[thin,<-] (1.75,3.4) --(1.2,3.6) node[above]{исходный заряд};
  \draw[thin,<-] (6.3,2.5) -- (6.7,2.5) node[below,rotate=90] {оставим этот диод};
  \draw[dashed] (1.2,5)-- (1.2,6);
  \draw (1.2,6)to[short,-o](1.2,6.1)node[left] {\large{-}};
  \draw[dashed] (2.8,5)-- (2.8,6);
  \draw (2.8,6)to[short,-o](2.8,6.1)node[right] {\large{+}};
;\end{circuitikz}

Искусственная коммутация $\cong$ принудительная коммутация $\cong$ ёмкостная коммутация.
Искусственная коммутация и  принудительная коммутация -- синонимы.
\begin{circuitikz}
\begin{scope}[scale=0.75]
  \draw[dashed] (1.2,5)-- (1.2,6);
  \draw (1.2,6)to[short,-o](1.2,6.1)node[left] {\large{-}};
  \draw[dashed] (2.8,5)-- (2.8,6);
  \draw (2.8,6)to[short,-o](2.8,6.1)node[right] {\large{+}};
  \end{scope}
  ;\end{circuitikz} -- Кратковременно подключить, искусственно включить, принудительно
включить источник. Чаще всего таким источником является заряженный конденсатор. 

\begin{circuitikz}
\begin{scope}[scale=0.75]
  \draw (1.2,5)-- (1.2,6);
  \draw (1.2,6)to[short,-o](1.2,6.1)node[left] {\large{-}};
  \draw (2.8,6)to[ospst] (2.8,5);
  \draw (2.8,6)to[short,-o](2.8,6.1)node[right] {\large{+}};
  \draw[thin,red,->] (0,5.2) -- (0.4,5.2) arc(-90:0:0.2) -- (0.6,6)  
  arc(180:90:0.6)  --(2.8,6.6) node[midway,above] {ток}
  arc(90:0:0.6)-- (3.4,5.4) arc(180:270:0.2) -- (3.9,5.2);
  \end{scope}
;\end{circuitikz} -- источник перехватывает \underline{ток} нагрузки.
Но главная задача -- отключить нагрузку.

Это делается в два этапа:
\begin{itemize}
\item запереть тиристор
\item отключить нагрузку
  \end{itemize}

\begin{circuitikz}
\begin{scope}[scale=0.75] 
  \draw (0,5)-- (1.2,5)-- (1.2,6) to[C,l_=$C$](2.8,6)--(2.8,5) -- (4,5);
  \draw[thin,red,->] (0,5.2) -- (0.4,5.2) arc(-90:0:0.2) -- (0.6,6)  
  arc(180:90:0.6)  --(2.8,6.6) node[midway,above] {ток}
  arc(90:0:0.6)-- (3.4,5.4) arc(180:270:0.2) -- (3.9,5.2);
  \end{scope}
;\end{circuitikz} -- конденсатор идеально подходит для обоих этапов. Конденсатор перезаряжается
и перехватывает энергию. Ёмкостная коммутация -- частный случай искуссственной коммутации,
когда источником является конденсатор.

Можем использовать импульсный трансформатор. 
