%Краткое напоминаю, что в конце предыдущей лекции мы изучали реверсивные
%преобразователи.
Под реверсией понимают изменение чего-либо на противопожное.
Если иметь ввиду двигатель, то реверс означает вращение в другую сторону.
Напряжение реверсируется даже в нереверсивном преобразователе.
Для того чтобы ток был реверсивным нужно как минимум два комплекта вентилей.
В классификации реверсивных преобразователей в первом классе были
преобразователи с одной группой вентилей. Переключение тока происходило
посредством переключения полюсов нагрузки с полюсами питания.
В истинно реверсивных преоюразователях существуют две группы вентилей.

\subsection{перекрестрая схема}
исторически самая первая.

\begin{figure}[H]
\begin{circuitikz}\draw
  (0,3.5) to[L] ++ (0,-1)
  to[Ty,l_=$\begin{array}{c}
      \textcyrillic{1я группа}\\
      \textcyrillic{вентилей}\end{array}$] ++ (0,-1.5)
  -- ++ (1,0)
  (0,3.5) -- ++ (1,0)
  to[L] ++ (0,-1)
  to[Ty,-*] ++ (0,-1.5)
  -- ++ (1,0)
  (1,3.5) -- ++ (1,0)
  to[L] ++ (0,-1)
  to[Ty,-*] ++ (0,-1.5)
  % уравнительный реактор Ур1
  -- ++ (1.5,0)
  to[L,l={$\textcyrillic{Ур1}$}] ++ (0,1)
  -- ++ (1.5,1.5)
  -- ++ (1,0)
  % 2я группа вентилей
  to[L] ++ (0,-1)
  to[Ty,-*] ++ (0,-1.5)
  -- ++ (1,0)
  (6,3.5) -- ++ (1,0)
  to[L] ++ (0,-1)
  to[Ty,-*] ++ (0,-1.5)
  -- ++ (1,0)
  (7,3.5) -- ++ (1,0)
  to[L] ++ (0,-1)
  to[Ty,l=$\begin{array}{c}
      \textcyrillic{2я группа}\\
      \textcyrillic{вентилей}\end{array}$] ++ (0,-1.5)
  % уравнительный реактор Ур2
  (6,1) -- ++ (-1.5,0)
  to[L,l_={\textcyrillic{Ур2}}] ++ (0,1)
  -- ++ (-1.5,1.5)
  -- ++ (-1.5,0)
  % мотор
  (1,1) -- ++ (0,-1.5)
  -- ++ (2.5,0)
  (4,-0.5) circle (0.5 cm)
  (4,-0.5) node {M}
  (4.5,-0.5) -- ++ (2.5,0)
  -- ++ (0, 1.5)
  % первичная сторона
  (3, 5) to[L] ++ (0,-1)
  -- ++ (1,0)
  (4, 5) to[L] ++ (0,-1)
  -- ++ (1,0)
  (5, 5) to[L] ++ (0,-1)
  % core
  [thick] (0,3.74) rectangle (8,3.77)
  ;\end{circuitikz}
\caption{перекрестная схема реверсивного преобразователя}
\end{figure}

$\textcyrillic{Ур1}$ и $\textcyrillic{Ур1}$ -- уравнительные реакторы.
Силовые индуктивности рисуются как
\begin{circuitikz}\draw
  (0,0.8) -- ++ (0,-0.8)
  (0.5,0) arc(0:270:0.5)
  (0.5,0) -- ++ (-0.5,0)
  (0,-0.5) -- ++ (0,-0.3)
  ;\end{circuitikz}
 или как  
\begin{circuitikz}\draw
(0,0.8) --++ (0,-0.3)
(-0.5,0) arc(-180:90:0.5)
(-0.5,0)--++(0.5,0)
(0,0)--++(0,-0.8)  
  ;\end{circuitikz}

\begin{figure}[H]
  \begin{circuitikz}\draw
    (0,1) to[L,-*] ++(2,0)
    to[L] ++(2,0)
    --++(0,1)
    to[Ty,*-] ++(-1.3,0)
    --++(-1.4,0)
    to[Ty,*-*] ++(-1.3,0)
    --++(0,-1)
    (4,2)--++(0,1)
    to[Ty,*-*]++(-1.3,0)
    --++(-1.4,0)
    to[Ty,-*]++(-1.3,0)
    --++(0,-1)
    (4,3)--++(0,1)
    to[Ty]++(-1.3,0)
    to[short,-*]++(-0.7,0)
    --++(-0.7,0)
    to[Ty]++(-1.3,0)
    --++(0,-1)
    %LLL
    (1.3,2)--++(0,2)
    to[L]++(0,1.3) %La'
    --++(-0.4,0)
    --++(0,1.7)
    --++(1.8,0)
    (2,4) to[L]++(0,1.3) %Lb
    --++(-0.4,0)
    --++(0,1.5)
    --++(-0.3,0)
    (1.3,5.5) to[L]++(0,1.3) %La''

    (1.3,5.5)--++(0.7,0)
    to[L,*-]++(0,1.3) %Lb''
    (2.7,3)--++(0,1)
    to[L]++(0,1.3) %Lc'
    --++(-0.4,0)
    --++(0,1.5)
    --++(-0.3,0)
    (2,5.5)--++(0.7,0)
    to[L,*-]++(0,1.3) %Lc''
    --++(0,0.2)
    %мотор
    (2,1)--++(0,-0.5)
    (2,0) circle (0.5cm)
    (2,0) node {M}
    (2,-0.5)--++(0,-0.5)
    --++(2.3,0)
    --++(0,6.5)
    --++(-1.6,0)
    %core
    [thick] (0.9,7.2) rectangle (3.1,7.22)
    % первичная сторона
    (1.3,7.4) to[L]++(0,1.3)
    (2,7.4) to[L]++(0,1.3)
    (2.7,7.4) to[L]++(0,1.3)
    (1.3,7.4) -- (2.7,7.4)
    ;\end{circuitikz}
\caption{встречно-параллельная схема} 
\end{figure}


\begin{figure}[H]
  \begin{circuitikz}\draw
    (0,1) to[Ty] ++(1.3,0)
    --++(1.6,0)
    (4,1) to[Ty,-*]++(-1.3,0)
    %
    (0,1) --++(0,1)
    to[Ty,*-] ++(1.3,0)
    to[short,-*] ++ (0.7,0)
    --++(0.7,0)
    (4,1) --++ (0,1)
    to[Ty,*-] ++(-1.3,0)
    %
    (0,2) --++(0,1)
    to[Ty,*-*] ++(1.3,0)
    --++(1.6,0)
    (4,2) --++(0,1)
    to[Ty,*-] ++(-1.3,0)
    % LLL
    (1.3,-0.3) to[L] ++(0,1.3)
    --++(0,2)
    (2,-0.3) to[L] ++(0,1.3)
    --++(0,1)
    (2.7,-0.3) to[L] ++(0,1.3)
    % мотор
    (0,3) to[short,-*]++(0,1)
    --++(1.5,0)
    (2,4) circle (0.5cm)
    (2,4) node {M}
    (4,3) to[short,-*]++(0,1)
    --++(-1.5,0)
    %
    (0,4)--++(0,3)
    (4,4)--++(0,3)
    (1.3,5) to[Ty,-*]++(-1.3,0)
    (2.7,5) to[Ty,*-*]++(1.3,0)
    (1.3,5)--++(1.4,0)--++(0,2)
    %
    (1.3,6) to[Ty,-*]++(-1.3,0)
    (2.7,6) to[Ty,-*]++(1.3,0)
    (1.3,6) to[short,-*]++(0.7,0)--++(0.7,0)
    (2,6)--++(0,1)
    %
    (1.3,7) to[Ty,*-*]++(-1.3,0)
    (2.7,7) to[Ty,-*]++(1.3,0)
    (1.3,7)--++(1.4,0)
    % LLL
    (1.3,7) to[L]++(0,1.3)
    (2,7)   to[L]++(0,1.3)
    (2.7,7) to[L]++(0,1.3)
    % Lурав
    (1.3,-0.3)--++(3.7,0)
    --++(0,3.8)
    to[L,l_={$L_\textcyrillic{урав}$}]++(0,1)
    --++(0,3.8)
    --++(-3.7,0)
    % core
    [thick] (1,8.4) rectangle (3,8.43)
    % первичная сторона
    (1.3,8.53)--++(1.4,0)
    (1.3,8.53) to[L]++(0,1.3)
    (2,8.53) to[L]++(0,1.3)
    (2.7,8.53) to[L]++(0,1.3)
    ;\end{circuitikz}
\caption{H-схема}
\end{figure}  
